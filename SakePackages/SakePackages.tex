% $Author$
% $Date$
% $Revision$

% HISTORY:
% 2008-09-29 - Damien Cassou started
% 2009-01-21 - Stef revised

%=================================================================
\ifx\wholebook\relax\else
% --------------------------------------------
% Lulu:
	\documentclass[a4paper,10pt,twoside]{book}
	\usepackage[
		papersize={6.13in,9.21in},
		hmargin={.75in,.75in},
		vmargin={.75in,1in},
		ignoreheadfoot
	]{geometry}
	\input{../common.tex}
	\setboolean{lulu}{true}
% --------------------------------------------
% A4:
%	\documentclass[a4paper,11pt,twoside]{book}
%	\input{../common.tex}
%	\usepackage{a4wide}
% --------------------------------------------
    \graphicspath{{figures/} {../figures/}}
	\begin{document}
\fi
%=================================================================
%\renewcommand{\nnbb}[2]{} % Disable editorial comments
\sloppy
%=================================================================
\chapter{Sake and Sake-Packages}
\chalabel{sake}

Sake is a library to define tasks based on pre-conditions and
dependent tasks. It allows people to describe tasks (as done in the
Ant/Java world) and execute these tasks when necessary.  Sake-Packages is
another library which uses Sake to implement a packaging system with
dependencies, as in the GNU/Linux world. Then, it describes a lot of
Squeak packages with their dependencies to ease their installation.

To install Sake and Sake-Packages:

\begin{code}
  Installer install: 'Packages'
\end{code}

This uses Installer (see \charef{installer}) which will take care of
installing both Sake-Packages and its dependent library Sake.

\section{Sake}
\seclabel{sake}

Sake proposed a domain specific language in Smalltalk to declare
targets as in Make and Ant. It is most similar to the Ruby
implementation Rake proposed by Martin
Fowler\footnote{\url{http://rake.rubyforge.org}}.

Sake defines the concept of a task (in the class \clsind{SakeTask}) with:
\begin{description}
\item[An action] a block of instructions;
\item[A condition] a block whose value will determine if the action
  needs to be executed;
\item[Dependencies] a set of other tasks to run before evaluating the
  current one.
\end{description}

The algorithm to run a task is defined in
\cmind{SakeTask}{privateRunTakeOrdering:} and can be described like:

\begin{enumerate}
\item if the task has already been ran then stop;
\item run all dependent tasks;
\item evaluate the condition, if it is false then stop;
\item execute the action.
\end{enumerate}

The condition ensures that the task is executed only if necessary,
\eg{} if the task installs a package inside the image, the condition
will verify that the package is not already in.

A dependent task may also have dependent tasks and the previous
algorithm will take care of running everything in the right order.

A task is defined in a method like in:

\begin{method}{Typical task}
MakeTheTea>>fillCupOfTea
  ^ SakeTask define: 
                   [ :task |
                        task dependsOn: { self taskBoilWater }.
                        task if: [ Cup isEmpty ].
                        task action: [ Cup add: 'tea'; add: 'water'. ].
\end{method}

The method \ct{fillCupOfTea} returns a new task whose role is to fill
an empty cup with tea and boiled water. The method \ct{taskBoilWater}
returns a new task which is used as the sole dependency of the
\ct{fillCupOfTea} task.

The methods \ct{addDependency:} and \ct{addAction:} can be used for
respectively adding a new dependency or a new action to a task, instead of or in conjunction with \ct{dependsOn:} and \ct{if:}.

Tasks can also be added as conditions of another task (in the \ct{if:}
block). In that case, if the task succeeds then the action is
performed, if the task fails then the action is considered not needed.

To run this task you would execute

\begin{code}{}
  | aTeaMaker |
  aTeaMaker := MakeTheTea new.
  aTeaMaker fillCupOfTea run.
\end{code}

The last line will first run the task \ct{taskBoilWater}, then ensures
the Cup is empty and if it is add the tea and the water.

Alternatively \ct{runStepping}, or \ct{runLogging} are available for
debugging.

To see the hierarchy of tasks that will be traversed execute

\begin{code}{}
  aTeaMaker fillCupOfTea what explore.
\end{code}

\important{For implementation reasons, a method can only define at
  most one task!}

\subsection{Class-based tasks}

Sake also provides mechanisms to deal with Smalltalk classes inside
tasks. This is implemented by the class \clsind{SakeClassTask}.

The following describes some examples of class-based tasks.

\begin{code}{}
myTask: className
  ^ self define: [:task |
      task
        if: {SakeClassTask exists: className};
        action: [...]
    ]
\end{code}

The previous task action will be executed if and only if there is a
class named \ct{className} in the system. This code also illustrates
that tasks can be parametrized with a variable (here \ct{className} is
a method parameter).

\begin{code}{}
(SakeTask class: #A) subclass: #B
\end{code}

This returns a task whose evaluation will result in the creation of a
new class named \ct{B} sub-classing the class \ct{A}. This task
depends on the existence of the class \ct{A}.

\begin{code}{}
(SakeTask class: #Cup) removeSelector: #tmp
\end{code}

This returns a class which, if ran, will remove the method \ct{tmp}
from the class \ct{Cup}.

\subsection{Controlling evaluation}
\seclabel{contr-eval}

Facilities exist to control the execution of a task. In the action block you could use the following:

\begin{code}{}
task step: 'Are you sure?'
\end{code}

This will ask confirmation from the user before going further. If the user cancels the dialog box, the task is canceled.

\begin{code}{}
task stop: 'Error message'
\end{code}

The method \ct{stop} displays an error message and cancels the task.

The method \ct{answer:with:} is used to provide an automatic answer to a question asked by some code your task will ultimately run.

\begin{code}{}
SakeTask define: [:task |
  task if: (SakeTask class: #WAKom) missing.
  task answer: '.*username.*' with: 'admin'.
  task answer: '.*password.*' with: 'seaside'.
  task answer: '.*port number.*' with: '8080'.
  task action: [Installer install: 'Seaside']]
\end{code}

\section{Sake-Packages}
\seclabel{sake-packages}



%=============================================================
\ifx\wholebook\relax\else
   \bibliographystyle{jurabib}
   \nobibliography{scg}
   \end{document}
\fi
%=============================================================




%-----------------------------------------------------------------

%%% Local Variables:
%%% coding: utf-8
%%% mode: latex
%%% TeX-master: t
%%% TeX-PDF-mode: t
%%% ispell-local-dictionary: "english"
%%% End: