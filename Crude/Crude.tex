% $Author: ducasse $
% $Date: 2009-08-24 10:17:33 +0200 (Mon, 24 Aug 2009) $
% $Revision: 28563 $

%=================================================================
\ifx\wholebook\relax\else
% --------------------------------------------
% Lulu:
	\documentclass[a4paper,10pt,twoside]{book}
	\usepackage[
		papersize={6.13in,9.21in},
		hmargin={.75in,.75in},
		vmargin={.75in,1in},
		ignoreheadfoot
	]{geometry}
	\input{../common.tex}
	\setboolean{lulu}{true}
% --------------------------------------------
% A4:
%	\documentclass[a4paper,11pt,twoside]{book}
%	\input{../common.tex}
%	\usepackage{a4wide}
% --------------------------------------------
     \graphicspath{{figures/} {../figures/}}
	\begin{document}
\fi
%=================================================================
%\renewmessage{\nnbb}[2]{} % Disable editorial comments
\sloppy

%=================================================================
%\renewcommand{\nnbb}[2]{#2} % Disable editorial comments


\chapter{Crude in Pharo}
\chapterauthor{\authorsven{}}

This is a tutorial showing how to implement a small but non-trivial web application in Smalltalk using Seaside, Glorp and PostgreSQL. Reddit, is web application where users can post interesting links that get voted up or down. The idea is that the 'best' links end up with the most points automatically. Many other websites exist in the area of social bookmarking, like Delicious, Digg and Hacker News.

The idea of the Chapter is to show you it is perfectly possible to write nice web applications in Pharo and really fast and 
Reddit.st adds persistency in a relational database, unit tests as well as web application components to the mix.

The 10 main sections of this article follow the development of the 10 classes making up the application. The focus of the Smalltalk version is not so much on the small size or the high developer productivity, but more on the fact that we can cover so much ground using such powerful frameworks, as well as the natural development flow from model over tests and persistence to web GUI.

The material shown in this Chapter was originally written by Sven Van Caekenberghe and we thanks him for his permission to use it to create this book chapter.  We assume that you understand what web applications are and how Seaside basically works. If not, you should the Seaside Chapter for an introduction \sd{should be in the other book I guess or in omnibus}. We also will assume that you have a basic understanding of relational databases and/or SQL.


\section{First: a Model}

The central object of our application will be \ct{RedditLink}, an interesting URL with a title, a created timestamp and a number of points. It has the following properties: \ct{id url title created points}.

These are naturally instance variables of our class. Create a new class inheriting from \ct{Object} subclass by editing the class template.

\begin{code}{}
Object subclass: #RedditLink 
   instanceVariableNames: !\textbf{'id url title created points'}! 
   classVariableNames: '' 
   poolDictionaries: '' 
   category: 'Reddit'
\end{code}

Next,  use the class refactoring tool to automatically generate accessors (getters and setters) for all our instance variables. With these implemented we can write our \ct{initialize} and \ct{printOn:} methods.

\begin{code}{}
RedditLink>>initialize
    self initialize.
    self points: 0.
    self created: TimeStamp now

RedditLink>>printOn:
    super printOn: stream. 
    stream nextPut: $(. 
    self url printOn: stream. 
    stream nextPut: $,.
    self title printOn: stream. 
    stream nextPut: $)
\end{code}

We also add a method named \ct{posted}  that will return the \ct{Duration} of time the link now exists. 
We will need that when rendering links later on. 
\begin{code}{}
RedditLink>>posted
    ^ TimeStamp now - self created
\end{code}

Apart from creating and displaying RedditLinks, users should be able to vote them up and down. Therefore, we add two action methods, \ct{voteUp} and \ct{voteDown}.

\begin{code}{}
RedditLink>>voteUp
    self points: self points + 1

RedditLink>>voteDown
    self points > 0 ifTrue: [ self points: self points - 1 ]
\end{code}


We also introduce the class method \ct{#withUrl:title:} to create new instances as follows

\begin{code}{}
RedditLink class>>withUrl: url title: title
   ^ self new url: url; title: title; yourself
\end{code}

The core of the RedditLink object is now finished. Everything is ready to make instances and use them.


\section{RedditLinkTests}
\sd{may be we should write it tests first: we will see}

Units tests are very important, not so much in small examples like this one, but especially in larger applications. Having a good set of unit tests with descent coverage helps protect the code during changes. At the same time, unit tests function as working documentation. Instead of writing scratch test code in some workspace, you can just as well write a unit test. We created the class \ct{RedditLinkTests} as a subclass of \ct{TestCase} and add 3 test methods.


\begin{code}{}
RedditLinkTests>>testInitialState
| link | link := RedditLink new. link assertContractUsing: self. self assert: link points isZero
\end{code}

\begin{code}{}
RedditLinkTests>>testCreate
| link url title | url := 'http://www.seaside.st'. title := 'Seaside'. link := RedditLink withUrl: url title: title. link assertContractUsing: self. self assert: link points isZero. self assert: link url = url. self assert: link title = title
\end{code}

\begin{code}{}
RedditLinkTests>>testVoting
| link | link := RedditLink new. link assertContractUsing: self. self assert: link points isZero. link voteUp. self assert: link points = 1. link voteDown. self assert: link points isZero. link voteDown. self assert: link points isZero
\end{code}


\section{}
\section{}

\section{}
\section{}
\section{}

\section{}
\section{}

\section{}


%=========================================================
\ifx\wholebook\relax\else
    \bibliographystyle{jurabib}
    \nobibliography{scg}
    \end{document}
\fi
%=========================================================



%%% Local Variables: 
%%% coding: utf-8
%%% mode: latex
%%% TeX-master: Lint.tex
%%% TeX-PDF-mode: t
%%% End:
