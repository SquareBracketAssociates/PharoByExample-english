% $Author$
% $Date$
% $Revision$

% HISTORY:
% 2007-10-03 - Stef started chapter
% 2009-04-26 - Stef started writing again

%=================================================================
\ifx\wholebook\relax\else
% --------------------------------------------
% Lulu:
	\documentclass[a4paper,10pt,twoside]{book}
	\usepackage[
		papersize={6.13in,9.21in},
		hmargin={.75in,.75in},
		vmargin={.75in,1in},
		ignoreheadfoot
	]{geometry}
	\input{../common.tex}
	\setboolean{lulu}{true}
% --------------------------------------------
% A4:
%	\documentclass[a4paper,11pt,twoside]{book}
%	\input{../common.tex}
%	\usepackage{a4wide}
% --------------------------------------------
    \graphicspath{{figures/} {../figures/}}
	\begin{document}
\fi
%=================================================================
%\renewcommand{\nnbb}[2]{} % Disable editorial comments
\sloppy
%=================================================================
\chapter{Concurrency}\chalabel{basic}

\pharo as any Smalltalk is a sequential language since at one point in time there is only 
one computation carried on. However, Smalltalk has the ability to run programs concurrently by interleaving their executions. The idea behind Smalltalk was to propose a complete OS and as such a Smalltalk run-time offers the possibility to execute different processes (or threads) that are scheduled by the Smalltalk process scheduler. 

Smalltalk's concurrency is \emph{collaborative} and \emph{preemptive}. It is preemptive because a process with higher priority can interrupt the current running one. It is collaborative because the current process should explicitly release the control to give a chance to the other processes of the same priority can get executed by the scheduler. 

In this chapter we present how processes are created and their lifetime. We will show how the process scheduler manages the system. We will present one basic abstraction proposed by:  Semaphore and the critical section. 

The subsequent chapter will present the other abstractions offered by \pharo: Mutex, Monitor, Delay...


%=================================================================
\section{Processes}
\pharo supports the concurrent execution of multiple programs using independent processes. These processes are lightweight processes as they share a common memory space. Such processes are instances of the class \ct{Process}. In other languages there are often called \ct{threads} or green threads.

\paragraph{Launching a process. }
To execute concurrently a program, we specify such a program in a block and send it the message \ct{fork}.

\begin{code}{}
[1 to: 10 do: [:i | Transcript show: i  printString ; cr]] fork
\end{code}

This expression creates an instance of the class \ct{Process}. It is added to the list of scheduled processes of the process scheduler. We say that this process is \emph{runnable}, it can be potentially executed. It will be executed when the process scheduler will schedule it as the current running process. At this moment the block of this process will be executed. 


We can also create a process which is not scheduled using the message \ct{newProcess}. 

\begin{code}{}
| pr | 
pr := [1 to: 10 do: [:i | Transcript show: i  printString ; cr]] newProcess
pr resume
\end{code}

This process is not \emph{runnable}, it is not added to the list of the scheduled processes of the process scheduler. It can be scheduled sending it the message \ct{resume}. 

You can also pass arguments to a process with the message \ct{newProcessWith:} as follows: 

\begin{code}{}
| pr | 
pr := [ :max |
		1 to: max do: [:i | Transcript show: i  printString ; cr]] newProcessWith: #(20).
pr resume
\end{code}



A process can also be temporarily suspended (\ie stopped) using the message \ct{suspend}. A suspended processed can be rescheduled using the message \ct{resume}. Now we can also terminate a process using the message terminate. 
A terminated process cannot be scheduled any more. 

\section{Process possible states}
From the description above we can see that a process can be in four different states: \emph{running}, \emph{runnable}, \emph{suspended}, and \emph{terminated}. As we will see in the following of this chapter, a process can also be \emph{waiting} when a semaphore does not allow it to proceed. A process can be 5 different states (See Figure~\ref{fig:processState}).

\begin{figure}
\begin{center}
\includegraphics[width=6cm]{ProcessState}
\caption{Process possible states.\figlabel{processState}}
\end{center}
\end{figure}


\begin{description}
\item [Runnable.] A process has been created but not run. It is in the scheduler lists and could get the execution control. In Figure~\ref{fig:processStateInContext}, the processes \ct{P4} and \ct{P8} are runnable. The processor can select one of them as the current active process. 

\item [Running.] A process has been created and scheduled. Now it is running and executing a program. At the end of its execution it will be terminated. As we will see later it may release  the control (by sending the message \ct{yield} to \ct{Processor} the process scheduler) so that other processes may be scheduled and run.  In such a case it will be runnable, \ie waiting to get again the control. 

\item [Terminated.] When a process receives the message \ct{terminate}, it stops running and cannot be run anymore. 

\item [Suspended.] A running is stopped to run and it is not in the list of the scheduler nor in semaphore waiting list. It may be runnable again by sending it the message \ct{resume}. In Figure~
\ref{fig:processStateInContext}, the process \ct{P0} got suspended, it became a suspended process. Such suspended process can be resume, in such a case it will added to the runnable processes of the scheduler. 

\item [Waiting.] A process is not in the scheduler list but waiting in a semaphore list of waiting process. It may be rescheduled once the semaphore will receive enough messages \ct{signal} to wake up it (see Section~\ref{sec:semaphore}). In Figure~
\ref{fig:processStateInContext}, the process \ct{P3} wait for a resource controlled by a semaphore, the semaphore when signaled, resumes its waiting processes \ct{P1} and \ct{P2}. 
\end{description}

\begin{figure}
\begin{center}
\includegraphics[width=10cm]{ProcessStateInContext}
\caption{Process possible states in context.\figlabel{processStateInContext}}
\end{center}
\end{figure}


Suspended and waiting state are similar: the main difference is that a suspended process may be rescheduled using the message \ct{resume}, while a waiting process is only rescheduled when the semaphore that puts it on wait, reschedule it (see Section~\ref{sec:semaphore}).

To sum up, there are 3 messages sent to a Process which change its state: \ct{resume}, \ct{suspend} and \ct{terminate}. In addition, the messages \ct{wait} and \ct{signal} sent to a semaphore have effect on processes. The message \ct{wait} affects the currently running process and the message \ct{signal} sent to a semaphore will resume of the waiting semaphore processes. As we will see soon, a process may release its control to give a chance to other processes to run using \ct{Processor yield}.


\section{Process Scheduler}

Processes are scheduled, \ie prioritized by the unique instance of the class \ct{ProcessScheduler}. This instance can be accessed using the global variable
\ct{Processor}. The process scheduler has the responsibilities to rank and schedule processes depending on their priority and their order.

\begin{table}
\begin{tabular}{l|c|c}
	name& priority & Purpose\\ \hline
	TimingPriority&80&\\ 
	HighIOPriority&70&\\ 
	LowIOPriority&60&\\ 
	UserInterruptPriority&50&\\
	UserSchedulingPriority&40\\
	UserBackgroundPriority&30&\\ 
	SystemBackgroundPriority&20&\\ 
	SystemRockBottomPriority&10&\\ \hline
\end{tabular}
\end{table}

\paragraph{Priority and Preemption}
Smalltalk processes are preemptive: process whose priority is higher have a higher chance to get run than those of lower priority. A process is run until it either releases explicit the control using \ct{Processor yield}, it gets waiting on a semaphore or it terminates its execution. 

When a process is run, all the other processes are suspended. 


Tous les autres processus sont mis en attente. Dans le cas où plusieurs processus
ont la même priorité, alors l'ordre de réveil est utilisé. Le premier à être réveillé sera exécuté jusqu'à sa terminaison ou mise en sommeil. Les autres processus de même priorité ou de priorité inférieure seront mis en attente.
A titre d'exemple, supposons que nous disposions d'un objet compte bancaire référencé par la variable \verb!monCompte! utilisée dans les trois processus correspondant aux trois lignes suivantes~:

\begin{code}{}
[Transcript show: 'First at 30' ;cr ] forkAt: 30.
[Transcript show: 'Second at 40' ;cr ] forkAt: 40.
[Transcript show: 'Third at 40' ;cr ] forkAt: 40.
\end{code}

\begin{code}{}
Second at 40
Third at 40
First at 30
\end{code}

[monCompte depot: 50] forkAt: 30.
[monCompte solde] forkAt: 40.
[monCompte retrait: 200] forkAt: 40.


Dans chacune de ces trois lignes de code, un bloc donne les traitements réalisés par le processus. Le processus est ensuite créé et réveillé en envoyant le message forkAt: au bloc. L'argument de ce message correspond à la
priorité du processus.

Une fois la première ligne évaluée, le processus de dépôt est lancé. Cependant, il est rapidement mis en attente car la ligne suivante lance le processus de consultation de solde à une priorité plus élevée. La troisième ligne lance un processus de retrait à la même priorité que pour la consultation. De ce fait, il reste en attente. Lorsque la
consultation est complètement terminée, le processus de retrait est activé car il a la priorité la plus forte. Ce n'est que lorsque tous les processus de priorité supérieure à 30 sont terminés que notre processus de dépôt est activé.

En plus de ces règles, le programmeur Smalltalk peut créer des processus de même niveau de priorité qui ne se bloquent pas les uns les autres. Pour ce faire, chacun des processus doit régulièrement demander au gestionnaire de processus Processor d'activer les éventuels autres processus de même priorité mis en attente.
Cette demande se fait à l'aide du message yield (Processor yield).

Mise en sommeil
Une autre solution consiste en une mise en sommeil. Pour ce faire, Smalltalk intègre la classe Delay qui permet de suspendre le processus actif pendant un laps de temps. Dans l'exemple suivant, le processus exécute une boucle de consultations de solde espacées de 30 secondes.

[ | attente |
    attente := Delay forSeconds: 30.
    100 timesRepeat: [
             attente wait.  "Mise en sommeil"
             monCompte solde.]
] forkAt: 40.


Nous faisons l'hypothèse que monCompte est une variable qui référence un compte bancaire. Le processus commence par creer une instance de Delay qui correspond à l'intervalle d'attente entre deux consultations de solde. Il réalise ensuite 100 fois (message timesRepeat: envoyé à l'entier 100) une consultation précédée par une attente.









\section{Semaphore Basic Principle}\seclabel{semaphore}
A semaphore is a basic mechanism for synchronizing processes.  A semaphore will either suspend or resume processes depending on its internal state. It has a list of waiting processes. 
When a process sends the message \ct{wait} to a semaphore, depending on the internal state of the semaphore, the calling process is put on wait (the process is added to the process waiting list of the semaphore). When a process has finished --- to use the resource for example, it signals it to the semaphore, which resumes one of its waiting processes. \sd{check the phrasing of xavier for this part}

Figure~\ref{fig:sema} shows a semaphore which is associated with a resource. The process \ct{P0} is currently using the resource. The process \ct{P2} and  \ct{P3}  are suspended and waiting in the list of suspended processes of the semaphore. 

\begin{figure}
\begin{center}
\includegraphics[width=6cm]{ASemaphoreAndProcesses}
\caption{ A process is using a resource, the other processes are waiting in a semaphore list of suspended processes. \figlabel{sema}}
\end{center}
\end{figure}

A semaphore is often used to control the access  to a resource by multiple processes. For example, using a semaphore we can ensure that only one process accesses the semaphore at a time. 
When we want to create a protected resource, the method of the resource should send the messages \ct{wait} or  when the resource is invoked or \ct{signal} when the resource use is terminated. 
This way a process wanting to use a resource will call one of the protected methods and the semaphore
will control its access by suspending the calling process. In addition the semaphore will resume the suspended 
process by rescheduling it when necessary -- and this without the need for the process to poll the semaphore.


We can also see a semaphore is a kind of clever counter that manages processes. 
This counter can be initialized with a value;  you cannot read the current value but you can only increment or decrement by one such a value. When a process decrements a semaphore, if the result is negative, the 
semaphore suspends the process and this process cannot continue until another process increments 
the semaphore. When a process increments the semaphore, if there are other processes waiting, one of them gets resumed.



\paragraph{Illustrating Semaphore.}
Image that we want to make sure that a given resource cannot be used by two processes at the same time.
We will attach a semaphore to the resource. Figure~\ref{fig:sema} shows that the process \ct{P0} is using the resource and that other processes needed it so they are waiting. When a new process \ct{P4} requires the resource, it notifies it, which leads to send the message \ct{wait} to the semaphore. The process is then suspended and added to the list of the waiting processed of the semaphore (Figure~\ref{fig:SemaphoreSendingWait}). 


When \ct{P0} has finished to work with the resource, it notifies the resource which has for effect to notify the semaphore with the message \ct{signal}. The next waiting process is resumed, it is added to the scheduler processes list and will be executed 


\begin{figure}
\begin{center}
\includegraphics[width=6cm]{SemaphoreSendingWait2}
\caption{A new process wants to access the resources, it is suspended and added to the wait list of the semaphore. \figlabel{SemaphoreSendingWait}}
\end{center}
\end{figure}


\begin{figure}
\begin{center}
\includegraphics[width=5.5cm]{SemaphoreSendingSignal}\includegraphics[width=3.5cm]{ProcessToInSchedulerQueue}\includegraphics[width=3.5cm]{SemaphoreResuming}
\caption{  process terminates, a suspended process is resumed (added to the scheduler list).\figlabel{ProcessToInSchedulerQueue}}
\end{center}
\end{figure}




\paragraph{Semaphore creation.}
A semaphore is created using the message \ct{new}




\paragraph{Some implementation aspects.} The implementation of the method \ct{wait} of the Semaphore class,
implements the counter behavior we mentioned early. When the number of signals is negative, then the current process (ie the one that sends the message wait) is suspended. 

\begin{code}{}
Semaphore>>>wait
	"Primitive. The active Process must receive a signal through the receiver 
	before proceeding. If no signal has been sent, the active Process will be 
	suspended until one is sent. Essential. See Object documentation 
	whatIsAPrimitive."
		<primitive: 86>
	self primitiveFailed

	"excessSignals>0  
		ifTrue: [excessSignals := excessSignals-1]  
		ifFalse: [self addLastLink: Processor activeProcess suspend]
\end{code}
% For all intents and purposes, \clsindmain{Object} is the root of the inheritance hierarchy. Actually, in \pharo the true root of the hierarchy is \clsind{ProtoObject}, which is used to define minimal entities that masquerade as objects, but we can ignore this point for the time being.
% % (more on this later in the chapter on reflection).
% 
% \ct{Object} can be found in the \scatind{Kernel-Objects} category. Astonishingly, there are some 400 methods to be found here (including extensions).  In other words, every class that you define will automatically provide these 400 methods, whether you know what they do or not. Note that some of the methods should be removed and new versions of \pharo may remove some of the superfluous methods. 
% 
% \sd{I do not like to quote something that can change and that people can find simply in the image but let us keep it for now.}
% The class comment for the \ct{Object} states:
% 
% \ct{Object>>>printOn:} is very likely one of the methods that you will most frequently override. This method takes as its argument a \clsind{Stream} on which a \clsind{String} representation of the object will be written. The default implementation simply write the class name preceded by ``\ct{a}'' or ``\ct{an}''. \ct{Object>>>printString} returns the \ct{String} that is written:
% 
% For example, the class \clsind{Browser} does not redefine the method \ct{printOn:} and sending the message printString to an instance executes the methods defined in \ct{Object}. 
% \begin{code}{@TEST}
% Browser new printString --> 'a Browser'
% \end{code}
% 
% The class \ct{TTCFont} shows an example of \mthind{TTCFont}{printOn:}
% specialization. It prints the name of the class followed by the family
% name, the size and the subfamily name of the font as shown by the code
% below which prints an instance of the class.
% 
% \begin{method}[zork]{printOn: redefinition.}
% TTCFont>>>printOn: aStream
%         aStream nextPutAll: 'TTCFont(';
% 		nextPutAll: self familyName; space;
% 		print: self pointSize; space;
% 		nextPutAll: self subfamilyName;
% 		nextPut: $)
% \end{method}\ignoredollar$
% 
% % \begin{code}{@TEST}
% \begin{code}{} % ON: THIS IS FRAGILE -- breaks in Pharo
% TTCFont allInstances anyOne printString --> 'TTCFont(BitstreamVeraSans 6 Bold)'
% \end{code}
% 
% \begin{method}{Hash must be reimplemented for complex numbers}
% Complex>>>hash
%     "Hash is reimplemented because = is implemented."
%     ^ real hash bitXor: imaginary hash.
% \end{method}

\ifx\wholebook\relax\else
   \bibliographystyle{jurabib}
   \nobibliography{scg}
   \end{document}
\fi

\section{Protocol}


\pharo comme tous les Smalltalks offre la possibilité d'exécuter différents programmes en parallele. 
La concurrence en Smalltalk est préemptive et collaborative. Préemptive car les processus de priorité superieure 
prennent précédence sur ceux de priorité inferieure. Collaborative car pourqu'un processus ait une chance de s'executer alors qu'un processus de meme priorité est deja entrain de s'executer, celui-ci doit explicitement relacher le contrôle. Dans cet article nous montrons les élements de base et dans un prochain article nous montrerons les abstractions essentielles telleque Monitor, Delay.

Les processus
Pour les besoins de la programmation concurrente, Smalltalk offre une panoplie de classes qui permettent la manipulation de processus légers (threads) instances de la classe Process [sharp97]. Un processus est caractérisé par un bloc (instance de BlockContext) qui décrit les opérations qu'il doit exécuter et une priorité d'exécution.


\section{Synchronisation}
Afin de gérer les accès concurrents aux ressources, Smalltalk dispose d'objets verrous. Ce sont des instances de la classe Semaphore qui permettent la synchronisation des sections critiques du code. Pour ce faire, tous les processus qui partagent une même ressource doivent envoyer au sémaphore le message wait avant l'accès et le message signal après l'accès. En effet, l'exécution du processus qui envoie un wait est suspendue. Lorsque le sémaphore reçoit un signal, il réveille le plus ancien processus (i.e. le premier à avoir envoyé wait).

CompteBancaireCourant subclass: #CompteSynchronise
   instanceVariableNames: `verrou'!\\
   classVariableNames: `'!\\
04&\tab \verb!poolDictionaraies: `'!\\
05&\tab \verb!category: `Comptes Bancaires'!\\
\hline
06&{\bf initialize}\\
07&\tab   \verb!super initialize.!\\
08&\tab   \verb!verrou := Semaphore forMutualExclusion!\\
\hline
09&{\bf depot\verb!:! montant}\\
10&\tab   \verb!verrou critical: [super depot: montant]!\\
\hline
11&{\bf retrait\verb!:! montant}\\
12&\tab    \verb!verrou critical: [super retrait: montant]!\\
\hline
13&{\bf solde}\\
14&\tab   \verb!verrou critical: [^super solde]!\\
\hline
\end{tabular}
}
    \caption{Définition de la classe {\tt CompteSynchronise}}
     \figlabel{compteSynchronise}
   \end{center}
\end{figure}

Par exemple, si nous souhaitons disposer de comptes courants dont les dépôts, retraits et consultation de solde sont synchronisés, nous définissons la sous-classe de CompteBancaireCourant comme suit~:

 La sous-classe CompteSynchronise ajoute une variable d'instance
verrou! qui représente un sémaphore (ligne~2, figure~\ref{fig:compteSynchronise})

L'initialisation de la variable d'instance verrou est opérée dans la méthode d'initialisation initialize (lignes~6 à~8, figure~\ref{fig:compteSynchronise}). Cette méthode crée un sémaphore à l'aide le message forMutualExclusion qui (par opposition au new) retourne un sémaphore qui ne suspend pas le premier processus qui envoie un message wait. Le but étant de ne suspendre les processus qu'en cas d'accès concurrents, un tel sémaphore est un sémaphore normal qui a reçu un signal juste après sa création.

Les méthodes depot: (lignes~9 et~10, figure~), retrait: (lignes~11 et~12, figure ) et solde (lignes~13 et~14, figure) sont redéfinies pour synchroniser les accès.
A noter qu'au lieu d'utiliser un couple wait/signal, nous utilisons le message critical: qui réalise le même traitement. L'avantage de ce dernier est d'envoyer toujours le signal et donc la libération de la ressource même en cas d'exception.


Afin de simplifier les échanges de données (des objets) entre processus, Smalltalk introduit des files synchronisées, instances de la classe SharedQueue. Plusieurs processus producteurs peuvent ainsi injecter des données dans la file, pendant que des processus consommateurs les retirent. Le recours aux sémaphores est inutile, puisque la file assure la synchronisation. En particulier, elle suspend les processus consommateurs lorsqu'elle est vide et les réveille quand des producteurs injectent des données.

Ce comportement simple peut bien sûr être étendu. Par exemple, il est possible d'enrichir le comportement FIFO ({\em first in first out}) avec des priorités \cite{sharp97}.
Chaque objet injecté dans la file est marqué d'une priorité.
Les objets les plus prioritaires sont les premiers à être extraits de la file.


Liens 
[BLUEB] Smalltalk-80: The Language and its Implementation : http://stephane.ducasse.free.fr/FreeBooks/BlueBook/ 


\section{semaphore}
	Boolean vs integer
	
\section{Critical Section}

\section{Scheduler}
How does it work?

\section{Delay}

\chapter{Avanced Abstractions}

\section{Monitor -- Broken}

\section{SharedQueue}

\section{Mutex - Reentrant Semaphore}
RecursionLock in VW


%=============================================================
\ifx\wholebook\relax\else
   \bibliographystyle{jurabib}
   \nobibliography{scg}
   \end{document}
\fi
%=============================================================

%-----------------------------------------------------------------

%%% Local Variables:
%%% coding: utf-8
%%% mode: latex
%%% TeX-master: t
%%% TeX-PDF-mode: t
%%% ispell-local-dictionary: "english"
%%% End: