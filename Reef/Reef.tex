% $Author: ducasse $
% $Date: 2009-08-24 10:17:33 +0200 (Mon, 24 Aug 2009) $
% $Revision: 28563 $

%=================================================================
\ifx\wholebook\relax\else
% --------------------------------------------
% Lulu:
	\documentclass[a4paper,10pt,twoside]{book}
	\usepackage[
		papersize={6.13in,9.21in},
		hmargin={.75in,.75in},
		vmargin={.75in,1in},
		ignoreheadfoot
	]{geometry}
	\input{../common.tex}
	\setboolean{lulu}{true}
% --------------------------------------------
% A4:
%	\documentclass[a4paper,11pt,twoside]{book}
%	\input{../common.tex}
%	\usepackage{a4wide}
% --------------------------------------------
    \graphicspath{{figures/} {../figures/}}
	\begin{document}
\fi
%=================================================================
%\renewmessage{\nnbb}[2]{} % Disable editorial comments
\sloppy

%=================================================================
%\renewcommand{\nnbb}[2]{#2} % Disable editorial comments

\chapter{Reef}

Reef is a framework to build dynamic components. As a complement to Seaside, Reef does not replace it, instead, it makes easier the use of AJAX/Javascript inside it. A reef component can be embedded inside a Seaside one (we often call Reef components, component part for this reason), but a Seaside component can be embedded inside a Reef one. 

In this Chapter, you are going to create a dynamic search of classes, to show basic AJAX interaction.


\section{Installation}

To get started with download the latest Seaside image that you can find on \url{http://www.seaside.st} then execute the following expression:

\begin{code}{Loading Reef}
Gofer it
	squeaksource: 'Reef';
	package: 'ConfigurationOfReef';
	load.

"Use  this if you already have Seaside30 and JQueryWidgetBox installed"
((Smalltalk at: #ConfigurationOfReef) project version: #bleedingEdge) load. 

"This loads everything, including Seaside30 and JQueryWidgetBox"
((Smalltalk at: #ConfigurationOfReef) project version: #bleedingEdge) load: 'All'. 
\end{code}


\section{Creating a search component}

We will create a Reef component and embed it inside a Seaside one. 


\subsection{A Simple Reef Component Form}

To create a form in Reef, you need to subclass from \ct{REForm}. \ct{REForm} is the class responsible for creating forms in Reef. 


\begin{code}{}
REForm subclass: #RTSearchPart
	...

\end{code}

\ct{REForm} is a subclass of \ct{REContainer} which is the root of the components that can have children. Other reef container components are \ct{REPanel}, \ct{REGrid}, and \ct{REList}. Of course such components can be further specialized: hence \ct{REPanel} is the superclass of \ct{REDialog}, \ct{REAccordion}, \ct{RECarousel}, and \ct{RETabs}. All these components can then contain other components as shown in Figure~\ref{review}. 


\begin{figure}[h]
\begin{center}
\includegraphics[width=3cm]{REViewHierarchy}
\caption{The \ct{REView} hierarchy: \ct{REContainer} is the root of the component container components.\label{review}}
\end{center}
\end{figure}


Now let us go back to our \ct{RTSearchPart} class. You need to override the method \ct{initializeContents} to add children components to your container widget. We use the message 
\ct{add:} to add subcomponents.
 

\begin{code}{}
RTSearchPart>>initializeContents
	self add: 'Search'.
	self add: RETextField new.
\end{code}

\sd{should avoid to harcode class name}

And this is it for our first 'part component' (we usually call reef components \emph{parts}, as a convention to make explicit the fact that a Reef component is often just a part of a Seaside component.


\subsection{A Host Seaside Component}
Now to see our creation in action we will define a simple traditional Seaside component.

\begin{code}{}
WAComponent subclass: #RTWebApplication
	instanceVariableNames: 'searchComponent'
	...
\end{code}

\sd{the name sucks!}

\begin{code}{}
RTWebApplication>>initialize
	super initialize.
	searchComponent := RTSearchPart new asComponent.
\end{code}

Since the part will be a children of the application, we should return it as part of the children protocol. 

\begin{code}{}
RTWebApplication>>children
	^ Array with: searchComponent
\end{code}

Finally we just render the part and do nothing special. 

\begin{code}{}
RTWebApplication>>renderContentOn: html
	html render: searchComponent
\end{code}

As you may have noticed, search component is not just holding an instance of RTSearchPart. Indeed we send the asComponent to this instance before assigning it to the instance variable. And indeed a part of the magic here is this \ct{asComponent} message send. This message says to any Reef component to be wrapped into a Seaside component. This component can now be added to your Seaside application as any other component.

\paragraph{Registering.} As any Seaside application we should register it. Execute in a workspace or define as class \ct{initialize}method the following expression and execute it. 

\begin{code}{Registering our application as a Reef application.}
REApplication 
	registerAsApplication: 'simpleTutorial'
	root: RTWebApplication
\end{code}


You may wonder why we should register our application as a Reef application and not as a simple Seaside application. This is because
Reef supports client side updates and other cool feature that we will present later. 
In addition, a Reef application needs to be configured with some elements: 1) we need some jQuery libraries 2) We need a div tag always present in our browser (a ''dispatcher area'' to process AJAX requests) \ct{REApplication class>>registerAsApplication:root:} does that for you, but you could do this by yourself and use the common application registering mechanism.
\sd{check that this is class level}


\paragraph{First version.}

So, executing your application in \url{http://localhost:8080/simpleTutorial}, we get the result shown in Figure~\ref{first}:

\begin{figure}[h]
\begin{center}
\includegraphics[width=6cm]{SearchPartV1}
\caption{A simple component (version 1).\label{first}}
\end{center}
\end{figure}

Ok, no so great at the moment, but we are going to add more functionalities.
Right now, our Reef component doesn't do anything (just render a form with a text field), but this is 
just the first step, now we are going to add behavior to our text field widget.


\begin{figure}[h]
\begin{center}
\includegraphics[width=6cm]{SearchPartV2}
\caption{A simple component (version 2).\label{second}}
\end{center}
\end{figure}


\begin{code}{}
RTSearchPart>>initializeContents
	self add: 'Search'.
	self add: (RETextField new
		onKeyPressed: [:me | me triggerThenDo: [self inform: text] ];
		callback: [:v | text := v]
\end{code}

\sd{what is text? a new instance variable in SearchPart}

Changes introduced: 
\begin{enumerate}
\item We added a \ct{callback:} message. The \ct{callback:} message looks like a standard seaside brush...
and in fact is the same.

\item We added an \ct{onKeyPress:} message. This is different from Seaside. In plain Seaside,
you need to send a Javascript string to this kind of messages (tag events). 
In Reef, that's not what you do. Instead, you use a block with Smalltalk code (at least in most cases). 
In this case, the \ct{onKeyPress:} block is getting the text field as a parameter (it is optional), and we are saying: trigger this widget and then execute another action.
\sd{wht is trigger this widget means?}
\end{enumerate}
Let's test it! When you enter a character you should get the inform as shown in Figure~\ref{second}.

\paragraph{Adding a feedback panel.}

Now we want to add a panel to show the classes whose name matches the input. 
For this we are going to modify our \ct{RESearchPart} class again. We add a new instance 
variable named \ct{resultsPanel} that will display all the matching results.

\begin{code}{}
REForm subclass: #RTSearchPart
	instanceVariableNames: 'text resultsPanel'
	...
\end{code}



\begin{code}{Initializing the panel showing the results}
RESearchPart>>initializeContents
	resultsPanel := REPanel new.
	self add: 'Search'.
	self add: (RETextField new
		onKeyPress: [:me | me triggerThenDo: [self search] ];
		callback: [:v | text := v]).
	self add: resultsPanel
\end{code}



\begin{code}{}
RESearchPart>>search
	resultsPanel removeAll.
	text ifNotEmpty: [ 
		resultsPanel
			addAll: (Smalltalk classNames
					select: [:each | each beginsWith: text]
					thenCollect: [ :each | 
							REPanel new 
								add: each ; yourself ]);
			refresh ]
\end{code}

apparently this one is better
\begin{code}{}
RESearchPart>>search
	resultsPanel removeAll.
	text ifNotEmpty: [
		resultsPanel
			addAll: (Smalltalk classNames select: [:each | each beginsWith: text]);
			refresh ]
\end{code}


\sd{I do not see why we are adding a new panel? because we removeall the items
so we could just add the items inside}

What changes we introduced?
\begin{enumerate}
\item we added a panel instance variable (\ct{resultPanel}) 
\item then, we modified \ct{onKeyPress:} to call \ct{search} instead of showing a dialog. 
\item The \ct{search} method does a refresh of the panel by doing:
\begin{itemize}
\item remove all children (if any) 
\item add all class names with a name starting with ``text'', as a new panel (we need a
panel because we want a name per line...) 
\item send a \ct{refresh} message, which actually performs the refresh of the panel
content. 
\end{itemize}
\end{enumerate}


Once you refreshed your application, you should see Figure~\ref{third}.

\begin{figure}[h]
\begin{center}
\includegraphics[width=6cm]{SearchPartV3}
\caption{With the matching list of classes as feedback (version 3).\label{third}}
\end{center}
\end{figure}

So we are done with your first Reef component. What you should notice is that we did not talk about JavaScript since it is encapsulated in Reef components. 



\section{Dealing with Callbacks}
One of the strengths of Reef is that we can control the interaction between the client and the server. 

\subsection{Client-Server interaction in Reef}

One of the most important features in Seaside is it's natural flow, achieved through the use of closures under the forms of \textit{callbacks}. One of the problems on using a client-side library, such as \ct{jQuery} or any JavaScript library, is the loose of this natural feeling.

In Reef, we try to keep this natural flow, as much as we can, by hiding most of the JavaScript complexity, and using smalltalk closures also for JavaScript behavior. 
For that, we have redefined the event behavior to match a little more with what a Smalltalk programmer can expect. Let us see the different kind of interactions.



\subsection{Three different kind of interactions}

There are three kinds of  possible interactions between a page in a client and the Seaside server:

\begin{enumerate}
\item The client application needs to access the server to do something, and then update the status of the page, but without rendering a new page because it can be too costly. As we think this kind of interaction is the hardest to manage, and the most commonly used, this is the default Reef behavior.

\item The client application needs to call server with some functionality, and then it renders a new page. This is the ``normal behavior'' of a web application, and in Reef we support it as a way to ``get back'' to normal flow, while inside a Reef flow.

\item The page needs to change the DOM (for instance, to hover current selected row), without going to server at all. This is, for us, the most difficult to solve, because the DOM manipulation needs to be a JavaScript sentence.
\end{enumerate}

These three different interactions are defining the ways an application can interact with the user, creating different results, but just one final feeling: we are in a ``desktop'' application.
We are going to explain how Reef manages these three approaches.

\subsection{Updating the server, then update the client}
As we said before, we think this is the most common behavior you can find in a web-application, and for that reason we are focusing in this mode. How does it work?
Well, let's suppose you have a validation in a text field that needs to be validated on blur event (once you lost focus), you have something like this:

\begin{code}{}
	...
	self add: (RETextField new
				onBlur: [ :me | me triggerThenDo: 
							[ self validate ifFalse: [ self error: 'Error validating' ] ] ]; 
				on: #value of: self)
	...
\end{code}

As you can see, the flow is more or less like it would be in any usual Smalltalk application, with just one  difference: you need to tell the client that you are going to use the value just typed in this field, \sd{is it validate?} and just then we can for real interact with the client side value. Using this kind of callback is the default behavior in any Reef component, with the unique exception of the argument of \ct{callback:} in form elements (those who will become in a FORM element in HTML, like an INPUT, TEXTAREA, etc.).

\sd{totally unclear to me }


\subsection{Update the server, then render a new page}

This functionality is useful after pressing a button, or after doing some operations that we decide needs a new page. What it does is to skip the client/server AJAX interaction and render a full new page: it works just like a common anchor or button press in plain HTML. To tell Reef that the result of a callback should be executed in another page, we use the protocol \ct{asReefPageCallback}. See next example:

\begin{code}{}
	... 
	add: (REButton new
			label: 'Login'; 
			callback: [ self login ] asReefPageCallback)
	...
\end{code}

In this case, pressing the button acts just like pressing a regular button in Seaside.

\sd{ (well... not exactly equal, but act like that in almost any case). either we explain it or we remove it}

\subsection{Not updating the server, just the client}
This third case is useful to perform client side validations, or highlighting something on hover. This callback executes its contents at render time, and prepare the page with the resulting JavaScript, so it can act on client side. To tell Reef that it should execute the callback on client side, we use the protocol: \ct{asReefClientCallback}. See the example:

\sd{we should put the complete code}

\begin{code}{}
	... 
	self add: (REPanel new
		class: 'panel'; 
		onMouseMove: [ :me | me asClient addClass: 'hover' ] asReefClientCallback; 
		onMouseOut: [ :me | me asClient removeClass: 'hover' ] asReefClientCallback; 
		add: 'Panel with hover'; yourself).
	...
\end{code}

As you can see, we introduced the protocol \ct{asReefClientCallback}, which answers a representation in JavaScript, and you can then send messages of the jQuery package.
\sd{where? I only saw asReefClientCallback no asClient}



\subsection{Hybrids}

There is still one scenario to deal with: The one in which we want to do something and in some cases the opposite. For example, if we want to validate some inputs and on success render a new page, but on failure stay in the current page, and show a message. 

To allow us execute different client/server interactions inside a flow, there are the protocols: \ct{inPage:}, \ct{inClient:}, both allow us to perform different actions to change the rendering logic. 

\sd{why inPage: and not inServer: (onServer: and onClient: would be better)}

See the example:

\sd{we should put the complete code}

\begin{code}{}
	... 
	self add: (REButton new
			label: 'Login'; 
			callback: [
				self triggerThenDo: [ self login
							ifTrue: [ self inPage: [ self show: aComponent ] ] 
							ifFalse: [ self inClient: [ self asClient addClass: 'error' ] ] ] ].
\end{code}							


In this case, a new page (with aComponent contents) will be rendered if the login is successful, and an error class will be added to form if it doesn�t.

\subsection{API}

Here is the list

\ct{[ ... ]} A block in Reef is a callback who leads to an interaction between client and server, without render a new page. Exception: callbacks on form elements.

\ct{[ ... ] asReefPageCallback} Contents of block are executed, but all rendering is do it in a new page.

\ct{[ ... ] asReefClientCallback} Contents of block are executed in rendering phase, and the result must be a JQuery Stream or a String.

\ct{[ :me | ... ]}

\ct{triggerThenDo:} 	Triggers the form element (if applied to a form triggers full form).

\ct{asClient} 	jQuery object representation of reef component.

\ct{inPage:} 	Renders the block contents in a new page.

\ct{inClient:} 	Renders the block, in same page as we are now. It works just like \ct{asReefClientCallback}.

\section{Interaction in Action}

Now we will show you a simple application showing in practice the different interaction. 

\section{Reef Defaults widgets}

\subsection{Accordion}
\subsection{Carousel}

\subsection{Grids}

\subsection{Calendar}

\section{Reef Core Explained}



\begin{code}{}
REForm new 
	add: (RETextInput new 
		autocomplete: [ :text | Smalltalk allClasses select: [ :each | each name asLowercase beginsWith: text ] ];
		yourself);
	yourself.
\end{code}



%=========================================================
\ifx\wholebook\relax\else
   \bibliographystyle{jurabib}
   \nobibliography{scg}
   \end{document}
\fi
%=========================================================



%%% Local Variables: 
%%% coding: utf-8
%%% mode: latex
%%% TeX-master: Lint.tex
%%% TeX-PDF-mode: t
%%% End:

