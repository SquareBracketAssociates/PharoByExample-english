% $Author: ducasse $
% $Date: 2009-08-24 10:17:33 +0200 (Mon, 24 Aug 2009) $
% $Revision: 28563 $

%=================================================================
\ifx\wholebook\relax\else
% --------------------------------------------
% Lulu:
	\documentclass[a4paper,10pt,twoside]{book}
	\usepackage[
		papersize={6.13in,9.21in},
		hmargin={.75in,.75in},
		vmargin={.75in,1in},
		ignoreheadfoot
	]{geometry}
	\input{../common.tex}
	\setboolean{lulu}{true}
% --------------------------------------------
% A4:
%	\documentclass[a4paper,11pt,twoside]{book}
%	\input{../common.tex}
%	\usepackage{a4wide}
% --------------------------------------------
    \graphicspath{{figures/} {../figures/}}
	\begin{document}
\fi
%=================================================================
%\renewmessage{\nnbb}[2]{} % Disable editorial comments
\sloppy

%=================================================================
%\renewcommand{\nnbb}[2]{#2} % Disable editorial comments

\chapter{Reef}

Reef is a framework to build dynamic components. As a complement to Seaside, Reef does not replace it, instead, it makes easier the use of AJAX/Javascript inside it. A reef component can be embedded inside a Seaside one (we often call Reef components, component part for this reason), but a Seaside component can be embedded inside a Reef one. 

In this Chapter, you are going to create a dynamic search of classes, to show basic AJAX interaction.


\section{Installation}

To get started with download the latest Seaside image that you can find on \url{http://www.seaside.st} then execute the following expression:

\begin{code}{Loading Reef}
Gofer new
	squeaksource: 'Reef';
	package: 'Reef';
	load
\end{code}

\sd{I got This package depends on the following classes:
  JQAlphaNumeric
You must resolve these dependencies before you will be able to load these definitions: 
  JQAlphaNumeric>>beDecimal
  JQAlphaNumeric>>beInteger}

\section{Creating a search component}

We will create a Reef component and embed it inside a Seaside one. 


\subsection{A Simple Reef Component Form}

To create a form in Reef, you need to subclass from \ct{REForm}. \ct{REForm} is the class responsible for creating forms in Reef. 


\begin{code}{}
REForm subclass: #RTSearchPart
	...

\end{code}

\ct{REForm} is a subclass of \ct{REContainer} which is the root of the components that can have children. Other reef container components are \ct{REPanel}, \ct{REGrid}, and \ct{REList}. Of course such components can be further specialized: hence \ct{REPanel} is the superclass of \ct{REDialog}, \ct{REAccordion}, \ct{RECarousel} and \ct{RETabs}. All these components can then contain other components as shown in Figure~\ref{review}. 


\begin{figure}[h]
\begin{center}
\includegraphics[width=3cm]{REViewHierarchy}
\caption{The REView hierarchy: The REContainer is the root of the component container components.\label{review}}
\end{center}
\end{figure}


Now let us go back to our \ct{SearchPart}.  You need to override the method \ct{initialize} to add children components to your container widget. 

\sd{why this is initializeContents and not initialize?}
\begin{code}{}
RTSearchPart>>initializeContents
	self add: 'Search'.
	self add: RETextField new.
\end{code}

\sd{Should explain what add: is}
\sd{should avoid to harcode class name}

And this is it for our first 'part component' (we usually call reef components \emph{parts}, as a convention to make explicit the fact that a Reef component is often just a part of a Seaside component.


\subsection{A Host Seaside Component}
Now to see our creation in action we will define a simple traditional Seaside component.

\begin{code}{}
WAComponent subclass: #RTWebApplication
	instanceVariableNames: 'searchComponent'
	...
\end{code}

\sd{the name sucks!}

\begin{code}{}
RTWebApplication>>initialize
	super initialize.
	searchComponent := RTSearchPart new asComponent.
\end{code}

Since the part will be a children of the application, we should return it as part of the children protocol. 

\begin{code}{}
RTWebApplication>>children
	^ Array with: searchComponent
\end{code}

Finally we just render the part and do nothing special. 

\begin{code}{}
RTWebApplication>>renderContentOn: html
	html render: searchComponent
\end{code}

As you may have noticed, search component is not just holding an instance of RTSearchPart. Indeed we send the asComponent to this instance before assigning it to the instance variable. And indeed a part of the magic here is this \ct{asComponent} message send. This message says to any Reef component to be wrapped into a Seaside component. This component can now be added to your Seaside application as any other component.

\paragraph{Registering.} As any Seaside application we should register it. Execute in a workspace or define as class \ct{initialize}method the following expression and execute it. 

\begin{code}{Registering our application as a Reef application.}
REApplication 
	registerAsApplication: 'simpleTutorial'
	root: RTWebApplication
\end{code}


You may wonder why we should register our application as a Reef application and not as a simple Seaside application. This is because
Reef supports client side updates and other cool feature that we will present later. 
In addition, a Reef application needs to be configured with some elements: 1) we need some jQuery libraries 2) We need a div tag always present in our browser (a ''dispatcher area'' to process AJAX requests) \ct{REApplication class>>registerAsApplication:root:} does that for you, but you could do this by yourself and use the common application registering mechanism.
\sd{check that this is class level}


\paragraph{First }

So, executing your application in \url{http://localhost:8080/simpleTutorial}, we get the result shown in Figure~\ref{first}:

\begin{figure}[h]
\begin{center}
\includegraphics[width=2cm]{REViewHierarchy}
\caption{A simple component (version 1).\label{first}}
\end{center}
\end{figure}

Ok, no so great at the moment, but we are going to add more functionalities.
Right now, our Reef component doesn't do anything (just render a form with a text field), but this is 
just the first step, now we are going to add behavior to our text field widget.


\begin{figure}[h]
\begin{center}
\includegraphics[width=2cm]{REViewHierarchy}
\caption{A simple component (version 2).\label{first}}
\end{center}
\end{figure}

\begin{code}{}
RTSearchPart>>initializeContents
	self add: 'Search'.
	self add: (RETextField new
		onKeyPressed: [:me | me triggerThenDo: [self inform: text] ];
		callback: [:v | text := v]
\end{code}

Changes introduced: 
\begin{enumerate}
\item We added a \ct{callback:} message. The \ct{callback:} message looks like a standard seaside brush...
and in fact is the same.

\item We added an \ct{onKeyPress:} message. This is different from Seaside. In plain Seaside,
you need to send a Javascript string to this kind of messages (tag events). 
In Reef, that's not what you do. Instead, you use a block with Smalltalk code (at least in most cases). 
In this case, the \ct{onKeyPress:} block is getting the text field as a parameter (it is optional), and we are saying: trigger this widget and then execute another action.
\sd{wht is trigger this widget means?}
\end{enumerate}
Let?s test it!




%=========================================================
\ifx\wholebook\relax\else
   \bibliographystyle{jurabib}
   \nobibliography{scg}
   \end{document}
\fi
%=========================================================



%%% Local Variables: 
%%% coding: utf-8
%%% mode: latex
%%% TeX-master: Lint.tex
%%% TeX-PDF-mode: t
%%% End:

