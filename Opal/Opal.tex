% $Arnaud Jean-baptiste$
% $22 juin 2011 $
% $v1$

% HISTORY: []
% 2011-june-22 :: first commit


%=================================================================
\ifx\wholebook\relax\else
% --------------------------------------------
% Lulu:
	\documentclass[a4paper,10pt,twoside]{book}
	\usepackage[
		papersize={6.13in,9.21in},
		hmargin={.75in,.75in},
		vmargin={.75in,1in},
		ignoreheadfoot
	]{geometry}
	\input{../common.tex}
	\pagestyle{headings}
	\setboolean{lulu}{true}
% --------------------------------------------
% A4:
%	\documentclass[a4paper,11pt,twoside]{book}
%	\input{../common.tex}
%	\usepackage{a4wide}
% --------------------------------------------
    \graphicspath{{figures/} {../figures/}}
	\begin{document}
	\renewcommand{\nnbb}[2]{} % Disable editorial comments
	\sloppy
\fi


\newcommand{\opal}[0]{Opal~Compiler } % Ducasse
%=================================================================
\chapter{Opal Compiler} \chalabel{opal}

%\alex{I addressed lukas comments using editorial macros. Please, if you read this, remove this comment and remove the editorial macros, else their purpose is diminished}

\indexmain{opal}
\opal is a Smalltalk to Bytecode compiler for Pharo. This project was initiate for replace the old Compiler. The old Compiler was made in 80s. The old compiler was design in one process Scanner/Parser. The result is the old compiler is hard to understand, to extend and completely unadapted to the new needs. That why \opal project was made.

\opal process, is made in 3 step,
\begin{itemize}
\item Source Code to Abstract Syntax Tree
\item Abstract Syntax Tree to Intermediary Representation
\item Intermediary Representation to Bytecode
\end{itemize}

\begin{figure}[ht]\centering
	\includegraphics[width=\linewidth]{fullProcess}
	\caption{complete process of compilation with \opal \figlabel{fullProcess}}
\end{figure}

\section{Source Code to Abstract Syntax Tree}
The Abstract Syntax Tree (AST) used in \opal come from Refactoring Browser using RBParser to generate. 
\begin{figure}[ht]\centering
	\includegraphics[width=\linewidth]{SourceToAnnotatedAST}
	\caption{Source to AST Annotated \figlabel{SourceToAnnotatedAST}}
\end{figure}

\section{Abstract Syntax Tree to Intermediary Representation}

\begin{figure}[ht]\centering
	\includegraphics[width=\linewidth]{AnnotatedASTToIR}
	\caption{AST Annotated to Intermediary Representation \figlabel{AnnotatedASTToIR}}
\end{figure}

\section{Intermediary Representation to Bytecode}
\begin{figure}[ht]\centering
	\includegraphics[width=\linewidth]{IRToBytecode}
	\caption{Intermediary Representation to Bytecode \figlabel{IRToBytecode}}
\end{figure}
A program that manipulates other programs (or even itself) is a \emphind{metaprogram}.
For a programming language to be reflective, it should support both \ind{introspection} and \ind{intercession}.
Introspection is the ability to \emph{examine} the data structures that define the language, such as objects, classes, methods and the execution stack.
Intercession is the ability to \emph{modify} these structures, in other words to change the language semantics and the behavior of a program from within the program itself.
\emph{Structural reflection} is about examining and modifying the structures of the run-time system, and \emphind{behavioural reflection} is about modifying the interpretation of these structures.

In this chapter we will focus mainly on \ind{structural reflection}.
We will explore many practical examples illustrating how \st supports introspection and metaprogramming.

%=========================================================
\ifx\wholebook\relax\else
   \bibliographystyle{jurabib}
   \nobibliography{scg}
   \end{document}
\fi
%=========================================================

%Other stuff:
%- anonymous classes (uses compile: and primitiveChangeClassTo:) ???
%- collect direct senders; class collaborations
%- Object primitiveChangeClassTo: become: and becomeForward: (see tests and slides with minimal object example)
%- PointerFinder?
%- anonymous classes (see slides) ?

%Test  Coverage using ObjectsAsMethodsWrap package:
%\begin{code}{}
%category := 'SCGPier'.
%w := (ObjectAsOneTimeMethodWrapper installOnClassCategory: category).
%tr := TestRunner new.
%ToolBuilder open: tr.
%[tr
%	categoryAt: (tr categoryList indexOf: 'SCGPier') put: true;
%	selectAllClasses;
%	runAll.]
%ensure: [[w do: [:each| each uninstall ]] valueUnpreemptively].
%((w select: [:each | each executed not ])
%	collect: [:each | each wrappedClass name, '>>', each selector name ]) explore.
%\end{code}
