% $Author$
% $Date$
% $Revision$

% HISTORY:
% Chapter started by Damien C (2009-09-02)

%=================================================================
\ifx\wholebook\relax\else
% --------------------------------------------
% Lulu:
	\documentclass[a4paper,10pt,twoside]{book}
	\usepackage[
		papersize={6.13in,9.21in},
		hmargin={.75in,.75in},
		vmargin={.75in,1in},
		ignoreheadfoot
	]{geometry}
	\input{../common.tex}
	\setboolean{lulu}{true}
% --------------------------------------------
% A4:
%	\documentclass[a4paper,11pt,twoside]{book}
%	\input{../common.tex}
%	\usepackage{a4wide}
% --------------------------------------------
    \graphicspath{{figures/} {../figures/}}
	\begin{document}
\fi
%=================================================================
%\renewcommand{\nnbb}[2]{} % Disable editorial comments
\sloppy
%=================================================================
\chapter{Glamour}
\chalabel{glamour}

Browsers are a crucial instrument to understand complex systems or
models. Each problem domain is accompanied by an abundance of browsers
that are created to help analyze and interpret the underlying
elements. Thee issue with these browsers is that they are frequently
rewritten from scratch, making them expensive to create and burdensome
to maintain. While many frameworks exist to ease the development of
user interfaces in general, they provide only limited support to
simplifying the creation of browsers.

Glamour is a dedicated framework to describe the navigation flow
of browsers. Thanks to its declarative language, Glamour allows to
quickly define new browsers for their data.

In this chapter we will first detail the creation of some example
browsers to have an overview of the Glamour framework. In a second
part, we will describe Glamour in more details.

\section{Installation and first browser}

To install Glamour on your \pharo{} image execute the following code:

\begin{code}{}
ScriptLoader
  loadLatestPackage: 'GlamourLoader'
  from: 'http://www.squeaksource.com/Glamour'.
(Smalltalk classNamed: #GlamourLoader) load
\end{code}

Now that Glamour is installed, we are going to build a first browser
in order to dive into Glamour's declarative language. What about
building an Apple's Finder-like file browser? This browser is built
using the Miller Columns browsing technique, displaying hierarchical
elements in a series of columns. The principle of such a browser is
that a column always reflects the content of the element selected in
the previous column, the first column-content being chosen on opening.

\damien{insert screenshot of a finder-like browser (maybe not Apple's
  finder due to license restrictions).}

In our case of implementing a file browser, we want to display a list
of a particular directory's entries (each files and directories) in
the first column and then, depending on the user selection, appending
an other column:

\begin{itemize}
\item if the user selects a directory, the next column will display
  the entries of that particular directory;
\item if the user selects a file, the next column will display the
  content of the file.
\end{itemize}

This may look complex at first. However, Glamour provides a very
simple way of describing Miller Columns-based browsers. Glamour calls
that kind of browsers finders, referring to the Apple's Finder found
on Mac OS X. To create such a browser, we are going to use the
\clsind{GLMFinder} class and then tell Glamour that we want elements
to be in a list:

\dc{SFile class and its test can be found in SLICE10146 on task forces}

\begin{code}{}
browser := GLMFinder new.
browser list
	display: [:entry | entry files].
browser openOn: SFile anyRoot.
\end{code}

From this small piece of code you get a list of all entries (either
file or directory) found at the root of your file system, each line
representing either a file or a directory. If you click on a
directory, you can see the entries of this directory in the next
column.

\damien{insert a screenshot}

This code has some problems however. Each line displays the full print
string of the entry and this is probably not what you want. A typical
user would expect only names of each entry. This can easily be done by
customizing the list:

\begin{code}{}
browser list
  display: [:entry | entry files];
  format: [:entry | entry name].
\end{code}

This way, the message \ct{name} will be sent to each entry to get its
name. This makes the files and directory much easier to read.

\damien{insert a screenshot}

Another problem is that the code does not distinguish between files
and directories. If you click on a file, you will get an error because
the browser will send it the message \ct{files} that it does not
understand. To fix that, we just have to avoid displaying a list of
contained entries if the selected element is a file:

\begin{code}{}
browser list
  display: [:entry | entry files];
  format: [:entry | entry name];
  when: [:entry | entry isDirectory].
\end{code}

This works well but the user can't distinguish between a line
representing a file or a directory. This can be fixed by, for example,
adding a slash at the end of the file name if it is a directory:

\begin{code}{}
browser := GLMFinder new.
browser list
  display: [:entry | entry files];
  format: [:entry | entry isDirectory
                                   ifTrue: [entry name, '/']
                                   ifFalse: [entry name]];
  when: [:entry | entry isDirectory].
\end{code}

The last thing we might want to do is to display the contents of the
entry if it is a file. This gives the following final code:

\begin{code}{}
browser := GLMFinder new.
browser list
  display: [:entry | entry files];
  format: [:entry | entry isDirectory
                                   ifTrue: [entry name, '/']
                                   ifFalse: [entry name]];
  when: [:entry | entry isDirectory].
browser text
	display: [:entry | [entry contents]
                                   on: Exception
                                   do: ['Can''t display the content of this file']];
	when: [:entry | entry isFile].
browser openOn: SFile anyRoot
\end{code}

\dc{insert a screenshot}

This short introduction has just presented how to install Glamour and
how to use to create a simple file browser.

\section{Glamour in greater details}



%=============================================================
\ifx\wholebook\relax\else
   \bibliographystyle{jurabib}
   \nobibliography{scg}
   \end{document}
\fi
%=============================================================




%-----------------------------------------------------------------

%%% Local Variables:
%%% coding: utf-8
%%% mode: latex
%%% TeX-master: t
%%% TeX-PDF-mode: t
%%% ispell-local-dictionary: "english"
%%% End: