% $Author$
% $Date$
% $Revision$

% HISTORY:
% Chapter started by Stef (2008-07-26)

%=================================================================
\ifx\wholebook\relax\else
% --------------------------------------------
% Lulu:
	\documentclass[a4paper,10pt,twoside]{book}
	\usepackage[
		papersize={6.13in,9.21in},
		hmargin={.75in,.75in},
		vmargin={.75in,1in},
		ignoreheadfoot
	]{geometry}
	\input{../common.tex}
	\setboolean{lulu}{true}
% --------------------------------------------
% A4:
%	\documentclass[a4paper,11pt,twoside]{book}
%	\input{../common.tex}
%	\usepackage{a4wide}
% --------------------------------------------
    \graphicspath{{figures/} {../figures/}}
	\begin{document}
\fi
%=================================================================
%\renewcommand{\nnbb}[2]{} % Disable editorial comments
\sloppy
%=================================================================
\chapter{Announcements: an Object Dependency Framework}\chalabel{basic}


\section{A word about Component Coupling}

Here is an interesting question that often comes up often when writing
components. It is one that we faced when embedding our components. How
do the components communicate with each other in a way that doesn't
bind them together explicitly? That is, how does a child component
send a message to its parent component without explicitly knowing who
the parent is? Designing a component to refer to its parent is just a
part of the solution since the interfaces of different parents may be
different which would prevent the component from being reused in
different contexts.

There is a solution based on explicit dependencies also called the
change/update mechanism. Since early versions of Smalltalk, a
dependency mechanism based on a change/update protocol is available
and it is the foundation of the MVC framework itself. A component
registers its interest in some event and that event triggers a
notification. \lr{the reviewer wants to know if this is related to the
  observer pattern.}

There is a new framework called Announcement developed originally by
Vassili Bykov which has been ported to several Smalltalk
implementations. While the original dependency framework relied on
symbols for the event registration and notification, Announcement
promotes an object-oriented solution. Events are plain objects.

The main idea behind the framework is to set up announcers, define or
reuse some announcements, let clients register interest in events and
signal events. An event is an object representing an occurrence of a
specific event. It is the place to define all the information related
to the event occurrence. An announcer is responsible for registering
interested clients and announcing events. 

\paragraph{An Example.} Here is an example taken from Ramon Leon's
very good Smalltalk blog. This example shows how we can use
announcements to manage the communication between a parent component
and its children as for example in the context of a menu and its menu
items.

\begin{method}{Defining an announcer.}
MySession>>announcer
    ^ announcer ifNil: [announcer := Announcer new]
\end{method}

Then subclass \clsind{Announcement} for any interesting thing that
might happen. The announcement subclass is the place to add any extra
information about the specific announcement such as a context, the
objects involved \etc This is why announcement objects are both more
powerful and simpler than using symbols.

\begin{classdef}{Defining a specific Announcement.}
Announcement subclass: #RemoveChild
    instanceVariableNames: 'child'

RemoveChild class>>child: aChild
    ^self new
        child: aChild;
        yourself

RemoveChild>>child: anChild
    child := anChild

RemoveChild>>child
    ^child
\end{classdef}

Any component interested in this announcement registers its interest
using the method \mthind{on:do:}{on: anAnnouncementClass do: aBlock}
or \mthind{on: anAnnouncementClass send: aSelector to: anObject}{on:
  anAnnouncementClass send: aSelector to: anObject}.  The messages
\ct{on:do:} and \ct{on:send:to:} are strictly equivalent to the
messages \mthind{subscribe:do:}{subscribe: anAnnouncementClass do:
  aValuable}  and
\mthind{subscribe:send:to:}{subscribe: anAnnouncementClass send:
  aSelector to: anObject}. You can also ask an announcer to
\ct{unsubcribe:} an object.

In the following example, when a parent component is created, it
expresses interest in the \clsind{RemoveChild} event and specifies the
action that it will perform when such an event happens.

\begin{method}{}
Parent>>initialize
    super initialize.
    !\textbf{self session announcer on: RemoveChild do: [:it | self removeChild: it child] }!

Parent>>removeChild: aChild
    self children remove: aChild
\end{method}

And any component that wants to fire this event simply announces it by
sending in an instance of that custom announcement object.

\begin{method}{Announcing an event.}
Child>>removeMe
    self session announcer announce: (RemoveChild child: self)
\end{method}

Note that depending on where you place the announcer, you can even
have different sessions sending events to each other, or different
applications.

Announcements are not always the best way to establish communication
between components.  On one hand, announcements let you create loosely
coupled components and thus maximize reusability.  On the other hand,
they introduce additional complexity when you may be able solve your
communication problem with a simple message send.

%=============================================================
\ifx\wholebook\relax\else
   \bibliographystyle{jurabib}
   \nobibliography{scg}
   \end{document}
\fi
%=============================================================




%-----------------------------------------------------------------

%%% Local Variables:
%%% coding: utf-8
%%% mode: latex
%%% TeX-master: t
%%% TeX-PDF-mode: t
%%% ispell-local-dictionary: "english"
%%% End: