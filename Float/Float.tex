% $Author: ducasse $
% $Date: 2009-08-24 10:17:33 +0200 (Mon, 24 Aug 2009) $
% $Revision: 28563 $

% HISTORY:
% 2011-09-11 - Migrated to PharoBox: svn checkout https://XXX@scm.gforge.inria.fr/svn/pharobooks/PharoByExampleTwo-Eng


%=================================================================
\ifx\wholebook\relax\else
% --------------------------------------------
% Lulu:
	\documentclass[a4paper,10pt,twoside]{book}
	\usepackage[
		papersize={6.13in,9.21in},
		hmargin={.75in,.75in},
		vmargin={.75in,1in},
		ignoreheadfoot
	]{geometry}
	\input{../common.tex}
	\setboolean{lulu}{true}
% --------------------------------------------
% A4:
%	\documentclass[a4paper,11pt,twoside]{book}
%	\input{../common.tex}
%	\usepackage{a4wide}
% --------------------------------------------
    \graphicspath{{figures/} {../figures/}}
	\begin{document}
\fi
%=================================================================
%\renewmessage{\nnbb}[2]{} % Disable editorial comments
\sloppy
%=================================================================

\chapter{Fun with Floats}
\chapterauthor{\authornicolas{}}

Floats are inexact by nature and this can confuse programmers. In this chapter we present 
an introduction to this problem. The basic message is that Floats are what they are: inexact but fast numbers.


\section{Never test equality on floats}
The first basic principle is to never compare float equality. 
Let's take a simple case: the addition of two floats may not be equal to the float representing their
sum. For example \ct{0.1 + 0.2} is not equal to \ct{0.3}.

\begin{code}{}
(0.1 + 0.2) = 0.3
	returns false
\end{code}

Hey, this is unexpected, you did not learn that in school, did you? This behavior is surprising indeed, but it's normal since floats are inexact numbers. What is important to understand is that the way floats are printed is also impacting our understanding. Some approaches prints a simpler representation of reality than others. In early versions of Pharo printing \ct{0.1 + 0.2} were printing \ct{0.3}, now  printing it returns \ct{0.30000000000000004}.
This change was guided by the idea that it is better not to lie to the user. Showing the inexactness of float is better than hiding because one day or another we can be deeply bitten by them. 

\begin{code}{}
(0.2 + 0.1) printString
	returns '0.30000000000000004' 

0.3 printString
	returns	'0.3'
\end{code}	

We can see that we are in presence of two different numbers by looking at the hexadecimal values. 

\begin{code}{}
(0.1+0.2) hex 
	returns '3FD3333333333334'
0.3 hex 
	returns '3FD3333333333333' 
\end{code}

	
The method \ct{storeString} also conveys that we are in presence of two different numbers.
		
\begin{code}{}
(0.1+0.2) storeString 
	returns 	'0.30000000000000004' 
0.3 storeString 
	returns	'0.3'
\end{code}	
		
\paragraph{About \ct{closeTo:}.} One way to know if two floats are probably close enough to look like the same number is to use the message \ct{closeTo:}

\begin{code}{}
(0.1 + 0.2) closeTo: 0.3
	returns true

0.3 closeTo: (0.1 + 0.2)
	returns true
\end{code}		
		
		
\paragraph{About Scaled Decimals.} There is a solution if you absolutely need exact floating point numbers, use Scaled Decimals. Scaled Decimals are exact numbers so they exhibit the behavior you expected.

\begin{code}{}
0.1s2 + 0.2s2 = 0.3s2
	returns true
\end{code}		
		

\section{Disecting a Float}
To understand what operation is involved in above addition, we must know how Floats are represented internally in the computer: Pharo's Float format is a wide spread standard found on most computers - IEEE 754-1985 double precision on 64 bits
(See \url{http://en.wikipedia.org/wiki/IEEE_754-1985} for more details).
In this format, a Float is represented in base 2 by this formula: \[ sign \cdot mantissa \cdot 2^{exponent} \]
\begin{itemize}
\item The sign is represented on 1 bit.
\item The exponent is represented on 11 bits.
\item The mantissa is a fractional number in base two, with a leading 1 before decimal point, and with 52 binary digits after fraction point.
\end{itemize}

For example, a serie of 52 bits:\\
\phantom{\ct{1.}}\ct{0110010000000000000000000000000000000000000000000000}\\
 means the mantissa is:\\
 \ct{1.0110010000000000000000000000000000000000000000000000}\\
 which also represents the following fractions:  \[ 1 + \frac{0}{2} +  \frac{1}{2^2} +  \frac{1}{2^3} +  \frac{0}{2^4} +  \frac{0}{2^5} +  \frac{1}{2^6} + \cdots  +  \frac{0}{2^{52}} \]
The mantissa value is thus between 1 (included) and 2 (excluded) for normal numbers.



\paragraph{Building a Float.}
Let us construct such a mantissa:
\begin{code}{}
(#(0 2 3 6) detectSum: [:i | (2 raisedTo: i) reciprocal]) asFloat.
	returns 1.390625
\end{code}

Now let us multiply by $2^3$ to get a non null exponent:
\begin{code}{}
(#(0 2 3 6) detectSum: [:i | (2 raisedTo: i) reciprocal]) asFloat timesTwoPower: 3.
	returns 11.125
\end{code}

In Pharo, you can retrieve these informations:
 \begin{code}{}
11.125 sign.
	returns 1
	
11.125 significand.
	returns 1.390625
	
11.125 exponent.
	returns 3
\end{code}

In Pharo, there is no message to handle the normalized mantissa directly. Instead it is possible to handle the mantissa as an Integer after a 52 bits shift to the left. There is one good reason for this: operating on Integer is easier because arithmetic is exact. The result includes the leading 1 and should thus be 53 bits long for a normal number (that's the float precision):
 \begin{code}{}
11.125 significandAsInteger printStringBase: 2.
	returns '10110010000000000000000000000000000000000000000000000'
	
11.125 significandAsInteger highBit.
	returns 53
	
Float precision.
	returns 53
\end{code}

You can also retrieve the exact fraction corresponding to the internal representation of the Float:
 \begin{code}{}
 11.125 asTrueFraction.
	returns  (89/8)

(#(0 2 3 6) detectSum: [:i | (2 raisedTo: i) reciprocal]) * (2 raisedTo: 3).
	returns  (89/8)
\end{code}

Until there we retrieved the exact input we've injected into the Float. Are Float operations exact after all? Hem, no, we only played with fractions having a power of 2 as denominator and a few bits in numerator. If one of these conditions is not met, we won't find any exact Float representation of our numbers. In particular, it is not possible to represent 1/5 with a finite number of binary digits. Consequently, a decimal fraction like 0.1 cannot be represented exactly with above representation.
 \begin{code}{}
(1/5) asFloat = (1/5).
	returns false
	
(1/10) = 0.1
	returns false
\end{code}

Let us see in detail how we could get the fractional bits of \ct{1/10} \ie \mbox{\ct{2r1/2r101}}. For that, we must lay out the division:

\[
\begin{tabular}{r|l}
\multicolumn{1}{r|}{1\phantom{00000000}} & 101 \\  \cline{2-2}
10\phantom{0000000} & 0.00110011 \\ 
100\phantom{000000}  & \\
1000\phantom{00000}  & \\
-101\phantom{00000}  & \\
   11\phantom{00000}  & \\
   110\phantom{0000} & \\
  -101\phantom{0000} & \\
        1\phantom{0000} & \\
        10\phantom{000} & \\
        100\phantom{00} & \\
        1000\phantom{0} & \\
         -101\phantom{0} & \\
             11\phantom{0} & \\
             110 & \\
            -101 & \\
                  1 & \\
\end{tabular} 
 \]

What we see is that we get a cycle: every 4 Euclidean divisions, we get a quotient \ct{2r0011} and a remainder \ct{1}. That means that we need an infinite series of this bit pattern \ct{0011} to represent 1/5 in base 2. Let us see how Pharo dealt to convert (1/5) to a Float:

 \begin{code}{}
(1/5) asFloat significandAsInteger printStringBase: 2.
	returns '11001100110011001100110011001100110011001100110011010'
	
(1/5) asFloat exponent.
	returns -3
\end{code}

That's the bit pattern we expected, except the last bits \ct{001} have been rounded to upper \ct{010}. This is the default rounding mode of Float, round to nearest even. We now know why \ct{0.2} is represented inexactly in machine. It's the same mantissa for \ct{0.1}, and its exponent is \ct{-4}.

So, when we entered \ct{0.1} + \ct{0.2}, we didn't get exactly (1/10) + (1/5).
Instead of that we got:

 \begin{code}{}
0.1 asTrueFraction + 0.2 asTrueFraction.
	returns  (10808639105689191/36028797018963968)
\end{code}

But that's not all the story... Let us inspect the bit pattern of above fraction, and check the span of this bit pattern, that is the position of highest bit set to 1 (leftmost) and position of lowest bit set to 1 (rightmost):
 \begin{code}{}
10808639105689191 printStringBase: 2.
	returns  '100110011001100110011001100110011001100110011001100111'
	
10808639105689191 highBit.
	returns  54
	
10808639105689191 lowBit.
	returns  1

36028797018963968 printStringBase: 2.
	returns  '10000000000000000000000000000000000000000000000000000000'
\end{code}

The denominator is a power of 2 as we expect, but we need 54 bits of precision to store the numerator... Float only provides 53. There will be another rounding error to fit into Float representation:

\begin{code}{}
(0.1 asTrueFraction + 0.2 asTrueFraction) asFloat = (0.1 asTrueFraction + 0.2 asTrueFraction).
	returns false
	
(0.1 asTrueFraction + 0.2 asTrueFraction) asFloat significandAsInteger.
	returns '10011001100110011001100110011001100110011001100110100'
\end{code}

To summarize what happened, including conversions of decimal representation to Float representation:\\
\begin{tabular}{lll}
(1/10) asFloat & 0.1 & inexact (rounded to upper) \\
(1/5) asFloat & 0.2 & inexact (rounded to upper) \\
(0.1 + 0.2) asFloat & $\cdots$ & inexact (rounded to upper)  \\
\end{tabular}

3 inexact operations occurred, and, bad luck, the 3 rounding operations were all to upper, thus they did cumulate rather than annihilate.
On the other side, interpreting \ct{0.3} is causing a single rounding error (3/10) asFloat.
We now understand why we cannot expect \ct{0.1 + 0.2 = 0.3}.

As an exercise, you could show why \ct{1.3} *  \ct{1.3} $\neq$ \ct{1.69}.


\section{With floats, printing is inexact}

One of the biggest trap we learned with above example is that despite the fact that \ct{0.1} is printed \ct{'0.1'} as if it were exact, it's not.
The name \ct{absPrintExactlyOn:base:} used internally by \ct{printString} is a bit confusing, it does not print exactly,
but it prints the shortest decimal representation that will be rounded
to the same Float when read back (Pharo always converts the decimal representation to the nearest Float).

Another message exists to print exactly, you need to use \ct{printShowingDecimalPlaces:} instead.
As every finite Float is  represented internally as a Fraction with a
denominator being a power of 2, every finite Float has a decimal
representation with a finite number of decimals digits (just multiply
numerator and denominator with adequate power of 5, and you'll get the
digits). Here you go:
\begin{code}{}
0.1 asTrueFraction denominator highBit.
	returns  56
\end{code}
This means that the fraction denominator is $2^{55}$ and that you need 55 decimal digits after the decimal point to really print internal representation of \ct{0.1} exactly.
\begin{code}{}
0.1 printShowingDecimalPlaces: 55.
	returns '0.1000000000000000055511151231257827021181583404541015625'
\end{code}
And you can retrieve the digits with:
\begin{code}{}
0.1 asTrueFraction numerator * (5 raisedTo: 55).
	returns  1000000000000000055511151231257827021181583404541015625
\end{code}
You can just check our result with:
\begin{code}{}
1000000000000000055511151231257827021181583404541015625/(10 raisedTo: 55) =  0.1 asTrueFraction
	returns true
\end{code}
You see that printing the exact representation of what is implemented in machine would be possible but would be cumbersome. Try printing \ct{1.0e-100} exactly if not convinced.



\section{Float rounding is also inexact}
While float equality is known to be evil, you have to pay attention to other aspects of floats. Let us illustrate that point with the following example.

\begin{code}{Float truncation can be surprising}
2.8 truncateTo: 0.01
	returns 2.8000000000000003

2.8 roundTo: 0.01
	returns 2.8000000000000003
\end{code}

It is surprising but not false that \ct{2.8 truncateTo: 0.01} does not return \ct{2.8} but \ct{2.8000000000000003}. This is because \ct{truncateTo:} and \ct{roundTo:} perform several operations on floats: inexact operations on inexact numbers can lead to cumulative rounding errors as you saw above, and that's just what happens again.

Even if you perform the operations exactly and then round to nearest Float, the result is inexact because of the initial inexact representation of \ct{2.8}  and  \ct{0.01}. 
\begin{code}{}
(2.8 asTrueFraction roundTo: 0.01 asTrueFraction) asFloat
	returns 2.8000000000000003
\end{code}

Using \ct{0.01s2} rather than  \ct{0.01} let this example appear to work:
\begin{code}{}
2.80 truncateTo: 0.01s2
	returns  2.80s2
	
2.80 roundTo: 0.01s2
	returns  2.80s2
\end{code}

But it's just a case of luck, the fact that \ct{2.8} is inexact is enough to cause other surprises as illustrated below:
\begin{code}{}
2.8 truncateTo: 0.001s3.
	returns 2.799s3
	
2.8 < 2.800s3.
	returns true
\end{code}

Truncating in the Float world is absolutely unsafe. Though, using a ScaledDecimal for rounding is unlikely to cause such discrepancy, except when playing with last digits.



\section{Fun with Inexact representations}
To add a nail to the coffin, let's play a bit more with inexact representations. Let us try to see the difference between different numbers: 

\begin{code}{}
{
((2.8 asTrueFraction roundTo: 0.01 asTrueFraction) - (2.8 predecessor)) abs -> -1.
((2.8 asTrueFraction roundTo: 0.01 asTrueFraction) - (2.8)) abs -> 0.
((2.8 asTrueFraction roundTo: 0.01 asTrueFraction) - (2.8 successor)) abs -> 1.
} detectMin: [:e | e key ]

returns the pair
	0.0->1
\end{code}

The following expression returns \ct{0.0->1}, which means that:
 \ct{(2.8 asTrueFraction roundTo: 0.01 asTrueFraction) asFloat = (2.8 successor)}

But remember that 

\begin{code}{}
(2.8 asTrueFraction roundTo: 0.01 asTrueFraction) ~= (2.8 successor)
\end{code}

It must be interpreted as the nearest Float to \ct{(2.8 asTrueFraction roundTo: 0.01 asTrueFraction)} is \ct{(2.8 successor)}.

If you want to know how far it is, then get an idea with:

\begin{code}{}
((2.8 asTrueFraction roundTo: 0.01 asTrueFraction) - (2.8 successor asTrueFraction)) asFloat
	returns -2.0816681711721685e-16
\end{code}



\section{Conclusion}
Floats are inexact numbers. Pay attention when you manipulate them.
There are much more things to know about floats, and if you are advanced enough, it would be a good idea to check this link from the wikipedia page \href{http://www.validlab.com/goldberg/paper.pdf}{"What Every Computer Scientist Should Know About Floating-Point Arithmetic"}.


%=========================================================
\ifx\wholebook\relax\else
   \bibliographystyle{jurabib}
   \nobibliography{scg}
   \end{document}
\fi

%=================================================================
\ifx\wholebook\relax\else\end{document}\fi
%=================================================================

%-----------------------------------------------------------------

%%% Local Variables:
%%% coding: utf-8
%%% mode: latex
%%% TeX-master: t
%%% TeX-PDF-mode: t
%%% ispell-local-dictionary: "english"
%%% End: