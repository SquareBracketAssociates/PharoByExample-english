% $Author: oscar $
% $Date: 2007-06-06 15:10:38 +0200 (Wed, 06 Jun 2007) $
% $Revision: 23297 $
%=============================================================
% Demo and test the common macros
%=================================================================
\documentclass[a4paper,10pt,twoside]{book}
%=============================================================
\input{common.tex}
\graphicspath{{figures/}} % for \dothis
%=============================================================
\begin{document}
\sloppy
\mainmatter
%=============================================================
\section*{Code environments using the listings package}
%=============================================================
\paragraph{Sample code environment}
Use the script, method, classdef and example environments.
Each takes a name and an optional label.
With labels we can have cross references to
\verb|\egref{history}| \egref{history},
\verb|\scrref{helloworld}| \scrref{helloworld},
\verb|\clsref{myclass}| \clsref{myclass},
and \verb|\mthref{doit}| \mthref{doit}.

\begin{verbatim}
\begin{example}[history]{The first thing that Smalltalk could do}{}
3 + 4 --> 7
\end{example}
\end{verbatim}
\begin{example}[history]{The first thing that Smalltalk could do}{}
3 + 4 --> 7
\end{example}

\begin{verbatim}
\begin{script}[helloworld]{The first thing you should try}
Transcript show: 'hello world'
\end{script}
\end{verbatim}
\begin{script}[helloworld]{The first thing you should try}
Transcript show: 'hello world'
\end{script}

\begin{verbatim}
\begin{classdef}[myclass]{MyClass is defined}
Object subclass: #MyClass
	instancevariables: '...'
	...
\end{classdef}
\end{verbatim}
\begin{classdef}[myclass]{MyClass is defined}
Object subclass: #MyClass
	instancevariables: '...'
	...
\end{classdef}

\begin{verbatim}
\begin{method}[doit]{A doit method}
MyClass>>>doit
<tab>^ super doit
\end{method}
\end{verbatim}
\begin{method}[doit]{A doit method}
MyClass>>>doit
	^ super doit
\end{method}

The plain code environment has no title, label or numbering:

\begin{verbatim}
\begin{code}{}
just some plain code
\end{code}
\end{verbatim}
\begin{code}{}
just some plain code
\end{code}

%=============================================================
\paragraph{Listings environments and macros}
The code environments 

\begin{verbatim}
\begin{code}{}
    ...
\end{code}
\end{verbatim}
take plain, verbatim code,
and translate some special characters like $\wedge$ to \ct{^}. Even tabs are handled, (which is not true for verbatim).
\begin{code}{}
"All handled correctly: ^ $ ' % \\ << >> _ { }"
"NB: If you !{\bf really}! want an exclamation mark you must spell it BANG"
"
If you really want a caret to look like this: CARET instead of this: ^, you must spell it out."
| y |
true & false not & (nil isNil) ifFalse: [self halt].
y _ self size + super size.
#($a #a 'a' 1 1.0)
	do: [:each | Transcript 
			show: (each class name); 
                     show: ' ';
                     show: (each printString).
{ 1 + 2 . 3 \\ 4 . 1 << 3. 2 >> 5 . 1 % 2 }.
^ x < y 
\end{code}

QWERTY layout:
\begin{code}{}
BANG @ # $ % ^ & * ( ) UNDERSCORE +
1 2 3 4 5 6 7 8 9 0 - =
 Q W E R T Y U I O P { }
 q w e r t y u i o p [ ]
  A S D F G H J K L : "" |   		 (twice " to turn off italics")
  a s d f g h j k l ; ' \
   Z X C V B N M < > ?
   z x c v b n m , . /
\end{code}

LaTeX escape:
\begin{verbatim}
\begin{code}{}
plain code and !\textbf{bolded text}!
\end{code}
\end{verbatim}

\begin{code}{}
plain code and !\textbf{bolded text}!
\end{code}

% ^ $ \\ % # ' 

In-line code with \verb|\ct| is typed like this \verb|\ct{1 + 2 --> 3}| and looks like this: \ct{1 + 2 --> 3}, text can follow immediately.  The ``brackets'' around \verb|\ct| can be any matching pair of characters, useful if you want \ct${ and }$ in the code.

\subsection{Special chars with $\backslash$ct}
\ct=^ ~ # $ ' % \\ << >> _ {  } ! -- --> =\\
\verb|\ct=^ ~ # $ ' % \\ << >> _ {  } ! -- --> =|

\subsection{Special conventions}

\verb$\ct{Class>>>method}$ prints as \ct{Class>>>method}.\\
\verb$\ct{3 + 4 - 5 --> 2}$ prints as \ct{3 + 4 - 5 --> 2}.

Use \verb$@TEST$ to include the code in the automatic tests and use \verb$-->$ to represent the expected result
\begin{verbatim}
\begin{code}{@TEST}
true       --> true
3@4     --> 3@4
$a         --> $a
#(1 2 3) --> #(1 2 3)
\end{code}
\end{verbatim}

%=============================================================
%\paragraph{New code environment}

%Experiment:

%\begin{testenv}{testing}
%self doit
%\end{testenv}

%=============================================================
\section*{Other macros}
%=============================================================
\paragraph{url}
\url{SqueakByExample.org}
%=============================================================
\paragraph{names}
\SUnit
\xUnit
\st
\squeak
\pharo
%=============================================================
\paragraph{editorial}
\fix \ugh{please rephrase this}
\ins{please insert this text}
\del{delete this}
and
\chg{change this}{to this}
\ab{...}
%=============================================================
\paragraph{abbreviation}
\ie
\eg
\etc
%=============================================================
\paragraph{Smalltalk macros}
sep: \sep
%=============================================================
\paragraph{scat and prot}
System category \scat{Kernel-Objects} and protocol \prot{accessing}.
%=============================================================
\paragraph{menu}
\menu{World menu \go open ...}
%=============================================================
\paragraph{button}
\button{Create}
%=============================================================
\paragraph{do this}
\verb|\dothis|
\ct{-->}
\dothis{Download and install \pharo.}
%=============================================================
\paragraph{Keyboard shortcut}
\verb|\short{d}|
\ct{-->}
\short{d}
%=============================================================
\section*{Footnote citations}
There is a great book on \squeak by Ducasse\cite{Duca05j}.
%=============================================================
\section*{Important stuff}
\important{This is really important}
%=============================================================
% \bibliographystyle{plain}
%\bibliographystyle{jurabib}
%\nobibliography{scg}

%
% This bibliography was produced by using jurabib.bst
%
\begin{thebibliography}{}

\bibitem[{Ducasse\jbdy {2005}}%
         {}%
         {{0}{}{book}{2005}{}{}{}{}%
          {APress\bibbdsep {} 2005}}%
         {{Squeak: Learn Programming with Robots}%
          {}{}{2}{}{}{}{}{}}%
        ]{Duca05j}
 \jbbibargs {\bibnf {Ducasse} {St\'ephane} {S.} {} {}} {St\'ephane Ducasse}
  {au} {\bibtfont {Squeak: Learn Programming with Robots}\bibatsep\  \apyformat
  {APress\bibbdsep {} 2005} \jbnote {1} {ISBN: 1-59059-491-6} } {\bibhowcited}
  \jbendnote {ISBN: 1-59059-491-6} \jbdoitem {{Ducasse}{St\'ephane}{S.}{}{}} {}
  {} \bibAnnote {book}

\end{thebibliography}

%=============================================================
\end{document}
%=============================================================
