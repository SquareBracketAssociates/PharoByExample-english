% $Author$
% $Date$
% $Revision$
%=================================================================
\ifx\wholebook\relax\else
% --------------------------------------------
% Lulu:
	\documentclass[a4paper,10pt,twoside]{book}
	\usepackage[
		papersize={6in,9in},
		hmargin={.75in,.75in},
		vmargin={.75in,1in},
		ignoreheadfoot
	]{geometry}
	\input{../common.tex}
	\pagestyle{headings}
	\setboolean{lulu}{true}
% --------------------------------------------
% A4:
%	\documentclass[a4paper,11pt,twoside]{book}
%	\input{../common.tex}
%	\usepackage{a4wide}
% --------------------------------------------
    \graphicspath{{figures/} {../figures/}}
	\begin{document}
	\renewcommand{\nnbb}[2]{} % Disable editorial comments
	\sloppy
\fi
%=================================================================
%Noury's commands
\newcommand{\nouryComment}[1]{\textcolor{red}{\textbf{\emph{#1}}}}

%=================================================================

\chapter{Sockets}\label{cha:sockets}
Modern softwares are often distributed on multiple devices and collaborate through a network in order to achieve some task.
The basic approach to set up such a collaboration is to use {\em sockets}.
A typical use is in the World Wide Web.
Browsers and servers do interact through HTTP sockets.

The concept of socket was first introduced by researchers from Berkley University in the \nouryComment{1960's}.
They defined the first socket API for the C programming language in the context of Unix operating systems.
Since then, the concept of socket was spread out to other operating systems.
It's API was ported to other programming languages including Smalltalk.

In this chapter, we present the socket's API in the context of Pharo Smalltalk.
We show through some examples how to to use sockets for building both clients and servers.

\section{Basic Concepts}
\subsection{Socket}
A remote communication involves at least two softwares exchanging some data bytes through a network.
Each software accesses the network through at least one socket (see figure~\ref{fig:socketConcept}).
A socket can then be defined as a {\em plug on a communication network}.

\begin{figure}[ht]\centering
	\includegraphics[width=.75\linewidth]{socketConcept.pdf}
	\caption{Softwares Remote Communication Through Sockets.}
	\label{fig:socketConcept}
\end{figure}

Sockets are used to achieve a bidirectional communication.
They allow both sending and receiving data.
Such interaction can be done according to some communication protocol which is encapsulated by the used sockets.
On the Internet and other networks such as ethernet LANs\footnote{Local Area Networks.}, two basic protocols widely used are {\em TCP/IP} and {\em UDP/IP}.

\subsection{TCP/IP vs UDP/IP}
TCP/IP stands for {\em Transmission Control Protocol / Internet Protocol} (TCP for short).
The use of TCP guarantees a reliable communication (no data loss).
It requires that softwares involves in the communication get connected before actually communicating. 
Once a connection is established interacting parties can send and receive an arbitrary amount of bytes.
This is often referred to as {\em stream communication}.
Data reach the destination in the same order of their sending.

UDP/IP stands for {\em User Datagram Protocol / Internet Protocol} (UDP for short).
Datagrams are chunks of data which size can not exceed 64KB.
UDP is an unreliable protocol because of two reasons.
First, UDP does not guarantee that  datagrams will actually reach there destination.
The second reason why UDP is qualified as unreliable is that the order of receptions of multiple datagrams from a single sender to some particular receiver may arrive in an arbitrary order.
However UDP is faster than TCP since no connection is required before sending data.
A typical use of UDP is  ``heart-beating'' as used in server-based social software, where clients need to notify the server their status (e.g. Requesting interactions, or Invisible).

\nouryComment{In a future version we will present UDP. But, so  far UDP does not work in \pharo.}
In the reminder of this chapter we will focus on TCP Sockets.
We first will present how to develop a simple client application using a socket
 
\section{Simple TCP Client}
We call {\em TCP client}  a software that initiates a TCP connection in order to exchange data with another software: the {\em server}.
It is important to mention that the client and the server may be developed in different languages.
The life-cycle of such a client in \pharo decomposes into 4 steps:
\begin{enumerate}
\item Create a TCP socket.
\item Connect the socket to some server.
\item Exchange data with the server through the socket.
\item Close the socket.
\end{enumerate}

\subsection{Create a TCP Socket}
\pharo provides a single socket class.
At creation, the socket type (TCP or UDP) is provided to the socket plugin of the virtual machine.
To create a TCP socket, you need to evaluate the following message 
\begin{center}
\ct!Socket newTCP!.
\end{center}
Method \ct!newTCP! hands out the tcp type (which is stored in the class variable \ct!TCPSocketType!) to set up a TCP socket.

\subsection{Connect a TCP Socket to some Server}
In order to connect a TCP Socket to some server, you need to have the object representing the IP address of the server.
This address is a instance of \ct!SocketAddress!.
A handy way to create it is to use \ct!NetNameResolver! that  provides IP style network name lookup and translation facilities.

\scrref{creatingSocketAddress} provides two examples of socket address creation.
The first one creates an address from a string from the server name (\ct!'www.esug.org'!), while the second does the creation from a string representing the server's IP address (\ct!'127.0.0.1'!).
Note that to use the \ct!NetNameResolver! you need a network connection.
The only exception is for retrieving the local host address.

\nouryComment{There are dedicated methods for retreiving the localhost address but they are buggy at the time of writing this text.}
\nouryComment{The NameLookupFailure exception is used ONLY in NetNameResolver class>>addressForName:timeout:. It should be more widely used.}

\begin{script}[creatingSocketAddress]{Creating a Socket Address}
| esugAddress localAddress |
esugAddress := NetNameResolver addressForName: 'www.esug.org'.
localAddress := NetNameResolver addressForName: '127.0.0.1'.
\end{script}

Now we can connect our TCP socket to our server as shown in \scrref{connectingTcpSocket}.
Message \ct!connectTo:port:! attempts to connect the socket to the server which address and port are provided as parameters.
\nouryComment{SocketAddress can also store the port of the server. 
However, NetNameResolver does NOT support building socket address WITH port.}
This message returns immediately after issuing to the system (through a primitive call) the request to connect the socket.
Message \ct!waitForConnectionFor: 10! suspends the current process until the socket is connected to the server.
It waits at most \ct!10! seconds as provided in the parameter.
If the socket is not connected after 10 seconds, the \ct!ConnectionTimedOut! exception is signaled. 
Otherwise, the execution can carry on by evaluating the expression \ct!clientSocket isConnected! which obviously answers \ct!true!.

\begin{script}[connectingTcpSocket]{Connecting a TCP Socket to some Server.}
| clientSocket serverAddress |
clientSocket := Socket newTCP.
serverAddress := NetNameResolver addressForName: 'www.esug.org'.
clientSocket 
	connectTo: serverAddress port: 80;
	waitForConnectionFor: 10.
clientSocket isConnected
\end{script}

\subsection{Exchange Data with Server}
\label{sec:exchangeData}
Once connection is established, the client can send/receive data to/from the server.
By data we mean a \ct!ByteString!\footnote{One can send a \ct!ByteArray!, but the receiving socket will return a \ct!ByteString! however.}.
Typically, the client sends some request to the server and then expects receiving some response.
Web browsers act according to this schema.
A web browser is a client that issues a request to some web server identified by the URL.
Such request is often the path to some resource on the server such as an html file or a picture.
Then, the browser awaits the server response (e.g. html code, picture bytes).

\begin{script}[dataExhangeWithTcpSocket]{Exchanging Data with some Server through a TCP Socket.}
|clientSocket data|
... ``create and connect the TCP clientSocket''
clientSocket sendData: 'Hello server'.
data := clientSocket receiveData.
... ``Process data''
\end{script}

\scrref{dataExhangeWithTcpSocket} shows the protocol to send and receive data through a client socket.
Here, we send the string \ct-'Hello server!'- to the server using the the \ct!sendData:! message.
Next, we send the \ct!receiveData! to our client socket to make it wait for data reception.
Then, the content of variable \ct!data! is processed.

\begin{script}[dataReceptionTimeOut]{Bounding the Maximum Time for Data Reception.}
|clientSocket data|
... ``create and connect the TCP clientSocket''
[data := clientSocket receiveDataTimeout: 5.
... ``Process data''
] on: ConnectionTimedOut 
do: [:timeOutException|
	Transcript 
		cr; 
		show: 'No data receivedBANG';
		space;
		show: 'Network connection is too slow or server is down.']
\end{script}

%Timeout
Note that by using  \ct!receiveData!, the client waits until the server either sends no more data, or close the connection.
This means that the client may wait indefinitely.
An alternative would be to have the client signal a \ct!ConnectionTimedOut! exception if it had waited too much as shown in \scrref{dataReceptionTimeOut}.
We use message \ct!receiveDataTimeout:! to ask the client socket to wait for 5 seconds.
If data is received during this period of time, it is processed silently.
But if no data is received during the 5 seconds, a \ct!ConnectionTimedOut! is signaled.
So, we log a description of what happened on the \ct!Transcript!.

\subsection{Close the Socket}
A socket remains alive while devices at both ends are connected.
Once the interaction is over, either the server or the client can decide to close the socket.
This can be done by sending the \ct!close! message to the socket.
The image where this message is evaluated will then send a close request to the other side.
The socket remains connected until the other side close it.
However, this may last indefinitely when there is an network failure or when the other side is down.
This is why sockets also answer the \ct!destroy! message, that frees system resources required by the socket.

In practice we do use \ct!closeAndDestroy!.
It first attempts to close the socket by sending the \ct!close! message.
Then, if the socket is still connected after a duration of 20 seconds, the socket is destroyed.
Note that there exist a variant \ct!closeAndDestroy: seconds! which gets as a parameter the duration to wait before destroying the socket.

\begin{script}[closeAndDestroy]{Closing a TCP Socket After Connection to a Web Site.}
| clientSocket serverAddress httpQuery htmlText |
httpQuery := 'GET / HTTP/1.1', String crlf, 
	'Host: www.esug.org:80', String crlf, 
	'Accept: text/html', String crlfcrlf.
Transcript cr; cr; show: 'Attempt to get a web page...'.
serverAddress := NetNameResolver addressForName: 'www.esug.org'.
clientSocket := Socket newTCP.
[clientSocket 
	connectTo: serverAddress port: 80;
	waitForConnectionFor: 10.
 clientSocket sendData: httpQuery.
 htmlText  := clientSocket receiveDataTimeout: 5.
 Transcript cr; show: htmlText.
] ensure: [clientSocket closeAndDestroy].
Transcript cr; show: '...Done'
\end{script}

To summarize all steps described so far, we use the example of getting a web page from a server in \scrref{closeAndDestroy}.
First, we retrieve the IP address of the \url{www.esug.org} server.
Then, we create a TCP socket and connect it to the server.
We use the IP address we get in the previous step and the default port for web servers: 80.
Next we forge the HTTP\footnote{HyperText Transfer Protocol used for web communications.} query.
The string corresponding  to our query starts with the \ct!GET! keyword, followed by a slash saying that we would like to get the root file of the server.
Follows the protocol version \ct!HTTP/1.1!.
The second line recalls the host name and port.
The third and last line of the HTTP query refers to format accepted by our client. 
Since, we intend to display the result of our query on the \ct!Transcript!, we state that our client accepts texts with html format.
After sending the http query, we wait at most 5 seconds for the html text that we display on the \ct!Transcript!.
Socket connection, query sending and html reception are inside a block which execution is ensured to end with cleaning up socket related resources, by means of the \ct!closeAndDestroy! message.

\section{Simple TCP Server}
A {\em TCP Server} is a software that awaits TCP connections from TCP clients.
Once connection established, both the server and the client can send a receive data in any order.
A big difference between the server and the client is that the server uses at least two sockets.

\subsection{TCP Socket Server Life-cyle}
The life-cycle of a TCP server in \pharo  has 5 steps:
\begin{enumerate}
\item Create a first TCP socket, let's call it {\em socket$_1$}.
\item Wait for connections by making {\em socket$_1$}  listen on some port.
\item Accept a client request for connection. As a result,  {\em socket$_1$} will build a second socket, let's call it {\em socket$_2$}.
\item Exchange data with the client through {\em socket$_2$}. In the meanwhile, {\em socket$_1$} can continue to wait for connections, and possibly create new sockets to exchange data with other clients. 
\item Close {\em socket$_2$}.
\item Close {\em socket$_1$} when we decide to kill the server and stop accepting client connections.
\end{enumerate}

Concurrence is implicit in this life-cycle.
The server listens for incoming client's connection requests through {\em socket$_1$}, while exchanging data with some client through {\em socket$_2$}.
The server can even simultaneously exchange data with multiple clients through different sockets.
In the following, we first illustrate the socket serving machinery.
Then, we give a complete server class and explain the server's life-cycle and related concurrency issues.

\subsection{Serving Basic Example}
We illustrate the serving basics through a simple example of an echo TCP server that accepts a single client request.
It sends back to clients whatever data it received and quits.
The code is provided by \scrref{servingBasicExample}.

\begin{script}[servingBasicExample]{Basic Echo Server.}
| connectionSocket interactionSocket |
connectionSocket := Socket newTCP.
connectionSocket listenOn: 9999 backlogSize: 10.
interactionSocket := connectionSocket waitForAcceptFor: 60.
connectionSocket closeAndDestroy.
receivedData := interactionSocket receiveData.
Transcript cr; show: receivedData.
interactionSocket sendData: 'ECHO: ', receivedData.
interactionSocket closeAndDestroy.
\end{script}

First, we create the socket that we will use for handling incoming connections.
We set it up to listen on port 9999.
The \ct!backlogSize! is set to 10 meaning, that we allocate a buffer for 10 connection requests.
This backlog will not be actually used in this example.
But, a more realistic server will have to handle connections and then process connection requests in the backlog.

Once the connection socket is prepared, with make it starting listening for client connections.
The \ct!waitForAcceptFor: 60! message makes the socket wait connection requests for 60 seconds.
If no client attempts to connect during these 60 seconds, the message answers \ct!nil!.
Otherwise, we get a new socket \ct!interactionSocket! connected the client's socket.
At this point, we don't need the connection socket anymore, so we can close it (\ct!closeAndDestroy! message).

Since the interaction socket is already connected connected to the client, we can use it to exchange data.
Messages \ct!receiveData! and \ct!sendData:! presented above (see section~\ref{sec:exchangeData}) can be used to this end.
In our example, we wait for data from the client, next we display it on the \ct!Transcript!, and last we send it back to the client prefixed with the \ct!'ECHO: '! string.
Last, we finish the interaction with the client by closing the interaction socket.

\begin{script}[echoClient]{Echo Client.}
| clientSocket serverAddress echoString |
serverAddress := NetNameResolver addressForName:'127.0.0.1'.
clientSocket := Socket newTCP.
[clientSocket 
	connectTo: serverAddress port: 9999;
	waitForConnectionFor: 10.
 clientSocket sendData: 'Hello PharoBANG'.
 echoString  := clientSocket receiveDataTimeout: 5.
 Transcript cr; show: echoString.
] ensure: [clientSocket closeAndDestroy].
\end{script}

In order to test the above example, you have to use two different images, one that runs the server code and one for the client code.
Indeed, since we use the user interaction process, the \pharo UI will be frozen at some points, such as during the \ct!waitForAcceptFor:!.
\scrref{echoClient} provides the code to run on the client image.
Note that you have to run the server code first.

%An alternative to using the client is to use the netcat on Unix machines

\subsection{Echo Server Class}
We define here the  \ct!EchoServer! class that deals with concurrency issues.
It allows dealing concurrently with multiple client queries and it does not freeze the UI.
As we can see in the definition labelled as \clsref{echoServerClassDef}, the \ct!EchoServer! declares three instance variables.
The first one (\ct! connectionSocket!) refers to the socket used for listening to client connections.
The two last instance variables (\ct!isRunning! and \ct!isRunningLock!) are used to manage the server process life-cycle while dealing with synchronization issues.
 
\begin{classdef}[echoServerClassDef]{\ct!EchoServer! Class Definition}
Object subclass: #EchoServer
	instanceVariableNames: 'connectionSocket isRunning isRunningLock'
	classVariableNames: ''
	poolDictionaries: ''
	category: 'SimpleSocketServer'
\end{classdef}

\begin{method}[meth:serverInitialize]{The \ct!EchoServer>>>initialize! Method}
EchoServer>>>initialize 
	super initialize.
	isRunningLock := Mutex new.
	self isRunning: false
\end{method}

\begin{method}[meth:isRunningRead]{The \ct!EchoServer>>>isRunning! Read Accessor}
EchoServer>>>isRunning
	^isRunningLock critical: [isRunning]
\end{method}

\begin{method}[meth:isRunningWrite]{The \ct!EchoServer>>>isRunning:! Write Accessor}
EchoServer>>>isRunning: aBoolean
	isRunningLock critical: [isRunning := aBoolean]
\end{method}

The \ct! isRunning! instance variable is a flag that is set to \ct!true! while the serving is running.
As we'll see below, it can be accessed by different processes.
Therefore, we use the mutex (see \mthref{meth:serverInitialize}) stored in the \ct!isRunningLock! instance variable in order to ensure that only one process can read or write the value of \ct!isRunning! (use of the \ct!critical:! message in \mthref{meth:isRunningRead} and \mthref{meth:isRunningWrite}).

\begin{method}[meth:stop]{The \ct!EchoServer>>>stop! Method}
EchoServer>>>stop
	self isRunning: false
\end{method}

\begin{method}[meth:start]{The \ct!EchoServer>>>start! Method}
start
	isRunningLock critical: [
		self isRunning ifTrue: [^self].
		self isRunning: true].
	connectionSocket := Socket newTCP.
	connectionSocket listenOn: 9999 backlogSize: 10.
	[self serve] fork
\end{method}

In order to manage the life-cycle of our server, we introduced two methods \ct!EchoServer>>>start! and \ct!EchoServer>>>stop!.
Let's start with the simplest one \ct!EchoServer>>>stop! which definition is provided as \mthref{meth:stop}.
It simply sets the \ct!isRunning! flag  to false.
This will have the consequence of stopping the serving process created by the \ct!EchoServer>>>start! (see the last line of \mthref{meth:start}).
%%%%%%CONTINUE FROM HERE
In order to avoid duplicating servers...

\section{SocketStream}
\section{Network Interface Initialization?}
NetNameResolver class>>initializeNetwork
\section{A Basic Chat Application}
\section{Distributed Object Applications}
\section{Chapter summary}
%Reference to chapter remoteSmalltalk (rST) � faire!


%=================================================================
\ifx\wholebook\relax\else\end{document}\fi
%=================================================================
