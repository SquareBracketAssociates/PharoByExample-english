% $Author: ducasse $
% $Date: 2009-08-24 10:17:33 +0200 (Mon, 24 Aug 2009) $
% $Revision: 28563 $

%=================================================================
\ifx\wholebook\relax\else
% --------------------------------------------
% Lulu:
	\documentclass[a4paper,10pt,twoside]{book}
	\usepackage[
		papersize={6.13in,9.21in},
		hmargin={.75in,.75in},
		vmargin={.75in,1in},
		ignoreheadfoot
	]{geometry}
	\input{../common.tex}
	\setboolean{lulu}{true}
% --------------------------------------------
% A4:
%	\documentclass[a4paper,11pt,twoside]{book}
%	\input{../common.tex}
%	\usepackage{a4wide}
% --------------------------------------------
    \graphicspath{{figures/} {../figures/}}
	\begin{document}
\fi
%=================================================================
%\renewmessage{\nnbb}[2]{} % Disable editorial comments
\sloppy

%=================================================================
%\renewcommand{\nnbb}[2]{#2} % Disable editorial comments

\chapter{First stop: VM's SCM and related}

You want to compile your own VM, don't you? Compiling the VM just for compiling it and following some instructions is not really helpful, otherwise why don't you directly download the VM binary?  But to understand and learn about the VM, it is important 

To compile the VM, you will have to deal with the problem of the VM's Software Configuration Management. The first time I tried to compile the Pharo/Squeak VM was like 2 years ago. After that, I tried few times more, and most of the times I got some troubles. In addition, in the last months not only there have been a lot of changes related to code versioning and management, but also Cog VM come into play. A lot of people is confused where each part of the VM is committed, or what is needed to compile each VM. I will clarify all that so that in the next post we can finally compile the VM by ourself.

Since the Interpreter VM and Cog VM are a little different regarding the code management, I will split them.

Note that a new infrastructure is put in place to simplify all this. It is based on git and connected to hudson build servers. 

\section{Interpreter VM}

\paragraph{Downloading code.}
If you remember from the previous chapter, we have 2 parts: VMMaker with the core of the VM, and the platform code. For VMMaker it is easy: it is the VMMaker package in squeaksource (\url{http://www.squeaksource.com/VMMaker.html}). The platform code is the official SVN. This sounds pretty straightforward, doesn't it?  But this is not totally true. There are several problems (some may probably be because of my ignorance) that I have found with this approach:

\begin{enumerate}
\item The package VMMaker is not self contained, i.e, it has dependencies on other packages (some packages in the same repository and some in others). First problem, you need to know which other packages you need. For example, to build the VM you may need also the packages: 'FFI-Pools', 'SharedPool-Speech', 'MemoryAccess', 'SlangBrowser', 'Balloon3D-Plugins', 'Plugin-XXX', etc.

\item Similar to the previous item, there is not only the problem of knowing which packages are needed, but instead which version. So...how do you know that for 'VMMaker-dtl.221' you need 'FFI-Pools-eem.2', 'MemoryAccess-dtl.3', 'Balloon3D-Plugins-ar.6', etc ?  Using just the last version of every package does work all the times.

\item Sync between VMMaker and platform code. How do you know for each VMMaker version which SVN version you need of the platform code? or vice-versa how do you know which VMMaker version you need for a specific SVN version? once again, relying in the last version is not a reliable solution.


\item Similar to 3) there is yet another problem: the platform code, as you can imagine, is split in one folder for every platform (see SVN): there is one for UNIX, one for Windows, for MacOS, and for iOS (but forget this one for the moment). Each platform has a ''leader'' or ''maintainer'', which is the person in charge of implementing/modifying the code. The problem raises when there are changes in VMMaker for example, that require changes in all platform code, and this is not changed in all of them. So for example, in UNIX the changes are committed, but not in Mac OS. Each platform code is not always in sync with the rest. 

Note that I am not complaining: this is all open-source and we all do our best. I am just telling you the problems I have seen so far.

\item The previous problem happens not only for the commits in the repository, but also for the VM releases. Most of the times, they are not in sync. Maybe there is a particular platform that releases 5 times in a year, and maybe there is another one every 1 year and a half 
The version of every VM are not in sync. So for Mac for example you have Squeak 4.2.5beta1U, Squeak 5.7.4.1, Squeak 5.8b4, etc. For UNIX, Squeak-4.4.7.2357, Squeak-vm-3.7-7, Squeak 4.0.3.2202, etc.  In Windows, SqueakVM-Win32-4.1.1, SqueakVM-Win32-3.11.5, SqueakVM-Win32-3.10.9, etc. So as you can see, they are all completely different, and for me this is complicated since you cannot just refer to a unique VM version.


\item The SVN repository is restricted, so you cannot commit unless you have authorized access. This could be a good and bad point at the same time.
\end{enumerate}

Scary isn't it? 


I want to make it clear: I am not complaining against this, I am just telling the problems I have found, and how certain infrastructure that has been done in the last months helped with some of these issues.

\paragraph{Metacello to the rescue.}
So now you know that VMMaker is just another Monticello package, and you also know that you have to manage versions, dependencies, why not groups and others. Does that ring a bell with anyone?  YEEES! Of course, Metacello    So, one thing we did in Pharo (although I guess it is/was also used in Squeak), is to create a Metacello Configuration for VMMaker: ConfigurationOfVMMaker. For those that doesn't know what Metacello is, it is a Package Management System on top of Monticello, where the ConfigurationOfVMMaker is a class where you can define versions, dependencies, etc, about your project. If you are a Smalltalker and you don't know anything about Metacello I recommend you to take a look and read the chapter available on \url{http://www.pharobyexample.org}

Anyhow, with ConfigurationOfVMMaker we solved the first two problems. With Metacello baselines we define all the structural information of the Interpreter VM: which packages are needed (the dependencies), possible groups (not everybody wants to load the same packages), repositories, etc. And with Metacello versions, we can define a whole set of working versions. For example, for ConfigurationOfVMMaker version 1.5 we load 'VMMaker-dtl.221', 'MemoryAccess-dtl.3', 'FFI-Pools-eem.2', etc. This is a set of frozen versions that we known to work properly all together. Notice that creating versions for ConfigurationOfVMMaker should be done by the ''VM developers''. 

In fact, it was done by people like Torsten,  Laurent and me. But the important thing is that the user doesn't need to do that. The only thing the user needs to do to load VMMaker with all its dependencies, and all loading a working version of every package, is to load the Metacello version. Do you want to try by yourself?  Just take this Pharo (\url{https://gforge.inria.fr/frs/download.php/28435/Pharo-1.2.1-11.04.03.zip}) image, and evaluate:

\begin{code}{}
1 Deprecation raiseWarning: false.
2 Gofer new
3    squeaksource:'MetacelloRepository';
4    package:'ConfigurationOfVMMaker';
5    load.
6 ((Smalltalk at: #ConfigurationOfVMMaker) project version: '1.5') load.
\end{code}

Why I told you to download that particular Pharo image? and why I am explicitly loading the version 1.5 instead of using the last one?  Because I want that my posts are reproducible. If you evaluate this instead:

\begin{code}{}
(Smalltalk at: #ConfigurationOfVMMaker) project lastVersion load.
\end{code}

I cannot guarantee that everything will be working properly. The same with the Pharo image. If you took any Pharo image 1.0, or 1.1 or 1.2, or Squeak 4.2, I am not sure that VMMaker will load correctly. The same if you load another version than 1.5. So...in this case, I am sure (because I test it before posting) that with that Pharo image and that version of ConfigurationOfVMMaker, VMMaker is working properly.

The last point 3) is not yet solved, since you cannot know that for a certain SVN version you need XXX version of ConfigurationOfVMMaker, or vice-versa. But we will come to this later on...The rest of the problems are not solved either.

\paragraph{Towards a bright future.}
Again note that a new infrastructure developed by the Pharo team and Igor Stasenko is in place to simplify all this. It is based on git and connected to hudson build servers. So a lot of the problems would be avoided if people would use the new infrastructure. 


\section{Generating the VM}

You need two things to compile the VM: the C platform code that is directly committed in SVN and the generated C code from the VMMaker. Do you always need to translate VMMaker to C? Not necessary, because the generated code is also committed in the SVN, usually under the ''/src'' folder, for example \ct{http://www.squeakvm.org/cgi-bin/viewcvs.cgi/trunk/platforms/unix/src/}. 
It is there so that if someone wants to compile, just download both parts and with GCC compiles the VM. No need to take a Smalltalk image, load VMMaker, and generate sources. So when is it really needed to generate sources from VMMaker?

\begin{enumerate}
\item When the /src in the SVN is outdated in relation to the platform code.
\item When you did changes in VMMaker. You can do changes in VMMaker just for fun, for your own project, for testing, etc.
\item For learning purpose 
\end{enumerate}


So how do you compile the VM?  yes, of course, using a C compiler but that is not enough information! For example, usually you need to place the /src folder (where the output of the generated VMMaker sources go) in a certain place so that it is found by the makefiles. Even more, the problem is that each platform has its own instructions of how to compile. You can find the instructions for UNIX here, for Windows here, and for Mac OS (after searching this info for a long time) it seems (if it is not this please let me know) to be here.

Not only each platform has its own instructions to build the VM, but also some lack support for IDE. For example, it is not easy to b able to compile the VM out of the box with Microsoft Visual Studio or with Appel's XCode. For example, for XCode, you need a \ct{.xcodeproj} file for every project. The problem was that most of the times (at least when I tried) this file contained file locations of the committer (which of course is different from mine). So, at the end, I usually need to do some modifications to the project to be able to compile and run the VM from XCode. I am telling you all this so that you can understand the progress we (the community) did in the last months. 


\section{Internal and external plugins}
Before going further, let me do a little remark: did you remember I told you about the VM plugins?  Like FilePlugin, SocketPlugin, etc. Well, plugins can be compiled in two ways: internal or external. Internal plugins are linked together with the core classical VM, i.e, they are inside the binary file of the VM. External plugins are distributed as separate shared library and the cool thing is that you don't need to do anything at all to the VM. At runtime the normal/standard VM can just load an external plugin and use it. Whether you should compile a plugin as internal or external is out of scope of this post. What is important here is that:

\begin{itemize}
\item the normal guy that just wants to compiled the VM shouldn't need to know how each plugin must be compiled.
\item there are some plugins that only work when they are compiled in one of the two ways.
Generating the VMMaker sources
\end{itemize}

Now imagine that for any reason, you need to translate VMMaker package to C. How do you do that? The default approach with the Interpreter VM is by using a tool called VMMakerTool, which at the same time it is the name of the class.

\ct{VMMakerTool} is a class which is in the VMMaker repository and it is a UI that let you generate the sources. Here you can see a screenshot shown in Figure~\ref{vmMaker}:

\begin{figure}[!h]
\includegraphics[width=\linewidth]{vmMaker}
\caption{VMMaker.\label{vmMaker}}
\end{figure}



To get the window shown in Figure~\ref{vmMaker}, just evaluate:
\begin{code}{}
VMMakerTool openInWorld
\end{code}

This tool lets you to do a lot of things: choose which plugins to include and choose whether you want them internal or external, you can set the source output directory, the platform code directory, the CPU architecture (32 or 64 bits), etc. VMMakerTool is just the UI and it relies in the ''model'', which is the VMMaker class (yes, VMMaker is the name of the squeaksource repository, the name of one of the packages and also one of the classes). With the class VMMaker we can do the same of VMMakerTool but from code. Example:

\begin{code}{}
| sourcePath |
"The path where I load from SVN"
sourcePath := '/Users/mariano/Pharo/VM/svnSqueakTree/trunk'.
"Generate new sources"
VMMaker default
	platformRootDirectoryName: sourcePath, '/platforms';
	sourceDirectoryName: sourcePath, '/platforms/iOS/vm/src';
	internal: #(
		ADPCMCodecPlugin	
		B3DAcceleratorPlugin
		B3DEnginePlugin
		BalloonEnginePlugin
 
	"lots of plugins more.....I let few just for the example"

		SurfacePlugin
		UUIDPlugin
		DropPlugin)
	external: #();
	generateMainVM.
\end{code}


\section{Cog VM}
The infrastructure for the Cog  VM is a little messy for me so I would try to do my best to explain it. Cog started as a fork of the Interpreter VM. So imagine that you want to create a fork for VMMaker (in SqueakSource) and another fork in the SVN for the platform code. Monticello doesn't provide real and easy branch support, so Cog needed to do something weird (at least for me). Suppose that a regular version of the VMMaker package is 'VMMaker-dtl.161?. In this case 'dtl' is the initials of the committer, Dave Lewis. How does the Cog branch in VMMaker looks like?  They are just normal versions, but whose committer is 'oscog' (I guess this is because of Open-Source Cog). Example: 'VMMaker-oscog.54?. That means that in order to load Cog, you need to open the VMMaker package, and search for a version that matches 'VMMaker-oscog'. There is where Eliot Miranda commits the official Cog versions.

Exercise: Take a Monticello Browser, add the VMMaker repository and browse the version 'VMMaker-dtl.223?. Then, browse 'VMMaker-oscog.54? and notice the difference between them. For example, in 'VMMaker-oscog.54? there are several categories that are not even present in 'VMMaker-dtl.223?, like 'VMMaker-JIT', 'VMMaker-Multithreading', etc. Even more, the same categories contain different classes.

Now, regarding the branch in the platform code, this is much easier since it is a regular SVN branch which can be found at \url{http://www.squeakvm.org/svn/squeak/branches/Cog/}.

Fortunately, people have also developed a ConfigurationOfCog which follows the same idea of ConfigurationOfVMMaker.

One difference I found with the regular VM is that Cog is supposed to be translated to C using VMMaker class directly (not using VMMakerTool). You can see how to do at \url{http://www.squeakvm.org/svn/squeak/branches/Cog/image/Workspace.text}.


\begin{code}{}
This image is intended to build new CoInterpreter or StackInterpreter Cog VMs, and a Newspeak VM.
The following doits create a single; source tree (../src) for all platforms.  Since they use a relative path they will work out of the box.  Generate the entire VM using them.

x86 platforms:
	(VMMaker
		generate: CoInterpreter
		and: (Smalltalk
				at: ([:choices| choices at: (UIManager default chooseFrom: choices)
				ifAbsent: [^self]]
					value: #(SimpleStackBasedCogit StackToRegisterMappingCogit)))
		to: (FileDirectory default / '../src') fullName
		platformDir: (FileDirectory default / '../platforms') fullName
		excluding:#(BrokenPlugin SlangTestPlugin TestOSAPlugin
					FFIPlugin ReentrantARMFFIPlugin ReentrantFFIPlugin ReentrantPPCBEFFIPlugin
					NewsqueakIA32ABIPlugin NewsqueakIA32ABIPluginAttic))
other platforms:
	 (VMMaker
		generate: StackInterpreter
		to: (FileDirectory default / '../stacksrc') fullName
		platformDir: (FileDirectory default / '../platforms') fullName
		excluding: (InterpreterPlugin withAllSubclasses collect: [:ea| ea name]))

Newspeak VM:
	(VMMaker
		generate: NewspeakInterpreter
		to: (FileDirectory default / '../nssrc') fullName
		platformDir: (FileDirectory default / '../platforms') fullName
		including:#(	AsynchFilePlugin FloatArrayPlugin RePlugin BalloonEnginePlugin FloatMathPlugin
					SecurityPlugin BMPReadWriterPlugin NewsqueakIA32ABIPlugin SocketPlugin
					SoundPlugin BitBltSimulation JPEGReadWriter2Plugin SurfacePlugin DSAPlugin
					JPEGReaderPlugin UUIDPlugin DropPlugin LargeIntegersPlugin UnixOSProcessPlugin
					FileCopyPlugin Matrix2x3Plugin Win32OSProcessPlugin FilePlugin
					MiscPrimitivePlugin InflatePlugin VMProfileMacSupportPlugin))
To generate a single plugin you can use a VMMaker.

To rebuild this image
- start with a Squeak 4.1 or newer image (this started life as Squeak 4.1).  You need support for pragmas method tags.
0. Make sure you have an up-to-date definition of recreateSpecialObjectsArray, e.g. from System-cmm.330.mcz
1. load SharedPool-Speech (e.g. SharedPool-Speech-dtl.2 from http://www.squeaksource.com/Speech)
2. load FFI-Pools (e.g. FFI-Pools-ar.1 from http://source.squeak.org/FFI)
3. load Balloon-Engine-Pools (e.g. Balloon-Engine-Pools-eem.1 from http://??/Cog)
4. load  Qwaq-VMProfiling-Plugins (e.g.  Qwaq-VMProfiling-Plugins-eem.3 from http://??/Cog)
5. load VMMaker (e.g. VMMaker-eem-oscog.2 from http://??/Cog)
6. load Alien & Cog from ?? (for the BochsIA32Plugin for Cogit development/generation)
- save and open a VMMaker as described above and save
\end{code}

So, in summary, they way to compile Cog VM is more or less the same as the Interpreter VM: you take a Smalltalk image, you load Cog (you can use ConfigurationOfCog), you generate sources, you checkout SVN branch, and finally compile (the instructions of how to build each VM is in the same SVN). Generating the sources may not be necessary if the /src is in sync with /platforms.

Finally, notice also that Eliot usually uploads regular VM builds (Cog VM binaries for all OS) to this url.

%=========================================================
\ifx\wholebook\relax\else
   \bibliographystyle{jurabib}
   \nobibliography{scg}
   \end{document}
\fi
%=========================================================








New infrastructure
The same way we should thanks Teleplace for Eliot Miranda's work, we should also thanks INRIA for paying a Pharo engineer: Igor Stasenko. The good news is that since he started a couple of months ago he was not working for Pharo but instead for a new VM infrastructure . What is all this about? I'll give you only a quick introduction because in the next post we are going to compile the VM using such infrastructure. Disclaimer: this new infrastructure is only for Cog VM and all its variants, but not Interpreter VM. I guess that's because of the resources/time available.

So...in a nutshell, there are 3 big changes:

Use GIT instead of SVN. There is a new repository for the platform code which is versioned by GIT instead of SVN. There is a new account for CogVM in gitorious. It seems that nowadays if you are not in Git you are not cool, you do not exist. Ok, we are cool now    No one needs to ask for a blessing, everybody can clone, hack and push/share changes. People can pick the changes without having to have the permissions to publish.
Use CMakeVMMaker instead of VMMakerTool. CMakeVMMaker is a little tool that automates the build. It has two important things: 1) translate VMMaker to C, using the VMMaker class and 2) generate CMake files so that to ease the build. To do this  it automatically assumes (although it can be customized) which plugins are needed and how they are compiled (if internal or external), the needed compiler flags, the directories needed, etc. CMake is an excellent tool for doing cross-platform compiling and for automatic stuff....CMakeVMMaker generates all the necessary files for CMake. For those who doesn't know what CMake is, imagine one abstract step before makefiles. CMake is a cross-platform, open-source build system where you can define all necessary stuff like directories, compiler flags, etc, in text files. Once you have that, using CMake you can generate different outputs: normal makefiles where you can just use the command ''make'', Appel's XCode project or even Microsoft Visual Studio projects 
Continuous Integration for VMs!!  Can you imagine that for every GIT commit, Mr. Hudson takes the latest PharoCore image, loads the Cog VM, generates sources, and compiles the VM for Windows, Linux and Mac OS ? Come on! isn't this really cool?  Ok, you don't believe me? Go to the Pharo CI for CogVM.
In the next post we will see how to use this new infrastructure and how is solves some of the mentioned problems along this post. I want to thanks Esteban, Igor, Dave and all who answer my questions in the mailing lists 

Links:
\begin{enumerate}
\item CogVM in gitorious
\item Cog issue tracker
\item Cog binaries
\item Official Squeak VM SVN
\item Cog SVN branch
\item Squeak VM issue tracker
\item VM mailing list
\item VM-beginners mailing list
\item VMMaker page in squeak wiki
\item Screencast for compiling Interpreter VM in Pharo using SVN and VMMakerTool
\item Blog post for building a Pharo/Squeak VM from first Principles (using SVN and VMMakerTool)
\item Pharo collaborative book, chapter in Virtual Machine
\item Pharo Hudson server
\item Official UNIX instructions to compile the VM
\item Official Mac OS instructions to compiled the VM
\item Official Win32 instructions to compile the VM
\end{enumerate}


%=========================================================
\ifx\wholebook\relax\else
   \bibliographystyle{jurabib}
   \nobibliography{scg}
   \end{document}
\fi
%=========================================================



%%% Local Variables: 
%%% coding: utf-8
%%% mode: latex
%%% TeX-master: Lint.tex
%%% TeX-PDF-mode: t
%%% End:

