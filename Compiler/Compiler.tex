% $Author$
% $Date$

% HISTORY:
% 2007-07-18 - Matthieu started chapter (one paragraph only)

%=================================================================
\ifx\wholebook\relax\else
% --------------------------------------------
% Lulu:
	\documentclass[a4paper,10pt,twoside]{book}
	\usepackage[
		papersize={6.13in,9.21in},
		hmargin={.75in,.75in},
		vmargin={.75in,1in},
		ignoreheadfoot
	]{geometry}
	\input{../common.tex}
	\pagestyle{headings}
	\setboolean{lulu}{true}
% --------------------------------------------
% A4:
%	\documentclass[a4paper,11pt,twoside]{book}
%	\input{../common.tex}
%	\usepackage{a4wide}
% --------------------------------------------
    \graphicspath{{figures/} {../figures/}}
	\begin{document}
	\renewcommand{\nnbb}[2]{} % Disable editorial comments
	\sloppy
\fi
%=================================================================
\chapter{The old and the new compiler}
\chalabel{compiler}

% ONE PARAGRAPH BY MSUEN:
On part of the metalevel of squeal is the compiler. 
You probably notice that each time you add a method you compiled it and it is add to the class.
It make the way of programming in Smalltalk different from most of the language.
In Java you compiled the all class and the virtual machine take care of adding you class.
In Pharo your class is already inside the image and behavior are added bit by bit inside the image.
Each time you add a method you request the compiler.
The compiler is responsible of transforming Smalltalk into object oriented bytecode. 
So the bytecode can be understand by the virtual machine


%=============================================================
\ifx\wholebook\relax\else
   \bibliographystyle{jurabib}
   \nobibliography{scg}
   \end{document}
\fi
%=============================================================

%=================================================================
%%% Local Variables:
%%% coding: utf-8
%%% mode: latex
%%% TeX-master: t
%%% TeX-PDF-mode: t
%%% ispell-local-dictionary: "english"
%%% End: