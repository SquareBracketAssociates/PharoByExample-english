% $Author$
% $Date$
% $Revision$

% HISTORY:
% 2008-08-19 - Stef started chapter (outline only)

%=================================================================
\ifx\wholebook\relax\else
% --------------------------------------------
% Lulu:
	\documentclass[a4paper,10pt,twoside]{book}
	\usepackage[
		papersize={6.13in,9.21in},
		hmargin={.75in,.75in},
		vmargin={.75in,1in},
		ignoreheadfoot
	]{geometry}
	\input{../common.tex}
	\setboolean{lulu}{true}
% --------------------------------------------
% A4:
%	\documentclass[a4paper,11pt,twoside]{book}
%	\input{../common.tex}
%	\usepackage{a4wide}
% --------------------------------------------
    \graphicspath{{figures/} {../figures/}}
	\begin{document}
\fi
%=================================================================
%\renewcommand{\nnbb}[2]{} % Disable editorial comments
\sloppy
%=================================================================
\chapter{Debugging }\label{cha:basic}


\section{Halt}

\section{Conditional Halt}

\ct{InputState shiftPressed ifTrue: [ self halt ]}

\ct{self haltIf: [ ... particular test run ... ]}

\section{Halt Once}

On certain occasions, it is impractical to put an halt into a method.
For example, putting an halt statement into the \ct{MethodReference>>actualClass} method can 
simply kill the system since the \ct{actualClass} method is invoked by the browser when editing code.
Now it may be cumbersome to express a condition under which the halt should be raised.
Pharo offers a one shot halt. 



\begin{code}
	anObject setHaltOnce
	
	haltOnce
	inspectOnce
	
	anObject clearHaltOnce
	anObject toggleHaltOnce
	
\end{code}

\paragraph
\begin{code}
setHaltCountTo: int
halt: aString onCount: int 

inspectOnCount: int 
inspectUntilCount: int 
\end{code}

\section{Pointer Finder}

\section{Inspector Explorer}

% \section{Message Tallly} -- in Profiling chapter

\section{BreakPoints}

\section{Chasing Undeclared References}
Smalltalk cleanOutUndeclared. 

\section{VM logging}





%=================================================================
\ifx\wholebook\relax\else\end{document}\fi
%=================================================================

%-----------------------------------------------------------------

%%% Local Variables:
%%% coding: utf-8
%%% mode: latex
%%% TeX-master: t
%%% TeX-PDF-mode: t
%%% ispell-local-dictionary: "english"
%%% End: