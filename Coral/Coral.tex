% $Author: ducasse $
% $Date: 2009-08-24 10:17:33 +0200 (Mon, 24 Aug 2009) $
% $Revision: 28563 $

%% 2011-09-11 - Migrated to PharoBox: svn checkout https://XXX@scm.gforge.inria.fr/svn/pharobooks/PharoByExampleTwo-Eng

%=================================================================
\ifx\wholebook\relax\else
% --------------------------------------------
% Lulu:
	\documentclass[a4paper,10pt,twoside]{book}
	\usepackage[
		papersize={6.13in,9.21in},
		hmargin={.75in,.75in},
		vmargin={.75in,1in},
		ignoreheadfoot
	]{geometry}
	\input{../common.tex}
	\setboolean{lulu}{true}
% --------------------------------------------
% A4:
%	\documentclass[a4paper,11pt,twoside]{book}
%	\input{../common.tex}
%	\usepackage{a4wide}
% --------------------------------------------
    \graphicspath{{figures/} {../figures/}}
	\begin{document}
\fi
%=================================================================
%\renewmessage{\nnbb}[2]{} % Disable editorial comments
\sloppy

%=================================================================
%\renewcommand{\nnbb}[2]{#2} % Disable editorial comments

\chapter{Coral: Scripting in Pharo}

Have you ever dreamed to write your shell scripts in Smalltalk? 
Coral is then for your. 


\section{Getting Coral}

You can either use a prebuild image that you can find on the Pharo build server @@ here @@ or build
the Coral environment.

\begin{itemize}

\item Take a Pharo 1.3 image (here, for example:
\url{https://pharo-ic.lille.inria.fr/hudson/view/Pharo/job/Pharo Core 1.3/} \sd{probably wrong now.}

\item Open the image and load the package �ConfigurationOfCoral� from �Coral� SqueakSource repository. The fastest way is to evaluate:


\begin{code}{}
Gofer new 
	squeaksource: 'Coral'; 
	package: 'ConfigurationOfCoral'; load.
\end{code}

\item Then, evaluate (this line is written on the comment of the class \ct{ConfigurationOfCoral})

\begin{code}{}
(ConfigurationOfCoral project version: '0.4') load
\end{code}

\end{itemize}
At this point, Coral is installed from the Pharo perspective. But better install the helper files as explained in the next section.


\section{Helper Files}

Now to help you launching it. Coral can generate for you some scripts:

\begin{code}{}
Coralnstaller generateCoralScript.
Coralnstaller generateCoralDebugScript
\end{code}

This generates two shells that you can use as follows:

\begin{code}{}
./script.sh CoralScript.cst [arguments to Coral script]
\end{code}

\begin{itemize}
\item \ct{coral.sh} runs pharo in headless mode and quits when the Coral script is ended.
\item \ct{coralDebug.sh} runs pharo with UI and does not quit even if the Coral script is ended
\end{itemize}


Do not forget to give them the execute right:
\begin{code}{\sd{we should do that from coral!}}
chmod +x coral.sh coralDebug.sh
\end{code}

Note that these scripts remember the locations and names of your virtual machine and of your Coral image. You�ll have to generate them again if any of these information changes.

 
The last thing you can do is to generate all examples given with Coral:

\begin{code}{}
CoralInstaller generateAllExamples
\end{code}

 This message  takes all selectors in CoralScriptExamples class in the �script examples� category, and generate a file for each selector, which content will be the return value (a string containing a Coral script) of the evaluation of the selector.
 
 
At that point, you can execute any example. For example:
 
\begin{code}{}
./coral.sh scriptHappyFace.cst
\end{code}
 
\section{How to write Coral scripts?}
 
Coral syntax is almost the same as the Pharo syntax. 
 There are just two things added:
\begin{itemize}
\item  \ct{[ ]}: anything you put into square brackets will be executed.

\item Method bodies are surrounded by \ct{[ ]}.

For example here is the way to define a method on point which takes one argument
\begin{code}{}
Point>>myAreaScaled: scale
	[
	^ self x * self y * scale
	]
\end{code}
The only things that changes are \ct{Point>>} and \ct{[} after the parameter list and the closing one \ct{]}. 


For adding a method on class side, use: \ct{aClass class>>aSelectorName}
  \end{itemize}
  
  
  
\paragraph{Examples}
  
\begin{code}{\sd{we should use real examples}}
[ 
	Transcript show: 'This line is in two (unnamed) square brackets, so it is executed'. 
]

Object>>uselessMethod 
	[ 
	"This method does nothing but is added to the instance side of Object" 
	]
	
Object class>>anotherUselessMethod: aString 
	[
	"This method does nothing too, but it is added on class side. And it takes a paremeter". 
	]
\end{code}
  

\subsection{Class Definition} 
  
Since the definition of a subclass is quite long, there is a small syntax added

\begin{code}{}
Object < newSubClass
\end{code}

A hint to remember the order: if you put a pipe it looks like the UML triangle of generalization: \ct{<|}.


Take a look to the CoralScriptWriting class. It provides methods to access to the arguments given to the script, by such a command:

\begin{code}{}
./coral.sh myScript.cst foo -option 42
\end{code}


\section{Dealing with Files}

\section{Deamons}

\section{}


%=========================================================
\ifx\wholebook\relax\else
   \bibliographystyle{jurabib}
   \nobibliography{scg}
   \end{document}
\fi
%=========================================================




%%% Local Variables: 
%%% coding: utf-8
%%% mode: latex
%%% TeX-master: Lint.tex
%%% TeX-PDF-mode: t
%%% End:

