% $Author$
% $Date$
% $Revision$
%=================================================================
\ifx\wholebook\relax\else
% --------------------------------------------
% Lulu:
	\documentclass[a4paper,10pt,twoside]{book}
	\usepackage[
		papersize={6in,9in},
		hmargin={.75in,.75in},
		vmargin={.75in,1in},
		ignoreheadfoot
	]{geometry}
	\input{../common.tex}
	\pagestyle{headings}
	\setboolean{lulu}{true}
% --------------------------------------------
% A4:
%	\documentclass[a4paper,11pt,twoside]{book}
%	\input{../common.tex}
%	\usepackage{a4wide}
% --------------------------------------------
    \graphicspath{{figures/} {../figures/}}
	\begin{document}
	\renewcommand{\nnbb}[2]{} % Disable editorial comments
	\sloppy
\fi
%=================================================================
\chapter{Classes and metaclasses}
\label{cha:metaclasses}

\on{The section on responsibilities of Class, Behavior and Metaclass are weak, and need to be fleshed out with convincing examples. Marcus, can you help?}

As we saw in \charef{model}, in \st, everything is an object, and every object is an instance of a class. Classes are no exception:
classes are objects, and class objects are instances of other classes.
This object model captures the essence of object-oriented programming: it is lean, simple, elegant and uniform. 
However, the implications of this uniformity may confuse newcomers. The goal of this chapter is to show that there is nothing complex, ``magic'' or special here: just simple rules applied uniformly.
By following these rules you can always understand why the situation is the way that it is.

%=================================================================
\section{Rules for classes and metaclasses}

The Smalltalk object model is based on a limited number of concepts applied uniformly. 
Smalltalk's designers applied Occam's razor: any consideration leading to a model more complex than necessary was discarded.

To refresh your memory, here are the rules of the object model that we explored in \charef{model}.

\begin{enumerate}[label={\textbf{Rule \arabic{*}}.}, ref={Rule \arabic{*}}, leftmargin=*, widest=10]
% NB: rule labels must not be multiply defined across chapters!
\item{} % \label{rule:everything}
	Everything is an object.

\item{} % \label{rule:instance}
	Every object is an instance of a class.

\item{} % \label{rule:inheritance}
	Every class has a superclass.

\item{} % \label{rule:message}
	Everything happens by message sends.

\item{} % \label{rule:lookup}
	Method lookup follows the inheritance chain.

\end{enumerate}

As we mentioned in the introduction to this chapter, a consequence of \ref{rule:everything} is that \emph{classes are objects too}, so \ref{rule:instance} tells us that classes must also be instances of classes.
The class of a class is called a \emph{metaclass}.
\label{sec:metaclassIntro}
A \indmain{metaclass} is created automatically for you whenever you create a class.
Most of the time you do not need to care or think about metaclasses.
However, every time that you use the system browser to browse the ``\subind{system browser}{class side}'' of a class, it is helpful to recall that you are actually browsing a different class.
A class and its metaclass are two separate classes, even though the former is an instance of the latter. 

To properly explain classes and metaclasses, we need to extend the rules from \charef{model} with the following additional rules.

\begin{enumerate}[label={\textbf{Rule \arabic{*}}.}, ref={Rule \arabic{*}}, leftmargin=*, widest=10]
\setcounter{enumi}{5}
\item{} \label{rule:metaclass}
	Every class is an instance of a metaclass.

\item{} \label{rule:parallelhierarchy}
	The metaclass \subind{metaclass}{hierarchy} parallels the class hierarchy.

\item{} \label{rule:behavior}
	Every metaclass inherits from \clsind{Class} and \clsind{Behavior}.

\item{} \label{rule:metaclassclass}
	Every metaclass is an instance of \clsind{Metaclass}.

\item{} \label{rule:metaclassmetaclass}
	The metaclass of \ct{Metaclass} is an instance of \ct{Metaclass}.

\end{enumerate}

\noindent
Together, these 10 rules complete \st's object model.

We will first briefly revisit the 5 rules from \charef{model} with a small example.
Then we will take a closer look at the new rules, using the same example.

%=================================================================
\section{Revisiting the Smalltalk object model}

Since everything is an object, the color blue in \st is also an object.
\begin{code}{@TEST}
Color blue --> Color blue
\end{code}

\noindent
Every object is an instance of a class. The class of the color blue is the class \clsind{Color}:
\begin{code}{@TEST}
Color blue class --> Color
\end{code}

\noindent
Interestingly, if we set the \emph{alpha} value of a color, we get an instance of a different class, namely \clsind{TranslucentColor}:
\begin{code}{@TEST}
(Color blue alpha: 0.4) class --> TranslucentColor
\end{code}

\noindent
We can create a morph and set its color to this translucent color:
\begin{code}{}
EllipseMorph new color: (Color blue alpha: 0.4); openInWorld
\end{code}
\noindent
You can see the effect in \figref{translucentEllipse}.

\begin{center}
\begin{figure}[!ht]
\ifluluelse
	{\centerline {\includegraphics[scale=0.7]{TranslucentEllipse}}}
	{\centerline {\includegraphics[width=8cm]{TranslucentEllipse}}}
\caption{A translucent ellipse\label{fig:translucentEllipse}}
\end{figure}
\end{center}

By \ref{rule:inheritance}, every class has a superclass.
The superclass of \clsind{TranslucentColor} is \clsind{Color}, and the superclass of \ct{Color} is \clsind{Object}:
\begin{code}{@TEST}
TranslucentColor superclass --> Color
Color superclass                   --> Object
\end{code}

Everything happens by \ind{message send}{}s (\ref{rule:message}), so we can deduce that \mthind{Color class}{blue} is a message to \ct{Color}, \mthind{Object}{class} and \mthind{Color}{alpha:} are messages to the color blue, \mthind{Morph}{openInWorld} is a message to an ellipse morph, and \mthind{Behavior}{superclass} is a message to \ct{TranslucentColor} and \ct{Color}.
The receiver in each case is an object, since everything is an object, but some of these objects are also classes.

Method lookup follows the inheritance chain (\ref{rule:lookup}), so when we send the message \ct{class} to the result of \ct{Color blue alpha: 0.4}, the message is handled when the corresponding method is found in the class \ct{Object}, as shown in \figref{classmessage}.

\begin{center}
\begin{figure}[!ht]
\ifluluelse
	{\centerline{\includegraphics[width=\textwidth]{TranslucentClassMessage}}}
	{\centerline{\includegraphics[width=0.8\textwidth]{TranslucentClassMessage}}}
\caption{Sending a message to a translucent color\label{fig:classmessage}}
\end{figure}
\end{center}

The figure captures the essence of the \emphind{is-a} relationship.
Our translucent blue object \emph{is a} \ct{TranslucentColor} instance, but we can also say that it \emph{is a} \ct{Color} and that it \emph{is an} \ct{Object}, since it responds to the messages defined in all of these classes.
In fact, there is a message, \mthind{Object}{isKindOf:}, that you can send to any object to find out if it is in an \emph{is a} relationship with a given class:
\begin{code}{@TEST | translucentBlue |}
translucentBlue := Color blue alpha: 0.4.
translucentBlue isKindOf: TranslucentColor --> true
translucentBlue isKindOf: Color                    --> true
translucentBlue isKindOf: Object                  --> true
\end{code}

%=================================================================
\section{Every class is an instance of a metaclass}

% \ruleref{metaclass}

As we mentioned in \secref{metaclassIntro}, classes whose instances are themselves classes are called \ind{metaclass}{}es.

\paragraph{Metaclasses are implicit.}
Metaclasses are automatically created when you define a class. We say that they are \emph{implicit} since as a programmer you never have to worry about them. An \subind{metaclass}{implicit} metaclass is created for each class you create, so each metaclass has only a single instance.
% At a more advanced level, this implies that sharing between metaclasses is difficult except by subclassing. 

Whereas ordinary classes are named by global variables, metaclasses are anonymous.
However, we can always refer to them through the class that is their instance.
The class of \clsind{Color}, for instance, is \clsind{Color class}, and the class of \ct{Object} is \clsind{Object class}:
\begin{code}{@TEST}
Color class   --> Color class
Object class --> Object class
\end{code}

\noindent
\figref{translucentmetaclasses} shows how each class is an instance of its (\subind{metaclass}{anonymous}) metaclass.

\begin{center}
\begin{figure}[!ht]
\ifluluelse
	{\centerline {\includegraphics[width=\textwidth]{TranslucentMetaclasses}}}
	{\centerline {\includegraphics[width=0.8\textwidth]{TranslucentMetaclasses}}}
\caption{The metaclasses of Translucent and its superclasses\label{fig:translucentmetaclasses}}
\end{figure}
\end{center}

The fact that classes are also objects makes it easy for us to query them by sending messages. Let's have a look:

\begin{code}{@TEST}
Color subclasses                           --> {TranslucentColor}
TranslucentColor subclasses         --> #()
TranslucentColor allSuperclasses  --> an OrderedCollection(Color Object ProtoObject)
TranslucentColor instVarNames     --> #('alpha')
TranslucentColor allInstVarNames --> #('rgb' 'cachedDepth' 'cachedBitPattern' 'alpha')
TranslucentColor selectors             --> an IdentitySet(#alpha: #asNontranslucentColor #privateAlpha #pixelValueForDepth: #isOpaque #isTranslucentColor #storeOn: #pixelWordForDepth: #scaledPixelValue32 #alpha #bitPatternForDepth: #hash #convertToCurrentVersion:refStream: #isTransparent #isTranslucent #setRgb:alpha: #balancedPatternForDepth: #storeArrayValuesOn:)
\end{code}
\cmindex{Class}{subclasses}
\cmindex{Behavior}{allSuperclasses}
\cmindex{Behavior}{instVarNames}
\cmindex{Behavior}{allInstVarNames}
\cmindex{Behavior}{selectors}

%=================================================================
\section{The metaclass hierarchy parallels the class hierarchy}
% \ruleref{parallelhierarchy}

\ref{rule:parallelhierarchy} says that the superclass of a metaclass cannot be an arbitrary class: it is constrained to be the metaclass of the superclass of the metaclass's unique instance.

\begin{code}{@TEST}
TranslucentColor class superclass --> Color class
TranslucentColor superclass class --> Color class
\end{code}

\noindent
This is what we mean by the metaclass \subindmain{metaclass}{hierarchy} being parallel to the class hierarchy;  \figref{parallelHierarchies} shows how this works in the \clsind{TranslucentColor} hierarchy.

\begin{center}
\begin{figure}[!ht]
\ifluluelse
	{\centerline {\includegraphics[width=\textwidth]{TranslucentMetaclassHierarchy}}}
	{\centerline {\includegraphics[width=0.8\textwidth]{TranslucentMetaclassHierarchy}}}
\caption{The metaclass hierarchy parallels the class hierarchy.\label{fig:parallelHierarchies}}
\end{figure}
\end{center}

\begin{code}{@TEST}
TranslucentColor class                                     --> TranslucentColor class
TranslucentColor class superclass                   --> Color class
TranslucentColor class superclass superclass --> Object class
\end{code}

\paragraph{Uniformity between Classes and Objects.}
It is interesting to step back a moment and realize that there is no difference between sending a message to an object and to a class. 
In both cases the search for the corresponding method starts in the class of the receiver, and proceeds up the inheritance chain.

Thus, messages sent to classes must follow the metaclass inheritance chain.
Consider, for example, the method \mthind{Color class}{blue}, which is implemented on the \subind{system browser}{class side} of \ct{Color}.
If we send the message \ct{blue} to \ct{TranslucentColor}, then it will be looked-up the same way as any other message.
The lookup starts in \ct{TranslucentColor class}, and proceeds up the metaclass hierarchy until it is found in \ct{Color class} (see \figref{metaclasslookup}).

\begin{code}{@TEST}
TranslucentColor blue --> Color blue
\end{code}
\noindent
Note that we get as a result an ordinary \ct{Color blue}, and not a translucent one\,---\,there is no magic!

\begin{center}
\begin{figure}[!ht]
\ifluluelse
	{\centerline {\includegraphics[width=\textwidth]{TranslucentColorBlue}}}
	{\centerline {\includegraphics[width=0.8\textwidth]{TranslucentColorBlue}}}
\caption{Message lookup for classes is the same as for ordinary objects.\label{fig:metaclasslookup}}
\end{figure}
\end{center}

Thus we see that there is one uniform kind of method \subind{method}{lookup} in Smalltalk. Classes are just objects, and behave like any other objects. 
Classes have the power to create new instances only because classes happen to respond to the message \ct{new}, and because the method for \ct{new} knows how to create new instances.
Normally, non-class objects do not understand this message, but if you have a good reason to do so, there is nothing stopping you from adding a \ct{new} method to a non-metaclass.

Since classes are objects, we can also inspect them.
\index{inspector}

\dothis{Inspect \ct{Color blue} and \ct{Color}.}

\noindent
Notice that in one case you are inspecting an instance of \ct{Color} and in the other case the \ct{Color} class itself.  
This can be a bit confusing, because the title bar of the inspector names the \emph{class} of the object being inspected.

The inspector on \ct{Color} allows you to see the superclass, instance variables, method \subind{method}{dictionary}, and so on, of the \ct{Color} class, as shown in \figref{inspectingColor}.

\begin{center}
\begin{figure}[!ht]
\ifluluelse
	{\centerline{\includegraphics[width=\textwidth]{InspectingColor}}}
	{\centerline{\includegraphics[width=10cm]{InspectingColor}}}
\caption{Classes are objects too.\label{fig:inspectingColor}}
\end{figure}
\end{center}

%=================================================================
\section{Every metaclass Inherits from \lct{Class} and \lct{Behavior}}
% \ruleref{behavior}

Every \ind{metaclass} \emphind{is-a} class, hence inherits from \clsind{Class}.
\ct{Class} in turn inherits from its superclasses, \clsind{ClassDescription} and \clsind{Behavior}.
Since everything in \st \emph{is-an} object, these classes all inherit eventually from \ct{Object}.
We can see the complete picture in \figref{inheritbehavior}.

\begin{center}
\begin{figure}
\ifluluelse
	{\centerline{\includegraphics[width=\textwidth]{TranslucentBehavior}}}
	{\centerline{\includegraphics[width=0.8\textwidth]{TranslucentBehavior}}}
\caption{Metaclasses inherit from \lct{Class} and \lct{Behavior}\label{fig:inheritbehavior}}
\end{figure}
\end{center}

\paragraph{Where is \lct{new} defined?}
To understand the importance of the fact that metaclasses inherit from \ct{Class} and \ct{Behavior}, it helps to ask where \ct{new} is defined and how it is found. When the message \ct{new} is sent to a class it is looked up in its metaclass chain and ultimately in its superclasses \ct{Class}, \ct{ClassDescription} and \ct{Behavior} as shown in \figref{sendingnew}.

The question \emph{``Where \ct{new} is defined?''} is crucial. \mthind{Behavior}{new} is first defined in the class \ct{Behavior}, and it can be redefined in its subclasses, including any of the metaclass of the classes we define, when this is necessary. Now when a message \ct{new} is sent to a class it is looked up, as usual, in the metaclass of this class, continuing up the superclass chain right up to the class \ct{Behavior}, if it has not been redefined along the way.

Note that the result of sending \ct{TranslucentColor new} is an instance of \clsind{TranslucentColor} and \emph{not} of \ct{Behavior}, even though the method is looked-up in the class \ct{Behavior}!  \ct{new} always returns an instance of \self, the class that receives the message, even if it is implemented in another class.

\begin{code}{@TEST}
TranslucentColor new class --> TranslucentColor    "not Behavior!"
\end{code}

\begin{center}
\begin{figure}
\ifluluelse
	{\centerline{\includegraphics[width=\textwidth]{TranslucentSendingNew}}}
	{\centerline{\includegraphics[width=0.8\textwidth]{TranslucentSendingNew}}}
\caption{\lct{new} is an ordinary message looked up in the metaclass chain.\label{fig:sendingnew}}
\end{figure}
\end{center}

A common mistake is to look for \ct{new} in the superclass of the receiving class. The same holds for \ct{new:}, the standard message to create an object of a given size.
For example, \lct{Array new: 4} creates an array of 4 elements.
You will not find this method defined in \ct{Array} or any of its superclasses.
Instead you should look in \ct{Array class} and its superclasses, since that is where the lookup will start.

\on{The text below needs more details and convincing examples ...}

\paragraph{Responsibilities of \lct{Behavior}, \lct{ClassDescription} and \lct{Class}.}
\clsind{Behavior} provides the minimum state necessary for objects that have instances: this includes a superclass link, a method dictionary, and a description of the instances (\ie representation and number).
\on{not sure I understand the last point}
\ct{Behavior} inherits from \ct{Object}, so it, and all of its subclasses, can behave like objects. 

\ct{Behavior} is also the basic interface to the compiler.
It provides methods for creating a method dictionary,
compiling methods,
creating instances (\ie \mthind{Behavior}{new}, \mthind{Behavior}{basicNew}, \mthind{Behavior}{new:}, and \mthind{Behavior}{basicNew:}),
manipulating the class hierarchy (\ie \mthind{Behavior}{superclass:}, \mthind{Behavior}{addSubclass:}),
accessing methods (\ie \mthind{Behavior}{selectors}, \lmthind{Behavior}{allSelectors}, \mthind{Behavior}{compiledMethodAt:}),
accessing instances and variables (\ie \mthind{Behavior}{allInstances}, \mthind{Behavior}{instVarNames}\ldots),
accessing the class hierarchy (\ie \mthind{Behavior}{superclass}, \mthind{Behavior}{subclasses})
and querying (\ie \mthind{Behavior}{hasMethods}, \mthind{Behavior}{includesSelector}, \mthind{Behavior}{canUnderstand:}, \mthind{Behavior}{inheritsFrom:}, \mthind{Behavior}{isVariable}).


\clsind{ClassDescription} is an abstract class that provides facilities needed by its two direct subclasses, \clsind{Class} and \clsind{Metaclass}.
\clsind{ClassDescription} adds a number of facilities to the basis provided by \ct{Behavior}:
named instance variables,
the categorization of methods into protocols,
the notion of a name (abstract),
the maintenance of change sets and the logging of changes, and
most of the mechanisms needed for filing-out changes.

\clsind{Class} represents the common behaviour of all classes.
It provides a class name, compilation methods, method storage, and instance variables.
It provides a concrete representation for class variable names and shared pool variables (\mthind{Class}{addClassVarName:}, \mthind{Class}{addSharedPool:}, \mthind{Class}{initialize}).
\ct{Class} knows how to create instances, so all metaclasses should inherit ultimately from \ct{Class}.

%=================================================================
\section{Every metaclass is an instance of \lct{Metaclass}}
% \ruleref{metaclassclass}

Metaclasses are objects too; they are instances of the class \clsind{Metaclass} as shown in \figref{metaclassclass}. The instances of class \ct{Metaclass} are the anonymous metaclasses, each of which 
has exactly one instance, which is a class.

\begin{center}
\begin{figure}
\ifluluelse
	{\centerline{\includegraphics[width=\textwidth]{TranslucentMetaclassClass}}}
	{\centerline{\includegraphics[width=0.8\textwidth]{TranslucentMetaclassClass}}}
\caption{Every metaclass is a \lct{Metaclass}.\label{fig:metaclassclass}}
\end{figure}
\end{center}

\ct{Metaclass} represents common metaclass behaviour.
It provides methods for instance creation (\lct{sub\-class\-Of:})
creating initialized instances of the metaclass's sole instance,
initialization of class variables,
metaclass instance,
% (\ct{name:inEnvironment:subclassOf:})
% Actually, this is in ClassBuilder, it seems!
method compilation, % (different semantics can be introduced)
and class information (inheritance links, instance variables, \etc).
\ab{This is too cryptic for me.}

%=================================================================
\section{The metaclass of \lct{Metaclass} is an Instance of \lct{Metaclass}}
% \ruleref{metaclassmetaclass}

The final question to be answered is: what is the class of \clsind{Metaclass class}?

The answer is simple: it is a metaclass, so it must be an instance of \ct{Metaclass}, just like all the other metaclasses in the system (see \figref{metaclassclassclass}).

\begin{center}
\begin{figure}
\ifluluelse
	{\centerline{\includegraphics[width=\textwidth]{TranslucentMetaclassClassClass}}}
	{\centerline{\includegraphics[width=0.8\textwidth]{TranslucentMetaclassClassClass}}}
\caption{All metaclasses are instances of the class \lct{Metaclass}, even the metaclass of \lct{Metaclass}.\label{fig:metaclassclassclass}}
\end{figure}
\end{center}

The figure shows how all metaclasses are instances of \ct{Metaclass}, including the metaclass of \ct{Metaclass} itself.
If you compare Figures \ref{fig:metaclassclass} and \ref{fig:metaclassclassclass} you will see how the metaclass \subind{metaclass}{hierarchy} perfectly mirrors the class hierarchy, all the way up to \ct{Object class}.

The following examples show us how we can query the class hierarchy to demonstrate that \figref{metaclassclassclass} is correct.
(Actually, you will see that we told a white lie\,---\,\ct{Object class superclass -->} {\clsind{ProtoObject class}, not \ct{Class}. In \sq, we must go one superclass higher to reach \ct{Class}.)

\begin{example}{The class hierarchy}{@TEST}
TranslucentColor superclass --> Color
Color superclass                   --> Object
\end{example}

\needlines{4}
\begin{example}{The parallel metaclass hierarchy}{@TEST}
TranslucentColor class superclass   --> Color class
Color class superclass                     --> Object class
Object class superclass superclass --> Class    "NB: skip ProtoObject class"
Class superclass                              --> ClassDescription
ClassDescription superclass            --> Behavior
Behavior superclass                         --> Object
\end{example}

\begin{example}{Instances of Metaclass}{@TEST}
TranslucentColor class class --> Metaclass
Color class class                   --> Metaclass
Object class class                 --> Metaclass
Behavior class class              --> Metaclass
\end{example}
\begin{example}{Metaclass class is a Metaclass}{@TEST}
Metaclass class class --> Metaclass
Metaclass superclass --> ClassDescription
\end{example}

%=================================================================
\section{Chapter summary}
Now you should understand better how classes are organized and the impact of a uniform object model. If you get lost or confused, you should always remember that message passing is the key: you look for the method in the class of the receiver. 
This works on \emph{any} receiver. 
If the method is not found in the class of the receiver, it is looked up in its superclasses.

\begin{itemize}
\item Every class is an instance of a metaclass.
	Metaclasses are implicit. A Metaclass is created automatically when you create the class that is its sole instance.

\item The metaclass hierarchy parallels the class hierarchy.
	Method lookup for classes parallels method lookup for ordinary objects, and follows the metaclass's superclass chain.

\item Every metaclass inherits from \ct{Class} and \ct{Behavior}.
	Every class \emph{is a} \ct{Class}. Since metaclasses are classes too, they must also inherit from \ct{Class}.
	\ct{Behavior} provides behaviour common to all entities that have instances.

\item Every metaclass is an instance of \ct{Metaclass}.
	\ct{ClassDescription} provides everything that is common to \ct{Class} and \ct{Metaclass}.

\item The metaclass of \ct{Metaclass} is an instance of \ct{Metaclass}.
	The \emph{instance-of} relation forms a closed loop, so \ct{Metaclass class class --> Metaclass}.
\end{itemize}
%=================================================================
\ifx\wholebook\relax\else\end{document}\fi
%=================================================================

%-----------------------------------------------------------------

