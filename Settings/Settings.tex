% $Author: ducasse $
% $Date: 2009-08-24 10:17:33 +0200 (Mon, 24 Aug 2009) $
% $Revision: 28563 $

% HISTORY:
% 2010-02-19 - Stef started

%=================================================================
\ifx\wholebook\relax\else
% --------------------------------------------
% Lulu:
	\documentclass[a4paper,10pt,twoside]{book}
	\usepackage[
		papersize={6.13in,9.21in},
		hmargin={.75in,.75in},
		vmargin={.75in,1in},
		ignoreheadfoot
	]{geometry}
	\input{../common.tex}
	\setboolean{lulu}{true}
% --------------------------------------------
% A4:
%	\documentclass[a4paper,11pt,twoside]{book}
%	\input{../common.tex}
%	\usepackage{a4wide}
% --------------------------------------------
    \graphicspath{{figures/} {../figures/}}
	\begin{document}
\fi
%=================================================================
%\renewmessage{\nnbb}[2]{} % Disable editorial comments
\sloppy
%=================================================================
\chapter{The Settings Framework}

When an application gets mature we often needs to provide variations such as the default selection color, the default font and the default font size... Often such variations represent user preferences of possible software customizations. 
Since the 1.1 release, Pharo contains and uses the Settings framework to manage its preferences. However Settings is not limited to manage Pharo preferences but can be used by any applications. The nice thing about Settings is that it is 
not intrusive, it supports a modular decomposition of software and can be added to an application after its inception. 
This is what we will see now.

\section{Settings in a nutshell}
A preference is a particular \emph{value} which is usually accessible from everywhere. Basically such a preference value is stored in a class variable or in an instance variable of a singleton and is directly managed through the use of simple accessors.

Pharo contains numerous preferences. As examples, the user interface theme, the desktop background color or a boolean flag to allow or prohibit the use of sound are currently declared as preferences. Pharo users need to browse existing preferences and eventually change their value, this is the major role of the settings browser presented in section~\ref{sec:TheSettingBrowser}. 

A setting is a \emph{declaration} (description) of a preference value. To be viewed and updated through the setting browser, a preference value must be described by a setting. Such a setting is built by a particular method tagged with a pragma. The section~\ref{sec:DeclaringASetting} explains how to declare a setting.

Depending on where and when you are using Pharo, some preferences can change often. As an example, when you are doing a demonstration, you may want to have bigger fonts, at work you may need to set a proxy whereas at home none is needed. Having to change a set of preferences depending on where you are and what you are doing can be very tedious and boring. With the setting browser it is possible to save the current set of preference values in a named style that can be reloaded later. Setting style management is presented in Section~\ref{sec:SettingStylesManagement}.

\section{The Setting browser}
\label{sec:TheSettingBrowser}
The setting browser, shown in figure~\figref{fig:TheSettingBrowser}, mainly allows the browsing of all currently declared settings and to change related preference values. To open it, just use the World menu (\menu{World \go System \go Settings}) or evaluate the following expression:
\begin{code}{}
SettingBrowser open
\end{code}
The settings are presented in several trees in the middle panel. Setting searching and filtering is available from the top toolbar whereas the bottom panels show currently selected setting description (left bottom panel) and current package set (right bottom panel). 
\begin{figure}[tbh]
\begin{center}
\includegraphics[scale=0.3]{SettingBrowser}
\caption{The setting browser}
\figlabel{fig:TheSettingBrowser}
\end{center}
\end{figure}

\subsection{Browsing and changing preference values}
Setting declarations are organized in trees which can be browsed in the middle panel.
\section{Declaring a setting}
\label{sec:DeclaringASetting}



\section{Setting styles management}
\label{sec:SettingStylesManagement}


\section{Design of the Settings Framework}

The design of the Settings framework is based on the three following points: (1) a preference is not defined in a global class but local to the package that uses it, (2) settings can be declared independently from the application they refer to, (3) any setting declaration can be loaded even if the Settings framework is not loaded. 

Let's explain now these two points since they have an impact on the modular structure of Pharo. 

\begin{figure}[tbh]
\begin{center}
\includegraphics[scale=0.3]{Principles}
\caption{The Settings principles}
\figlabel{principles}
\end{center}
\end{figure}

\paragraph{Local value and with a local flow.}

The Settings framework supports the idea that a preference value is local to a package. A package should define either a singleton or a class variable defined somewhere on a class. The methods of the class are able to refer to the variable. The package should provide some way to get and set the value of the preference.

here give an example

Understanding the difference to previous design as implemented in Squeak3.9 or Pharo1.0 is important.
In previous versions, the class Preferences was the place where the preferences were defined as well as methods to change their values. This implies that code using preferences was during its execution referencing the class Preferences.
The flow of control was clearly not local the class using the preferences but always executing some methods on the Preferences class.



@@
here show an example of the old way
@@


\paragraph{Declaration definition.}
While the actual value of a setting should be defined and used locally to a package, the setting declaration (so that the setting value) can be defined externally to the package using class extension or other mecanisms.
In essence a setting declaration is description of a value: it describes how to get its value, how to change it...

This means that you are not forced to define setting declaration in a package that defines a preference value.

@@ here an example@@


\paragraph{No explicit dependency on the Setting Framework.}
Finally when declaring a setting, the code does not refer explicitly to the setting class. This has the good property 
that you can load code containing setting declaration even if the Settings framework is not loaded. This way we make sure 
that we get a modular system. In case the settings framework is not loaded, the method containing the setting declaration 
is just not used by the system.


%\begin{code}{}
%| working |
%working := FSDiskFilesystem current working.
%\end{code} 
%
%Type  the above code into a workspace and evaluate it. It assigns a reference of the current working directory to the variable \ct{working}. References are the central object of the framework and provide the primary mechanism of working with files and directories. They are instances of the class \ct{FSReference}.
%
%Note that you not use platform specific classes such as \ct{FSUnixFilesystem} or \ct{FSWindowsFilesystem}. All code below works on \ct{FSReference} instances.
%
%
%\section{Navigating the Filesystem}
%
%Now let's do some more interesting things. To list children of your working directory evaluate the following expression:
%
%\begin{code}{}
%working := FSDiskFilesystem current working.
%working children.
%\end{code} 
%
%To access all the children of the current directory you can use \ct{allChildren}
%
%\begin{code}{}
%working allChildren.
%\end{code}
%
%
%To get only jpeg files you can for example 
%\begin{code}{}
%working allChildren select: [ :each | each basename endsWith: 'jpeg' ]
%\end{code} 
% 
%To get a reference to a specific file or directory within your working directory use the slash operator:
%
%\begin{code}{}
% cache := working / 'package-cache'.
%\end{code} 
%
%Navigating back to the parent is easy:
%
% cache parent.
%You can check for various properties of the cache directory by evaluating the following expressions:
%\begin{code}{}
% cache exists.             "--> true"
% cache isFile.             "--> false"
% cache isDirectory.        "--> true"
% cache basename.           "--> 'package-cache'"
%\end{code} 
% 
%To get additional information about the filesystem entry evaluate:
%
%\begin{code}{}
% cache entry creation.     "--> 2010-02-14T10:34:31+00:00"
% cache entry modification. "--> 2010-02-14T10:34:31+00:00"
% cache entry size.         "--> 0 (directories have size 0)"
%\end{code} 
%The framework also supports locations, late-bound references that point to a file or directory. When asking to perform a concrete operation, a location behaves the same way as a reference. Currently the following locations are supported:
%
%\begin{code}{}
% FSLocator desktop.
% FSLocator home.
% FSLocator image.
% FSLocator vmBinary.
% FSLocator vmDirectory.
%\end{code} 
%
%If you save a location with your image and move the image to a different machine or operating system, a location will still resolve to the expected directory or file.
%
%\subsection{Opening Read- and Write-Streams}
%
%To open a file-stream on a file ask the reference for a read- or write-stream:
%
%\begin{code}{}
% stream := (working / 'foo.txt') writeStream.
% stream nextPutAll: 'Hello World'.
% stream close.
% stream := (working / 'foo.txt') readStream.
% stream contents.
% stream close.
%\end{code}
%
%Please note that \ct{writeStream} overrides any existing file and \ct{readStream} throws an exception if the file does not exist. There are also short forms available:
%
%\begin{code}{}
% working / 'foo.txt' writeStreamDo: [ :stream | stream nextPutAll: 'Hello World' ].
% working / 'foo.txt' readStreamDo: [ :stream | stream contents ].
%\end{code}
%
%Have a look at the streams protocol of FSReference for other convenience methods.
%
%Renaming, Copying and Deleting Files and Directories
%
%You can also copy and rename files by evaluating:
%
%\begin{code}{}
% (working / 'foo.txt') copyTo: (working / 'bar.txt').
%\end{code} 
%
%To create a directory evaluate:
%\begin{code}{}
% backup := working / 'cache-backup'.
% backup createDirectory.
%\end{code} 
%
%And then to copy the contents of the complete package-cache to that directory simply evaluate:
%
% cache copyAllTo: backup.
%Note, that the target directory would be automatically created, if it was not there before.
%
%To delete a single file evaluate:
%
% (working / 'bar.txt') delete.
%To delete a complete directory tree use the following expression. Be careful with that one though.
%
% backup deleteAll.
% 
%Thats the basic API of the Filesystem library. If there is interest we can have a look at other features and other filesystem types in a next iteration.
%
%
%
% working / 'foo.txt' readStreamDo: [ :stream | stream nextPutAll: 'Hello World' ].
% working / 'foo.txt' writeStreamDo: [ :stream | stream contents ].
%
%
%\section{Design }
%\sd{should add class comments and a uml diagram}
%
%\subsection{Path}
%
%Paths are the most fundamental element of the Filesystem API. They represent filesystem paths in a very abstract sense, and provide a high-level protocol for working with paths without having to manipulate Strings. Here are some examples using the methods that FSPath provides:
%
%\begin{code}{}
%    "absolute path"
%    FSPath / 'plonk' / 'feep'       => /plonk/feep
%    
%    "relative path"
%    FSPath * 'plonk' / 'feep'       => plonk/feep
%
%    "relative path with extension"
%    FSPath * 'griffle' , 'txt'      => griffle.txt
%    
%    "changing the extension"
%    FSPath * 'griffle.txt' , 'jpeg'     => griffle.jpeg
%    
%    "parent directory"
%    (FSPath / 'plonk' / 'griffle') parent   => /plonk
%    
%    "resolving a relative path"
%    (FSPath / 'plonk' / 'griffle') resolve: (FSPath * '..' / 'feep')
%                        => /plonk/feep
%    
%    "resolving an absolute path"
%    (FSPath / 'plonk' / 'griffle') resolve: (FSPath / 'feep')
%                        => /feep
%                        
%    "resolving a string"
%    (FSPath * 'griffle') resolve: 'plonk'   => griffle/plonk
%                        
%    "comparing"
%    (FSPath / 'plonk') contains: (FSPath / 'griffle' / 'nurp')
%                        => false
%\end{code}
%
%\subsection{Filesystem}
%
%A filesystem is an interface to some hierarchy of directories and files. "The filesystem," provided by the host operating system, is embodied by FSDiskFilesystem and it's platform-specific subclasses. But other kinds of Filesystems are also possible. FSMemoryFilesystem provides a RAM diska filesystem where all files are stored as ByteArrays in the image. FSZipFilesystem represents the contents of a zip file.
%
%Each filesystem has its own working directory, which it uses to resolve any relative paths that are passed to it. Some examples:
%
%\begin{code}{}
%    fs := FSMemoryFilesystem new.
%    fs workingDirectory: (FSPath / 'plonk').
%    griffle := FSPath / 'plonk' / 'griffle'.
%    nurp := FSPath * 'nurp'.
%    
%    fs resolve: nurp            => /plonk/nurp
%    
%    fs createDirectory: (FSPath / 'plonk')  => "/plonk created"
%    (fs writeStreamOn: griffle) close.  => "/plonk/griffle created"
%    fs isFile: griffle.         => true
%    fs isDirectory: griffle         => false
%    fs copy: griffle to: nurp       => "/plonk/griffle copied to /plonk/nurp"
%    fs exists: nurp             => true
%    fs delete: griffle          => "/plonk/griffle" deleted
%    fs isFile: griffle          => false
%    fs isDirectory: griffle         => false
%\end{code}
%	
%\subsection{Reference}
%
%Paths and filesystems are the lowest level of the Filesystem API. An FSReference combines a path and a filesystem into a single object which provides a simpler protocol for working with files. It implements the same operations as FSFilesystem , but without the need to track paths and filesystem separately:
%
%
%\begin{code}{}
%    fs := FSMemoryFilesystem new.
%    griffle := fs referenceTo: (FSPath / 'plonk' / 'griffle').
%    nurp := fs referenceTo: (FSPath * 'nurp').
%    
%    griffle isFile              
%    griffle isDirectory 
%    
%    griffle parent ensureDirectory.     
%    griffle writeStreamDo: [:s]         
%    griffle copyTo: nurp            
%    griffle delete              
%\end{code}    
%
%References also implement the path protocol, with methods like \ct{/} , \ct{parent} and \ct{resolve:}.
%
%\subsection{Locator}
%
%Locators could be considered late-bound references. They're left deliberately fuzzy, and only resolved to a concrete reference when some file operation needs to be performed. Instead of a filesystem and path, locators are made up of an origin and a path. An origin is an abstract filesystem location, such as the user's home directory, the image file, or the VM executable. When it receives a message like \ct{isFile}, a locator will first resolve its origin, then resolve its path against the origin.
%
%Locators make it possible to specify things like "an item named 'package-cache' in the same directory as the image file" and have that specification remain valid even if the image is saved and moved to another directory, possibly on a different computer.
%
%\begin{code}{}
%    locator := FSLocator image / 'package-cache'.
%    locator printString             => '{image}/package-cache'
%    locator resolve                 => /Users/colin/Projects/Mason/package-cache
%    locator isFile                  => false
%    locator isDirectory             => true
%\end{code}	
%
%The following origins are currently supported:
%
%\ct{image} - the image file
%\ct{changes} - the changes file
%\ct{vmBinary} - the executable for the running virtual machine
%\ct{vmDirectory} - the directory containing the VM application (may not be the parent of \ct{vmBinary})
%\ct{home} - the user's home directory
%\ct{desktop} - the directory that hold the contents of the user's desktop
%\ct{documents} - the directory where the user's documents are stored
%
%
%Applications my also define their own origins, but the system will not be able to resolve them automatically. Instead, the user will be asked to manually choose a directory. This choice is then cached so that future resolution requests won't require user interaction.
%
%\subsection{Enumeration}
%
%References and Locators also provide simple methods for dealing with whole directory trees:
%
%\begin{description}
%\item[allChildren]
%
%This will answer an array of references to all the files and directories in the directory tree rooted at the receiver. If the receiver is a file, the array will contain a single reference, equal to the receiver.
%
%\item[allEntries]
%This method is similar to \ct{allChildren}, but it answers an array of \ct{FSDirectoryEntries}, rather than references.
%
%\item[copyAllTo: aReference]
%
%This will perform a deep copy of the receiver, to a location specified by the argument. If the receiver is a file, the file will be copied; if a directory, the directory and its contents will be copied recursively. The argument must be a reference that doesn't exist; it will be created by the copy.
%
%\item[deleteAll]
%
%This will perform a recursive delete of the receiver. If the receiver is a file, this has the same effect as \ct{delete}.
%\end{description}
%
%\subsection{Visitors}
%
%The above methods are sufficient for many common tasks, but application developers may find that they need to perform more sophisticated operations on directory trees.
%
%The visitor protocol is very simple. A visitor needs to implement \ct{visitFile:} and \ct{visitDirectory:}. The actual traversal of the filesystem is handled by a guide. A guide works with a visitor, crawling the filesystem and notifying the visitor of the files and directories it discovers. There are three Guide classes, \ct{FSPreorderGuide}, \ct{FSPostorderGuide} and \ct{FSBreadthFirstGuide} , which traverse the filesystem in different orders. To arrange for a guide to traverse the filesystem with a particular visitor is simple. Here's an example:
%
%\begin{code}{}
%    FSBreadthFirstGuide show: aReference to: aVisitor
%\end{code}	
%
%The enumeration methods described above are implemented with visitors; see \ct{FSCopyVisitor}, \ct{FSDeleteVisitor} and \ct{FSCollectVisitor} for examples.



%=========================================================
\ifx\wholebook\relax\else
   \bibliographystyle{jurabib}
   \nobibliography{scg}
   \end{document}
\fi

%=================================================================
\ifx\wholebook\relax\else\end{document}\fi
%=================================================================

%-----------------------------------------------------------------

%%% Local Variables:
%%% coding: utf-8
%%% mode: latex
%%% TeX-master: t
%%% TeX-PDF-mode: t
%%% ispell-local-dictionary: "english"
%%% End: