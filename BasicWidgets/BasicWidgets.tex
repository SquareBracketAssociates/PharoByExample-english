% $Author: ducasse $
% $Date: 2009-08-24 10:17:33 +0200 (Mon, 24 Aug 2009) $
% $Revision: 28563 $

% HISTORY:


%=================================================================
\ifx\wholebook\relax\else
% --------------------------------------------
% Lulu:
	\documentclass[a4paper,10pt,twoside]{book}
	\usepackage[
		papersize={6.13in,9.21in},
		hmargin={.75in,.75in},
		vmargin={.75in,1in},
		ignoreheadfoot
	]{geometry}
	\input{../common.tex}
	\setboolean{lulu}{true}
% --------------------------------------------
% A4:
%	\documentclass[a4paper,11pt,twoside]{book}
%	\input{../common.tex}
%	\usepackage{a4wide}
% --------------------------------------------
    \graphicspath{{figures/} {../figures/}}
	\begin{document}
\fi
%=================================================================
%\renewmessage{\nnbb}[2]{} % Disable editorial comments
\sloppy
%=================================================================

\chapter{Basic Widgets}

\section{The window}
\subsection{Opening a window}
The basic class for managing a window is \ct{StandardWindow}. Let's start with an empty one. To create and open an empty window just try the following:
\begin{code}{}
StandardWindow new openInWorld
\end{code}
You should see a window with a topbar that you can move with the mouse (see FIG~\ref{fig:emptyWindow}). 

\begin{figure}[ht]\centering
	\includegraphics[width=6cm]{EmptyWindow}
	\caption{An empty window}
	\label{fig:emptyWindow}
\end{figure} 

On one side, the topbar has three buttons for closing, collapsing and expanding the window. On the other side, the topbar has a menu button with a default set of items.

\subsection{A window and its model}

To set a model to a window, it's quite easy:

\begin{code}{}
StandardWindow new model: myModel
\end{code}

By specifying a model to a window, and by providing specifics methods, the model will be able to control some of the window's behaviors.

\subsubsection{Little example:}

First, let's create the model class

\begin{classdef}{Defining a specific Model.}

Object subclass: #MyModel
	instanceVariableNames: ''
	classVariableNames: ''
	poolDictionaries: ''
	category: 'PBE2-Examples'

MyModel>>#initialExtent

	^ 200@200
\end{classdef}

Let's see the result:

\begin{code}{}
StandardWindow new openInWorld.
StandardWindow new model: (MyModel new); openInWorld.
\end{code}

So you should see a window with the same size than the previous one, and a small window which size is exactly what you have specified in the method \ct{initialExtent} (see FIG~\ref{fig:withAndWithoutModel}).

\begin{figure}[ht]\centering
	\includegraphics[width=7cm]{WithAndWithoutModel}
	\caption{Two windows: one without a model and one with a model}
	\label{fig:withAndWithoutModel}
\end{figure}

You can also defined this method:

\begin{method}{Defined if the model is ok to change}
okToChange

	^ false
\end{method}

Now, when you try to close the small window nothing happens.

\subsection{Your own topbar menu}

\subsection{Menubar}

Menubar allow you to add direct and simple shortcuts.

First, your window must have a model, then you simply have to implement the method \ct{} and to fill up the menu provided as parameter as following:

\begin{method}{}

addModelItemsToWindowMenu: aMenu
	"Add model-related items to the window menu"
	
	" First, we add a separator "
	aMenu addLine.
	
	" Then, we add our items "
	aMenu
		add: 'Label of the entry'
		target: receiverOfTheFollowingSelector
		action: #selectorWeWantToBeExecuted.

\end{method} 

\subsubsection{Little Example}

\begin{method}{Define the extra entries of the menu}
MyModel>>addModelItemsToWindowMenu: aMenu
	"Add model-related items to the window menu"
	"super addModelItemsToWindowMenu: aMenu."
	
	" First, we add a separator "
	aMenu addLine.
	
	" Then, we add our items "
	aMenu
		add: 'Open an inspector on me'
		target: self
		action: #inspect.
\end{method} 

So when you click on the menu button, you see at the end of the list the label we typed just before (see FIG~\ref{fig:menuBar}), and when you click on it, an inspector is opened (see FIG~\ref{fig:windowAndInspector})

\begin{figure}[ht]\centering
	\includegraphics[width=7cm]{MenuBar}
	\caption{Menu bar with our extra item at the end}
	\label{fig:menuBar}
	\includegraphics[width=7cm]{WindowAndInspector}
	\caption{The inspector is well opened}
	\label{fig:windowAndInspector}
\end{figure}

\subsection{Taskbar}

\section{Buttons}

\section{Text fields}

\section{Text editor}

\section{Panes and layout managing}

\section{List widgets}

\section{Tree widgets}

\section{Conclusion}

%=========================================================
\ifx\wholebook\relax\else
   \bibliographystyle{jurabib}
   \nobibliography{scg}
   \end{document}
\fi

%=================================================================
\ifx\wholebook\relax\else\end{document}\fi
%=================================================================

%-----------------------------------------------------------------

%%% Local Variables:
%%% coding: utf-8
%%% mode: latex
%%% TeX-master: t
%%% TeX-PDF-mode: t
%%% ispell-local-dictionary: "english"
%%% End: