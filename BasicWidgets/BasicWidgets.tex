% $Author: ducasse $
% $Date: 2009-08-24 10:17:33 +0200 (Mon, 24 Aug 2009) $
% $Revision: 28563 $

% HISTORY:


%=================================================================
\ifx\wholebook\relax\else
% --------------------------------------------
% Lulu:
	\documentclass[a4paper,10pt,twoside]{book}
	\usepackage[
		papersize={6.13in,9.21in},
		hmargin={.75in,.75in},
		vmargin={.75in,1in},
		ignoreheadfoot
	]{geometry}
	\input{../common.tex}
	\setboolean{lulu}{true}
% --------------------------------------------
% A4:
%	\documentclass[a4paper,11pt,twoside]{book}
%	\input{../common.tex}
%	\usepackage{a4wide}
% --------------------------------------------
    \graphicspath{{figures/} {../figures/}}
	\begin{document}
\fi
%=================================================================
%\renewmessage{\nnbb}[2]{} % Disable editorial comments
\sloppy
%=================================================================

\chapter{Basic Widgets}

\section{The window}
\subsection{Opening a window}
The basic class for managing a window is \ct{StandardWindow}. Let's start with an empty one. To create and open an empty window just try the following:
\begin{code}{}
StandardWindow new openInWorld
\end{code}
You should see a window with a topbar that you can move with the mouse (see FIG~\ref{fig:emptyWindow}). 

\begin{figure}[ht]\centering
	\includegraphics[width=6cm]{EmptyWindow}
	\caption{An empty window}
	\label{fig:emptyWindow}
\end{figure} 

On one side, the topbar has three buttons for closing, collapsing and expanding the window. On the other side, the topbar has a menu button with a default set of items.

\subsection{A window and its model}

To set a model to a window, it's quite easy:

\begin{code}{}
StandardWindow new model: myModel
\end{code}

By specifying a model to a window, and by providing specifics methods, the model will be able to control some of the window's behaviors.

\subsubsection{Example:}

First, let's create the model class

\begin{classdef}{Defining a specific Model.}

Object subclass: #MyModel
	instanceVariableNames: ''
	classVariableNames: ''
	poolDictionaries: ''
	category: 'PBE2-Examples'

MyModel>>#initialExtent

	^ 200@200
\end{classdef}

Let's see the result:

\begin{code}{}
StandardWindow new openInWorld.
StandardWindow new model: (MyModel new); openInWorld.
\end{code}

So you should see a window with the same size than the previous one, and a small window which size is exactly what you have specified in the method \ct{initialExtent} (see FIG~\ref{fig:withAndWithoutModel}).
\sd{should we always specifies a model?}
\ben{If you want to specify the initial extent, either you define a model or you subclass (as far as I know)}
\sd{what should be put in the model vs. the Morph itself?}
\ben{I think/hope it will be clear when the API section will be complete}

\begin{figure}[ht]\centering
	\includegraphics[width=7cm]{WithAndWithoutModel}
	\caption{Two windows: one without a model and one with a model}
	\label{fig:withAndWithoutModel}
\end{figure}

You can also defined this method:

\begin{method}{Define if the model is ok to change or not}
okToChange

	^ false
\end{method}

Now, when you try to close the small window nothing happens.

\subsection{Your own topbar menu}

Here, we will show you how to easily add direct and simple shortcuts in the window's menu.

First, your window must have a model, then you simply have to implement the method \mthind{addModelItemsToWindowMenu:}{addModelItemsToWindowMenu: aMenu} and to fill up the menu provided as parameter as following:

\begin{method}{}
addModelItemsToWindowMenu: aMenu
	"Add model-related items to the window menu"
	
	"First, we add a separator"
	aMenu addLine.
	
	"Then, we add our items"
	aMenu
		add: 'Label of the entry'
		target: receiverOfTheFollowingSelector
		action: #selectorWeWantToBeExecuted.

\end{method} 

\subsubsection{Example:}

\begin{method}{Define the extra entries of the menu}
MyModel>>addModelItemsToWindowMenu: aMenu
	"Add model-related items to the window menu"
	super addModelItemsToWindowMenu: aMenu.
	aMenu addLine.
	aMenu
		add: 'Open an inspector on me'
		target: self
		action: #inspect.
\end{method} 

So when you click on the menu button, you see at the end of the list the label we typed just before (see FIG~\ref{fig:menuBar}), and when you click on it, an inspector is opened (see FIG~\ref{fig:windowAndInspector})

\begin{figure}[ht]\centering
	\includegraphics[width=7cm]{MenuBar}
	\caption{Menu with our extra item at the end}
	\label{fig:menuBar}
	\includegraphics[width=7cm]{WindowAndInspector}
	\caption{The inspector is well opened}
	\label{fig:windowAndInspector}
\end{figure}

\subsection{Main Window API}

Here is the list of the main API for a window

\begin{itemize}
\item Title
\item Color
\item roundcorners?
\item minimalExtent
\item 

\begin{code}{}
SystemWindow new
	maximumExtent: 200@100; openInWorld

pareil pour 

StandardWindow new
	unexpandedFrame: (0@0 extent: 200@100) ; openInWorld
\end{code}

je ne comprends pas pourquoi je peux alors avoir une fenetre immense?

\item unresizeable?
\item Action on close?
\end{itemize}


\section{Buttons}

\section{Text fields}

\section{Text editor}

\section{Panes and layout managing}

\section{List widgets}

\section{Tree widgets}

\section{Add submorphs to a morph}

%Now that we have our window as we like, we will fill it up with some other morphs.

\subsection{With the default layout}

The default layout is ProportionalLayout. To use it, you have to use the method \mthind{addMorph:frame:}{addMorph: aMorph frame: aFrame}
\begin{code}{}
aWindow
	addMorph: morphToAdd
	frame: (x0@y0 corner: x1@y1)
\end{code}
where x0, y0, x1 and y1 are defined as shown in the figure.
Note that those values are between 0 and 1.

\ben{add a figure here as soon as Omnigraffle works again}

Let's do some examples

\subsubsection{Examples:}

%| morphToAdd1 morphToAdd2 morphToAdd3 |
First, let's define our objects
\begin{code}{Objects definition}
| container redMorph blueMorph greenMorph |

redMorph := Morph new 
				color: (Color red);
				yourself.
				
blueMorph := Morph new 
				color: (Color blue);
				yourself.
				
greenMorph := Morph new 
				color: (Color green);
				yourself.
				
container := PanelMorph new.
\end{code}

\paragraph{Now let's see different configurations and their result}

\begin{code}{}
window
	addMorph: redMorph
	frame: (0@0 corner: 1@1).
	
redMorph color: Color red.	
	
window openInWorld
\end{code}
Note: you have to reset the color of the morph after having added it because the window set the default color. Here it seems strange but for more complicated morphs (like buttons, list \dots) it sets the background color for a better integration.

As a result, you can see that the red morph is stretched the fill the space both vertically and horizontally (see FIG~\ref{fig:simpleLayoutExample1}).

\begin{figure}[ht]\centering
	\includegraphics[width=7cm]{SimpleLayoutExample1}
	\caption{The red morph fill the whole space}
	\label{fig:simpleLayoutExample1}
\end{figure}

\paragraph{Let's try a little more complicated configuration}

\begin{code}{}
window
	addMorph: redMorph
	frame: (0@0 corner: 0.33@1).

window
	addMorph: blueMorph
	frame: (0.33@0 corner: 0.66@1).

window
	addMorph: greenMorph
	frame: (0.66@0 corner: 1@1).

redMorph color: Color red.
blueMorph color: Color blue.
greenMorph color: Color green.

windows openInWorld
\end{code}

As a result, you can see three bands of color where each is horizontally a third of the window size and fill the space vertically (see FIG~\ref{fig:simpleLayoutExample2}).

\begin{figure}[ht]\centering
	\includegraphics[width=7cm]{SimpleLayoutExample2}
	\caption{Three bands of colors}
	\label{fig:simpleLayoutExample2}
\end{figure}

Note that if you resize the window, proportions are kept.
Also note that each band can be resized horizontally.

\paragraph{So let's do the last example}

\begin{code}{}
window
	addMorph: redMorph
	frame: (0@0 corner: 0.5@0.5).

window
	addMorph: blueMorph
	frame: (0.5@0 corner: 1@0.5).

window
	addMorph: greenMorph
	frame: (0@0.5 corner: 1@1).

redMorph color: Color red.
blueMorph color: Color blue.
greenMorph color: Color green.
	
window openInWorld.
\end{code}

As you have predicted, the result is composed by two squares above and the green rectangle below (see FIG~\ref{fig:simpleLayoutExample3}).

\begin{figure}[ht]\centering
	\includegraphics[width=7cm]{SimpleLayoutExample3}
	\caption{Two squares above and a rectangle below}
	\label{fig:simpleLayoutExample3}
\end{figure}

Note that like for the previous example, you can resize each part.

So basically, you now know everything about the default layout.

\subsection{More complicated layouts}

Now that we have seen the default layout, I will introduce you quickly the other layouts:
\begin{itemize}
	\item LayoutFrame
	\item RowLayout
	\item StackLayout
	\item TableLayout
\end{itemize}

\subsubsection{LayoutFrame}

This layout is used when you have to specified both a fix part \footnote{independent of the size of the window} and a proportional part\footnote{like the ProportionalLayout}.



\section{Conclusion}

%=========================================================
\ifx\wholebook\relax\else
   \bibliographystyle{jurabib}
   \nobibliography{scg}
   \end{document}
\fi

%=================================================================
\ifx\wholebook\relax\else\end{document}\fi
%=================================================================

%-----------------------------------------------------------------

%%% Local Variables:
%%% coding: utf-8
%%% mode: latex
%%% TeX-master: t
%%% TeX-PDF-mode: t
%%% ispell-local-dictionary: "english"
%%% End: