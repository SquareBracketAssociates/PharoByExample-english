% $Author: ducasse $
% $Date: 2009-08-24 10:17:33 +0200 (Mon, 24 Aug 2009) $
% $Revision: 28563 $

% HISTORY:


%=================================================================
\ifx\wholebook\relax\else
% --------------------------------------------
% Lulu:
	\documentclass[a4paper,10pt,twoside]{book}
	\usepackage[
		papersize={6.13in,9.21in},
		hmargin={.75in,.75in},
		vmargin={.75in,1in},
		ignoreheadfoot
	]{geometry}
	\input{../common.tex}
	\setboolean{lulu}{true}
% --------------------------------------------
% A4:
%	\documentclass[a4paper,11pt,twoside]{book}
%	\input{../common.tex}
%	\usepackage{a4wide}
% --------------------------------------------
    \graphicspath{{figures/} {../figures/}}
	\begin{document}
\fi
%=================================================================
%\renewmessage{\nnbb}[2]{} % Disable editorial comments
\sloppy
%=================================================================

\chapter{Basic Widgets}

This chapter is about the basic widow gadgets (aka widgets) used to compose Graphic User Interfaces in Pharo using the Morphic layer. Since Morphic is based on a composite pattern,  you can easily assemble different widgets to build complex interfaces. So we will explain how the basic widgets (the one that should be mainly used) work and then how to assemble them together. To help you to understand such mechanisms, an example will be used trough the whole chapter.

\section{The window}
\index{Morphic!window}

\ja{explain the basis of Widget ? What is a widget in pharo ? Which classes should I know ? }

\subsection{Opening a window}
The basic class for managing a window is \clsind{StandardWindow}. Let's start with an empty one.

\dothis{To create and open an empty window just try the following:}
\begin{code}{}
StandardWindow new openInWorld
\end{code}
You should see a window with a top bar \ja{title bar ?} that you can move with the mouse (see \figref{fig:emptyWindow}). 

\begin{figure}[ht]
\begin{center}
	\includegraphics[width=6cm]{EmptyWindow}
	\caption{An empty window}
	\figlabel{fig:emptyWindow}
\end{center}
\end{figure} 

On one side, the topbar has three buttons for closing, collapsing and expanding the window. On the other side, the topbar has a menu button with a default set of items \ja{which ones ?}.

\subsection{A window and its model}
\index{Morphic!window model}

\ben{introduce MVC}
To set a model to a window \ja{what is a model ? why should I set a model to a window ?}, it's quite easy:

\begin{code}{}
StandardWindow new model: myModel
\end{code}

By specifying a model to a window, and by providing specifics methods, the model will be able to control some of the window's behaviors.

\subsubsection{Example:}

First, let's create the model class

\begin{classdef}{Defining a specific Model.}
Object subclass: #MyModel
	instanceVariableNames: ''
	classVariableNames: ''
	poolDictionaries: ''
	category: 'PBE2-Examples'

MyModel>>#initialExtent

	^ 200@200
\end{classdef}

Let's see the result:

\begin{code}{}
StandardWindow new openInWorld.
StandardWindow new model: (MyModel new); openInWorld.
\end{code}

So you see a window with the same size than the previous one, and a small window which size is exactly what you have specified in the method \mthind{Morph}{initialExtent} (see \figref{fig:withAndWithoutModel}).
\sd{should we always specifies a model?}
\ben{If you want to specify the initial extent, either you define a model or you subclass (as far as I know)}
\sd{what should be put in the model vs. the Morph itself?}
\ben{I think/hope it will be clear when the API section will be complete}
\ja{if you are sure about your answers, just include them in the real text.}

\begin{figure}[ht]
\begin{center}
	\includegraphics[width=7cm]{WithAndWithoutModel}
	\caption{Two windows: one without a model and one with a model}
	\figlabel{fig:withAndWithoutModel}
\end{center}
\end{figure}
\ben{move that into Model API part}
You can also override this method \ja{what is the goal ? I do not want to define a method for nothing... you are writing a story :)}:

\begin{method}{Define if the model is ok to change or not}
MyModel>>#okToChange

	^ false
\end{method}

Now, when you try to close the small window the window will not be closed since the model explicitly precise (through the method \mthind{okToChange}{okToChange} that we just override) that the widget is not ok to be changed.

\subsection{Your own topbar menu}

Here, we will show you how to easily add direct and simple shortcuts in the window's menu.

First, your window must have a model \ja{blah blah... why must have a model}, then you simply have to override the method \mthind{Object}{addModelItemsToWindowMenu: aMenu} and to fill up the menu provided as parameter as following:
\apl{addModelItemsToWindowMenu: is in Object !!}

\begin{method}{}
MyModel>>#addModelItemsToWindowMenu: aMenu
	"Add model-related items to the window menu"
	
	"First, we add a separator"
	aMenu addLine.
	
	"Then, we add our items"
	aMenu
		add: 'Label of the entry'
		target: receiverOfTheFollowingSelector
		action: #selectorWeWantToBeExecuted.

\end{method} 

\subsubsection{Example:}

\begin{method}{Define the extra entries of the menu}
MyModel>>#addModelItemsToWindowMenu: aMenu
	"Add model-related items to the window menu"
	super addModelItemsToWindowMenu: aMenu.
	aMenu addLine.
	aMenu
		add: 'Open an inspector on me'
		target: self
		action: #inspect.
\end{method}

So when you click on the menu button, you see at the end of the list the label we typed just before in the method \ben{how to ref to method env}  (see \figref{fig:menuBar}), and when you click on it, an inspector is opened (see \figref{fig:windowAndInspector})

\begin{figure}[ht]
\begin{center}
	\includegraphics[width=7cm]{MenuBar}
	\caption{Menu with our extra item at the end}
	\figlabel{fig:menuBar}
\end{center}
\end{figure}

\begin{figure}[ht]
\begin{center}
	\includegraphics[width=7cm]{WindowAndInspector}
	\caption{The inspector is well opened}
	\figlabel{fig:windowAndInspector}
\end{center}
\end{figure}

\subsection{Main Window API}

\ja{some blahblah} Here is the list of the main API for a window

\begin{itemize}
\item Title
\item Color
\item roundcorners?
\item minimalExtent
\item 

\begin{code}{}
SystemWindow new
	maximumExtent: 200@100; openInWorld

pareil pour 

StandardWindow new
	unexpandedFrame: (0@0 extent: 200@100) ; openInWorld
\end{code}

je ne comprends pas pourquoi je peux alors avoir une fenetre immense? \ja{what ?}

\item unresizeable? \ja{what ?}
\item Action on close? \ja{what ?}
\end{itemize}

\paragraph*{Title.}
\mthind{StandardWindow}{title:}
The default title of the window is simply 'Window'. You can change this title by sending \mthind{StandardWindow}{title:} to a window with a String as argument:
\begin{code}{}
w := StandardWindow new title: 'My first window'.
w openInWorld
\end{code}
\paragraph*{Position and size.}
In order to change the position and the size of a window, you can send the messages \mthind{StandardWindow}{extent:} and \mthind{StandardWindow}{position:} to a newly opened window. 
\needlines{4}
\begin{code}{}
w := StandardWindow new title: 'My first window'.
w openInWorld.
w extent: 100@50.
w position: 0@0
\end{code}
\dothis{Now try the following: first create a window but without opening it; then set its size and its position and then open it by sending \ct{openInWorld} to it.}
\needlines{4}
\begin{code}{}
w := StandardWindow new title: 'My first window'.
w extent: 100@50.
w position: 0@0
w openInWorld.
\end{code}
You should observe that the size and the position are not set with your values. Don't worry, you did nothing wrong, its a normal behavior regarding the way window opening is implemented in Morphic. In fact you have to know that a window internal state is overwritten with default values when it is opened. This is one of the responsibilities of \clsind{RealEstateAgent}. 


\section{Buttons}
\subsection{Push button}
\subsection{Check button}
\subsection{Radio button}

\section{Text fields}
\subsection{get text}
\subsection{set text}
\subsection{selection}

\section{Text editor}

\section{Panes and layout managing}
\subsection{newRow:}
\subsection{newColumn:}
\subsection{newGroup:}

\section{List widgets}
\subsection{getting list}
\subsection{getting selections}
\subsection{setting selections}

\hjo{outlining with subsections.}

\section{Tree widgets}
\subsection{getting list}
\subsection{getting selections}
\subsection{setting selections}
\subsection{collapsing and expanding}

\section{Layout management}
\index{Morphic!Layout management}

\ben{Maybe we should write a section about layouts to describe their API ?}
\sd{yes excellent idea. Do you know that luarent and hilaire wrote some text }
\ben{in fact, only TableLayout have a specific API}

A morph is basically a composite. As such it can be composed of sub-morphs. As it is explained below, adding a morph into a parent is very simple: just use the \mthind{Morph}{addMorph:} message which is sent to a parent morph with a sub-morph passed as argument. The problem is to understand how a morph can manage its own layout according to its size and the morphs that it contains. There are two main possibilities: (1) a morph can be added to another morph with a fixed position or (2) the placement of a morph is automatically managed with regards to its parent morph.

Of course, the second solution is predominantly chosen. In this case, a layout manager is to be used to manage the space of a morph. But one have to decide if the placement is managed from the parent-morph or from the sub-morph point of view:
\begin{description}
\item[top-down placement:] the placement is managed from the point of view of the parent morph which makes use of a layout manager that decides how to place the sub-morphs;
\item[bottom-up placement:] the placement is managed locally from the point of view of each sub-morph, the layout manager decides how to place a sub-morph according the parent extent.
\end{description}
Historically, these two possibilities are respectively provided by the \clsind{TableLayout} and by the \clsind{ProportionalLayout} classes. As it is explained in this chapter, each of these layout manager has its own variants and now a lot of possibilities are offered by the morphic framework.

%Adding a morph to another one is simple using the \ct{addMorph:} message. Now we often want to place a morph at a given place and in particular that the morph occupies space even if its container morph is resized. For that we use layout. Layout are objects that control how to morphs occupy space. Several layouts exist: proportional, 
\subsection{Composing a morph}
To add a morph into a parent morph, just make use of the \mthind{Morph}{addMorph:} message. Together with the use of the \ct{extent:} and of the \ct{position:} messages, it is straightforward to compose a morph.

\dothis{Declare a morph with a small extent and its parent with a bigger extent, then assign a color to the parent (different from blue), set the position of the parent, add the first morph to the parent and open the parent morph}

As an example, you can try with the following code:

\begin{code}
| sub parent |
sub := Morph new extent: 30 @ 30. "the sub-morph with a small extent"
parent := Morph new extent: 100 @ 80. "the parent morph with a bigger extent"
parent position: 10 @ 10; color: Color orange. "set the parent position and color"
parent addMorph: sub. "add the sub-morph"
parent openInWorld "open the parent morph"
\end{code}

\begin{figure}[htbp]
\begin{center}
	\includegraphics[width=7cm]{composingMorph1}
	\caption{Composing a morph: a first try}
	\figlabel{fig:composingMorph1}
\end{center}
\end{figure}

The result is shown in \figref{fig:composingMorph1}. You may be surprised because the sub-morph is drawn outside it's parent. There are two reasons for this:
\begin{itemize}
\item first, you must recall that a particularity of the morph system is to consider a morph's position as an absolute one. Here, the sub-morph position is \ct{0@0} because it has not been explicitly changed. It implies that the sub-morph is drawn starting at the top left of the world.
\item second, the parent position has been set \textbf{before} the sub-morph has been added to it.
\end{itemize}

\dothis{To be convinced, try the same experiment but this time, set the parent morph before adding the sub-morph}

\begin{code}
| sub parent |
sub := Morph new extent: 30 @ 30. 
parent := Morph new extent: 100 @ 80. 
parent addMorph: sub. !\textbf{"add the sub-morph"}!
parent position: 10 @ 10; color: Color orange. !\textbf{"set the parent position after"}!
parent openInWorld 
\end{code}
The result of this second try is shown in \figref{fig:composingMorph2}.
\begin{figure}[htbp]
\begin{center}
	\includegraphics[width=7cm]{composingMorph2}
	\caption{Composing a morph: a second try}
	\figlabel{fig:composingMorph2}
\end{center}
\end{figure}

For this second solution, the sub-morph is well physically inside its parent.

\dothis{Move the parent morph with the mouse or by sending to it the message \ct{position:} with a point as argument. Do this for the two solutions}

You can observe that the sub-morph is also moved whatever the solution you consider. If you inspect the sub-morph after having moved its parent you can see that its position has changed accordingly. So we can say that the sub-morph is logically in its parent but this doesn't mean that it is physically inside its parent. The reason why the second solution sub-morph is drawn inside its parent is only because the parent position has been changed after the sub-morph adding. 

The consequence is that if you want a submorph to be placed inside the area occupied by its parent then you have to explicitly calculate its position according to the parent position.

\dothis{Open the morph halo on the parent morph, then change the parent extent by clicking on the bottom right yellow button and moving the mouse around with the mouse button down. Do the same but with the sub-morph}

You observe that, if you change the parent morph extent, then the sub-morph extent stays unchanged and that you can make the parent morph smaller that its sub-morph. The same thing occurs if you change the sub-morph extent. It means that sub-morph and parent morph extents are managed independently.

So, if you want a morph extent to be changed according to a particular rule, for example according to a parent or a sub-morph extent change, then you have to implement it.

\dothis{To exercise yourself, try to implement the following: a red morph containing one row with three adjacent white sub-morphs, each containing itself a \ct{StringMorph} showing its position. see \figref{fig:composingMorph3}}.

\begin{figure}[htbp]
\begin{center}
	\includegraphics[width=7cm]{composingMorph3}
	\caption{Composing a morph with a row of sub-morphs}
	\figlabel{fig:composingMorph3}
\end{center}
\end{figure}
Here is a solution:
\begin{code}
| parent |
parent := Morph new color: Color red.
1 to: 3
 do: [:i | 	| sub |
	sub := Morph new color: Color white; borderWidth: 1; extent: 20 @ 20; yourself.
	sub addMorphCentered: (StringMorph contents: i asString).
	parent addMorph: sub.
	sub position: (Point x: i - 1 * sub width y: 0)].
parent openInWorld
\end{code}


Now, imagine that you want the parent morph extent to exactly fit its row extent, that you want to be able to add a sub-morph and have the parent morph extent updated accordingly. It is clear that one can't implement it like that. But don't be despaired of this, don't give up!.  Of course, there are many way do automatically manage morph positions and area extents as it is explained in next sections.


%On remarque que, bien que le morph bleu soit ajouté dans le morph orange, le morph bleu est visuellement à l'extérieur de l'orange. Par contre, si on déplace le morph orange, on déplace bien son contenu, le morph bleu, aussi. L'envoi du message #position: à un morph provoque le déplacement du morph et des morphs contenus. Le décalage observé entre nos deux morphs existe parce-que le message #position: a été envoyé au morph orange avant l'ajout du morph bleu.
%
%Quand on ajoute un morph dans un autre sans indiquer de position avec le message #position:, le morph est ajouté en position absolue 0@0. Si on ajoute un second morph, alors il recouvre le premier, toujours en position 0@0.
%
%La taille du morph orange peut être changée (à l'aide du halo par exemple) sans conséquence vis à vis du bleu. De même, la taille du bleu peut être changée sans conséquence vis à vis de l'orange.
%
%Dans Squeak, certaines IHM sont construites de cette façon (voir le panneau compilation de l'éditeur de Smacc) mais ce n'est pas à conseiller, l'utilisation de positions fixes rend difficile la maintenance et l'adaptabilité d'un tel code.
%
%Voici par exemple le code d'un morph contenant d'autres morphs dont la position est calculée de façon à constituer une ligne régulière :
%
%alig0
%	"MUIDAlignmentTest new alig0"
%	| alig sm |
%	alig := Morph new.
%	alig color: Color orange.
%	1
%		to: 3
%		do: [:i | 
%			alig addMorphBack: (sm := self cellMorphNamed: i printString).
%			sm
%				position: (Point x: i - 1 * sm width y: 0)].
%	alig openInWorld
%
%Les sous-morphs sont créés par la méthode suivante :
%
%cellMorphNamed: aString 
%	| m |
%	m := Morph new color: Color lightBlue;
%				 borderColor: Color black;
%				 borderWidth: 1;
%				 extent: 20 @ 20;
%				 yourself.
%	m
%		addMorphCentered: (StringMorph contents: aString font: Preferences windowTitleFont).
%	^ m
%
%Le résultat ci-contre montre que la largeur du morph orange contenant est inférieure à celle des morphs contenus ce qui n'est généralement pas souhaité. De plus, on ne peux pas facilement changer les positions des morphs contenus ou changer leur taille sans intervenir dans le code. La réactivité du morph est très limité et son amélioration nécessite beaucoup de codage lourd et non maintenable. Par exemple, pensez au modifications nécessaires pour que les morphs contenus soient tous séparés par un espace élastique qui augmente ou diminue suivant la taille du contenant orange. Ou encore, si on veut que la taille du contenant soient toujours en cohérence vis à vis des morphs contenus.



\subsection{With the default layout into a window}

This section  explains how to add morph into a window with the default layout.
\apl{Se serait mieux d'utiliser un parent Morph plutot qu'un Window}
The default layout is ProportionalLayout. To use it, you have to use the method \mthind{addMorph:frame:}{addMorph: aMorph frame: aFrame}
\begin{code}{}
aWindow
	addMorph: morphToAdd
	frame: (x0@y0 corner: x1@y1)
\end{code}
where x0, y0, x1 and y1 are defined as shown in  Figure~\ref{fig:frameExplanation}. Note that their values are floats between 0 and 1.

\begin{figure}[ht]\centering
	\includegraphics[width=6cm]{DefaultFrame}\includegraphics[width=6cm]{FrameExplanation}
	%\caption{How frames are defined}
	%\label{fig:defaultFrame}
	\caption{An example with a frame \ct{0@0 corner: 0.5@0.5}.}
	\label{fig:frameExplanation}
\end{figure}

\subsubsection{Some Morphs}
First, let's define some morphs to illustrate and experiment with.
\begin{code}{Objects definition}
| container redMorph blueMorph greenMorph |
redMorph := Morph new color: Color red; yourself.				
blueMorph := Morph new color: Color blue; yourself.
greenMorph := Morph new  color: Color green; yourself.
container := PanelMorph new.
\end{code}

We will not repeat their definition in the future except in the first code snippet so that you can 
get its full definition. 

\paragraph{A first configuration.}
The following snippet of code asks the red morph to occupy all the space of its container.
\begin{code}{}
| window |
window := SystemWindow new.
redMorph := Morph new color: Color red; yourself.	
window
	addMorph: redMorph
	frame: (0@0 corner: 1@1).
	
redMorph color: Color red.	
window openInWorld
\end{code}
Note: you have to reset the color of the morph after having added it because the window set the default color. Here it seems strange but for more complicated morphs (like buttons, list \dots) it sets the background color for a better integration.

\begin{figure}[ht]\centering
	\includegraphics[width=6cm]{SimpleLayoutExample1}
	\caption{The red morph fill the whole space.}
	\label{fig:simpleLayoutExample1}
\end{figure}

As a result, you can see that the red morph is stretched the fill the space both vertically and horizontally (see Figure~\ref{fig:simpleLayoutExample1}).


\paragraph{A little more complicated configuration.}
Now we add three morphs of different colors. 

\begin{code}{}
window
	addMorph: redMorph
	frame: (0@0 corner: 0.33@1).
window
	addMorph: blueMorph
	frame: (0.33@0 corner: 0.66@1).
window
	addMorph: greenMorph
	frame: (0.66@0 corner: 1@1).
\end{code}

As a result, you can see three stripes of color where each is horizontally a third of the window size and fills the space vertically (see Figure~\ref{fig:simpleLayoutExample2}).

\begin{figure}[ht]\centering
	\includegraphics[width=6cm]{SimpleLayoutExample2}
	\caption{Three color stripes}
	\label{fig:simpleLayoutExample2}
\end{figure}

Note that if you resize the window, proportions are kept. Also note that each stripe can be resized horizontally.

\paragraph{A last example.}
Now we change the configuration of 
\begin{code}{}
window
	addMorph: redMorph
	frame: (0@0 corner: 0.5@0.5).

window
	addMorph: blueMorph
	frame: (0.5@0 corner: 1@0.5).

window
	addMorph: greenMorph
	frame: (0@0.5 corner: 1@1).
\end{code}

As you may guess, the result is composed by two squares above a green rectangle (see Figure~\ref{fig:simpleLayoutExample3}).

\begin{figure}[ht]\centering
	\includegraphics[width=6cm]{SimpleLayoutExample3}
	\caption{Two squares above and a rectangle below}
	\label{fig:simpleLayoutExample3}
\end{figure}

Note that like for the previous example, you can resize each part. So basically, you now know everything about the default layout.


%----------------------------------------------------------------------------------------------
%
%				More Complicated Layouts
%
%----------------------------------------------------------------------------------------------


\subsection{More complicated layouts}

This section explains how to use different layouts and to use them to add morphs into another morph, which is a window or not.

Now that we have seen the default layout, we introduce you quickly the other layouts:
\begin{itemize}
	\item LayoutFrame
	\item RowLayout
	\item StackLayout
	\item TableLayout
\end{itemize}

\subsubsection{LayoutFrame}

This layout is used when you have to specified both a fix part \footnote{independent of the size of the window} and a proportional part\footnote{like the ProportionalLayout}.

This layout is used by example to add a toolbar.

To use this layout, you will have to use the method \mthind{addMorph:fullFrame:}{addMorph: aMorph fullFrame: aLayout} as follows:

\begin{code}{}
toolBarHeight := 100.
window
	addMorph: redMorph
	fullFrame: (LayoutFrame
				fractions: (0@0 corner: 1@0) "proportional part"
				offsets: (0@0 corner: 0@toolBarHeight)). "fix part"
				
				"Here, fractions: (0@0 corner: 1@0) means that the morph will fit the while width of the morph, but that the height is not dynamic (both y values are 0)"
				"and offsets: (0@0 corner: 0@toolBarHeight) means that the height is static and values toolBarHeight"
				
window
	addMorph: blueMorph
	fullFrame: (LayoutFrame
				fractions: (0@0 corner: 1@1) "proportional part"
				offsets: (0@toolBarHeight corner: 0@0)). "fix part"
				
	"Here fractions: (0@0 corner: 1@1) means that the moprh will	fit the whole container"
	"but offsets: (0@toolBarHeight corner: 0@0) precise that the top of the morph will always be at toolBarHeight from the top of the container"
\end{code}

As a result, you can see the red static part and the blue part which fill the space. When you resize the window, the red part will always stay the same (see Figure~\ref{fig:layoutFrame}).

\begin{figure}[ht]\centering
	\includegraphics[width=7cm]{LayoutFrame}
	\caption{Fix red part and dynamic blue part}
	\label{fig:layoutFrame}
\end{figure}

Here is another example showing the fixed sized bar at the bottom of the window.
\begin{code}{}
toolBarHeight := 100.
window
	addMorph: redMorph
	fullFrame: (LayoutFrame
				fractions: (0@0 corner: 1@1) "proportional part"
				offsets: (0@0 corner: 0@(toolBarHeight negated))). "fix part"
window
	addMorph: blueMorph
	fullFrame: (LayoutFrame
				fractions: (0@1 corner: 1@1) "proportional part"
				offsets: (0@(toolBarHeight negated) corner: 0@0)). "fix part"
\end{code}

Here you can see that the fix part (the blue one) is below (see Figure~\ref{fig:layoutFrame2}).

\begin{figure}[ht]\centering
	\includegraphics[width=7cm]{LayoutFrame2}
	\caption{Fix blue part and red blue part}
	\label{fig:layoutFrame2}
\end{figure}

Let's try a bit more complicated example:
\begin{code}{}
toolBarHeight := 100.
window
	addMorph: redMorph
	fullFrame: (LayoutFrame
				fractions: (0@0 corner: 1@0.4) "proportional part"
				offsets: (0@0 corner: 0@0)). "fix part"
window
	addMorph: greenMorph
	fullFrame: (LayoutFrame
				fractions: (0@0.4 corner: 1@0.4)
				offsets: (0@0 corner: 0@toolBarHeight)).				
window
	addMorph: blueMorph
	fullFrame: (LayoutFrame
				fractions: (0@0.4 corner: 1@1) "proportional part"
				offsets: (0@toolBarHeight corner: 0@0)). "fix part"
window openInWorld.
\end{code}

So now, the fix part is the green morph which stick in the middle of the window (see Figure~\ref{fig:layoutFrame3}).

\begin{figure}[ht]\centering
	\includegraphics[width=7cm]{LayoutFrame3}
	\caption{Dynamic red and blue parts and green fix part}
	\label{fig:layoutFrame3}
\end{figure}

\begin{figure}[!ht]\centering
	\includegraphics[width=6cm]{LayoutFrame4}
	\caption{Dynamic red and blue parts and green fix part}
	\label{fig:layoutFrame4}
\end{figure}

Of course, you can do the same vertically:
\begin{code}{}
toolBarWidth := 50.
window
	addMorph: redMorph
	fullFrame: (LayoutFrame
				fractions: (0@0 corner: 0.4@1) "proportional part"
				offsets: (0@0 corner: 0@0)). "fix part"				
window
	addMorph: greenMorph
	fullFrame: (LayoutFrame
				fractions: (0.4@0 corner: 0.4@1)
				offsets: (0@0 corner: toolBarWidth@0)).				
window
	addMorph: blueMorph
	fullFrame: (LayoutFrame
				fractions: (0.4@0 corner: 1@1) "proportional part"
				offsets: (toolBarWidth@0 corner: 0@0)). "fix part"
\end{code}

So the result is the same but vertically (see Figure~\ref{fig:layoutFrame4}).



\subsection{RowLayout}
The row layout is used to add submorphs in a single row where each submorph will be equally dispatched.

So let's try to use it. The message \ct{layoutPolicy:} specifies the new layout.
\begin{code}{}
window layoutPolicy: RowLayout new.

window 
	addMorph: redMorph;
	addMorph: blueMorph;
	addMorph: greenMorph.

window openInWorld.
\end{code}

As you can see, the window title bar is misplaced because it is added by the system using the same method once the layout has been changed (see Figure~\ref{fig:rowLayout1}).

\begin{figure}[ht]\centering
	\includegraphics[width=7cm]{RowLayout1}
	\caption{The title bar is misplaced}
	\label{fig:rowLayout1}
\end{figure}

To fix that, we will use a container instance of \ct{PanelMorph}.
\begin{code}{}
container := PanelMorph new.
container layoutPolicy: RowLayout new.
container 
	addMorph: redMorph;
	addMorph: blueMorph;
	addMorph: greenMorph.
window
	addMorph: container
	frame: (0@0 corner: 1@1).
window openInWorld.
\end{code}

As you can see, it's a bit better, but the morphs do not fill the whole space (see FIG~\ref{fig:rowLayout2}).

\begin{figure}[ht]\centering
	\includegraphics[width=7cm]{RowLayout2}
	\caption{The title bar is well placed, but morphs do not fill the whole space}
	\label{fig:rowLayout2}
\end{figure}

So let's fix that using \ct{hResizing:} and \ct{vResizing:}. 

\begin{code}{}
container := PanelMorph new.
container layoutPolicy: RowLayout new.
redMorph
	hResizing: #spaceFill;
	vResizing: #spaceFill.
blueMorph
	hResizing: #spaceFill;
	vResizing: #spaceFill.
greenMorph
	hResizing: #spaceFill;
	vResizing: #spaceFill.
container 
	addMorph: redMorph;
	addMorph: blueMorph;
	addMorph: greenMorph.
window
	addMorph: container
	frame: (0@0 corner: 1@1).
window openInWorld.
\end{code}

So for each submorph we have specified that both vertically and horizontally it should fill the space.
The result is what we have expected (see Figure~\ref{fig:rowLayout3}).

\begin{figure}[ht]\centering
	\includegraphics[width=7cm]{RowLayout3}
	\caption{Finally what we expected}
	\label{fig:rowLayout3}
\end{figure}

Note that contrary to the ProportionalLayout, here you can't resize any part.

\subsection{StackLayout}

\ben{I do not know how it works ... Should ask Igor}
\sd{or  may be we should drop it. Ask igor}

\subsection{TableLayout}
This layout is used to align submorphs following a row or a column and taking in account a direction.
So by default, the layout builds a column directed from top to bottom.
\begin{code}{}
container := PanelMorph new.
container 
	layoutPolicy: TableLayout new;
	listDirection: #topToBottom.
{ redMorph. blueMorph. greenMorph } do: [:each |
	each 
		hResizing: #spaceFill;
		vResizing: #spaceFill ].
container 
	addMorph: redMorph;
	addMorph: blueMorph;
	addMorph: greenMorph.
window
	addMorph: container
	frame: (0@0 corner: 1@1).
window openInWorld.
\end{code}

So we obtain a column where any morph is dispatched equally (see Figure~\ref{fig:tableLayout1}).

\begin{figure}[ht]\centering
	\includegraphics[width=7cm]{TableLayout1}
	\caption{TableLayout by default}
	\label{fig:tableLayout1}
\end{figure}

Note that you can't resize any submorph and that you have add in order:
red, blue and green and you get 
green, blue and  red.

The explanation is that when you add a morph following the direction, it's done like in Tetris, you make them fall following the direction. But the direction is kept when there is space to keep.


The following example show a situation  where we do not force each morph to expand vertically
\begin{code}{}
container := PanelMorph new.
container 
	layoutPolicy: TableLayout new;
	listDirection: #topToBottom.
{ redMorph. blueMorph. greenMorph } do: [:each |
	each hResizing: #spaceFill ].
container 
	addMorph: redMorph;
	addMorph: blueMorph;
	addMorph: greenMorph.
window
	addMorph: container
	frame: (0@0 corner: 1@1).
window openInWorld.
\end{code}

Here you see that the space is kept below submorphs (see Figure~\ref{fig:tableLayout2}).

\begin{figure}[ht]\centering
	\includegraphics[width=7cm]{TableLayout2}
	\caption{The space is left below submorphs}
	\label{fig:tableLayout2}
\end{figure}

You can also experiment the other directions

\begin{figure}[ht]\centering
	\includegraphics[width=4cm]{TableLayout3}
	\caption{Bottom To Top}
	\label{fig:tableLayout3}
	\includegraphics[width=4cm]{TableLayout4}
	\caption{Left To Righ}
	\label{fig:tableLayout4}
	\includegraphics[width=4cm]{TableLayout5}
	\caption{Right To Left}
	\label{fig:tableLayout5}
\end{figure}


TableLayout is also good to get a FlowLayout using its \ct{wrapDirection}.
By flow layout we mean  which works the same way usually text is handled:
All submorphs are put on a single line and wrapped when the border is hit.

\begin{code}{}
|builder|
builder := UITheme builder.
(builder newRow: {
   Morph new color: Color red.
   Morph new color: Color yellow.
   Morph new color: Color green.
   Morph new color: Color blue.
   Morph new color: Color orange})
      cellInset: 10;
      wrapDirection: #topToBottom;
      openInWindow
\end{code}

\begin{figure}[ht]\centering
	\includegraphics[width=4cm]{FlowLayout}
	\caption{TableLayout can be used to implement FlowLaout To Left}
	\label{fig:tableLayout5}
\end{figure}


\newpage
\section{Some Examples}


I would like to create a Text Morph that wraps the text horizontally, and expands the height to fit the text. Thus, this morph would never offer scrolling. Figure~\ref{fig:paneMorph}.


\begin{code}{}
| textMorph |
textMorph := UITheme builder newText: ''.
textMorph
    hResizing: #spaceFill;
    borderWidth: 1.
(UITheme builder newColumn: {textMorph}) openInWindow.
textMorph contentsWrapped: 'Some text here

Get a halo and inspect the text morph
then use #contentsWrapped: to change text'
\end{code}

\begin{figure}[ht]\centering
	\includegraphics[width=7cm]{paneMorph}
	\caption{PaneMorph}
	\label{fig:paneMorph}
\end{figure}


If I place this one in a surrounding pane / expander / tab, how do I get the surrounding morph to resize when the text changes? See Figure~\ref{fig:expanders}.

\begin{code}{}
|textMorph|
textMorph := UITheme builder newText: ''.
textMorph
    hResizing: #spaceFill;
    borderWidth: 1.
(UITheme builder newColumn: {
    UITheme builder newExpander: 'One' for: textMorph.
    UITheme builder newExpander: 'Two' for: Morph new}) openInWindow.

textMorph contentsWrapped: 'Some text here
use #contentsWrapped: to change text'
\end{code}

\begin{figure}[ht]\centering
	\includegraphics[width=5cm]{expanded}\includegraphics[width=5cm]{expanded2}
	\caption{Two Expanders }
	\label{fig:expanders}
\end{figure}


In general, if the surrounding morph has \ct{#shrinkWrap} constraints then changes to the (minimum) dimensions of its submorphs will propagate resulting in a change of size for the surrounding morph.



\section{Conclusion}

%=========================================================
\ifx\wholebook\relax\else
   \bibliographystyle{jurabib}
   \nobibliography{scg}
   \end{document}
\fi

%=================================================================
\ifx\wholebook\relax\else\end{document}\fi
%=================================================================

%-----------------------------------------------------------------

%%% Local Variables:
%%% coding: utf-8
%%% mode: latex
%%% TeX-master: t
%%% TeX-PDF-mode: t
%%% ispell-local-dictionary: "english"
%%% End: