% $Author: ducasse $
% $Date: 2009-08-24 10:17:33 +0200 (Mon, 24 Aug 2009) $
% $Revision: 28563 $

% HISTORY:


%=================================================================
\ifx\wholebook\relax\else
% --------------------------------------------
% Lulu:
	\documentclass[a4paper,10pt,twoside]{book}
	\usepackage[
		papersize={6.13in,9.21in},
		hmargin={.75in,.75in},
		vmargin={.75in,1in},
		ignoreheadfoot
	]{geometry}
	\input{../common.tex}
	\setboolean{lulu}{true}
% --------------------------------------------
% A4:
%	\documentclass[a4paper,11pt,twoside]{book}
%	\input{../common.tex}
%	\usepackage{a4wide}
% --------------------------------------------
    \graphicspath{{figures/} {../figures/}}
	\begin{document}
\fi
%=================================================================
%\renewmessage{\nnbb}[2]{} % Disable editorial comments
\sloppy
%=================================================================

\chapter{Basic Widgets}

\section{The window}
\subsection{Opening a window}
The basic class for managing a window is \ct{StandardWindow}. Let's start with an empty one. To create and open an empty window just try the following:
\begin{code}{}
StandardWindow new openInWorld
\end{code}
You should see a window with a topbar that you can move with the mouse. 

On one side, the topbar has three buttons for closing, collapsing and expanding the window. On the other side, the topbar has a menu button with a default set of items.

\subsection{A window and its model}

\subsection{Your own topbar menu}

\subsection{Menubar}

\subsection{Taskbar}

\section{Buttons}

\section{Text fields}

\section{Text editor}

\section{Panes and layout managing}

\section{List widgets}

\section{Tree widgets}

\section{Conclusion}

%=========================================================
\ifx\wholebook\relax\else
   \bibliographystyle{jurabib}
   \nobibliography{scg}
   \end{document}
\fi

%=================================================================
\ifx\wholebook\relax\else\end{document}\fi
%=================================================================

%-----------------------------------------------------------------

%%% Local Variables:
%%% coding: utf-8
%%% mode: latex
%%% TeX-master: t
%%% TeX-PDF-mode: t
%%% ispell-local-dictionary: "english"
%%% End: