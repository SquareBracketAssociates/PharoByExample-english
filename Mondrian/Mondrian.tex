% $Author$
% $Date$
% $Revision$

% HISTORY:
% Chapter started by Damien C (2009-09-02)

%=================================================================
\ifx\wholebook\relax\else
% --------------------------------------------
% Lulu:
	\documentclass[a4paper,10pt,twoside]{book}
	\usepackage[
		papersize={6.13in,9.21in},
		hmargin={.75in,.75in},
		vmargin={.75in,1in},
		ignoreheadfoot
	]{geometry}
	\input{../common.tex}
	\setboolean{lulu}{true}
% --------------------------------------------
% A4:
%	\documentclass[a4paper,11pt,twoside]{book}
%	\input{../common.tex}
%	\usepackage{a4wide}
% --------------------------------------------
    \graphicspath{{figures/} {../figures/}}
	\begin{document}
\fi
%=================================================================
%\renewcommand{\nnbb}[2]{} % Disable editorial comments
\sloppy
%=================================================================
\chapter{Mondrian}
\chalabel{mondrian}


Mondrian is an information visualization engine that lets the
visualization be specified via a script.  It is based on a graph model
and works directly with the objects to be represented.

Mondrian received the second place at the ESUG 2006 Innovation Award.

The original version has been primarily written by Michael Meyer as
part of his Masters Thesis. Currently, Mondrian is developed by Tudor
Girba, Alexandre Bergel and Simon Denier.

\section{Installation and first visualization}

To install Mondrian on your image, you have to execute the following
snippet:

\begin{code}{}
ScriptLoader loadLatestPackage: 'MondrianLoader' fromSqueaksource: 'Mondrian'.

MondrianLoader load
\end{code}

Lets get started by visualizing the Smalltalk Collection
hierarchy. First, you need to retreive

\begin{code}{}
| view classes |
classes := Collection withAllSubclasses.
view := MOViewRenderer new.
view nodes: classes.                  "Creates a node for each class in the Collection hierarchy"
view edges: classes            
     from: [:each | each superclass] 
     to: [:each | each].              "Creates an edge for each class from the superclass to itself"
view treeLayout.                      "Arranges the view in a tree"
view open.
\end{code}

\section{Tutorial}

\section{Mondrian in greater details}



%=============================================================
\ifx\wholebook\relax\else
   \bibliographystyle{jurabib}
   \nobibliography{scg}
   \end{document}
\fi
%=============================================================




%-----------------------------------------------------------------

%%% Local Variables:
%%% coding: utf-8
%%% mode: latex
%%% TeX-master: t
%%% TeX-PDF-mode: t
%%% ispell-local-dictionary: "english"
%%% End: