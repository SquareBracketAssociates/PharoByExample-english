% $Author: ducasse $
% $Date: 2009-08-24 10:17:33 +0200 (Mon, 24 Aug 2009) $
% $Revision: 28563 $

% HISTORY:


%=================================================================
\ifx\wholebook\relax\else
% --------------------------------------------
% Lulu:
	\documentclass[a4paper,10pt,twoside]{book}
	\usepackage[
		papersize={6.13in,9.21in},
		hmargin={.75in,.75in},
		vmargin={.75in,1in},
		ignoreheadfoot
	]{geometry}
	\input{../common.tex}
	\setboolean{lulu}{true}
% --------------------------------------------
% A4:
%	\documentclass[a4paper,11pt,twoside]{book}
%	\input{../common.tex}
%	\usepackage{a4wide}
% --------------------------------------------
    \graphicspath{{figures/} {../figures/}}
	\begin{document}
\fi
%=================================================================
%\renewmessage{\nnbb}[2]{} % Disable editorial comments
\sloppy
%=================================================================

\chapter{Little Numbers}




Let's be dummy for a while.

\section{Power of 2}

Let's start with some simple maths. 

\begin{code}{}
2 raisedTo: 0
	returns 1

2 raisedTo: 2
	returns 4
	
2 raisedTo: 8
	returns 256
\end{code}

\begin{figure}[h]
\begin{center}
\includegraphics[width=0.65\textwidth]{16bits-number}
\caption{Power of 2 and their numerical equivalence.}
\end{center}
\end{figure}

How to encode 13? It cannot be higher than $2^{4}$ because $2^{4}$ = 16. So it should be 8 + 4 + 1, $2^{3} + 2^{2} + 2^{0}$.

\begin{figure}[h]
\begin{center}
\includegraphics[width=0.65\textwidth]{16bits-number13}
\caption{13 = $2^{3} + 2^{2} + 2^{0}$.}
\end{center}
\end{figure}


\begin{code}{}
(2 raisedTo: 0)
	returns 1

2 raisedTo: 2
	returns 4
	
2 raisedTo: 8
	returns 256
\end{code}

\subsection*{Binary notation}
\begin{code}{}
2r01101
	returns 13

13 printStringBase: 2
	returns '01101'
	
Integer readFrom: '01101' base: 2 	
	returns 13
\end{code}



\section{Bit shifting is multiplying by 2}

\begin{code}{Using \ct{>>} and \ct{<<}}
2r000001000 
	returns 8

2r000001000 >> 1
	returns 4
	
2r000001000 << 1
	returns 16
\end{code}

\begin{figure}[h]
\begin{center}
\includegraphics[width=0.65\textwidth]{16bits-numberMultiplication}
\caption{Multiply by 2 or dividing by 2.}
\end{center}
\end{figure}

\begin{code}{Using \ct{bitShift:}}
2r000001000 
	returns 8

2r000001000 bitShift: -1
	returns 4
	
2r000001000 bitShift: 1
	returns 16
\end{code}

\begin{code}{}
2r000001100 
	returns 12

2r000001100 bitShift: -1
	returns 2r00000110
	returns 6 
	
2r000001100 bitShift: 1
	returns 2r000011000
	returns 24
\end{code}

\begin{figure}[h]
\begin{center}
\includegraphics[width=0.65\textwidth]{16bits-numberMultiplication2}
\caption{Mutiply by 2 or dividing by 2. We can move several bits in one direction.}
\end{center}
\end{figure}

\subsection*{Shifting bits}


\begin{code}{}
2 raisedTo: 8
	returns 256
	
1 << 8
	returns 256
	
2r100000000
	returns 256
		
\end{code}

\begin{figure}[h]
\begin{center}
\includegraphics[width=0.65\textwidth]{16bits-1shifted8}
\caption{We move 8 times 1 to the left. So from $2^{0}$ we get $2^{8}$.}
\end{center}
\end{figure}

\begin{code}{}
(2 raisedTo: 8) + (2 raisedTo: 10) 
	returns 1280
	
2r010100000000
	returns 1280
	
2r010100000000 >> 8
	returns	2r0101
	returns 5
\end{code}

\begin{figure}[h]
\begin{center}
\includegraphics[width=0.65\textwidth]{16bits-1280shifted8}
\caption{We move 8 times to the right. So from 1280 gets 5.}
\end{center}
\end{figure}







\section{Bit Access}

\begin{code}{}
2r000001101 bitAt: 0
	returns 0 
	
2r000001101 bitAt: 1
	returns 1 
	
2r000001101 bitAt: 2
	returns 0 
			
2r000001101 bitAt: 3
	returns 1
	 
2r000001101 bitAt: 5
	returns 0 	 
\end{code}


\begin{code}{}
Integer>>bitAt: anInteger
	"Answer 1 if the bit at position anInteger is set to 1, 0 otherwise.
	self is considered an infinite sequence of bits, so anInteger can be any strictly positive integer.
	Bit at position 1 is the least significant bit.
	Negative numbers are in two-complements.
	
	This is a naive implementation that can be refined in subclass for speed"
	
	^(self bitShift: 1 - anInteger) bitAnd: 1
\end{code}

We shift to the right from an integer minus one (hence \ct{1 - anInteger})
and with a \ct{bitAnd:} we know whether there is a one or zero in the location.

%

\section{More here}

\section{2 complement}


\section{Hexadecimal}


\begin{code}
15 hex
	returns '16rF'
	
15 printStringHex 'F'

16rF
	returns 15
\end{code}


\begin{code}
(1 to: 15) collect: [:each | each -> each hex] 

{(1->'16r1'). (2->'16r2'). (3->'16r3'). (4->'16r4'). (5->'16r5'). (6->'16r6'). (7->'16r7'). (8->'16r8'). (9->'16r9'). (10->'16rA'). (11->'16rB'). (12->'16rC'). (13->'16rD'). (14->'16rE'). (15->'16rF')}
\end{code}


\begin{code}{}
16rF printStringBase: 2
	returns '1111'
\end{code}


\subsection*{Selecting a subpart}

\begin{code}{}
2r1100010100000000 
	returns 50432
\end{code}

Imagine that only 4 bits interest us starting at the 9th one.
We shift out the 8 first and clear all the others after the next 4.  

\begin{code}{}
(2r1100010100000000 >> 8)
	returns 2r11000101
	returns 197

(2r1100010100000000 >> 8) bitAnd: 16rF	
	returns 0101	
	returns 5
	
2r11000101 bitAnd: 1111
	returns 0101	
	returns 5
\end{code}


\section{Conclusion}

%=========================================================
\ifx\wholebook\relax\else
   \bibliographystyle{jurabib}
   \nobibliography{scg}
   \end{document}
\fi

%=================================================================
\ifx\wholebook\relax\else\end{document}\fi
%=================================================================

%-----------------------------------------------------------------

%%% Local Variables:
%%% coding: utf-8
%%% mode: latex
%%% TeX-master: t
%%% TeX-PDF-mode: t
%%% ispell-local-dictionary: "english"
%%% End: