
% $Author: oscar $
% $Date: 2009-01-13 10:57:05 +0100 (Tue, 13 Jan 2009) $
% $Revision: 23887 $


% HISTORY: [see also Metaprogramming2.tex]
% 2007-05-22 - Damien Pollet started new chapter (translation from french article)
% 2007-06-20 - translation complete
% 2008-01-09 - Alex Bergel added Methods as Objects
% 2008-12-13 - Oscar started editing
% 2009-05-29 - Oscar started editing again

%=================================================================
\ifx\wholebook\relax\else
% --------------------------------------------
% Lulu:
	\documentclass[a4paper,10pt,twoside]{book}
	\usepackage[
		papersize={6.13in,9.21in},
		hmargin={.75in,.75in},
		vmargin={.75in,1in},
		ignoreheadfoot
	]{geometry}
	\input{../common.tex}
	\pagestyle{headings}
	\setboolean{lulu}{true}
% --------------------------------------------
% A4:
%	\documentclass[a4paper,11pt,twoside]{book}
%	\input{../common.tex}
%	\usepackage{a4wide}
% --------------------------------------------
    \graphicspath{{figures/} {../figures/}}
	\begin{document}
	% \renewcommand{\nnbb}[2]{} % Disable editorial comments
	\sloppy
\fi
%=================================================================
\chapter{Dynamic metaprogramming}\chalabel{metaprog}

Here I would like to have a more dynamic reflective chapter.
May be we should move some code from the other chapters.

	
\begin{itemize}
\item Creating new little kernel


\item DynamicDeadCodeFinder 

\item Object as Wrapper

\item ObjectOut aka disk wrapper

\item Actalk or may be in another chapter

\item replay object from adrian
\end{itemize}


\section{ObjectTracer and friends}

ObjectViewers offers the same kind of interception of messages (via doesnotUnderstand:) as ObjectTracers, but instead of just being wrappers, they actually replace the object being viewed.  This makes them a lot more dangerous to use, but one can do amazing things.  For instance, the example below actually intercepts the InputSensor object, and prints the mouse coordinates asynchronously, every time they change:
	Sensor evaluate: [Sensor cursorPoint printString displayAt: 0@0]
		wheneverChangeIn: [Sensor cursorPoint].
To exit from this example, execute:
	Sensor xxxUnTrace

\begin{code}{}
ObjectViewer class>>on: viewedObject evaluate: block1 wheneverChangeIn: block2
	^ self new xxxViewedObject: viewedObject evaluate: block1 wheneverChangeIn: block2

ObjectViewer>>xxxViewedObject: viewedObject evaluate: block1 wheneverChangeIn: block2
	"This message name must not clash with any other (natch)."
	tracedObject := viewedObject.
	valueBlock := block2.
	changeBlock := block1.
	recursionFlag := false

ObjectViewer>>doesNotUnderstand: aMessage 
	"Check for change after sending aMessage"
	| returnValue newValue |
	recursionFlag ifTrue: [^ aMessage sentTo: tracedObject].
	recursionFlag := true.
	returnValue := aMessage sentTo: tracedObject.
	newValue := valueBlock value.
	newValue = lastValue ifFalse:
		[changeBlock value.
		lastValue := newValue].
	recursionFlag := false.
	^ returnValue
\end{code}


An ObjectTracer can be wrapped around another object, and then give you a chance to inspect it whenever it receives messages from the outside.  For instance...
	(ObjectTracer on: Display) flash: (50@50 extent: 50@50)
will give control to a debugger just before the message flash is sent.
Obviously this facility can be embellished in many useful ways.
See also the even more perverse subclass, ObjectViewer, and its example.

\begin{code}{}
ObjectTracer>>doesNotUnderstand: aMessage 
	"All external messages (those not caused by the re-send) get trapped here"
	"Present a dubugger before proceeding to re-send the message"

	ToolSet debugContext: thisContext
				label: 'About to perform: ', aMessage selector
				contents: nil.
	^ aMessage sentTo: tracedObject.

ObjectTracer>>xxxUnTrace
	tracedObject become: self

ObjectTracer>>xxxViewedObject
	"This message name must not clash with any other (natch)."
	^ tracedObject

ObjectTracer>>xxxViewedObject: anObject
	"This message name must not clash with any other (natch)."
	tracedObject := anObject

ObjectTracer>>on: anObject
	^ self new xxxViewedObject: anObject
\end{code}






\section{Minimal Object class}

Pharo ProtoObject class is a minimal class in the sense that it only defines the bare minimum for an object to survive in the environement. This class is handy to build tools based on the handling of doesNotUnderstand:.
In fact it is sometimes nice that any messages sent to an object get trapped and send the doesNotUnderstand: message.
This way we can get a grip on the messages sent and do new things with them such as serializing obbjects, doing remote calls.

Now as a simple illustration we would like to have a way to know what is the methods defined in Object used in a given tasks. 
For that define the class MinimalObject

\begin{classdef}{}
ProtoObject subclass: #MinimalObject
	instanceVariableNames: ''
	classVariableNames: ''
	poolDictionaries: ''
	category: 'MiniActalk'
\end{classdef}

and define the method doesNotUnderstand: so that if a message is not understood, it is copied from the class \ct{Object}
to the MinimalObject one. 

\begin{code}{}
doesNotUnderstand: aMessage

	| sel category |
	sel := aMessage selector. 
	Transcript show: 'Does not understand ',  aMessage selector printString ; cr.
	category := (Object organization categoryOfElement: sel).
	self class compile: (Object sourceCodeAt: sel) classified: category.
	^ aMessage sentTo: self.
\end{code}


Notice the use of the message sentTo: which given a message can execute it on a given receiver. 

Now if you open a Transcript and execute the following instruction \ct{MinimalObject new inspect}
and interact with the inspector, you will see slowly the methods used for the interaction with the inspector
compiled into your class. 

Note that this technique does not really work for subclasses since a method may not be defined in a class
but in one of its superclass and we will not know that.
In a subsequent section, we will propose an extension of this code to deal with this problem but recompiling superclass method to raise
errors and get replaced during the recompilation.


\section{}


\section{Actalk}


An actor will be a class which treats concurrently the message it receives. So we will need to get the messages and queue them in a kind of mailbox. Therefore we will use the doesNotUnderstand: trick again. We will define in fact two classes: one that will be public actor object that will get the messages and treat them as asynchronously and another one that will represent the activity or behavior of the actor.
 We defined the class \ct{TalkActor} as follows:



\begin{classdef}{}
ProtoObject subclass: #TalkActor
	instanceVariableNames: 'mailBox behavior'
	classVariableNames: ''
	poolDictionaries: ''
	category: 'MiniActalk'
\end{classdef}

\begin{method}{}
TalkActor>>mailBox
	^ mailBox
	
TalkActor>>initialize
	mailBox := SharedQueue new
	
TalkActor>>initializeBehavior: aBehavior
	behavior := aBehavior.
	behavior initializeWithActor: self
\end{method}

\begin{method}{}
TalkActor>>asynchronousSend: aMessage
	mailBox nextPut: aMessage

TalkActor>>doesNotUnderstand: aMessage
 	self asynchronousSend: aMessage 
\end{method}

\begin{method}{}
TalkActor class>>behavior: aBehavior
	^ self new initializeBehavior: aBehavior
\end{method}
The class TalkActorBehavior is plain normal class. Its key behavior will be that an instance will spawned a process
whose job will be to wait for messages put in the mailbox and to execute them one by one when present. 
So the message send to an Actor will be redirected to an actor behavior and executed by this object. 

\begin{classdef}{}
Object subclass: #TalkActorBehavior
	instanceVariableNames: 'actor'
	classVariableNames: ''
	poolDictionaries: ''
	category: 'MiniActalk'
\end{classdef}

\begin{method}{}
TalkActorBehavior>>actor
	^ ATalkActor behavior: self
\end{method}

\begin{method}{}
TalkActorBehavior>>initializeWithActor: anActor

	actor := anActor.
	self setProcess
TalkActorBehavior>>setProcess
	"should have a way to kill all the actors storing the processes forked"
	
	[[true ] 
		whileTrue: [self acceptNextMessage]] fork
\end{method}


\begin{method}{}
TalkActorBehavior>>acceptNextMessage
	self acceptMessage: actor mailBox next

TalkActorBehavior>>acceptMessage: aMessage
	^ self perform: aMessage selector withArguments: aMessage arguments
\end{method}


To make sure that you can use the debugger and inspector on your actor we mix the approach 
presented before to copy down Object methods. 

\begin{method}{}
TalkActor>>doesNotUnderstand: aMessage

	| sel |
	sel := aMessage selector. 
	(Object respondsTo: sel)
		ifFalse: [ ^ self asynchronousSend: aMessage ] 
		ifTrue: [	
				Transcript show: 'Does not understand ',  aMessage selector printString ; cr.
				self class compile: (Object sourceCodeAt: sel) classified: 'copied down'.
				^ aMessage sentTo: self.]
\end{method}


\subsection{Counter Actor}
A first actor is defined by subclassing the behavior of an actor.

\begin{classdef}{}
TalkActorBehavior subclass: #TalkCounter
	instanceVariableNames: 'count'
	classVariableNames: ''
	poolDictionaries: ''
	category: 'MiniActalk'
\end{classdef}

\begin{method}{}
TalkCounter>>initialize
	count := 0.

TalkCounter>>increment
	count := count + 1.
	Transcript show: 'increment '; cr.
\end{method}
	
	
\begin{method}{}
TalkCounter class>>example1
	"self example1"

	| anActorCounter |
	anActorCounter := TalkCounter new actor.
	anActorCounter increment; increment.	
\end{method}



Note that here even if we would add method count that returns the count number
we would not get it by adding anActorCounter count since the message would be added into the queue
but we do not have a mean to get the value to us. We will fix that in a moment.

Now to check that the counter is really increased we can temporary violate the asynchronous
nature of our actor and add the method \ct{behavior}. Then using it and waiting enough we can check that the 
inner state of the actor changes as shown by the example2. 

\begin{method}{}
TalkActor>>behavior
	^ behavior
\end{method}

\begin{method}{}
TalkCounter class>>example2
	"self example2"
	
	| anActorCounter |
	anActorCounter := TalkCounter new actor.
	anActorCounter increment; increment.
	(Delay forSeconds: 2) wait.
	^ anActorCounter behavior count. 
\end{method}

\section{Adding a printing actor}
\begin{classdef}{}
TalkActorBehavior subclass: #TalkPrinter
	instanceVariableNames: ''
	classVariableNames: ''
	poolDictionaries: ''
	category: 'MiniActalk'
\end{classdef}


\begin{method}{}
TalkPrinter>>reply: value
	Transcript show: ' > ', value printString; cr  
\end{method}

\begin{method}{}
TalkCounter>>consultAndReplyTo: replyDestination
	replyDestination reply: count
\end{method}


\begin{code}{}
| p |
	p := TalkPrinter new actor. 
	Transcript open.
	TalkCounter new actor
		increment; increment; consultAndReplyTo: p.
\end{code}

This way we do not query with a synchronous call the value of the counter
but the actor execute a message on his mailbox which has as effect to 
send another asynchronous message to another actor that will print the result. 
	
\subsection{Adding future}


%=================================================================
\ifx\wholebook\relax\else\end{document}\fi
%=================================================================

%-----------------------------------------------------------------
\subsection{XXX}
