% $Author: ducasse $
% $Date: 2009-07-06 11:55:40 +0200 (Mon, 06 Jul 2009) $
% $Revision: 27889 $

% HISTORY:
% 2008-08-19 - Stef started chapter (outline only)
% 2011-09-11 - Migrated to PharoBox: svn checkout https://XXX@scm.gforge.inria.fr/svn/pharobooks/PharoByExampleTwo-Eng

%=================================================================
\ifx\wholebook\relax\else
% --------------------------------------------
% Lulu:
	\documentclass[a4paper,10pt,twoside]{book}
	\usepackage[
		papersize={6.13in,9.21in},
		hmargin={.75in,.75in},
		vmargin={.75in,1in},
		ignoreheadfoot
	]{geometry}
	\input{../common.tex}
	\setboolean{lulu}{true}
% --------------------------------------------
% A4:
%	\documentclass[a4paper,11pt,twoside]{book}
%	\input{../common.tex}
%	\usepackage{a4wide}
% --------------------------------------------
    \graphicspath{{figures/} {../figures/}}
	\begin{document}
\fi
%=================================================================
%\renewcommand{\nnbb}[2]{} % Disable editorial comments
\sloppy
%=================================================================
\chapter{Checking and Transforming Programs with Rewrite rules}\chalabel{rewrite}


If you really just want the string 'tabs', the string 'tabs' with the
quotes is the search expression.

If you want to find it as part of a substring use something along:
\begin{code}{}
   `#string `{ :node | node value isString and: [ node value
includesSubString: 'tabs' ] }

The `#string is a literal pattern (booleans, characters, arrays,
strings, numbers, ...) and `{ ... adds a constraint on the preceeding
match.
\end{code}



\begin{code}{}
| className realClass replacer category |

className := #MyClass.
realClass := Smalltalk at: className.
category := #accessing.

replacer := RBParseTreeRewriter new
				replace: '`receiver oldMessage' with: '`receiver newMessage';
				yourself.
(realClass organization listAtCategoryNamed: category)
	collect: [:sel |
		| parseTree |
		parseTree := ( realClass >> sel) parseTree.
		(replacer executeTree: parseTree)
			ifTrue: [ realClass compile: replacer tree newSource " [1] " ] ]

Now some questions:

1) In [1] "( RBClass existingNamed: className ) compileTree: replacer
tree" would be more appropiate? I get a MNU when evaluated with latest
AST/Refactoring.
2) The above script have some problems, if you want to get replaced

'this is a string' asString.

to

'this is a string' asText

it won't match. The same happens with: " self model asString " and
other patterns. 

answer>>>`#literal asString


How to modify the rewriter to match
oldMessage/newMessage at any point?

>>>``@obj oldMessage


i.e. [: tmp | anObject1 blabla1 blabla2 oldMessage blabla3 ]

Thanks in advance,
\end{code}


answer from lukas
\begin{code}{}
RBClass is a class private to the framework, you should never need to
instantiate it directly. And if you do, only through an instance of
RBNamespace that provides a delta to the current system state. In your
example you don't need to do that.

\end{code}

%=================================================================
\ifx\wholebook\relax\else\end{document}\fi
%=================================================================

%-----------------------------------------------------------------

%%% Local Variables:
%%% coding: utf-8
%%% mode: latex
%%% TeX-master: t
%%% TeX-PDF-mode: t
%%% ispell-local-dictionary: "english"
%%% End: