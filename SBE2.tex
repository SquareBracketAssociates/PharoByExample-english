% $Author$
% $Date$
% $Revision$
%=================================================================
% This is the main file for the Squeak By Example book.
% The individual chapters can also be latexed by themselves.
% Note that the \end{document} marker occurs near the
% middle of this file, to leave out additional material
% for a future version of the book.
%=================================================================
\documentclass[a4paper,10pt,twoside]{book}
\usepackage[
	papersize={6in,9in},
	hmargin={.75in,.75in},
	vmargin={.75in,1in},
	ignoreheadfoot
]{geometry}
\input{common.tex}
\setboolean{lulu}{true}
%=================================================================
% Add the path for the figures of each chapter here:
\graphicspath{
	{figures/}
	% {Announcements/figures/}
	{Compiler/figures/}
	% {Concurrency/figures/}
	% {Debugging/figures/}
	{Exceptions/figures/}
	{Installer/figures/}
	% {Magma/figures/}
	{Metaprogramming/figures/}
	{Monticello/figures/}
	{Omnibrowser/figures/}
	{Profiling/figures/}
	{Regex/figures/}
	{Seaside/figures/}
}
%=================================================================
\let\wholebook=\relax
\makeindex
\makeglossary
%=================================================================
% \renewcommand{\nnbb}[2]{} % Disable editorial comments
%=================================================================
\begin{document}
\frontmatter
%=================================================================
\setcounter{page}{1}
\pagestyle{headings}
%=================================================================
%:Inside cover
\author{
	Alexandre Bergel\quad
	Andrew P. Black\quad
	St\'ephane Ducasse\\[1ex]
	Oscar Nierstrasz\quad}
% (to be updated at the end)
\title{\Huge\bf More Squeak by Example\\[1ex]}
\isodate
\date{\emph{Version of \today}}
\maketitle
%=================================================================
%:TOC
\tableofcontents
% \listoffigures
% \listoftables
% \lstlistoflistings
\sloppy % To avoid LaTeX's annoying habit of letting lines stick over the margins!
\mainmatter
%=================================================================
% $Author$
% $Date$
% $Revision$

% HISTORY:
% 2006-10-05 - Oscar started
% 2007-05-28 - Stef edit
% 2007-06-06 - Oscar first draft
% 2007-08-14 - Stef corrections
% 2007-09-06 - Lukas review
% 2009-08-12 - Oscar rewrite for Pharo

%=================================================================
\ifx\wholebook\relax\else
% --------------------------------------------
% Lulu:
	\documentclass[a4paper,10pt,twoside]{book}
	\usepackage[
		papersize={6.13in,9.21in},
		hmargin={.75in,.75in},
		vmargin={.75in,1in},
		ignoreheadfoot
	]{geometry}
	\input{../common.tex}
	\pagestyle{headings}
	\setboolean{lulu}{true}
% --------------------------------------------
% A4:
%	\documentclass[a4paper,11pt,twoside]{book}
%	\input{../common.tex}
%	\usepackage{a4wide}
% --------------------------------------------
    \graphicspath{{figures/} {../figures/}}
	\begin{document}
	% \renewcommand{\nnbb}[2]{} % Disable editorial comments
	\sloppy
	\frontmatter
\fi
%=================================================================
\chapter{Preface}\chalabel{intro}

%=================================================================
\section*{What is \pharo?}

\pharo is a modern, open source, fully-featured implementation of the \st programming language and environment. \pharo is derived from \squeak\cite{Inga97a}, a re-implementation of the classic \st-80 system. Whereas \squeak was developed mainly as a platform for developing experimental educational software, \pharo strives to offer a lean, open-source platform for professional software development, and a robust and stable platform for research and development into dynamic languages and environments. \pharo serves as the reference implementation for the Seaside web development framework.

\pharo resolves some licensing issues with \squeak. Unlike previous versions of \squeak, the \pharo core contains only code that has been contributed under the MIT license. The \pharo project started in March 2008 as a fork of \squeak 3.9, and the first 1.0 beta version was released on July 31, 2009.

Although \pharo removes many packages from \squeak, it also includes numerous features that are optional in \squeak. For example, true type fonts are bundled into \pharo. \pharo also includes support for true block closures. The user interfaces has been simplified and revised.

\pharo is highly portable --- even its virtual machine is written entirely in \st, making it easy to debug, analyze, and change. \pharo is the vehicle for a wide range of innovative projects from multimedia applications and educational platforms to commercial web development environments. 

There is an important aspect behind \pharo: \pharo wants to make sure that it is not a copy of the past but really a reinvention of Smalltalk. Now big-bang approaches rarely succeed. \pharo will really favor evolutionary and incremental changes. We want to be able to experiment with important new features or libraries. Evolution means that \pharo accepts mistakes and is not not aiming for the next perfect solution in one big step -- even if we would love it. \pharo will favor small incremental changes but a multitude of them. The \pharo community will pay attention to your submissions to improve the system.

%=================================================================
\section*{Who should read this book?}

This book is based on \emph{Squeak by Example}\footnote{\sbe}, an open-source introduction to \squeak.
The book has been liberally adapted and revised to reflect the differences between \pharo and \squeak.
This book presents the various aspects of \pharo, starting with the basics, and proceeding to more advanced topics.

This book will not teach you how to program. The reader should have some familiarity with programming languages. Some background with object-oriented programming would be helpful.

This book will introduce the \pharo programming environment, the language and the associated tools.  You will be exposed to common idioms and practices, but the focus is on the technology, not on object-oriented design. Wherever possible, we will show you lots of examples. (We have been inspired by Alec Sharp's excellent book on Smalltalk\cite{Shar97a}.)
\index{Sharp, Alex}

There are numerous other books on \st freely available on the web but none of these focuses specifically on \pharo. See for example:
\url{http://stephane.ducasse.free.fr/FreeBooks.html}

\ifluluelse{}{\newpage} % layout hint
%=================================================================
\section*{A word of advice}

% http://www.surfscranton.com/architecture/KnightsPrinciples.htm

Do not be frustrated by parts of \st that you do not immediately understand.
You do not have to know everything!
Alan Knight expresses this principle as follows\footnote{\url{http://www.surfscranton.com/architecture/KnightsPrinciples.htm}}:
\index{Knight, Alan}
\important{{\bf Try not to care.}
Beginning \st programmers often have trouble because they think they need to understand all the details of how a thing works before they can use it. This means it takes quite a while before they can master \ct{Transcript show: 'Hello World'}. One of the great leaps in OO is to be able to answer the question ``How does this work?'' with ``I don't care''.}

%=================================================================
\section*{An open book}

This book is an open book in the following senses: 

\begin{itemize}

\item	The content of this book is released under the Creative Commons Attribution-ShareAlike (by-sa) license.
		In short, you are allowed to freely share and adapt this book, as long as you respect the conditions of the license available at the following URL: 
		\url{http://creativecommons.org/licenses/by-sa/3.0/}.

\item	This book just describes the core of \pharo.
		Ideally we would like to encourage others to contribute chapters
		on the parts of \pharo that we have not described.
		If you would like to participate in this effort, please
		contact us.  We would like to see this book grow!
\end{itemize}

For more details, visit \pbe.

%=================================================================
\section*{The \pharo community}

The \pharo community is friendly and active.
Here is a short list of resources that you may find useful:

\begin{itemize}
\item \url{http://www.pharo-project.org} is the main web site of \pharo.
environment built on top of \pharo but whose audience is elementary school teachers.)

\item \url{http://www.squeaksource.com} is the equivalent of SourceForge for \pharo projects.
Many optional packages for \pharo live here.
\end{itemize}

%=================================================================
\section*{Examples and exercises}

We make use of two special conventions in this book.

We have tried to provide as many examples as possible.
In particular, there are many examples that show a fragment of code which can be evaluated.  We use the symbol \ct{-->} to indicate the result that you obtain when you select an expression and \menu{print it}:

\begin{code}{@TEST}
3 + 4 --> 7    "if you select 3+4 and 'print it', you will see 7"
\end{code}

In case you want to play in \pharo with these code snippets, you can download a plain text file with all the example code from the book's web site: \pbe.

The second convention that we use is to display the icon \dothisicon{} to indicate when there is something for you to do:

\dothis{Go ahead and read the next chapter!}

%=================================================================
\section*{Acknowledgments}

We would first like to thank the original developers of \squeak for making this amazing \st development environment available as an open source project.

% We would like to thank various people who have contributed to this book.
% In particular, we thank
We would also like to thank Hilaire Fernandes and Serge Stinckwich who allowed us to translate parts of their columns on \st, and Damien Cassou for contributing the chapter on streams.

We especially thank Lukas Renggli and Orla Greevy for their comments on drafts of the first release.

We thank the University of Bern, Switzerland, for graciously supporting this open-source project and for hosting the web site of this book.

We also thank the Squeak community for their enthusiastic support of this project, and for informing us of the errors found in the first edition of this book.

%=============================================================
\ifx\wholebook\relax\else
   \bibliographystyle{jurabib}
   \nobibliography{scg}
   \end{document}
\fi
%=============================================================

%=================================================================
%:PLANNED for 2d edition
\part{SW Development}
% $Author $
% $Date$
% $Revision$

% HISTORY:
% 2007-10-29 - Alex first draft
% 2007-12-18 - Andrew review pass
% 2008-05-30 - Stef review pass
% 2009-04-22 - Oscar revised
% 2009-04-23 - Alex revised (noted Squeak vs Pharo)
% 2009-06-17 - Oscar migrated to Pharo
% 2009-07-15 - Oscar indexing
% 2011-09-11 - Migrated to PharoBox: svn checkout https://XXX@scm.gforge.inria.fr/svn/pharobooks/PharoByExampleTwo-Eng



%=================================================================
\ifx\wholebook\relax\else
% --------------------------------------------
% Lulu:
	\documentclass[a4paper,10pt,twoside]{book}
	\usepackage[
		papersize={6.13in,9.21in},
		hmargin={.75in,.75in},
		vmargin={.75in,1in},
		ignoreheadfoot
	]{geometry}
	\input{../common.tex}
	\pagestyle{headings}
	\setboolean{lulu}{true}
% --------------------------------------------
% A4:
%	\documentclass[a4paper,11pt,twoside]{book}
%	\input{../common.tex}
%	\usepackage{a4wide}
% --------------------------------------------
    \graphicspath{{figures/} {../figures/}}
	\begin{document}
	\renewcommand{\nnbb}[2]{} % Disable editorial comments
	\sloppy
	
\fi

	\newcommand{\Mont}{Monticello\xspace}
	\newcommand{\MCB}{\Mont browser\xspace}
	\newcommand{\RI}{repository inspector\xspace}

%=================================================================
\chapter{Versioning your code with \Mont}
\chapterauthor{\authoroscar{}}

\alex{Maybe a section on MC Configuration maps may be given}
\on{what are they?}

\on{Instead of a chapter on Installer, wouldn't a short section on Scripting Monticello suffice?}


\lr{
Subject: Re: [Pharo-project] Create a changeset for an entire package?\\
Reply-To: Pharo-project@lists.gforge.inria.fr\\
a) I don't wan't to include everything I changed from the package.
There are some things that I changed for debugging. So, I can't
fileout everything and I don't know if monticello will let me do
what I want.\\
It is possible, but it requires some manual work to get a custom
change-set from A to B:\\
1. Make sure that both versions A and B are stored as Monticello
Packages (probably to a local directory).\\
2. Load version A.\\
3. Load version B.\\
4. In the changes-browser there is now a new change-set named A->B.\\
5. Open a change browser on it and edit it as you like.\\
6. File-out the change-set.\\
What I usually do is different and I think goes better with the whole
workflow of Monticello:\\
1. In an image where I have the modified code I display the changes
between the previous version (A) and the current image (B).\\
2. Then I selectively revert the changes that I don't want to go into B.\\
3. And commit the package B.\\
b) Supose the change includes 3 separate packages, that should be
seen as a one. Loading only one package only would break things, so
I'd like the change to include all 3 in one operation.\\
It is possible to create custom PackageInfo subclasses (Seaside <= 2.8
did this), but I have rarely seen this and don't find it particularly
useful. I would stick to the original packages to make merging for the
maintainers easy.\\
Lukas
}

\mb{- in PBE, it would be nice to define the package from the start with
OB. The section is made for Squeak w/o package browser\\
- 'You should now be able to verify that the only the original (red)
tests are loaded.' typo before 'Branching' subsection\\
- in Branching subsection: 'Once again the tests should be green,
though our implementation of perfect numbers is slightly different.'
but there is only one test now so: 'The current test should be green...'\\
- (replace the merge tool by the current one in Pharo / add explanation)\\
- In Change subsection: the Patch Browser is mentionned in the figure
not in the text. At the end of the first paragraph, we should add 'via
the Patch Browser.'}

A versioning system helps you to store and log multiple versions of your code. In addition it may help you to manage concurrent accesses to a common source code repository. It keeps track of all changes to a set of documents and enables several developers to collaborate. As soon as the size of your software increases beyond a few classes, you probably need a versioning system.

Many different versioning systems are available. \ind{CVS}\footnote{\url{http://www.nongnu.org/cvs}} and \ind{Subversion}\footnote{\url{http://subversion.tigris.org}} are probably the most popular.
In principle you could use them to manage the development of \pharo software projects, but such a practice would disconnect the versioning system from the \pharo environment.
In addition, CVS-like tools only version plain text files and not individual packages, classes or methods. We would therefore lack the ability to track changes at the appropriate level of granularity. 
If the versioning tools know that you store classes and methods instead of plain text,
they can do a better job of supporting the development process.

%\ab{It would be better to say concretely why this would be a bad idea, rather than resorting to name-calling.  Many people think that files are good.  Let's tell the reader about the disadvantages!}

%\indmain{Monticello}
\index{SqueakSource}
\emph{\indmain{\Mont{}}} is  a versioning system for \pharo in which classes and methods, rather than lines of text,
are the units of change. \emph{\sqsrc{}} is a central online repository in which you can store versions of your applications using \Mont. \sqsrc is the equivalent of \ind{GForge}, and \Mont the equivalent of CVS. 

In this chapter, you will learn how to use use \Mont and \sqsrc to manage your software. We have already met \Mont briefly in earlier chapters\footnote{``A first application'' and ``The Pharo programming environment''}.
% These chapters are in book1, so we can't cross-reference them!
% (\charef{firstApp} and \charef{env}).
This chapter delves into the details of \Mont{} and describes some additional features that are useful for versioning large applications.

%=================================================================
\section{Basic usage}

We will start by reviewing the basics of creating a package and committing changes, and then we will see how to update and merge changes. 

%-----------------------------------------------------------------
\subsection{Running example --- perfect numbers}

We will use a small running example of perfect numbers\footnote{Perfect numbers were discovered by \ind{Euclid}. A perfect number is a positive integer that is the sum of its proper divisors. $6 = 1 + 2 + 3$ is the first perfect number.} in this chapter to illustrate the features of \Mont. We will start our project by defining some simple tests.

\dothis{Define a subclass of \clsind{TestCase} called \ct{PerfectTest} in the category \ct{Perfect}, and define the following test methods in the protocol \ct{running}:}
\begin{code}{}
PerfectTest>>>testPerfect
	self assert: 6 isPerfect.
	self assert: 7 isPerfect not.
	self assert: 28 isPerfect.
\end{code}

Of course these tests will fail as we have not yet implemented the \ct{isPerfect} method for integers. We would like to put this code under the control of \Mont as we revise and extend it.

%-----------------------------------------------------------------
\subsection{Launching \Mont}

\Mont is included in the standard \pharo distribution.
We will assume that \Mont is already installed in your image.
\menu{\Mont Browser} can be selected from the \emph{World} menu.
\index{Monticello!browser}

\begin{figure}[ht]\centering
	\includegraphics[width=\textwidth]{monticelloBrowser}
	\caption{The \Mont Browser.\figlabel{monticelloMain}}
\end{figure}

In \figref{monticelloMain} we see that the \Mont Browser consists of two list panes and one button pane. The left pane lists installed packages and the right panes shows known repositories.
Various operations may be performed via the button pane and the menus of the two list panes.

%-----------------------------------------------------------------
\subsection{Creating a package}

\index{Monticello!creating packages}
\Mont manages versions of \emph{packages}.  A \ind{package} is essentially a named set of classes and methods.
In fact, a package is an object\,---\,an instance of \clsind{PackageInfo}\,---\,that knows how to identify the classes and methods that belong to it.

We would like to version our \ct{PerfectTest} class. The right way to do this is to define a package\,---\,called \ct{Perfect}\,---\,containing \ct{PerfectTest} and all the related classes and methods we will introduce later. For the moment, no such package exists. We only have a \emph{category} called (not coincidentally) \ct{Perfect}. This is perfect, % (ugh!), 
since \Mont will map categories to packages for us.

\dothis{Press the \button{+Package} in the \MCB and enter \scat{Perfect}.}

\emph{Voil\`a!} You have just created the \pkg{Perfect} \Mont package. 

\begin{figure}[ht]\centering
	\includegraphics[width=\textwidth]{perfectPackage}
	\caption{Creating the Perfect package.\figlabel{perfect}}
\end{figure}

\Mont packages follow a number of important naming conventions for class and method categories.
Our new package named \pkg{Perfect} contains:

\begin{itemize}

\item All classes in the category \scat{Perfect}, or in categories whose names start with \scat{Perfect-}. For now this includes only our \ct{PerfectTest} class. \sd{Ask lukas because I do not see why Perfect- would be any different from PerfectZork - is not a special character.}

\item All methods belonging to \emph{any} class (in any category) that are defined in a protocol named \prot{*perfect} or \prot{*Perfect}, or in protocols whose names start with \prot{*perfect-} or \prot{*Perfect-}. Such methods are known as \emph{extensions}. We don't have any yet, but we will define some very soon.

\item All methods belonging to any classes in the category \scat{Perfect}, or in categories whose names begin with \scat{Perfect-}, \emph{except} those in protocols whose names start with \prot{*} (\ie those belonging to \emph{other} packages). This includes our \ct{testPerfect} method, since it belongs to the protocol \ct{running}.

\end{itemize}

%-----------------------------------------------------------------
\subsection{Committing changes}

\index{Monticello!committing changes}
Note in \figref{perfect} that the \button{Save} button is disabled (greyed out).

Before we save our \ct{Perfect} package, we need to specify \emph{where} we want to save it. A \emph{repository} is a package container, which may either be local to your machine or remote (accessed over the network). 
Various protocols may be used to establish a connection between your \pharo image and a repository. As we will see later (\secref{monti:repositories}), \Mont supports a large choice of repositories, though the most commonly used is HTTP, since this is the one used by \sqsrc.

\index{package-cache}
At least one repository, called \ct{package-cache}, is set up by default, and is shown as the first entry in the list of repositories on the right-hand side of your \MCB (see \figref{monticelloMain}).
The package-cache is created automatically in the local directory where your \pharo image is located. It will contain a copy of all the packages you download from remote repositories. 
By default, copies of your packages are also saved in the package-cache when you save them to a remote server.

Each package knows which repositories it can be saved to.
To add a new repository to the selected package, press the \button{+Repository} button. This will offer a number of choices of different kinds of repository, including HTTP. For the rest of the chapter we will work with the \ct{package-cache} repository, as this is all we need to explore the features of \Mont.

\dothis{Select the directory repository named \lct{package cache}, press \button{Save}, enter an appropriate log message, and \button{Accept} to save the changes.}

\begin{figure}[ht]\centering
	\includegraphics[width=.6\textwidth]{saving}
	\caption{You may set a new version name and a commit message when you save a version of a package.\figlabel{saving}}
\end{figure}

The \ct{Perfect} package is now saved in \ct{package-cache}, which is nothing more than a directory contained in the same directory as your \pharo image. Note, however, that if you use any other kind or repository (\eg{} HTTP, FTP, another local directory), a copy of your package will also be saved in the package-cache. 

\dothis{Use your favorite file browser (\eg Windows Explorer, Finder or XTerm) to confirm that a file \ct{Perfect-XX.1.mcz} was created in your package cache. \ct{XX} corresponds to your name or initials.\footnote{In the past, the convention was for developers to log their changes using only their initials. Now, with many developers sharing identical initials, the convention is to use an identifier based on the full name, such as ``apblack'' or ``AndrewBlack''.}}

%You may change the version name and add a comment.
%\ab{Experience with my students is that changing the version name is a really bad idea, since it can easily break \Mont.}
%\alex{I was not able to find a satisfying explanation, So I prefer to leave this point out}

A \emphind{version} is an immutable snapshot of a package that has been written to a repository. Each version has a unique version number to identify it in a repository.
Be aware, however, that this number is \emph{not} globally unique\,---\,in another repository you might have the same file identifier for a \emph{different snapshot}. For example, \ct{Perfect-onierstrasz.1.mcz} in another repository might be the \emph{final}, deployed version of our project!
When saving a version into a repository, the next available number is automatically assigned to the version, but you can change this number if you wish.
Note that version branches do not interfere with the numbering scheme (as with CVS or Subversion). As we shall see later, versions are by default ordered by their version number when viewing a repository.

%-----------------------------------------------------------------
\subsection{Class extensions}

Let's implement the methods that will make our tests green.

\dothis{Define the following two methods in the class \ct{Integer}, and put each method in a protocol called \ct{*perfect}. Also add the new boundary tests.  Check that the tests are now green.}

\begin{code}{}
Integer>>>isPerfect
	^ self > 1 and: [self divisors sum = self]

Integer>>>divisors
	^ (1 to: self - 1 ) select: [ :each | (self rem: each) = 0 ]

PerfectTest>>>testPerfectBoundary
	self assert: 0 isPerfect not.
	self assert: 1 isPerfect not.
\end{code}

Although the methods on \ct{Integer} do not belong to the \scat{Perfect} category, they \emph{do} belong to the \ct{Perfect} package since they are in a protocol whose name starts with \ct{*} and matches the package name. Such methods are known as \emphind{class extensions}, since they extend existing classes. These methods will be available \emph{only} to someone who loads the \ct{Perfect} package.

\alex{We could give a reference to the Cmd-p command in the OB browser. This keystroke is used to turn a method into a class-extension.}
\on{you mean it moves a method to an extension package}

%-----------------------------------------------------------------
\subsection{``Clean'' and ``Dirty'' packages}

\index{package!clean and dirty}
Modifying the code in a package with any of the development tools makes that package \emph{dirty}.
This means that the version of the package in the image is different from the version that has been saved or loaded. 

\begin{figure}[ht]\centering
	\includegraphics[width=\textwidth]{dirty}
	\caption{Modifying our Perfect package will ``dirty'' it.\figlabel{dirty}}
\end{figure}

In the \MCB,  a dirty package can be recognized by an asterix (\ct{*}) preceding its name.  This indicates which packages have uncommitted changes, and therefore need to be saved into a repository if those changes are not to be lost. Saving a dirty package cleans it. \ab{Why is it that when I click \button{Changes} on a dirty package, I sometimes get a dialog back that says ``no changes''?  I found this very confusing.  How can I see \emph{why} \Mont thinks a package is dirty?  How can I see \emph{which} repository it was originally loaded from or saved to?}

\dothis{Try the \button{Browse}, \button{History} and \button{Changes} buttons to see what they do\footnote{At the time of this writing, the \button{Scripts} button does not work.}.
\button{Save} the changes to the \ct{Perfect} package. Confirm that the package is now ``clean'' again.}

%\on{What does \button{Scripts} do?!}
%\subsection{Scripts}
% Script do not work!

%-----------------------------------------------------------------
\subsection{The Repository inspector} 

\index{Repository inspector|see{Monticello, repository inspector}}
\index{Monticello!repository inspector}
The contents of a repository can be explored using a \RI, which is launched using the \button{Open} button of \Mont (cf \figref{repositoryinspector}).

\index{package-cache}
\dothis{Select the \ct{package-cache} repository and open it. You should see something like \figref{repositoryinspector}.}

\begin{figure}[ht]\centering
	\includegraphics[width=\textwidth]{{repositoryinspector}}
	\caption{A \RI.\figlabel{repositoryinspector}}
\end{figure}

All the packages in the repository are listed on the left-hand side of the inspector:

\begin{itemize}
\item an \underline{underlined} package name means that this package is installed in the image;
\item a \underline{\bf bold underlined} name means that the package is installed, but that there is a more recent version in the repository;
\item a name in a normal typeface means that the package is not installed in the image.
\end{itemize}

\noindent
Once a package is selected, the right-hand pane lists the versions of the selected package:

\begin{itemize}
\item an \underline{underlined} version name means that this version is installed in the image;
\item a {\bf bold} version name means that this version is not an ancestor of the installed version. This may mean that it is a newer version, or that it belongs to a different branch from the installed version;
\item a version name displayed with a normal typeface shows an older version than the installed current one.
\end{itemize}

{\Actclick}ing the right-hand side of the inspector opens a menu with different sorting options. The \menu{unchanged} entry in the menu discards any particular sorting. It uses the order given by the repository.

%-----------------------------------------------------------------
\subsection{Loading, unloading and updating packages}

\index{package-cache}
\index{Monticello!loading, unloading, updating packages}
At present we have two versions of the \ct{Perfect} package stored safely in our \ct{package-cache} repository. We will now see how to unload this package, load an earlier version, and finally update it.

\dothis{Select the \ct{Perfect} package and its repository in the \MCB. \Actclick on the package name and select \menu{unload package}.}

\begin{figure}[th]\centering
	\includegraphics[width=\textwidth]{{unload}}
	\caption{Unloading a package.\figlabel{unload}}
\end{figure}

You should now be able to confirm that the \ct{Perfect} package has vanished from your image!

\dothis{In the \MCB, select the \ct{package-cache} in the repository pane, without selecting anything in the package pane, and \button{Open} the \RI.
Scroll down and select the \ct{Perfect} package. It should be displayed in a normal typeface, indicated that it is not installed.
Now select version 1 of the package and \button{Load} it.}

\begin{figure}[ht]\centering
	\includegraphics[width=\textwidth]{{loading}}
	\caption{Loading an earlier version.\figlabel{loading}}
\end{figure}

You should now be able to verify that the only the original (red) tests are loaded.

\dothis{Select the second version of the \ct{Perfect} package in the \RI and \button{Load} it.
You have now \emph{updated} the package to the latest version.}

Now the tests should be green again.

%-----------------------------------------------------------------
\subsection{Branching}

\index{Monticello!branching}
A \emphind{branch} is a line of development versions that exists independently of another line, yet still shares a common ancestor version if you look far enough back in time.

You may create new version branch when saving your package. Branching is useful when you want to have a new parallel development. For example, suppose your job is to maintain a software in your company. One day a different division asks you for the same software, but with a few parts tweaked for them, since they do things slightly differently. The way to deal with this situation is to create a second branch of your program that incorporate the tweaks, while leaving the first branch unmodified.

\dothis{From the \RI, select version 1 of the \ct{Perfect} package and \button{Load} it. Version 2 should again be displayed in bold, indicating that it no longer loaded (since it is not an ancestor of version 1).
Now implement the following two \ct{Integer} methods and place them in the \ct{*perfect} protocol, and also modify the existing \ct{PerfectTest} test method as follows:}

\begin{code}{}
Integer>>>isPerfect
	self < 2 ifTrue: [ ^ false ].
	^ self divisors sum = self

Integer>>>divisors
	^ (1 to: self - 1 ) select: [ :each | (self \\ each) = 0]

PerfectTest>>>testPerfect
	self assert: 2 isPerfect not.
	self assert: 6 isPerfect.
	self assert: 7 isPerfect not.
	self assert: 28 isPerfect.
\end{code}

Once again the tests should be green, though our implementation of perfect numbers is slightly different.

\dothis{Attempt to load version 2 of the \ct{Perfect} package.}

Now you should get a warning that you have unsaved changes.

\begin{figure}[ht]\centering
	\includegraphics[width=.8\textwidth]{{unsavedWarning}}
	\caption{Unsaved changes warning.\figlabel{unsavedWarning}}
\end{figure}

\dothis{Select \button{Abandon} to avoid overwriting your new methods.
Now \button{Save} your changes. You will get another warning that there may be newer versions.
Select \menu{Yes}, enter your log message, and \button{Accept} the new version.
}

\begin{figure}[ht]\centering
	\includegraphics[width=\textwidth]{{newerWarning}}
	\caption{Newer versions warning.\figlabel{newerWarning}}
\end{figure}

Congratulations! You have now created a new branch of the \ct{Perfect} package.

\begin{figure}[ht]\centering
	\includegraphics[width=\textwidth]{{branch}}
	\caption{Versions 2 and 3 are separate branches of version 1.\figlabel{branch}}
\end{figure}

\dothis{If you still have the \RI open, \button{Refresh} it to see the new version (\figref{branch}).}

%-----------------------------------------------------------------
\subsection{Merging}

\on{Should we be describing the new strategy instead?}

\index{Monticello!merging}
\index{Monticello!merge tool}
This section describes the conventional merging facility of Monticello.
To use it, make sure that the preference \ct{useNewDiffToolsForMC} is disabled.
You can either use the Preference Browser, or you can do this programmatically by evaluating the following expression:
\begin{code}{}
Preferences disable: #useNewDiffToolsForMC
\end{code}

% \alex{This subsection will need a major overhaul when turning the chapter for Pharo. The merging tool is radically different}\sd{Is it not. There is a preference to get the one shown in this section. We should just say useDiff....}
You can merge one version of a package with another using the \button{Merge} button in the \MCB. Typically you will want to do this when (i) you discover that you have been working on a out-of-date version, or (ii) branches that were previously independent have to be re-integrated. Both scenarios are common when multiple developers are working on the same package.

%\ap{Here I stopped}

Consider the current situation with our \ct{Perfect} package, as illustrated at the left of \figref{branching-merging}. We have published a new version 3 that is based on version 1.
Since version 2 is also based on version 1, versions 2 and 3 constitute independent branches.

\begin{figure}[ht]\centering
	\includegraphics[width=\textwidth]{branching-merging}
	\caption{Branching (left) and merging (right).\figlabel{branching-merging}}
\end{figure}

At this point we realize that there are changes in version 2 that we would like to merge with our changes from version 3.  Since we have version 3 currently loaded, we would like to merge in changes from version 2, and publish a new, merged version 4, as illustrated at the right of \figref{branching-merging}.

%\begin{figure}[ht]\centering
%	\includegraphics[width=.5\textwidth]{branching}
%	\caption{Version 3 is the current branch of version 1.\figlabel{branching}}
%\end{figure}


%\begin{figure}[ht]\centering
%	\includegraphics[width=.5\textwidth]{merging}
%	\caption{Version 4 merges changes from versions 2 and 3.\figlabel{merging}}
%\end{figure}

\begin{figure}[ht]\centering
	\includegraphics[width=\textwidth]{mergeButton}
	\caption{Select a separate branch (in bold) to be merged.\figlabel{mergeButton}}
\end{figure}

\dothis{Select version 2 in the repository browser, as shown in \figref{mergeButton}, and \click the \button{Merge} button.}

The merge tool is a tool that allows for fine-grained package version merging. Elements contained in the package to-be-merged are listed in the upper text pane. The lower text pane shows the definition of a selected element. 

\begin{figure}[ht]\centering
	\includegraphics[width=.8\textwidth]{mergeTool}
	\caption{Version 2 of the \ct{Perfect} package being merged with the current version 3.
	\on{This is a major fail! The old merge tool used to show deleted text striked through.
	Now, colour blind people will see no difference at all!}
	\figlabel{mergeTool}}
\end{figure}

In \figref{mergeTool} we see the three differences between versions 2 and 3 of the \ct{Perfect} package. The method \ct{PerfectTest>>>testPerfectBoundary} is new, and the two indicated methods of \ct{Integer} have been changed. In the lower pane we see the old and new versions of the source code of \ct{Integer>>>isPerfect}.
New code is displayed in red, removed code is barred and displayed in blue, and unchanged code is shown in black. 

A method or a class is in conflict if its definition has been altered. \figref{mergeTool} shows 2 conflicting methods in the class \ct{Integer}: \ct{isPerfect} and \ct{divisors}. A conflicting package element is indicated by being \underline{underlined}, \sout{barred}, or {\bf bold}. The full set of typeface conventions is as follows:

\begin{description}
\item[Plain=No Conflict.] A plain typeface indicates the definition is non-conflicting. For example, the method \ct{PerfectTest>>>testPerfectBoundary} does not conflict with an existing method, and can be installed.
\item[Bold=A method is conflicting.] A decision needs to be taken to keep the proposed change or reject it. The proposed method \ct{Integer>>>>isPerfect} is in conflict with an existing definition in the image. The conflict can be resolved by clicking \button{Keep} or \button{Reject}.
\item[Underlined=Repository replace current.] An \underline{underlined} element will be kept and replace the current element in the image. In \figref{keepReject} we see that \ct{Integer>>>isPerfect} from version 2 has been kept.
\item[Barred=Repository version rejected.] A \sout{barred} element has been rejected, and the local definition will not be replaced. In \figref{keepReject} \ct{Integer>>>divisors} from version 2 is rejected, so the definition from version 3 will remain.
\on{This appears to no longer be true! -- a rejected element is simply shown in plain font! -- should be fixed.}
\end{description}

\begin{figure}[ht]\centering
	\includegraphics[width=.8\textwidth]{keepReject}
	\caption{Keeping and rejecting changes.
	\figlabel{keepReject}}
\end{figure}

Note that the merge tool offers buttons to select all newer or all older changes, or to select all local or all remote changes that are still in conflict.

\dothis{Keep \ct{Integer>>>>isPerfect} and reject \ct{Integer>>>divisors}, and click the \button{Merge} button. Confirm that the tests are all green. Commit the new merged version of \ct{Perfect} as version 4.}

\on{Something is wrong here -- it complains that version 3 might be newer. When I saved it however everything seems fine.}

\begin{figure}[ht]\centering
	\includegraphics[width=0.8\textwidth]{merged}
	\caption{All older versions are now ancestors of merged version 4.
	\figlabel{merged}}
\end{figure}

If you now refresh the \RI, you will see that there are no more versions shown in bold, \ie all versions are ancestors of the currently loaded version 4 (\figref{merged}).

%=================================================================
\section{Exploring \Mont repositories}
\seclabel{monti:exploring}

\Mont has many other useful features. As we can see in \figref{monticelloMain}, the \MCB window has nine buttons. We have already used four of them\,---\,\button{+Package}, \button{Save}, \button{+Repository} and \button{Open}.
We will now look at \button{Browse}, \button{Changes} and \button{History}, which are used to explore the state and history of repositories

%-----------------------------------------------------------------
\subsection{Browse}

\index{Monticello!snapshot browser}
The \button{Browse} button opens a ``snapshot browser'' to display the contents of a package.
The advantage of the snapshot browser over the browser is its ability to display class extensions.

\dothis{Select the \ct{Perfect} package and click the \button{Browse} button.}

\begin{figure}[ht]
\centering
	\includegraphics[width=\textwidth]{{packageviewer}}
	\caption{The snapshot browser reveals that the \ct{Perfect} package extends the class \ct{Integer} with 2 methods.\figlabel{packageviewer}}
\end{figure}

\index{class extension}
For example, \figref{packageviewer} shows the class extensions defined in the \pkg{Perfect} package. Note that code cannot be edited here, though by {\actclick}ing, if your environment has been set up accordingly) on a class or a method name you can open a regular browser.

It is a good practice to always browse the code of your package before publishing it, to ensure that it really contains what you think it does.

\alex{David's browser shows class extensions. Although it does not supersede the snapshot browser, it should be mentioned here}


% \sd{could we get rid of this yellow and red clicking. it makes no sense. I remember that I propose a good naming convention for the mouse button -- it was also in one of the freebooks - select = left = you select a window to move it around, operate you want to get the menus and act on something and  meta is for the rest}

%-----------------------------------------------------------------
\subsection{Changes} 

\index{Monticello!patch browser}
The \button{Changes} button computes the difference between the code in the image and the most recent version of the package in the repository. 

\dothis{Make the following changes to \ct{PerfectTest}, and then click the \button{Changes} button in the \MCB.}

\begin{code}{}
PerfectTest>>>testPerfect
	self assert: 2 isPerfect not.
	self assert: 6 isPerfect.
	self assert: 7 isPerfect not.
	self assert: 496 isPerfect.

PerfectTest>>>testPerfectTo1000
	self assert: ((1 to: 1000) select: [:each | each isPerfect]) = #(6 28 496)
\end{code}

\begin{figure}[ht]\centering
	\includegraphics[width=\textwidth]{{patchbrowser}}
	\caption{The patch browser shows the difference between the code in the image and the most recently committed version.\figlabel{patchbrowser}}
\end{figure}

\figref{patchbrowser} shows that the \ct{Perfect} package has been locally modified with one changed method and one new method.
As usual, {\actclick}ing on a change offers you a choice of contextual operations.

%-----------------------------------------------------------------
\subsection{History} 

\index{Monticello!history browser}
The \button{History} button opens a version history viewer that displays the comments committed along with each version of the selected package (see \figref{historyviewer}).  
The versions of the package, in this case \ct{Perfect}, are listed on the left,
while information about the selected version is displayed on the right.

\dothis{Select the \ct{Perfect} package and click the \button{History} button.}

\begin{figure}[ht]\centering
	\includegraphics[width=\textwidth]{{historyviewer}}
	\caption{The version history viewer provides information about the various versions of a package.\figlabel{historyviewer}}
\end{figure}

By {\actclick}ing on a particular version, you can explore the changes with respect to the current working copy of the package loaded in the image, or spawn a new history browser relative to the selected version.

%=================================================================
\section{Advanced topics}
\seclabel{monti:advanced}

Now we will have a look at several advanced topics, including backporting, managing dependencies, and class initialization.

%-----------------------------------------------------------------
\subsection{Backporting}

\alex{I am wondering how useful this is. I personally never used it}

\index{Monticello!backporting}
Sometimes we want to port changes from one branch to another, without actually being forced to merge those branches.
Backporting is a process of applying selected changes from one version of a package to an ancestor so that these changes can be merged into later branches.  This is especially useful when corrections to software defects must be merged into multiple branches.

The process is illustrated in \figref{backport}.
Suppose that the main branch of our software system consists of versions A and B, maintained by Manny.
A contributor, Conny, has developed a separate experimental branch, C, with changes X and Y.
Change X fixes a nasty problem in versions A and B, so Manny asks Conny to prepare a backported branch D containing \emph{only} change X.
Now Manny can merge B and D to produce a new version E that fixes the defect.
Conny can continue to further develop her independent branch C.

\begin{figure}[ht]
\centering
	\includegraphics[width=\textwidth]{{backport}}
	\caption{Change X is backported from version C to version A, producing a new branch D. D can then be merged into B, without affecting C.\figlabel{backport}}
\end{figure}

The system records the fact that this new version was backported from a later version, and will make use of that information when merging.

To use \button{Backport}, you must have just saved your package\,---\,if your package is marked with the modified *, \button{Backport} is disabled.  When you press \button{Backport}, you will first be asked to pick the ancestor version you want to backport to.  You will then be presented with a multi-select list of all the changes between that ancestor and the current version.   Choose only the changes you want to backport, and then press \button{Select}.

Let us see how this works in practice. Recall that we earlier rejected the implementation of \ct{isPerfect} when we merged versions 2 and 3 of the \ct{Perfect} package. Now we will recover that change as a backport to version 1.  (Versions 1, 2 and 3 play the roles of versions A, B and C, respectively, in \figref{backport}.)

\dothis{Unload the \ct{Perfect} package. Now open a \RI on your \ct{package-cache} and load \ct{Perfect} version 3. In the \MCB select \ct{Perfect} and click on the \button{Backport} button. Select version 1 as the ancestor. You should be able to browse the changes between version 3 and 1, as shown in \figref{changes}.  Now select the \ct{Integer>>>isPerfect} method and click on \button{Select}.}

\begin{figure}[ht]
\centering
	\includegraphics[width=0.8\textwidth]{{changes}}
	\caption{Backporting changes from version 3 to version 1 of the \ct{Perfect} package.\figlabel{changes}}
\end{figure}


Congratulations! You have now backported the \ct{isPerfect} method from version 3 to version 1 of \ct{Perfect}.
Any changes you didn't select were reverted; that is, your image will now contain only the code from the ancestor version 1, plus the changes that you chose.
In the \MCB you should see that the currently loaded version of \ct{Perfect} is now version 1 (not version 3). If you click on \button{Changes}, you will see that the only change is the \ct{isPerfect} method.
You can now save this backported version, merge it into something else, or whatever you like.

%-----------------------------------------------------------------
\subsection{Dependencies}

\index{Monticello!dependency}
\index{package!required|see{Monticello, dependency}}
Most applications cannot live on their own and typically require the presence of other packages in order to work properly. For example, let us have a look at Pier\footnote{\url{http://source.lukas-renggli.ch/pier}}, a meta-described content management system. Pier is a large piece of software with many facets (tools, documentations, blog, catch strategies, security, ...). Each facet is implemented by a separate package. Most Pier packages cannot be used in isolation since they refer to methods and classes defined in other packages. \Mont provides a dependency mechanism for declaring the \emph{required packages} of a given package to ensure that it will be correctly loaded.

Essentially, the dependency mechanism ensures that all required packages of a package are loaded before the package is loaded itself. Since required packages may themselves require other packages, the process is applied recursively to a tree of dependencies, ensuring that the leaves of the tree are loaded before any branches that depend on them.
Whenever new versions of required packages are checked in, then new versions of the packages that depend on them will automatically depend on the new versions.

\important{\emph{Dependencies cannot be expressed across repositories}. All requiring and required packages must live in the same repository.}

\figref{dependencies} illustrates how this works in \ind{Pier}.
Package \ct{Pier-All} is an \emph{empty package} that acts as a kind of umbrella.
It requires \ct{Pier-Blog}, \ct{Pier-Caching} and all the other Pier packages.

\begin{figure}[ht]
\centering
	\includegraphics[width=\textwidth]{{dependencies}}
	\caption{Dependencies in Pier.\figlabel{dependencies}}
\end{figure}

Because of these dependencies, installing \ct{Pier-All} causes all the other Pier packages to be installed. Furthermore, when developing, the only package that needs to be saved is \ct{Pier-All}; all dependent dirty packages are saved automatically.

Let us see how this works in practice.  Our \ct{Perfect} package currently bundles the tests together with the implementation.  Suppose we would like instead to separate these into separate packages, so that the implementation can be loaded without the tests.  By default, however, we would like to load everything.

\dothis{Take the following steps:
\begin{itemize}\setlength{\itemsep}{0pt}
\item Load version 4 of the \ct{Perfect} package from the package cache
\item Create a new package in the browser called \ct{NewPerfect-Tests} and drag the class \ct{PerfectTest} to this package
% \item \sout{Rename the \ct{Perfect} category to \ct{Perfect-Tests} ({\actclick} on the category in the browser to rename it)} \on{There is no menu item to rename a package!}
\item Rename the \ct{*perfect} protocol of the \ct{Integer} class to \ct{*newperfect-extensions} (\actclick to rename it)
\item In the \MCB, add the packages \ct{NewPerfect-All} and \ct{NewPerfect-Extensions}.
\item Add \ct{NewPerfect-Extensions} and \ct{NewPerfect-Tests} as required packages to \ct{NewPerfect-All} ({\actclick} on \ct{NewPerfect-All})
\item Save package \ct{NewPerfect-All} in the package-cache repository.
	Note that \Mont prompts for comments to save the required packages too.
\item Check that all three packages have been saved in the package cache.
\item \Mont thinks that \ct{Perfect} is still loaded. Unload it and then load \ct{NewPerfect-All} from the \RI. This will cause \ct{NewPerfect-Extensions} and \ct{NewPerfect-Tests} to be loaded as well as required packages.
\item Check that all tests run.
\end{itemize}
}
\index{package-cache}

Note that when \ct{NewPerfect-All} is selected in the \MCB, the dependent packages are displayed in bold (see \figref{perfectDependencies}).

\begin{figure}[ht]
\centering
	\includegraphics[width=\textwidth]{{perfectDependencies}}
	\caption{\ct{NewPerfect-All} requires \ct{NewPerfect-Extensions} and \ct{NewPerfect-Tests}.\figlabel{perfectDependencies}}
\end{figure}

\important{If you further develop the \ct{Perfect} package, you should only load or save \ct{NewPerfect-All}, not its required packages.}

Here is the reason why:

\begin{itemize}
\item If you load \ct{NewPerfect-All} from a repository (package-cache, or anywhere else), this will cause \ct{NewPerfect-Extensions} and \ct{NewPerfect-Tests} to be loaded from the same repository.
\item If you modify the \ct{PerfectTest} class, this will cause the \ct{NewPerfect-Tests} and \ct{NewPerfect-All} packages to both become dirty (but not \ct{NewPerfect-Extensions}).
\item To commit the change, you should save \ct{NewPerfect-All}. This will commit a new version of \ct{NewPerfect-All} which then requires the new version of \ct{NewPerfect-Tests}. (It will also depend on the existing, unmodified version of \ct{NewPerfect-Extensions}.)  Loading the latest version of \ct{NewPerfect-All} will also load the latest version of the required packages.
\item If instead you save \ct{NewPerfect-Tests}, this will \emph{not} cause \ct{NewPerfect-All} to be saved.  This is bad because you effectively break the dependency.  If you then load the latest version of \ct{NewPerfect-All} you will not get the latest versions of the required packages. Don't do it!
\end{itemize}

\important{Do not name your top level package with a suffix (\eg \ct{Perfect}) that could match your subpackages. Do not define \ct{Perfect} as a required package of \ct{Perfect-Extensions} or \ct{PerfectTest}. You would get in trouble since \Mont would save all the classes for three packages while you only want two packages and an empty one at the top level.}
%-----------------------------------------------------------------
\subsection{Class initialization}

\index{class!initialization}
\index{Monticello!class initialization}
When \Mont loads a package into the image, any class that defines an \ct{initialize} method on the class side will be sent the \ct{initialize} message. The message is sent \emph{only} to classes that define this method on the class side. A class that does not define this method will not be initialized, even if \ct{initialize} is defined by one of its superclasses. NB: the initialize method is not invoked when you merely reload a package!
% martial: invoked method?? -> better with 'message \ct{initialize} is not sent??

%\on{OK, I confirmed this with a simple test}

Class initialization can be used to perform any number of checks or special actions.
A particularly useful application is to add new instance variables to a class.

Class extensions are strictly limited to adding new methods to a class.
Sometimes, however, extension methods may need new instance variables to exist.

Suppose, for example, that we want to extend the \ct{TestCase} class of SUnit with methods to keep track of the history of the last time the test was red.  We would need to store that information somewhere, but unfortunately we cannot define instance variables as part of our extension.

A solution would be to define an \ct{initialize} method on the class side of one of the classes:

\begin{code}{}
TestCaseExtension class>>initialize
	(TestCase instVarNames includes: 'lastRedRun') 
		ifFalse: [TestCase addInstVarName: 'lastRedRun']
\end{code}

When our package is loaded, this code will be evaluated and the instance variable will be added, if it does not already exist.


%As an illustration, let us suppose an extension of the Lights Out game to support unchangeable cells. Those cells are located randomly on the board. Generate numbers need to be generated. This can be done by adding a variable \ct{randomNumber} to the class \ct{LOGame}. This kind of extension is not well supported by \Mont. A package cannot define the variable \ct{randomNumber} as an extension of \ct{LOGame}. \Mont supports method extension only.
%\ab{Tell us what the problem is, before telling us the solution.  With an Example!}


%One way to circumvent this limitation is to define an initialize method on the class side, and manually add this variable. We could then define the following \ct{initialize} method on a class \ct{GameExtension} contained in the extending package:

%\begin{code}{}
%GameExtension class>>initialize
%	(LOGame instVarNames includes: 'randomNumber') 
%		ifFalse: [LOGame addInstVarName: 'randomNumber']
%\end{code}

%In the case that \ct{randomNumber} is not defined in \ct{LOGame}, this variable is added to this class.

%\ab{Isn't this more appropriate for a post-load \emph{do-it} than an \ct{initialize} method?  Do such things not exist (they did with change sets \ldots}
%\alex{Probably, but this seems to be broken with \Mont}

%=================================================================
\section{Getting a change set from two versions}
%:\alex{Change set were seen in Chapter 6, page 135 of the version 1 of PBE}
A Monticello version is the snapshot of one or more packages. A version contains the complete set of class and method definitions that constitute the underlying packages. Sometimes, it is useful to have a ``patch'' from two versions. A patch is the set of all necessary side effect in the system to go from one version A to another version B. 

\emph{Change set} is a \pharo built-in mechanism to define system patches. A change set is composed of global side effects on the system. New change set may be created and edited from the \emph{Change Sorter}. This tool is available from the \menu{World\go{}Tools} entry.

The difference between two Monticello versions may be easily captured by creating a new change set before loading a second version of a package. As an illustration, we will capture the differences between version 1 and 2 of the \pkg{Perfect} package:
\begin{enumerate}
\item Load version 1 of \pkg{Perfect} from the Monticello browser
\item Open a change sorter and create a new change set. Let's name it \lct{DiffPerfect}
\item Load version 2
\item In the change sorter, you should now see the difference between version 1 and 2. The change set may be saved on the filesystem by \actclick{ing} on it and selecting \menu{file out}. A \lct{DiffPerfect.X.cs} file is now located next to your \pharo image.
\end{enumerate}


%=================================================================
\section{Kinds of repositories}
\seclabel{monti:repositories}

\index{Monticello!repository}
Several kinds of repositories are supported by \Mont, each with different characteristics and uses. Repositories can be read-only, write-only or read-write. Access rights may be defined globally or can be tied to a particular user (as in \sqsrc, for example).
% and are related to the way of storing data used.

\index{Monticello!HTTP repository}
\paragraph{HTTP.} HTTP repositories are probably the most popular kind of repository since this is the kind supported by \sqsrc. % (Such servers can also be configured for read-only access. Saving versions via HTTP uses the PUT method \ab{Is ``put'' the right word}, which must be enabled on the server.)

The nice thing about HTTP repositories is that it's easy to link directly to specific versions from web sites. With a little configuration work on the HTTP server, HTTP repositories can be made browsable by ordinary web browsers, WebDAV clients, and so on.

HTTP repositories may be used with an HTTP server other than \ind{\sqsrc}. For example, a simple configuration\footnote{\url{http://www.visoracle.com/squeak/faq/monticello-1.html}} turns \ind{Apache} into a \Mont repository with restricted access rights:

\begin{code}{}
"My apache2 install worked as a Monticello repository right out of the box on my
RedHat 7.2 server.  For posterity's sake, here's all I had to add to my apache2 config:"
Alias /monticello/ /var/monticello/
<Directory /var/monticello>
  DAV on
  Options indexes
  Order allow,deny
  Allow from all
  AllowOverride None
  # Limit write permission to list of valid users.
  <LimitExcept GET PROPFIND OPTIONS REPORT>
    AuthName "Authorization Realm"
    AuthUserFile /etc/monticello-auth
    AuthType Basic
    Require valid-user
  </LimitExcept>
</Directory>
"This gives a world-readable, authorized-user-writable Monticello repository in
/var/monticello.  I created /etc/monticello-auth with htpasswd and off I went.
I love Monticello and look forward to future improvements."
\end{code}

\index{Monticello!FTP repository}
\paragraph{FTP.} This is similar to an HTTP repository, except that it uses an FTP server instead. An FTP server may also offer restricted access right and different FTP clients may be used to browse such \Mont repository.

\index{Monticello!GOODS repository}
\paragraph{GOODS.}
This repository type stores versions in a GOODS object database.
GOODS is a fully distributed object-oriented database management system that uses an active client model\footnote{\url{http://www.garret.ru/goods.html}}.
It's a read-write repository, so it makes a good ``working'' repository where versions can be saved and retreived. Because of the transaction support, journaling and replication capabilities offered by GOODS, it is suitable for large repositories used by many clients.  

\index{Monticello!directory repository}
\index{package-cache}
\paragraph{Directory.} A directory repository stores versions in a directory in the local file system. Since it requires very little work to set up, it's handy for private projects; since it requires no network connection, it's the only option for disconnected development. The \ct{package-cache} we have been using in the exercises for this chapter is an example of this kind of repository. Versions in a directory repository may be copied to a public or shared repository at a later time. \sqsrc supports this feature by allowing package versions (.mcz files) to be imported for a given project. Simply log in to \sqsrc, navigate to the project, and click on the \menu{Import Versions} link.

\paragraph{Directory with Subdirectories.}  A ``directory with subdirectories'' is very similar to ``directory'' except that it looks in subdirectories to retrieve list of available packages. Instead of having a flat directory that contains all package versions, such as repository may be hierarchically  structured with  subdirectories.

\index{Monticello!SMTP repository}
\paragraph{SMTP.} SMTP repositories are useful for sending versions by mail. When creating an SMTP repository, you specify a destination email address. This could be the address of another developer\,---\,the package's maintainer, for example\,---\,or a mailing list such as pharo-project. Any versions saved in such a repository will be emailed to this address.  SMTP repositories are write-only.

\paragraph{Programmatically adding repositories} For particular purposes, it may be necessary to programmatically add new repositories. This happens when managing configuration and large set of distributed monticello packages or simply customizing the entries available in the Monticello browser. For example, the following code snippet programmatically adds new directory repositories

\begin{code}
| repo |
{'/path/to/repositories/project-1/'. 
'/path/to/repositories/project-2/'. 
'/path/to/repositories/project-3/'. } do: 
[ :path |
	repo := MCDirectoryRepository new directory: 
		(FileDirectory on: path).
	MCRepositoryGroup default addRepository: repo ].
\end{code}

%\paragraph{\sqmap Release} This is a write-only repository used for publishing releases of a package to \sqmap. To configure the repository enter the name of the package on \sqmap, your \sqmap initials and your \sqmap password. Now any versions saved to the repository will be uploaded to your \sqmap account, and registered as a new release with \sqmap.  

%You need an account on \sqmap to add packages. New accounts may be created online\footnote{\url{http://map.squeak.org}}. The list of packages you add will appear on the \sqmap package list displayed when \sqmap is open.

%\alex{I added few sentences about adding package on squeakmap. But not many people are using it now. It is progressively being replaced by universe.}
%\ab{We havn't talked bout how to put stuff on \sqmap, only about how to load it.  So, we should add that material, either here, or in a subsequent subsection.}

%\paragraph{\sqmap Cache} When packages are installed via \sqmap, downloaded files are stored in a dedicated subdirectory in your working directory that acts as a cache. In order to make these files available to \Mont for loading, merging, \etc, a \sqmap Cache repository is created when these files are loaded for the first time.

%\on{why don't you first talk about the package cache?  I see you commented it out.}
%\alex{Because it does not appear in \pharo anymore. Maybe it was buggy or simply obsolete}

%\ab{One thing to say is how to get to versions in some \emph{other} package cache}
%\alex{I am not sure about what you mean.}. 

%\paragraph{package-cache}

%The package cache is a special repository that \Mont creates automatically. Like a directory repository, the package cache stores files in a directory on your local filesystem. See Elements of \Mont for more information.

%\ab{what about ``Directory with Subdirectories''?} 
%\alex{I added a new paragraph on it above}

%-----------------------------------------------------------------
\subsection{Using \sqsrc}

\indmain{\sqsrc} is a online repository that you can use to store your \Mont packages. An instance is running and accessible from \url{http://www.squeaksource.com}. At the time this chapter is being written, over 1500 projects are registered on \sqsrc and nearly 2000 people have an account. \figref{squeaksource} shows the main web page.  

\begin{figure}[ht]\centering
	\includegraphics[width=\textwidth]{squeaksource2}
	\caption{\sqsrc, the online \Mont code repository.\figlabel{squeaksource}}
\end{figure}

\dothis{Use a web browser to visit the \pbe project at
  \url{http://www.squeaksource.com/PharoByExample.html}. This project
  contains the Lights Out project from the first volume of this book.
  In the registration section on that web page you should see this
  \emph{repository expression:}}

\begin{code}{}
MCHttpRepository
    location: 'http://www.squeaksource.com/PharoByExample'
    user: ''
    password: ''
\end{code}
\noindent
\emph{Add this repository to \Mont by clicking \button{+Repository},
  and then selecting \menu{HTTP}. Fill out the template with the URL
  corresponding to the Lights Out project\,---\,you can copy the above
  repository expression from the web page and paste it into the
  template. Since you are not going to commit new versions of this
  package, you do not need to fill in the user and password.
  \button{Open} the repository, select the latest version and click
  \button{Load}.}

Pressing the \link{Register Member} link on the \sqsrc home page will probably be your first step if you do not have a \sqsrc account. 
Once you are a member, \link{Register Project} allows you to create a new project. 


\begin{figure}[ht]\centering
	\includegraphics[width=\textwidth]{squeaksourcesetting}
	\caption{Repositories under \sqsrc are highly configurable.\figlabel{squeaksourcesetting}}
\end{figure}

\Mont offers a large range of options (cf. \figref{squeaksourcesetting}) to configure a project repository: tags may be assigned, a license may be chosen, access for people who are not part of the project may be restricted (read/write, read, no access), emails may be sent upon commits, mailing list may be managed, and users may be defined to be members of the project (as administrator, developer, or guest).

% \on{This section seems too painfully obvious to include.}
%\paragraph{Troubleshooting} Not being able to remotely save your package may be due to a large range of causes. You will have to run through the usually network debugging routine:

%\begin{enumerate}
%\item Is your computer turned on?
%\item Is your network cable plugged in?
%\item Can you reach \url{http://squeaksource.com/} with a web browser from the same host as the image?
%\item Are people on IRC (or any other forum) ranting about \sqsrc being down again?
%\item Is \sqsrc accessible from other sites?\footnote{A handy link to check this is: \url{http://downforeveryoneorjustme.com/squeaksource.com}}
%\item Can you ping \url{http://squeaksource.com/} from Pharo?:\\
%   \ct{Socket pingPorts: #(80) on: 'squeaksource.com' timeOutSecs: 30}.
%\item Can you load packages using \Mont from \\\url{http://squeaksource.com/}?
%\end{enumerate}



%=================================================================
\section{The .mcz file format}

\index{Monticello!version}
\index{Monticello!mcz format}
\index{mcz format|see{Monticello, mcz format}}
Versions are stored in repositories as binary files.
These files are commonly call ``mcz files'' as they carry the extension .mcz.
This stands for ``\Mont zip'' since an mcz file is simply a zipped file containing the source code and other meta-data.

\important{An mcz file can be dragged and dropped onto an open image file, just like a change set. \pharo will then prompt you to ask if you want to load the package it contains. \Mont will not know which repository the package came from, however, so do not use this technique for development.}

% \sd{but you lose the fact that this is a package so don't do that.}

You may try to unzip such a file, for example to view the source code directly, but normally end users should not need to unzip these files themselves.
If you unzip it, you will find the following members of the mcz file.

\paragraph{File contents}
Mcz files are actually ZIP archives that follow certain conventions. Conceptually a version contains four things:

\begin{itemize}
\index{Monticello!package}
\item \emph{Package}. A version is related to a particular package. Each mcz file contains a file called ``package'' that contains information about the package's name.

\item \emph{VersionInfo}. This is the meta-data about the snapshot. It contains the author initials, date and time the snapshot was taken, and the ancestry of the snapshot. Each mcz file contains a member called ``version'' which contains this information.

A version doesn't contain a full history of the source code. It's a snapshot of the code at a single point in time, with a UUID identifying that snapshot, and a record of the UUIDs of all the previous snapshots it's descended from.

\index{Monticello!snapshot}
\item \emph{Snapshot}. A Snapshot is a record of the state of the package at a particular time. Each mcz file contains a directory named ``snapshot/''. All the members in this directory contain definitions of program elements, which when combined form the Snapshot. Current versions of \Mont only create one member in this directory, called ``source.st''.

\index{Monticello!dependency}
\item \emph{Dependencies}. A version may depend on specific version of other packages. An mcz file may contain a ``dependencies/'' directory with a member for each dependency. These members will be named after each package the \Mont package depends upon. For example, a \ct{Pier-All} mcz file will contains files named \ct{Pier-Blog} and \ct{Pier-Caching} in its dependencies directory.
\end{itemize}

\paragraph{Source code encoding}

The member named ``snapshot/source.st'' contains a standard fileout of the code that belongs to the package.

\paragraph{Metadata encoding}

The other members of the zip archive are encoded using S-expressions. Conceptually, the expressions represent nestable dictionaries. Each pair of elements in a list represent a key and value. For example, the following is an excerpt of a ``version'' file of a package named \ct{AA}:

\ct{(name 'AA-ab.3' message 'empty log message' date '10 January 2008' time '10:31:06 am' author 'ab' ancestors ((name 'AA-ab.2' message...)))}

It basically says that the version \ct{AA-ab.3} has an empty log message, was created on \ct{January 10, 2008}, by \ct{ab}, and has an ancestor named \ct{AA-ab.2}, ...

%\ab{butif it does?}

%\paragraph{Distributing mcz files}

%The metadata for a version ends up being fairly compact, so it's not unreasonable to distribute it with a release. \ab{What's a release?} It's also important that it be present if somebody decides to start hacking on your package \on{?}. Then they can create a mcz with their version of your package and it will have the correct ancestry information, enabling you to easily and correctly merge it back into your work.

%Stated another way,  So it's a perfect thing to distribute. \ab{This paragraph I understand.  Maybe just delete the previous one?  What is the connection with the paragraph title?}

%=================================================================
\section{Chapter Summary}

%This chapter explained how to use \Mont and \sqsrc to manage the source code of your application.

This chapter has presented the functionality of \Mont in detail.
The following points were covered:

\begin{itemize}
\item \Mont are mapped to Smalltalk categories and method protocols.
	If you add a package called \ct{Foo} to \Mont, it will include all classes in categories called \ct{Foo} or starting with \ct{Foo-}. It will also include all methods in those categories, except those in protocols starting with \ct{*}. Finally it will include all \emph{class extension} methods in protocols called \ct{*foo} or starting with \ct{*foo-} anywhere else in the system.

\item When you modify any methods or classes in a package, it will be marked as ``dirty'' in \Mont, and can be saved to a repository.

\item There are many kinds of repositories, the most popular being HTTP repositories, such as those hosted by \sqsrc.

\index{package-cache}
\item Saved packages are caches locally in a directory called \ct{package-cache}.

\item The \Mont \RI can be used to browse a repository. You can select which versions of packages to load or unload.

\item You can create a new \emph{branch} of a package by basing a new version on another version which is earlier than the latest version. The \RI keeps track of the ancestry of packages and can tell you which versions belong to separate branches.

\item Branches can be \emph{merged}. \Mont offers a fine degree of control over the resolution of conflicts between merged versions. The merged version will have as its ancestor the two versions it was merged from.

\item Alternatively, selected changes of a branch can be \emph{backported} to an arbitrary earlier version. This will create a new version that can be merged with any other version that needs those changes. The original backported branch remains independent in this case.

\item \Mont can keep track of dependencies between packages. When a package with dependencies to required packages is saved, a new version of that package is created, which then depends on the latest versions of all the required packages.

\item If classes in your packages have class-side \ct{initialize} methods, then \ct{initialize} will be sent to those classes when your package is loaded. This mechanism can be used to perform various checks or start-up actions. A particularly useful application is to add new instance variables to classes for which you are defining extension methods.

\item \Mont stores package versions in a special zipped file with the file extension \ct{.mcz}. The mcz file contains a snapshot of the complete source code of that version of your package, as well as files containing other important metadata, such a package dependencies.

\item You can drag and drop an mcz file onto your image as a quick way to load it.

%\item \Mont is used to manage your packages. \Mont is part of the Pharo standard distribution.

%\item \sqsrc is a remote online central repository in which you can store your code.

%\item \Mont packages are mapped to categories. 

%\item After having created a package in \Mont, you may store it in a remote repository such as that offered by \sqsrc.

%\item Merging and backporting are feature provided by \Mont that allow changes to cross different version line and to move along a same package version line.

%\item \Mont allows a package version different from the local version to be merged, creating a new branch and changing the local version of your package.
%\on{confusing sentence --- not sure what you want to say}
%\ab{ditto}
\end{itemize}

%ACK: Dale Henrichs
%=================================================================

%\section{Packages in \Mont: PackageInfo}

%The PackageInfo system is a simple, lightweight way of organizing Smalltalk source: it is nothing more than a naming convention, which uses (or abuses) the existing categorization mechanisms to group related code. Let me give you an example: say that you are developing a framework named PharoLink to facilitate using relational databases from Pharo. You will probably have a series of categories to contain all of your classes (e.g., category \cat{PharoLink-Connections} containing the classes \ct{OracleConnection}, \ct{MySQLConnection} and \ct{PostgresConnection})
%(\cat{PharoLink-Model} containing \ct{DBTable}, \ct{DBRow}  and \ct{DBQuery}) and so on. But not all of your code will reside in these classes\,---\,you may also have, for example, a series of methods to convert objects into an SQL friendly format: \mthind{Object}{asSQL},  \mthind{String}{asSQL} and \mthind{Date}{asSQL}.

%These methods belong in the same package as the classes in \cat{PharoLink-Connections} and \cat{PharoLink-Model}. You mark this by placing those methods in a method category (of \ct{Object}, \ct{String}, \ct{Date}, and so on) named \cat{*squeaklink} (note the initial star). The combination of the \cat{PharoLink-...} categories and the \cat{*squeaklink} method categories forms a package named "PharoLink".

%
%\section{Getting Started}

%\paragraph{Installing}

%

%\paragraph{Creating a Working Copy}

%The first thing you need to do is tell \Mont about the package you are interested in versioning. You do this by creating a Working Copy.

%\paragraph{From an .mcz version file}
%Open a FileList and navigate to the version file. Click on the 'Load' button to load the package into your image.

%\paragraph{From scratch}

%Click on the '+Package' button, and enter the name of a PackageInfo package. It doesn't matter whether or not the code for the package already exists.

%Once the Working Copy has been created, the name of the package will appear in the package list on the left side of the \MCB. If you loaded an existing version, the version name will be displayed in parenthesis after the package name, otherwise the parenthesis will be empty, indicating that your working copy has no ancestors.

%\paragraph{Connecting to a Repository}

%If you've already got a Working Copy, click on the package name on the left side of the \MCB, so that your repository will be associated with your package. To connect to a repository, click on the '+Repository' button in the \MCB. A pop-up menu will appear, allowing you to select the type of repository you want to connect to.

%The simplest repository type is 'directory.' When you select this type of repository, \Mont will open a FileList2 to allow you to select an existing directory in which to store versions. Other types of repositories typically require more configuration, and will open a text pane to allow you to enter it.

%\paragraph{Saving Changes}

%Changes to your working copy are automatically logged in your changes file, so you only need to create a new version of your package when you want to share the changes with others. Select the package on the left side of the \MCB and the repository to save to on the right, then click the 'Save' button. See Repositories for discussion of how to publish to shared repositories.

%\paragraph{Merging Changes}

%If you or some other developer have made changes to the same version of a package, load one version as your working set and then select the repository containing the other version in the \MCB, open a Repository Inspector and select the other version. Clicking the 'Merge' button will automatically load all non-conflicting changes from the other version. If you need to control which changes to accept, you may instead click 'Changes' to browse every difference.

%

%\section{Elements of \Mont}

%\paragraph{Packages}

%Packages are the units of versioning used by \Mont; the classes and methods they contain are recorded and versioned together. \Mont uses the packages defined by PackageInfo.

%\paragraph{Snapshots}

%A Snapshot is the state of a Package at a particular point in time

%\paragraph{Versions}

%A Version is a Snapshot of a Package and it's associated metadata\,---\,author initials, the date and time the snapshot was taken, and the Version's ancestry\,---\,the list of Versions from which it is derived.

%A Version is the standard currency of the system. You save them, load them, give them to others, merge them, delete... you get the picture. Versions are often stored in mcz files\,---\,see File Format

%\paragraph{Working Copies}

%Each package in an image that is being versioned with \Mont has a Working Copy. The Working Copy represents the Version of the package that is currently active in the image, and which may be modified by the Smalltalk development tools.

%\paragraph{Repositories}

%These are places to store your Versions. Unlike CVS, in which a Package is associated with one Repository, a \Mont Package can have Versions in many repositories. When adding a new Repository to use, you can choose from \sqmap Cache, FTP, HTTP (webdav), \sqmap Release, SMTP, or a directory somewhere on your hard drive (or network drive).

%For example, if I have six versions of package Foo, I could have Foo versions 1-4 being on my local harddrive, and 5-6 being on an ftp server. You could download version 5, make some changes and commit a new version (7) to your WebDAV repository. I can download and merge that version with my own work to produce version 8, which I save to my ftp repository.

%This is a key element of \Mont's distributed development model.

%\paragraph{Package cache}

%The package-cache is a local repository the \Mont uses to cache any package that is loaded into a particular image in a directory. That means it is filled with .mcz files, whether it is a package you create in your image, or one you download from somewhere else.

%When you use images in different directories you will have multiple package-caches, and may hold many of the same packages. If \Mont is loaded into an image which is subsequently moved, \Mont will continue to use the package-cache in the directory the image was moved from. Otherwise \Mont creates a new package-cache in the local directory. This can become a real mess and so some have used symlinks on unix systems to centralize it.

%\paragraph{Why cache packages at all?}

%When a Version is loaded into the image, it is likely to become the ancestor of new versions that are created as part of the development process. During merges, \Mont needs to examine the Snapshots of these ancestors in order to detect conflicts. By caching these ancestors as it loads them, \Mont reduces the chance that the necessary version will be unavailable\,---\,either because the repository it's in is no longer available or because it was loaded directly from a file and isn't in any repository.

%\section{The Snapshot Browser}

%The Snapshot browser is much like the standard Smalltalk Browser except that it displays the contents of a Snapshot, rather than the code that is active in the image. Since Snapshots are immutable, the Snapshot browser does not allow editiing.

%One difference between the Snapshot Browser and the familiar browsers is that the Snapshot browser uses the special category '*Extensions' to categorize classes that do not belong to the package, but which have extension methods that do.
%
%\section{More on PackageInfo}
%To get a feel for this, try filing the Refactoring Browser. The Refactoring Browser code uses PackageInfo's naming conventions, using "Refactory" as the package name. In a workspace, create a model of this package with  \ct{refactory := PackageInfo named: 'Refactory'}. 

%It is now possible to introspect on this package; for example, refactory classes will return the long list of classes that make up the Refactoring Browser. refactory coreMethods will return a list of MethodReferences for all of the methods in those classes. refactory extensionMethods is perhaps one of the most interesting queries: it will return a list of all methods contained in the Refactory package but not contained within a Refactory class. This includes, for example, \mthind{String}{expandMacrosWithArguments:} and \mthind{Behavior}{parseTreeFor:}.

%Since the PackageInfo naming conventions are based on those used already by \pharo, it is possible to use it to perform analysis even of code that has not explicitly adapted to work with it. For example, (PackageInfo named: 'Collections') externalSubclasses will return a list of all Collection subclasses outside the Collections categories.

%You can send fileOut to an instance of PackageInfo to get a changeset of the entire package. For more sophisticated versioning of packages, see the \Mont project.

%=========================================================
\ifx\wholebook\relax\else
   \bibliographystyle{jurabib}
   \nobibliography{scg}
   \end{document}
\fi
%=========================================================


% $Author$
% $Date$
% $Revision$

% HISTORY:
% 2008-08-19 - Stef started chapter (outline only)

%=================================================================
\ifx\wholebook\relax\else
% --------------------------------------------
% Lulu:
	\documentclass[a4paper,10pt,twoside]{book}
	\usepackage[
		papersize={6in,9in},
		hmargin={.75in,.75in},
		vmargin={.75in,1in},
		ignoreheadfoot
	]{geometry}
	\input{../common.tex}
	\setboolean{lulu}{true}
% --------------------------------------------
% A4:
%	\documentclass[a4paper,11pt,twoside]{book}
%	\input{../common.tex}
%	\usepackage{a4wide}
% --------------------------------------------
    \graphicspath{{figures/} {../figures/}}
	\begin{document}
\fi
%=================================================================
%\renewcommand{\nnbb}[2]{} % Disable editorial comments
\sloppy
%=================================================================
\chapter{Debugging }\label{cha:basic}


\on{Much of this is already in the Exceptions and Reflection chapters}
\sd{no and you can trust me on what I want to say}


\section{Halt}

\section{Conditional Halt}

\ct{InputState shiftPressed ifTrue: [ self halt ]}

\ct{self haltIf: [ ... particular test run ... ]}

\section{Halt Once}

\section{Pointer Finder}

\section{Inspector Explorer}

% \section{Message Tallly} -- in Profiling chapter

\section{BreakPoints}

\section{Chasing Undeclared References}
Smalltalk cleanOutUndeclared. 

\section{VM logging}





%=================================================================
\ifx\wholebook\relax\else\end{document}\fi
%=================================================================

%-----------------------------------------------------------------

%%% Local Variables:
%%% coding: utf-8
%%% mode: latex
%%% TeX-master: t
%%% TeX-PDF-mode: t
%%% ispell-local-dictionary: "english"
%%% End:
% $Author$
% $Date$
% $Revision$

% HISTORY:
% 2006-12-07 - Stef started chapter
% 2009-02-12 - Stef added examples
% 2010-02-24 - Alexandre begins the translation 

%=================================================================
\ifx\wholebook\relax\else
% --------------------------------------------
% Lulu:
	\documentclass[a4paper,10pt,twoside]{book}
	\usepackage[
		papersize={6.13in,9.21in},
		hmargin={.75in,.75in},
		vmargin={.75in,1in},
		ignoreheadfoot
	]{geometry}
	\input{../common.tex}
	\pagestyle{headings}
	\setboolean{lulu}{true}
% --------------------------------------------
% A4:
%	\documentclass[a4paper,11pt,twoside]{book}
%	\input{../common.tex}
%	\usepackage{a4wide}
% --------------------------------------------
    \graphicspath{{figures/} {../figures/}}
	\begin{document}
	% \renewcommand{\nnbb}[2]{} % Disable editorial comments
	\sloppy
\fi
%=================================================================
%\chapter{Profiling applications}

%\on{Some of this is now in the Reflection chapter}

%Profiling applications is not an obvious topics. Here we present a simple tutorial on using 
%\ct{MessageTally}. We thanks Andreas Raab for the original version of this tutorial.

%\sd{we re rewriting it}

%\on{needs a case study / running example}

%\sd{We should have a look at the example in VW profiler --- impact of String concatenation vs Stream usage}

%How to improve 
%\begin{code}{}
%	Collection>>select:thenCollect:
%	Collection>>select:thenDo:
%	Collection>>collect:thenSelect:
%	
%	Here are some optimized implementations: 

%	#select:thenDo: apply to Collection
%	#select:thenCollect: apply to OrderedCollection
%	#collect:thenSelect: apply to OrderedCollection

%	select:thenCollect
%	==================
%	" Unoptimized version results ---> between 1951 - 2005 ms"
%	| coll |
%	coll := #(1 2 3 4 5 6 7 8) asOrderedCollection. 
%	[ 100000 timesRepeat:[
%	   ( coll select:[:each | each > 5] ) collect:[:i | i * i]
%	  ]
%	] timeToRun

%	" Optimized version results ---> between 1229 - 1289 ms "
%	| coll |
%	coll := #(1 2 3 4 5 6 7 8) asOrderedCollection. 
%	[ 100000 timesRepeat:[
%	    coll select:[:each | each > 5] thenCollect:[:i | i * i]
%	  ]
%	] timeToRun

%	select:thenDo:
%	==============
%	" Unoptimized version results ---> between 3496 - 3573 ms"
%	" Optimized version results ---> between 2488 - 2619 "

%	coll := #(1 2 3 4 5 6 7 8 9 10 11 12 13 14 15 16 17 18 19 20).
%	[ 100000 timesRepeat:[
%	        coll select: [ : i | i even] thenDo: [ : i | i * i ]
%	     ]
%	] timeToRun  

%	collect:thenSelect:
%	===================
%	" Unoptimized version results ---> between 1678 - 1691 ms"
%	| coll |
%	Smalltalk garbageCollect.
%	coll := #(1 2 3 4) asOrderedCollection.   
%	[ 100000 timesRepeat:[
%	   ( coll collect:[:i | i * i]) select:[:each | each > 5]
%	   ]
%	] timeToRun  

%	" Optimized version results ---> between 974 - 979"
%	| coll |
%	coll := #(1 2 3 4) asOrderedCollection.   
%	[ 100000 timesRepeat:[
%	    coll collect:[:i | i * i] thenSelect:[:each | each > 5]
%	   ]
%	] timeToRun

%	
%	
%	
%	
%	
%	
%benchButton
%	| browser button canvas time |
%	browser := OBPackageBrowser openOnClass: Object selector: #yourself.
%	browser position: 0@0.
%	button := browser allMorphs detect: [:m | (m
%	isKindOf:PluggableButtonMorph ) and: [ m label = 'browse' ]].

%	canvas := World assuredCanvas.
%	time := [1000 timesRepeat: [ button fullDrawOn: canvas ]] timeToRun.
%	browser delete.
%	^ time
%\end{code}

%\section{MessageTally}

%The primary tool to measure performance, both for Squeak in general  is \ct{MessageTally}. \ct{MessageTally} acts on a particular expression (\ct{MessageTally spyOn:["your expression here"]}) and provides
%the following information:

%\begin{enumerate}
%\item a) A hierarchy showing how much time was spent where in the computation.
%\item b) A list showing which amount of time was spent in which leaf nodes
%\item c) Memory statistics, incl. the growth rate, garbage collection info etc.
%\end{enumerate}

%MessageTally uses a technique known as "pc-sampling". What that means is
%that a high-priority process is started which (based on a timer) samples
%the call stack of the process and allocates a time value (typically
%whatever it's using for sampling).

%It is important to notice that this is a statistical measure - given a
%large enough number of samples, the reported result will be
%statistically valid. On the other hand, small numbers of samples are
%typically statistically invalid - a single garbage collection can lead
%to a major change in an otherwise insignificant part of the computation.

%Here is an example:

%\ct{MessageTally spyOn:[100 raisedTo: 1000].}

%resulted in the following output:

%\begin{verbatim}
%**Tree**
%100.0% {3ms} SmallInteger(Number)>>raisedTo:
%100.0% {3ms} SmallInteger(Number)>>raisedToInteger:
%50.0% {2ms} LargePositiveInteger>>*
%50.0% {2ms} primitives
%\end{verbatim}

%These results claim that we're spending 50\% of the overall time in
%Multiplying large integers. Which is completely bogus since we have only
%two samples (the default sampling rate is 1ms). If we run this for a
%longer period of time, like here:

%\ct{MessageTally spyOn:[1000 timesRepeat:[100 raisedTo: 1000]]}

%\begin{verbatim}
%**Tree**
%100.0% {1007ms} SmallInteger(Number)>>raisedTo:
%99.9% {1006ms} SmallInteger(Number)>>raisedToInteger:
%96.8% {975ms} primitives
%3.0% {30ms} LargePositiveInteger>>*
%\end{verbatim}

%We see that indeed, we only spend roughly 3\% in multiplying large
%integers. The other 97\% are spent in "primitives" which, unfortunately,
%are not broken out separately in the measures (however, if such a
%measure is critical, then the primitives can to be factored into
%separate methods which then call the primitives themselves - this allows
%message tally to "see" the method frames and report the usage accordingly).

%In a more complex situation, the percentage tree is typically useful to
%figure out roughly the areas in which time is spent which can then be
%measured individually.

% \subsection{**Leaves**}

%
%The **Leaves* section reported by MessageTally is the amount of time spent in a
%method WITHOUT the time spent in the methods called from that method. In
%our above example the leaves are reported as:

%\begin{verbatim}
%**Leaves**
%96.8% {975ms} SmallInteger(Number)>>raisedToInteger:
%\end{verbatim}

%which is the overall time spent in \ct{Number>>raisedToInteger: (1006ms)}
%minus the time spent in \ct{LargePositiveInteger>>* (30ms)}. If a method
%shows up in the leaves it typically means that this method is
%computationally expensive or just gets called a large number of times.

% \subsection{**Memory**}

%
%The **Memory** statistics shown in MessageTally provide information
%about how various memory regions have changed:

%\begin{enumerate}
%\item old: Describes the "old space" in memory. This is the portion that
%will not be included in incremental garbage collection but only during a
%full garbage collection. See also the "tenure" information below.
%Extensively growing old space typically means that there is a problem
%with the allocation patterns or garbage collector settings.

%\item young: Describes the "young space" in memory, e.g., the region handled
%by the incremental garbage collector. Changes in young space are usually
%not relevant.

%\item used: Total amount of used memory.

%\item free: Total amount of unused memory.
%\end{enumerate}

%\subsection{**GCs**}

%The *GCs* statistics provide information about the garbage collector:

%\begin{enumerate}
%\item full: The number of "full" garbage collections and time spent in
%those. Automatic full garbage collections should be VERY rare, they are
%a sign that you're allocating huge amounts of memory repeatedly. Note
%that at times these garbage collections are manually triggered though
%(like in the checkpointing process) and a normal effect of the operation.

%\item incr: The number of "incremental" garbage collections and time spent
%in those. Generally speaking, IGCs should be quick (avg. < 2ms) and
%frequent (several times a second). However, the total time spent in IGCs
%should generally be less than 10\%, otherwise this is a sign of a problem
%with the allocation patterns. If the time spent in IGCs exceeds 25\%
%something is *definitely* wrong.

%\item tenures: Tenuring occurs when the number of surviving objects in young
%space exceeds a certain threshold. In this case, the young space
%boundary is increased (which adds to the size of "old space" mentioned
%above). Tenuring typically means that the working set of the application
%hasn't been reached. For example, in a space construction we would
%expect frequent tenuring until the space is fully constructed. Once the
%working set has been reached, tenuring should be rare to non-existent.
%Frequent tenuring in such cases means that the garbage collection
%parameters need to be adjusted.

%\item root table overflows: Root table overflows describe the (rare) case
%that the number of "roots" for the incremental garbage collector will
%overflow the internal table. This will force an immediate garbage
%collection plus tenuring. The measure is provided in order to be able to
%find such rare cases (which otherwise leave you wondering why the system
%is running full GCs all the time for no apparent reason)
%\end{enumerate}

%
%\subsection{Multiple processes}

%Historically, MessageTally measured and reported only the call stack of
%the current process. This had the disadvantage that if time was spent in
%a different process, it would be attributed to a bogus frame in current
%thread. For Croquet, I have changes this such that *all* processes are
%reported in order to be able to see "what else" is going on.

%For example, if we measure an expression like here:

%\ct{MessageTally spyOn:[(Delay forSeconds: 5) wait]}

%we will find that all of the time is reported here:

%\begin{verbatim}

%**Tree**
%99.5% {4975ms} ProcessorScheduler class>>startUp
%99.5% {4975ms} ProcessorScheduler class>>idleProcess
%\end{verbatim}

%The idle process is the process that is being activated when no other
%activity occurs (the implementation of the idle process requests the VM
%to sleep for a millisecond so that the VM isn't running a busy).
%Generally, time reported in idleProcess is time spent "doing nothing"
%(e.g., waiting for some activity).

%The other relevant system process that may show up is the finalization
%process. If the finalization process shows up, it means we're having a
%problem with too many weak references being finalized. This has been a
%*big* problem in the past, so keep an eye on it.

%=================================================================

\chapter{Optimizing Application}

Since the beginning of software engineering, programmers have faced issues related to application performance. Although there has been a great improvement on the programming environment to support better and faster development process, addressing performance issues when programming requires quite some dexterity.

Optimizing an application is not particularly difficult. The general idea is to make slow and frequently called methods either faster or less frequently called. Note that optimizing an application usually complexity the application. It is therefore recommended to optimize an application only when the requirements for it are well understood and addressed. In other term, you should optimize your application only when you are sure of what it is supposed to do. As Kent Beck famously formulated: 1 - Make It Work, 2 - Make It Right, 3 - Make It Fast.

\section{What does profiling mean?} 
Profiling an application is a term commonly employed that refers to obtaining dynamic information from a controlled program execution. The obtained information is intended to provide important hints on how to improve the program execution. These hints are usually numerical measurements, easily comparable from one program execution to another.

In this chapter, we will consider measurement related to method execution time and memory consumption. Note that other kind of information may be extracted from a program execution, in particular the method call graph.

It is interesting to observe that a program execution usually follows the universal 80-20 rule: only a few amount of the total amount of methods (let's say 20\%) consume the largest part of the available resources (80\% of memory and CPU consumption). Optimizing an application is essentially a matter of tradeoff therefore. In this chapter we will see how to use the available tools to quickly identify these 20\% of methods and how to measure the progress coming along the program enhancements we bring.

Experience shows that having unit tests is essential to ensure that we do not break the program semantics when optimizing it. When replacing an algorithm by another, we ought to make sure that the program still do what it is supposed to do.

%%%%%%%%%%%%
%%%%%%%%%%%%

\section{A simple example}

Consider the method \ct{Collection>>select:thenCollect:}. For a given collection, this method selects some element according to a predicate. It then returns a collection of applying a block function on each selected element. At the first sight, this behavior implies two runs over the collections: the one provided by the user of \ct{select:thenCollect:} then an intermediate one that contains the selected elements. However, this intermediate collection is not indispensable, since the selection and the function application can be done with only one run.

\paragraph{\ct{timeToRun}.} Profiling one program execution is usually not enough to fully identify and understand what has to be optimized. Comparing at least two different profiled executions is definitely more fruitful. The message \ct{timeToRun} may be sent to a bloc to obtain the time in milliseconds that it took to evaluate the block. In order to have a significant measurement, we need to ``amplify'' the profiling with a loop.

Here are some results:
\begin{code}{}
	| coll |
	coll := #(1 2 3 4 5 6 7 8) asOrderedCollection. 
	[ 100000 timesRepeat:[ ( coll select:[:each | each > 5] ) collect: [:i |i * i]]] timeToRun
	"Calling select:, then collect: ---> ~ 1951 - 2005 ms"

	| coll |
	coll := #(1 2 3 4 5 6 7 8) asOrderedCollection. 
	[ 100000 timesRepeat:[ coll select:[:each | each > 5] thenCollect:[:i |i * i]]] timeToRun
	"Calling select:thenCollect: ---> ~ 1229 - 1289 ms"
\end{code}

Although the difference between these two execution is only about few hundred of milliseconds, opting for one method instead of the other could significantly slow your application!

Let's scrutinize the definition of \ct{select:thenCollect:}. A naive and non-optimized implementation is found in \ct{Collection}. (Remember that \ct{Collection} is the root class of the Pharo collection library). A more efficient implementation is defined in \ct{OrderedCollection}, which takes into account the structure of an ordered collection to efficiently perform this operation.

\begin{code}{}
Collection>>select: selectBlock thenCollect: collectBlock
	"Utility method to improve readability."

	^ (self select: selectBlock) collect: collectBlock
\end{code}

\begin{code}{}
OrderedCollection>>select: selectBlock thenCollect: collectBlock
    " Utility method to improve readability.
	Do not create the intermediate collection. "

	| newCollection |
    newCollection := self copyEmpty.
    firstIndex to: lastIndex do:[:index |
		| element |
		element := array at: index.
		( selectBlock value: element ) 
			ifTrue:[ newCollection addLast: ( collectBlock value: element ) ]].
    ^ newCollection
\end{code}

As you have probably guessed already, other collection such as set and dictionary do not benefit from an optimized version. We leave as an exercise the cost of not having this optimization. No not forget to submit your contribution to Pharo if you come up with an optimized version of \ct{select:thenCollect:}.

%%%%%%%%%%%%
%%%%%%%%%%%%

\paragraph{\ct{bench}.} When sent to a block, the \ct{bench} message estimates how many times this block is evaluated per second. For example, the expression \ct{[ 1000 factorial ] bench} says that \ct{1000 factorial} may be executed approximately 350 times per second.

\begin{figure}
	\begin{center}
	\includegraphics[width=.8\linewidth]{MessageTallyOne}
	\caption{MessageTally en action.}
	\figlabel{MessageTallyOne}
	\end{center}
\end{figure}


\section{Code profiling in Pharo} 

The \ct{timeToRun} method is useful to tell how long an expression takes to be executed. But it is not really adequate to understand how the execution time is distributed over the computation triggered by evaluating the expression. Pharo comes with \ct{MessageTally}, a code profiler to precisely analyze the time distribution over a computation. 


\subsection{MessageTally}
\ct{MessageTally} is a implemented as a unique class having the same name. Using it is quite simple. A message \ct{spyOn:} needs to be sent to \ct{MessageTally} with a block expression as argument to obtained a detailed execution analysis. Evaluating \ct{MessageTally spyOn: ["your expression here"]} open a window that contains the following information:

\begin{enumerate}
\item a hierarchy list showing the methods executed with their associated execution time during the expression execution.

\item the method leaves of the execution. A method leave is a method implemented as a primitive. It does not execute any other method therefore.

\item statistic about the memory consumption and garbage collector involvement 

\end{enumerate}
Each of these points will be described later on.

\figref{MessageTallyOne} shows the result of the expression \ct{MessageTally spyOn: [20 timesRepeat: [Transcript show: 100 factorial printString]]}.
The message \ct{spyOn:} execute the provided block in a new process. The analysis focus on the process. The message \ct{spyAllOn:} relates all active processes during the execution. 

%Le message \ct{spyAt:on:} permet de s\'electionner le niveau de processus. Par exemple pour ne voir que le temps pris par l'expression utilisez \ct{MessageTally spyAt: 40 on: [20 timesRepeat: [Transcript show: 100 factorial printString]]}

A tool a bit less crude than \ct{MessageTally} is \ct{TimeProfileBrowser}. It shows the implementation of the executed method in addition (\figref{TimeProfiler}).  \ct{TimeProfileBrowser} understand the message \ct{spyOn:}.


\begin{figure}
	\begin{center}
	\includegraphics[width=.8\linewidth]{TimeProfiler}
	\caption{Le TimeProfiler utilise MessageTally et permet de consulter les m\'ethodes ex\'ecut\'ees.
	\ct{TimeProfileBrowser spyOn:  [20 timesRepeat: [Transcript show: 100 factorial printString]]}}
	\figlabel{TimeProfiler}
	\end{center}
\end{figure}


\subsection{Integration in the programming environment}
As shown previously, the profiler may be directly invoked by sending \ct{spyOn:} and \ct{spyAllOn:} to the \ct{MessageTally} class. It may be accessed through a number of additional ways.

\paragraph{Via the World menu.}
The World menu (obtained by clicking outside any Pharo window) offers some profiling facilities under the \ct{Debug} submenu (\figref{menu}). \ct{Start profiling all processes} creates a block from a text selection and invokes \ct{spyAllOn:}. The entry \ct{Start profiling UI} profiles the user interface process. This is quite handy when debugging an user interface!

\begin{figure}[h]
	\begin{center}
	\includegraphics[width=.6\linewidth]{menu}
	\caption{Acc\`es par le menu.}
	\figlabel{menu}
	\end{center}
\end{figure}



\paragraph{Via the Test Runner.}
As the size of an application grow, unit tests are usually becoming good candidate for code profiling. Running tests often is rather tedious when the time to run them is getting too long. The \ct{test runner} in \pharo offers a button \ct{Run Profiled} (\figref{testRunner}). 

Pressing this button runs the selected unit tests and generates a message tally report. 

\begin{figure}[h]
	\begin{center}
	\includegraphics[width=.8\linewidth]{testRunner}
	\caption{Acc\`es par le TestRunner}
	\figlabel{testRunner}
	\end{center}
\end{figure}


\section{Read and interpret the results} 
The message tally profiler essentially provides two kind of information:
\begin{itemize}
\item execution time is represented using a tree representing the profiled code execution (\ct{**Tree**}. Each node of this tree is annotated with the time spend in each leave method (\ct{**Leaves**}). 

\item memory activity contains the memory consumption (\ct{**Memory**} and the garbage collector usage (**GC**).
\end{itemize}

For illustration purpose, consider the following scenario: the string character \ct{'A'} is cumulatively appended 9 000 times to an initial empty string.

%Prenons le code suivant qui ajoute la chaine \ct{'A'} 9000 fois dans une chaine. Nous r\'ep\'etons ce code afin de prendre en compte la cr\'eation de la premi\`ere chaine. \alex{I do not get this, to take into account the creation of the first string?? The string '' is allocated at the compilation, not during the profiling. }

\begin{code}{}
MessageTally spyOn: 
     [ 500 timesRepeat: [
                     | str |  
                     str := ''. 
                     9000 timesRepeat: [ str := str, 'A' ]]].
\end{code} 

The complete result is:

\begin{footnotesize}
\begin{sf}
 - 19915 tallies, 19928 msec.

**Tree**
--------------------------------
Process: (40s)  1175: nil
--------------------------------
19.8% {3946ms} ByteString(SequenceableCollection)>>copyReplaceFrom:to:with:
  |14.0% {2790ms} primitives
  |5.9% {1176ms} ByteString class(String class)>>new:
7.7% {1534ms} primitives

**Leaves**
52.7% {10502ms} SmallInteger(Integer)>>timesRepeat:
14.0% {2790ms} ByteString(SequenceableCollection)>>copyReplaceFrom:to:with:
8.8% {1754ms} UndefinedObject>>DoIt
7.7% {1534ms} ByteString(SequenceableCollection)>>,
5.9% {1176ms} ByteString class(String class)>>new:

**Memory**
	old			+0 bytes
	young		+2,803,380 bytes
	used		+2,803,380 bytes
	free		-2,803,380 bytes

**GCs**
	full			0 totalling 0ms (0.0% uptime)
	incr		4500 totalling 923ms (5.0% uptime), avg 0.0ms
	tenures		0
	root table	0 overflows
\end{sf}
\end{footnotesize}

The first line gives the overall execution time and the number of samplings (also called \emph{tallies}, we will come back on sampling at the end of the chapter). 

\subsection{**Tree**: Cumulative information}

The \ct{**Tree**} section represents the execution tree per processes. The tree tells the time the \pharo interpreter spent in each method. It also tells  the different invocation using a call graph. Different execution flows are kept separated according to the process in which they have been executed. The process priority is also displayed, this helps distinguishing between different processes. The example tells:

\begin{sf}
\begin{small}
**Tree**
--------------------------------
Process: (40s)  1175: nil
--------------------------------
19.8% {3946ms} ByteString(SequenceableCollection)>>copyReplaceFrom:to:with:
  |14.0% {2790ms} primitives
  |5.9% {1176ms} ByteString class(String class)>>new:
7.7% {1534ms} primitives
\end{small}
\end{sf}

This tree shows that 19.8\% of the total execution time is spent in the method \ct{SequenceableCollection>>copyReplaceFrom:to:with:}. This method is called when concatenating character strings using the message comma (\ct{,}), itself indirectly invoking \ct{new:} and some virtual machine primitives.

The execution takes 19.8\% of the execution time, this means that the interpreter effort is shared with other processes. The invocation chain from the code to the primitives is relatively short. Reaching hundreds of nested calls is no exception for most of applications. We will optimize this example later on.


\subsection{**Leaves**: leaf methods}

The \ct{** Leaves**} part 

@@HERE 

La partie **Leaves** repr\'esente le temps pass\'e dans une m\'ethode
\emph{sans} le temps pass\'e dans les m\'ethodes appel\'ees par
celle-ci. Dans l'exemple pr\'ec\'edent, il apparait:

\begin{small}
\begin{sf}
**Leaves**
52.7% {10502ms} SmallInteger(Integer)>>timesRepeat:
14.0% {2790ms} ByteString(SequenceableCollection)>>copyReplaceFrom:to:with:
8.8% {1754ms} UndefinedObject>>DoIt
7.7% {1534ms} ByteString(SequenceableCollection)>>,
5.9% {1176ms} ByteString class(String class)>>new:
\end{sf}
\end{small}


Toutes les m\'ethodes n'apparaissent pas dans la section
**Leaves**. Une m\'ethode apparaissant dans cette section signifie
soit que la m\'ethode r\'ealise beaucoup d'op\'erations, soit qu'elle
est appell\'ee de nombreuses fois par d'autres m\'ethodes.
Ici on voit que la m\'ethode \ct{copyReplaceFrom:to:with:} qui prenait
pr\`es de 20\% du temps total prend seulement 14\% pour son ex\'ecution
propre. Regardez la d\'efinition de la m\'ethode pour voir qu'elle
fait effectivement plus que quelques envois de messages. 


 \subsection{**Memory**}

La partie statistique m\'emoire fournit des informations \`a propos
des changements observ\'es dans la m\'emoire. 
Pour comprendre ces informations, il faut savoir que le
ramasse-miettes (garbage collector) de Pharo est
un scavenging GC dont le principe est bas\'e sur la
remarque qu'un objet vieux \`a de moins fortes chances de ne plus
\^etre r\'ef\'erenc\'e et que les objets jeunes pour leur part sont
souvent rapidement d\'er\'ef\'erenc\'es. Ainsi plusieurs zones
m\'emoires sont consid\'er\'ees et une migration de l'espace des
objets jeunes vers les objets vieux est possible lorsqu'un objet
jeune a surv\'ecu quelques GC --- Il est promu (``tenured''). Ce qui veut dire,
par analogique aux universitaires am\'ericain, qu'il a obtenu un poste
fixe. 

Le profiler montre alors quatre \'etats de la m\'emoire :

\begin{enumerate}
\item old: d\'ecrit les "objets anciens". Cette valeur repr\'esente la
  partie qui n'est pas incluse dans le ramasse miettes incr\'emental
  (incremental GC),  mais seulement durant un ramasse miettes complet
  (full GC). Cet espace m\'emoire est li\'e au m\'ecanisme de tenure. Une large croissance de cet espace m\'emoire indique
  clairement un probl\`eme car cela veut dire que de nombreux objets  jeunes ont \'et\'e promus dans la m\'emoire vielle.

\item young: d\'ecrit l'espace "jeune" en m\'emoire. C'est la partie
  qui est scann\'ee par le  ramasse miettes incr\'emental. En
  g\'en\'eral les changements dans cet espace sont tr\`es fr\'equent
  et donc cette information est peu utile.

\item used: repr\'esente la taille de m\'emoire totale  utilis\'ee.

\item free: repr\'esente la taille de m\'emoire inutilis\'ee.
\end{enumerate}

Dans notre exemple, il n'y a pas de nouveaux objets dans la partie "vielle". Il y a
2 803 380 octets utilis\'es par le processus dans l'espace jeune. Il y a
donc au total 2 803 380 octets utilis\'es par l'ex\'ecution de
l'expression et donc autant d'octets occup\'es (-2 803 380 octets).

\subsection{**GCs**}

La partie statistique **GCs** fournit des informations sur le
ramasse-miettes lui-m\^eme. Les deux derni\`eres informations sont
pour des experts. 

\begin{enumerate}
\item full: repr\'esente le nombre de ramasse-miettes complets et le
  temps pass\'e. Avoir des ramasses-miettes complets reste en
  g\'en\'eral tr\'es rare. Ils sont souvent dus \`a l'allocation
  r\'ep\'et\'ee de gros blocs de m\'emoire. 

\item incr: repr\'esente le nombre de ramasse-miettes incrementaux. Le
  temps pass\'e par les ramasses-miettes incr\'ementaux sont rapides
  et souvent inf\'erieurs \`a 2ms en moyenne. De plus leurs lancements
  est fr\'equent `a savoir plusieurs fois par seconde. Cependant 
le temps total pass\'e en ramasse-miettes incr\'ementaux doit \^etre g\'en\'eralement inf\'erieur \`a 10\% autrement, c'est signe qu'il y a un probl\`eme. Si le temps est sup\'erieur \`a 25\%, il y a vraiment un probl\`eme.

\item tenures: La promotion d'objets jeunes en objet vieux est
  d\'eclench\'ee lors que l'espace des objects jeunes arrive a un
  seuil limite d'occupation. Dans ce cas, la taille de l'espace des
  jeunes est pass\'ee dans l'espace des vieux objets ce qui correspond
  au ``old'' pr\'ec\'edent. La promotion d'objets jeunes  est souvent
  le signe que votre application n'a pas atteint son stade de
  croisi\'ere --- que tous les objets n\'ecessaires n'ont pas pas
  \'et\'e cr\'e\'es et r\'ef\'erenc\'es. En g\'en\'eral apr\`es une
  phase de promotions, une application fait rarement des phases de
  promotions. 

\item root table overflows: La table des racines repr\'esente un
  ensemble d'objets \`a partir duquel les GC sont lanc\'es. Ce chiffre
  d\'ecrit les rares cas o\`u le nombre de "racines" destin\'ees au
GC incremental d\'epasse la table interne de racine. Ce cas force un
GC incr\'emental ainsi qu'une phase de promotion. Ce nombre est
affichait afin que vous puissiez comprendre les rares cas o\u` cela
peut arriver --- le syst\`eme faisant alors des GC complets sans
raison apparente. 
\end{enumerate}


Dans l'exemple on voit que seul le GC incr\'emental est utilis\'e. 
Comme nous allons le voir il est int\'eressant de mesurer la
quantit\'e d'objets cr\'ees car cela peut avoir des incidences sur les
performances.






\section{Illustrons une analyse}
Comprendre les r\'esultats du profiler est la premi\`ere \'etape pour
optimiser, cependant comme vous pouvez le voir il n'est pas simple de
comprendre si un algorithme est dispendieux. Nous montrons maintenant
au travers d'exemples comment la comparaison de l'ex\'ecution de
diff\'erentes expressions peut faire jaillir de la connaissance.

%La figure \ref{workspace} montre effectivement trois m\'ethodes de concat\'enation de chaines execut\'ees 9000 fois afin d'obtenir des r\'esultats plus caract\'eristiques. Attention toutefois dans le cas d'ajout d'\'el\'ements dans un dictionnaire par exemple. En effet les dictionnaires (comme toutes autres collections) ont ce que l'on peut appeler des tailles optimales pour lesquelles ils sont plus efficaces.

%\begin{figure}
% 	\begin{center}
% 	\includegraphics[width=.8\linewidth]{workspace}
% 	\caption{Acc\`es par lignes de code}
% 	\figlabel{workspace}
% 	\end{center}
% \end{figure}

L'utilisation de la m\'ethode "\ct{,}" est connue pour \^etre lente car
elle cr\'ee une nouvelle chaine r\'esultante de la
concat\'enation. Nous allons donc regarder comment l'utilisation d'une
Stream peut am\'eliorer les performances. Nous pouvons par exemple
comparer l'utilisation des m\'ethodes nextPut: et nextPutAll: ainsi
que mesurer l'impact de la pr\'eallocation des chaines r\'esultantes. 

\paragraph{Utilisation d'une Stream.}
Commen\c cons par utiliser une Stream. Alors que l'on pourrait croire
que la cr\'eation d'une Stream doit \^etre couteuse au point de ne pas
apporter un b\'en\'efice. En fait le r\'esultat du profiler est
\'eloquent, la nouvelle expression qui cr\'e\'e une Stream puis y
ajoute des caract\`eres est quasiment 10 fois plus rapide. Ceci est
compr\'ehensible car lorsque nous concat\'enons 9000 fois une chaine,
nous cr\'eons 8999 chaines intermediates et copions leur contenu alors
qu'avec une Stream nous ajoutons simplement un caract\`ere dans la
Stream \`a chaque it\'eration. 

\begin{code}{}
MessageTally spyOn: 
     [ 500 timesRepeat: [
                     | str |  
                     str := WriteStream on: (String new). 
                     9000 timesRepeat: [ str nextPut: $A ]]].
\end{code}

\begin{code}{}
 - 1790 tallies, 1790 msec.

**Tree**
--------------------------------
Process: (40s)  1175: nil
--------------------------------
39.1% {700ms} Character>>isOctetCharacter
16.2% {290ms} primitives

**Leaves**
39.1% {700ms} Character>>isOctetCharacter
22.1% {396ms} UndefinedObject>>DoIt
16.2% {290ms} WriteStream>>nextPut:
12.3% {220ms} SmallInteger(Integer)>>timesRepeat:

**Memory**
	old			+0 bytes
	young		+53,260 bytes
	used		+53,260 bytes
	free		-53,260 bytes

**GCs**
	full			0 totalling 0ms (0.0% uptime)
	incr		1140 totalling 198ms (11.0% uptime), avg 0.0ms
	tenures		0
	root table	0 overflows
\end{code}

\paragraph{Utiliser nextPutAll: \`a la place de nextPut:}
Nous allons maintenant \'etudier l'impact de la m\'ethode utilis\'ee
pour ajouter des \'el\'ements dans la Stream. En effet, la m\'ethode
\ct{nextPut: aCharacter} ajoute un caract\`ere. Essayons avec la m\'ethode
\ct{nextPutAll: aString}.

\begin{code}{}
MessageTally spyOn: 
    [ 500 timesRepeat: [
                    | str |  
                    str := WriteStream on: (String new). 
                    9000 timesRepeat: [ str nextPutAll: 'A' ]]].
\end{code}

On pourrait penser qu'ajouter un
caract\`ere est plus rapide qu'ajouter une chaine compos\'ee du m\^eme 
caract\`ere mais l'analyse de cette solution nous montre que notre
hypoth\`ese est fausse. On est plus rapide: 1610 ms contre 1790 ms
et ce m\^eme en occupant le double de m\'emoire. 

\begin{code}{}
 - 1617 tallies, 1618 msec.

**Tree**
--------------------------------
Process: (40s)  1175: nil
--------------------------------
50.8% {822ms} primitives

**Leaves**
50.8% {822ms} WriteStream>>nextPutAll:
20.3% {328ms} SmallInteger(Integer)>>timesRepeat:
18.5% {299ms} UndefinedObject>>DoIt

**Memory**
	old			+0 bytes
	young		+99,216 bytes
	used		+99,216 bytes
	free		-99,216 bytes

**GCs**
	full			0 totalling 0ms (0.0% uptime)
	incr		1139 totalling 190ms (12.0% uptime), avg 0.0ms
	tenures		0
	root table	0 overflows
\end{code}

\paragraph{Influence de la pr\'eallocation de la chaine.}
En Smalltalk, l'utilisation d'\ct{OrderedCollection} sans
pr\'e-allocation de la taille finale de la collection est connue pour
\^etre une op\'eration couteuse. En effet, \`a chaque fois que la
collection est pleine et doit grandir il faut copier une partie de la
collection. Maintenant regarder si la pr\'eallocation de la chaine
sur laquelle la stream va op\'erer a un impact. On utilise alors le
message \ct{new: aNumber} \`a la place de \ct{new}.


\begin{code}{}
MessageTally spyOn: 
    [ 500 timesRepeat: [
                    | str |  
                    str := WriteStream on: (String new: 10000). 
                    9000 timesRepeat: [ str nextPutAll: 'A' ]]].
\end{code}

\paragraph{Une exp\'erience.}
L'expression que l'on profile a clairement un impact sur le
r\'esultat. A titre d'exemple si l'on remplace les deux nombres 9000 par 500,
 on obtient des r\'esultats int\'eressants. Faites cette
manipulation sur les deux premi\`eres expressions. 
On obtient un facteur 2,7 (5100 ms contre 1850 ms) au lieu d'un facteur 10
entre la concat\'enation bas\'ee sur la m\'ethode \ct{,} et
l'utilisation d'une stream. On voit alors l'importance de
connaitre la longueur des chaines manipul\'ees. 

Notez que de la m\^eme mani\`ere la validit\'e du r\'esultat d\'epend
aussi de la dur\'ee d'\'echantillon. Comme le profiler utilise un
technique d'analyse de haut de pile (PC-Sampling), il est important de
s'assurer que l'on fait tourner l'expression de mani\`ere suffisante
pour que la probabilit\'e que l'\'echantillon en haut de pile soit significatif.





\section{Comptons les messages}
Il est aussi possible d'avoir un rapport d\'etaill\'e non plus bas\'e
sur l'\'echantillonnage de la pile d'ex\'ecution mais en
interpr\'etant le programme. En utilisant le message \ct{tallySends:},
on obtient ainsi une figure exacte des messages ex\'ecut\'es. La
\figref{sendTally} montre le r\'esultat obtenu en ex\'ecutant
l'expression suivante \ct{MessageTally tallySends:[ 1000 timesRepeat:  [3.14159 printString]]}:

\begin{figure}
	\begin{center}
	\includegraphics[width=.8\linewidth]{sendTally}
	\caption{Tous les messages execut\'es lors d'une ex\'ecution.}
	\figlabel{sendTally}
	\end{center}
\end{figure}

Faire un tallySend: prend usuellement plus de temps car cette m\'ethode utilise un
interpr\`ete de bytecode. 

%%%%%%%%%%%%
%%%%%%%%%%%%


\section{Fibonacci M\'emorisant}
Comme exercice nous vous proposons d'\'etudier l'impact d'une
m\'emorisation des calculs interm\'ediaires dans le cas de la suite de
fibonacci dont la d\'efinition est $fib (n) = fib (n-1) + fib(n-2)$
avec $fib(1)=1, fib(2)=1$.

Donnons d'abord une d\'efinition non m\'emorisante.
\begin{code}{}
Integer>>fibSlow
	self assert: self >= 1.
	(self == 1) ifTrue: [ ^1].
	(self == 2) ifTrue: [ ^1].
	^ (self - 1) fibSlow + (self - 2) fibSlow
\end{code}

Maintenant la version m\'emorisante stocke dans un cache (une collection ordonn\'ee) 
lorsque les r\'esultats ne sont pas connus  et utilise
ce cache lors des calculs. Au vu de la d\'efinition de fibonacci le
cache empeche donc de calculer une formule sur deux. Les caches peuvent
avoir bien plus d'impact lorsque le domaine
le permet. 
      
\begin{code}{}
Integer>>fib
	self assert: self >= 1.
	^ Self fibWithCache: OrderedCollection new.

Integer>>fibLookup: cache
	^ cache at: self ifAbsentPut: [ self fibWithCache: cache ] 

Integer>>fibWithCache: cache
	(self == 1) ifTrue: [ ^1].
	(self == 2) ifTrue: [ ^1].
	^ ((self - 1) fibLookup: cache) + ((self - 2) fibLookup: cache)
\end{code}

Profilez \ct{1200 fibSlow} et \ct{1200 fib} pour voir l'impact d'un tel
cache. Vous pouvez aussi mesurer l'impact de la pr\'eallocation de la
taille de la collection en utilisant \ct{OrderedCollection new: self}.

\begin{code}{}
29.8% {1843ms} OrderedCollection>>at:ifAbsentPut:
11.4% {705ms} OrderedCollection>>size
7.5% {464ms} SmallInteger(Integer)>>fibLookup:
4.8% {297ms} OrderedCollection>>at:put:
4.5% {278ms} SmallInteger(Integer)>>fibWithCache:
4.2% {260ms} OrderedCollection>>at:
2.2% {136ms} LargePositiveInteger>>+
2.1% {130ms} OrderedCollection>>makeRoomAtLast
2.0% {124ms} OrderedCollection>>addLast:
1.9% {118ms} OrderedCollection>>add:
1.9% {118ms} LargePositiveInteger(Integer)>>+
\end{code}

Il est int\'eressant de voir que la pr\'eallocation n'a que peu
d'effet et que la m\'ethode \ct{makeRoomAtLast} n'\'etant ex\'ecut\'ee
qu'une seule fois par calcul de \ct{fib} ne prend que peu de temps de
calcul. Par contre on peut voir que la m\'ethode \ct{at:ifAbsentPut:} 
prend une part importante. Nous allons donc proposer et \'evaluer une
 nouvelle impl\'ementation. Nous changeons donc la collection
 ordonn\'ee par un tableau pr\'e-allou\'e et nous d\'efinissons une
 nouvelle m\'ethode d'acc\'es au cache. 



\begin{code}{}
fib2
	self assert: self >= 1.
	^ self fibWithCache2: (Array new: self).

fibLookup2: cache
	|res|
	res := cache at: self.
	^ res ifNil: [cache at: self put: (self fibWithCache2: cache) ]
		
fibWithCache2: cache
 	(self == 1) ifTrue: [ ^1].
 	(self == 2) ifTrue: [ ^1].
 	^ ((self - 1) fibLookup2: cache) + ((self - 2) fibLookup2: cache)
\end{code}

Nous avons obtenus une diff\'erence int\'eressante qui illustre 
que
l'exp\'erimentation et la mesure sont la base de l'optimisation. 

\begin{code}{}
[1200 fib2] bench  '559.288142371526 per second.'
[1200 fib] bench  '191.2470023980816 per second.'
\end{code}



--------
\section{Consommation de m\'emoire par classe: SpaceTally}

Il est parfois important de connaitre le nombre d'instances d'une
classe ou sa consommation m\'emoire. La classe SpaceTally offre cette
fonctionalit\'e. 

\ct{SpaceTally new printSpaceAnalysis} montre la consommation
m\'emoire de la classe au niveau de ses m\'ethodes, du nombre
d'instances et de la m\'emoire utilis\'ees par les instances. Il n'est pas
surprenant de voir que les chaines et les m\'ethodes compil\'ees
prennent 30\% de la m\'emoire en Pharo.

\begin{code}{}
Class                           code space # instances  inst space percent
ByteString                           2217       91946       6325763    26.0
CompiledMethod               21186       60807       3704137    15.2
Bitmap                                  3893         319       3685532    15.1
Array                                     2478       96671       3015172    12.4
ByteSymbol                        920       40109       1009703     4.1
\end{code}

Vous pouvez aussi ex\'ecuter cette fonctionnalit\'e sur une s\'election de classes: 
\begin{code}{}
((SpaceTally new spaceTally: (Array with: TextMorph with: Point)) 
	asSortedCollection: [:a :b | a spaceForInstances > b spaceForInstances]) 
\end{code}



\section{Quelques conseils pour finir}

Nous vous avons montr\'e comment utiliser le profiler et montr\'e
quelques approches comme la comparaison de deux analyses pour
d\'eterminer une impl\'ementation plus judicieuse. L'utilisation d'un
cache est une technique tr\`es int\'eressante quand le domaine s'y
pr\^ete. Voici quelques conseils pour appr\'ehender des optimisations:
Ne commencez pas par optimiser les feuilles mais essayez de comprendre
l'algorithme dans son ensemble. Consid\'erez d'autres fa\c cons
d'obtenir le m\^eme r\'esultat.  Exploitez la m\'eta-information de
l'algorithme.
Consid\'erez les caract\'eristiques d'ex\'ecution des structures de
donn\'ees. Ainsi dans un graphe cyclique si on utilise une
collection ordonn\'ees ou une liste pour stocker les \'el\'ements d\'ej\`a
visit\'es, chaque ex\'ecution va potentiellement parcourir un grand nombre de fois
la liste pour savoir si l'\'el\'ement y est. Utiliser un
ensemble offre d\'ej\`a un temps d'acc\`es plus raisonable. Utiliser
un dictionaire peut \^etre aussi une solution lorsque les \'el\'ements ont
une bonne distribution de leur hash. 

N'oubliez pas de penser \`a la m\'emoire consomm\'ee lors de
l'algorithme. En effet ce n'est pas parce que le ramasse-miette peut
absorber la cr\'eation d'objets temporaires que le stresser est
anodin. Penser \`a la cr\'eation inutile de collection
interm\'ediaire.  En effet, ce n'est pas parce que Smalltalk poss\`ede
une superbe biblioth\`eques d'iterateurs qu'enchainer des select: et
collect: n'implique pas de multiples parcours de collections
ainsi que la g\'en\'eration de collections inutiles. 

\section {How MessageTally is implemented?}

%\bibliographystyle{alpha}
%{\small
%\bibliography{scg,lse}
%}



%=================================================================

\ifx\wholebook\relax\else\end{document}\fi
%=================================================================

%-----------------------------------------------------------------


% $Author$
% $Date$
% $Revision$

% HISTORY:
% 2008-01-19 - Stef started
% 2008-12-26 - Jannick Menanteau added text

%=================================================================
\ifx\wholebook\relax\else
% --------------------------------------------
% Lulu:
	\documentclass[a4paper,10pt,twoside]{book}
	\usepackage[
		papersize={6.13in,9.21in},
		hmargin={.75in,.75in},
		vmargin={.75in,1in},
		ignoreheadfoot
	]{geometry}
	\input{../common.tex}
	\setboolean{lulu}{true}
% --------------------------------------------
% A4:
%	\documentclass[a4paper,11pt,twoside]{book}
%	\input{../common.tex}
%	\usepackage{a4wide}
% --------------------------------------------
    \graphicspath{{figures/} {../figures/}}
	\begin{document}
\fi
%=================================================================
%\renewmessage{\nnbb}[2]{} % Disable editorial comments
\sloppy
%=================================================================
\chapter{Installing Packages with Installer }\chalabel{installer}

\on{Wouldn't it be enough just to add a few examples to the Monticello chapter? Does this really deserve a whole chapter on its own?}

This chapter presents the Installer package developed by Keith Hodges which greatly simplifies the loading of code in \pharo. Its official web page is at: \url{http://installer.pbwiki.com/Installer}.
Installer provides a simple Domain Specific Language for installing packages from different formats such as the Monticello package system or SqueakMap. It supports a large number of Squeak dialects, therefore it should work if you are using  \pharo or a Squeak distribution. 
It defines an infrastructure to define build scripts. It allows one to manage several sources code systems such Monticello packages, Squeakmap Catalogues, Sake, Universes or any pieces of Squeak code in files or web format. Now in the context of this book we will focus on Monticello package loading since this is the most convenient and used format and the one chosen in \pharo. 


%In my case, I use a Squeakmap script when I load a new Squeak image. This script loads all packages I want from Monticello automatically.
%\sd{really a squeakmap script?}

Most of the time people define simples script that loads all the packages they need to load automatically from Monticello. 
Installer offers a lot of possibilities and we will present its main API and functionality. 

\section{How to get Installer}
To download  the last version and install it, just load it from Monticello.
In Monticello, you need to add a new HTTP repository and put the following lines as parameters. After that, you can load \ct{Installer} as explain the the chapter dedicated to \ct{Monticello}.
\begin{code}{}
MCHttpRepository
	location: 'www.squeaksource.com/Installer'
	user: 'squeak'
	password: 'squeak'
\end{code}

You can also use \ct{ScriptLoader new installInstaller}.


\section{Installer Basic  API}
\sd{je ne regarderais pas les trucs commons car on ne sait pas si cela marche. Donc focus sur les truc simple 
et surtout sur MC.}

The class \ct{Installer} is a class factory of specific installers. Installer is based on a set of classes each 
supporting a common protocol but also offering a specific protocol for each format. The specific installers share
 a common API and each installer offers a specific protocol to install the format they can install. 

Here is an example to get a monticello  and a squeakmap installer.
\begin{code}{}
	| mc squeakmap | 
	mc := Installer monticello.
	squeakmap := Installer squeakmap.
\end{code}

\subsection{Most Important Messages}
When one uses Installer, some messages are the same if one uses monticello, squeakmap or others. These messages are basics but 
allow one to define most of the scripts.  In this section, common and most important messages to install a package are explained and illustrated with examples. Installing a package is specifying what to load and sending the message \ct{install} to such configured object: for that we have install/install: and addPackage:. 

\paragraph{Installing a package.} 
\begin{description}
\item \ct{addPackage: aPackageName} \quad
It adds a package to the collection of packages to be installed. When no version number is specified, the latest version of the package is loaded.
For example, here is the script to load. Note that Installer offers a large API and that for example \ct{addPackage:} is strictly equivalent to the message \ct{package:}  We suggest to only use \ct{addPackage:}. 

\begin{code}{}
	| monticello | 
	monticello := Installer monticello 
		http: 'http://www.squeaksource.com/eCompletion'.
	monticello addPackage: 'ECompletion'.	
\end{code}
Note that it does not install yet the package. for this you should use the \ct{install} message shown below.
When scripting Monticello you have to either pass the full path including the project in which the package is or
use the message \ct{project:} as show the two equivalent scripts below. 

\begin{code}{}
	| monticello | 
	monticello := Installer monticello 
		http: 'http://www.squeaksource.com/BreakOut'.
	monticello addPackage: 'BreakOut'.
	monticello install
\end{code}


\begin{code}{}
	| monticello | 
	monticello := Installer monticello 
		http: 'http://www.squeaksource.com/'.
	monticello project: 'BreakOut'.
	monticello addPackage: 'BreakOut'.
	monticello install
\end{code}


It is possible to chain the loading of several of packages by chaining multiple \ct{addPackage:} message, but pay attention that they should
be in the same project. 


It is possible to specify to load an exact version. When no version number is specified the latest version is loaded.
packages. 
\begin{code}{}
	| monticello | 
	monticello := Installer monticello 
		http: 'http://www.squeaksource.com/BreakOut'.
	monticello addPackage: 'BreakOut-sd.5'
\end{code}



\item \ct{install} 
It installs the packages defined with the message \ct{addPackage:}. Note that the message \ct{installQuietly} install the package without any warning message which could be important when you have running headless images. 
	
The following script shows how to load the elements \ct{OmniBrowser}  published on SqueakSource. When no version number is given, the installer loads the last version of the package.

%\begin{code}{}
%	| squeakmap | 
%	squeakmap := Installer squeakmap.
%	squeakmap package: 'DynamicBindings'.
%	squeakmap install.
%\end{code}

\begin{code}{}
	| monticello | 
	monticello := Installer monticello 
		http: 'source.lukas-renggli.ch/omnibrowser'.
	monticello package: 'OmniBrowser'.
	monticello install.
\end{code}


\item \ct{install: aPackageName} 

	The method \ct{install:} installs the corresponding package. Using this message is more compact than the messages \ct{addPackage:} followed by  \ct{install}. As before it is possible to perform a quiet installation using the message \ct{installQuietly:}.

\begin{code}{}
	| monticello | 
	monticello := Installer monticello 
		http: 'source.lukas-renggli.ch/omnibrowser'.
	monticello install: 'OmniBrowser';
		install: 'OB-Standard'.
\end{code}
\end{description}




%\sd{I would not stress squeakmap and use Monticello instead really few people use squeak map.}




\paragraph{Messages for SqueakMap and Monticello.} 
SqueakMap and Monticello have a specific user interface. It is possible to open this interface with \ct{Installer} by the use of message \ct{open}. 

They support version management. One can manage versions with Installer. The following messages are then specific to SqueakMap and Monticello.




\paragraph{Other messages.} 
\begin{description}
\item \ct{debug} 

It is also possible to specify whether a debugger should be open or not in case of problems. The messages \ct{debug} and \ct{noDebug} control the debugger opening

\begin{code}{}
	Installer debug.
\end{code}

And one can stop the debug mode. In such a case Installer switches in log mode. This mode don't shows a debugger, but informs the user by writting informations in a Transcript window.
\begin{code}{}
	Installer noDebug
\end{code}

Notice that methods \ct{debug} and \ct{noDebug} are class methods.
\end{description}

\paragraph{Class Factories.} 
Some class factories were used in examples above. Here is a list of some of them. 

\begin{code}{}
	Installer squeakmap.
	Installer monticello.
	Installer universe.
	Installer file.
	Installer web.
\end{code}
Each of these class methods creates a well-initialized installer. Each of these uses is explained in next sections.

\section{Scripting Monticello}
The installer supports scripting of Monticello packages using the message \mthind{monticello}{Installer class>>monticello} or using the shortcut message \mthind{mc}{Installer class>>mc}.

Since Monticello packages can be stored in http, ftp, magma, goods or directory databases we have to specify the database kind to load a monticello package. These are the repository options:

\begin{code}{}
	Installer monticello http: anUrl.
	Installer monticello http: anUrl user: name password: secret.
	Installer monticello ftp: host directory: dir user: name password: secret.
	Installer monticello magma: host port: aport
	Installer monticello goods: host port: aport.
	Installer monticello directory: stringOrFileDirectory.
\end{code}

\paragraph{Typical Usage.}
First you have to specify the repository using one of the messages presented above. Here as we want to access SqueakSource we use the  \mthind{http:}{http:} method. Then you should specify the squeak source project from which you will be loading your packages and finally send the message \mthind{install:}{install:} with the corresponding package name. Here you see that we request one specific version of \ct{'Comet'} while we will get the latest versions of the \ct{'Seaside'} and  \ct{'Scriptaculous'} packages. 

\begin{code}{}
	| monticello | 
	monticello := Installer monticello 
		http: 'source.lukas-renggli.ch/omnibrowser'.
	monticello install: 'OB-Morphic-lr.75'.
\end{code}

This code installs the version 75 of the package named \ct{'OB-Morphic'} posted by the author named lr.

Now SqueakSource is a common source of Monticello packages and Installer provides the \ct{'ss'} shortcut for squeaksource or \ct{'lukas'} for the lukas renggli sources repository. The previous script is equivalent to the following one: 

\begin{code}{}
 	Installer lukas project: 'omnibrowser';
    		install: 'OB-Morphic-lr.75'.
\end{code}

Here is a list of default repository shortcuts.

\begin{code}{}
	Installer ss. "squeaksource"
	Installer sf. "squeakFoundation"
	Installer lukas.
	Installer impara.
	Installer wiresong.
\end{code}

New shortcuts can be introduced using the message \ct{rememberAs:} as follows: 
\ct{Installer monticello http: 'http://www.myRepository.com'; rememberAs: #myrepo.}

\paragraph{Specifying a password.} 

Sometimes, to access to source code of a repository, user/password is required. One can specify this with the use of messages \ct{user:} and \ct{password:}. 

\begin{code}{}
Installer monticello 
      http: 'http://www.squeaksource.com';
      user: 'me'; password: 'asecret';
      install: 'something'.
\end{code}


\paragraph{Loading specific package version.} You can also specify a list of different authors of a package and the latest version available of the package will be installed. Installer takes the latest version of the selected authors and install it. 

\begin{code}{}
	Installer lukas project: 'omnibrowser';
		browse: #('OB-Tools-dkh' 
			'OB-Tools-EL' 
			'OB-Tools-damiencassou'). "either of these"
		install: 'OB-Tools-EL.60'.     "specific version"
\end{code}

In this example, when we have written the chapter, \ct{OB-Tools-dkh} has as latest version the 57, \ct{OB-Tools-EL} has 60 and \ct{OB-Tools-damiencassou} has 58. So in these three versions, Installer install the latedt, so the version \ct{OB-Tools-EL.60}.
\sd{I did not get it}

\paragraph{Getting Package information.} 

It is also possible to browse the packages without loading them in your image by using the message \ct{browse:}. 
\ct{browse} gives a snapshot of the content of the package (\figref{browse}).

\begin{code}{}
	| monticello | 
	monticello := Installer monticello 
		http: 'source.lukas-renggli.ch/omnibrowser'.
	monticello browse: 'OB-Morphic'.
\end{code}

\begin{figure}[ht]
	\begin{center}
	\includegraphics[width=1\linewidth]{browse}
	\caption{Result of message \ct{browse}}
	\figlabel{browse}
	\end{center}
\end{figure}


Moreover, one can obtain all information of the last version of a package by using the message \ct{view:}. These informations are the name, date of creation of the version, author, ancestors of the version and comment done by the author (\figref{view}).

\begin{code}{}
	| monticello | 
	monticello := Installer monticello 
		http: 'source.lukas-renggli.ch/omnibrowser'.
	monticello view: 'OB-Morphic'.
\end{code}

\begin{figure}[ht]
	\begin{center}
	\includegraphics[width=1\linewidth]{view}
	\caption{Result of message \ct{view}}
	\figlabel{view}
	\end{center}
\end{figure}

\paragraph{Providing automatic answers.} 
It is possible that a package request some information such as the default directory, admin and passwords or any configuration data. It is possible with Installer to specify answers to  questions that the package installation could ask. The message for this is \ct{answer:with:}

\begin{code}{}
Installer ss project: 'Seaside'.
	answer: 'myPassword' with: 'admin';
	install: 'Seaside2.8a1'
\end{code}

\paragraph{Looking for packages.} 
% %%%% move that elsewhere --- sd
% \sd{The following does not work and I do not understand why}
% 
% \begin{code}{}
% Installer ss packagesMatching: 'Moose*'. 
% Installer ss
% 	 match: 'Moose*'. 
% \end{code}
% 
% \begin{code}{}
% Installer monticello http: 'http://www.squeaksource.com'; 
% 	project: 'Moose';
% 	match: 'Moose*
% \end{code}
% 
% 


\begin{description}
\item \ct{open} This message opens the GUI of Monticello.

\begin{code}{}
	| monticello | 
	monticello := Installer monticello 
		http: 'www.squeaksource.com/Installer'.
	monticello open.
\end{code}



\item \ct{versions} 

The message \ct{versions} returns a collection of all versions of a squeakmap package. Each of them is a string made with the name of package and the version among parenthesis.

\begin{code}{}
	(Installer mc 
		http: 'http://www.squeaksource.com'; 
		project: 'Seaside'; 
		versions) explore. 
\end{code}

\end{description}


\paragraph{Make your own shortcut for repository.} 
You can save a shortcut for a new repository of monticello. For this, the message \ct{rememberAs:} create an acronym. 
\begin{code}{}
Installer monticello http: 'http://gjallar.krampe.se'; rememberAs: #gjallar.
\end{code}

After, you can use your shortcut like this:
\begin{code}{}
Installer gjallar install: 'Q2'.
\end{code}

% \paragraph{Abbreviated instantiation.} 
% 
% Installer offers shortcut messages for monticello. So one can use \ct{mc} instead of \ct{monticello}
% \begin{code}{}
% 	monticello := Installer mc.
% \end{code}

% \section{Sake}
% 
% \ja{to do}

%Sake-Packages n'a rien � voir avec MC. Sake-Packages est plus proche d'un syst�me comme Universes, apt-get ou mac ports. MC a pour r�le de faciliter le d�veloppement de projets en g�rant les diff�rentes versions et les branches. Un syst�me de package g�re quand � lui des distributions qui sont des ensembles de projets inter-d�pendants. Un syst�me de package a des fonctionnalit�s comme l'installation d'un package et des ses d�pendances, la mise � jour de la liste des packages disponibles (update) ainsi que la mise � jour des packages install�s (upgrade). Il doit aussi �tre capable de dire quels sont les packages install�s, quels sont ceux qui poss�dent une mise � jour installable. En th�orie, il doit aussi �tre capable de supprimer un package et tous les packages qui en d�pendent.


%Tasks are typically defined class side, but can be added to any class in
%the image. A typical task definition follows.

%MakeTheTea classSide >> #taskMakeTea
%SakeTask define: [ :task |
%	task dependsOn: { MakeTheTea taskBoilWater. }.
%	task if: [ Cup isEmpty ].
%	task action: [ Cup add: 'tea'; add: 'water'. ].
%].

%To run the task:
%MakeTheTea taskMakeTea run.

%"create a needed class"
%task dependsOn:{ (SakeTask class: Object) subclass: #Cup }
%"create a needed class"
%task dependsOn:{ (SakeTask class: Object)
%				subclass: #Cup category:'Demo'}
%task dependsOn:{ (SakeTask class: #Cup)
%				removeSelector: #tmp }
%task dependsOn:{ (SakeTask class: #Cup)
%				removeSelectorsMatching: '*' }
%task dependsOn:{ (SakeTask class: #Cup)
%				removeSelectorsMatching: '*' }
%task dependsOn:{ (SakeTask class: #Cup) ifExistsDo: [ :c | c rename:
%#Glass ]  }


%Notice that we have an unload block as well as a load block! This is a
%new and very experimental feature, we can now define unload scripts for
%each package, and Sake/Packages provides a way to capture and publish
%them for some or all image versions. Indeed this feature is barely
%tested, I mention it here simply to intoduce the concept.


%usage:
%PackagesSqueak310 new Seaside load.  or,  (PackagesSqueak310
%named:'Seaside') load.
%PackagesSqueak310 new Seaside unload. or,  (PackagesSqueak310
%named:'Seaside') unload.
%Packages/Sake is also supported by Installer in a manner similar to
%universes support, e.g.
%Installer sake addPackage: 'PackagesA'; addPackage: 'PackageB'; install.
%Addendum: Sake/Packages no longer uses blocks as standard, but the
%principle is the same.


\section{Loading from SqueakMap}
SqueakMap (SM) is a distributed catalog of source code artifacts: Monticello packages, SAR -- Squeak Archives a zip file-based format, changesets or even etoy projects.  Now the use of SqueakMap is declining because it is difficult to know if an catalog item will effectively work in a given version.  The community is using Monticello packages and Universes (see below) which acts as distribution of coherent set of packages.  SM lets us do the following actions: 
\begin{itemize}
\item Install simply Squeak packages. 
\item Upgrade an installed packages. 
\item Publish packages in SqueakMap which is immediately available for all to use.
\end{itemize}


\paragraph{Initializing Squeakmap.} 
First, we must update the SqueakMap cache. This allows us to have an updated list of packages present in the SqueakMap database. For this, execute the following code:
\begin{code}{}
Installer squeakmap update.
\end{code}

\paragraph{Loading a specific version of a package.} 
One can also specify a specific version of the catalog item by surrounding it with parentheses.
\begin{code}{}
	Installer sm install: 'Labby & Walker(17)'.
\end{code}
This code installs the version 17 of the catalog item named \ct{'Labby & Walker'}.

\paragraph{Getting package information.} 
We can obtain catalog item information such as for example its id, name, date of creation, download url... and many others.
\begin{code}{}
	Installer sm view: 'Labby & Walker(17)'.
\end{code}

Messages \ct{browse:} or \ct{view:} can be used to get informations of a catalog item. These two messages do the same thing  for Squeakmap.

% \paragraph{Abbreviated instantiation.} 
% Installer offers shortcut messages for squeakmap. So one can use \ct{sm} instead of \ct{squeakmap}
% \begin{code}{}
% 	squeakmap := Installer sm.
% \end{code}

We can see the difference \sd{which difference} in the SqueakMap GUI, the list of packages is composed by the name, and the window on the right of the list is the description with the fields \ct{author:}, \ct{name:}, \ct{summary:} and \ct{description:}.
\begin{figure}[ht]
	\begin{center}
 	\includegraphics[width=1\linewidth]{squeakmap}
	\caption{SqueakMap GUI} 
	\figlabel{squeakmap}
 	\end{center}
 \end{figure}


You can also look for squeakmap items as follows:
 \begin{description}
 \item \ct{match: aNamePatternString} This message returns the list of catalog item whose name is matching aPatternString. Any character is matched exactly except \ct{*} which represents any characters. \sd{what are the other magic characters?}
 	
 \begin{code}{}
 	(Installer squeakmap match: 'Labby*') explore.
 \end{code}
 
 The message \ct{explore} allows us to explore the list of concerned catalog item.
 
 \item \ct{search: aStringPattern}
 	This message searches the pattern string inside catalog item information. catalog item information can be its name, its author name, its summary and description. To match such as specific category one should use on the following strings as shown in the example \ct{author:}, \ct{name:}, \ct{summary:}, \ct{description:}. These properties are used as follows and are optional. 
 If there is only a string to search with no property, the search is made in the four categories.

\begin{code}{}
 	Installer sm search: 'author:*Bryant'. 
 	Installer sm search: 'name:Seaside'. 
 	Installer sm search: 'summary:*web components'. 
 	Installer sm search: 'description:*framework for building sophisticated web applications in Squeak*'. 
 	Installer sm search: 'seaside'.
\end{code}

Here, the last line searches all 'seaside' in all properties of all catalog item. The first lines searches all the catalog item whose author is matching 'Bryant'. The \ct{'*'} character means that a list of characters is matched. 
 Other lines search informations in categories name, summary and description. It is not possible to composed them.
\end{description}

The difference between \ct{match:} and \ct{search:} is that \ct{match:} searches catalog item name, and \ct{search:} searches in the catalog item descriptions.


\section{Universes}

Universes is a distribution of packages. The universe contains all packages that are known to load in the environment. In addition, package dependencies are resolved within the universe package. So, if one loads a package A which depends on a package B, the package B is automatically loads before. 
Universe looks like Monticello but for everyday work and dependency, the dependency resolution is too coarse-grained. 

For example, if one wants to download \ct{Pier} with some common plugins.
It depends on: Pier-OmniBrowser, Pier-Security, Pier-Documents, Pier-Tests, Magritte-Tests, Pier-Blog, Pier-Seaside, Pier-EditorEnh. So all of these packages are installed.

\begin{code}{}
Installer universe
	addPackage: 'Pier';
	install.
\end{code}

If one just wants to install the \ct{Pier-Blog} package, one don't need all of these packages. So, there are less dependancies. \ct{Pier-Blog} depends on RSRSS2 and Pier-Seaside packages. 
\begin{code}{}
Installer universe
	addPackage: 'Pier-Seaside';
	install.
\end{code}

\ct{install:} is always equivalent to \ct{addPackage:'pkg'; install}. For universes it makes sense to use the non abbreviated form, because it can process multiple packages in one install.
\begin{code}{}
Installer universe 
	addPackage: 'a pkg';
	addPackage: 'b pkg(1.1)';
	install.
\end{code}

\section{Others}
\paragraph{InstallerFile}
With Installer, you can install packages from script files.
The message file: needs the path of the file. (In example the path is '/MakeTestsGreen39.cs'). The message \ct{install} is used to install the script. 
\begin{code}{}
Installer file 
	file: '/MakeTestsGreen39.cs'; 
	install.
\end{code}

\paragraph{InstallerWeb}

There are some possibilities to install a package via an URL. The three codes after do the same thing, but they are decomposed differently.


\begin{code}{}
Installer web 
	url: 'http://wiki.squeak.org/squeak/uploads/5889/MakeTestsGreen39.cs'; 
	install.


Installer installUrl: 'http://wiki.squeak.org/squeak/uploads/5889/MakeTestsGreen39.cs'.


Installer web 
	url: 'http://wiki.squeak.org/squeak/uploads/5889/';
	install: 'MakeTestsGreen39.cs'.
\end{code}

%\paragraph{Embedded scripts.}
%Scripts embedded in html web pages are delimited by ...
%\begin{code}{}
%Installer web url: 'http://wiki.squeak.org/squeak/742'; install.
%\end{code}

%Custom markers for embedded scripts
%\begin{code}{}
%Installer web
%    url: 'http://wiki.squeak.org/742';
%    markers: 'beginning of script...end of script';
%    install.
%\end{code}
%Note: Scripts embedded in html or a swiki page may need to escape some entities.
%Supported entities are % \& \> \< * \" (see Installer-c-#entities)

\section{Conclusion}

\ja{to do}

%=========================================================
\ifx\wholebook\relax\else
   \bibliographystyle{jurabib}
   \nobibliography{scg}
   \end{document}
\fi

%=================================================================
\ifx\wholebook\relax\else\end{document}\fi
%=================================================================

%-----------------------------------------------------------------

%%% Local Variables:
%%% coding: utf-8
%%% mode: latex
%%% TeX-master: t
%%% TeX-PDF-mode: t
%%% ispell-local-dictionary: "english"
%%% End:

\part{Frameworks}
% $Author$
% $Date$
% $Revision$
% $Id$

% HISTORY:
% 2007-10-29 - Oscar started chapter
% 2007-11-30 - Oscar first draft
% 2007-12-07 - Orla Greevy reviewed
% 2007-12-09 - Lukas Renggli reviewed
% 2008-01-11 - Andrew revised
% 2009-04-17 - Fabrizio Perin reviewed
% 2009-04-18 - Jorge Ressia reviewed
% 2009-05-06 - Oscar converted to Pharo; fixed review comments

%=================================================================
\ifx\wholebook\relax\else
% --------------------------------------------
% Lulu:
	\documentclass[a4paper,10pt,twoside]{book}
	\usepackage[
		papersize={6.13in,9.21in},
		hmargin={.75in,.75in},
		vmargin={.75in,1in},
		ignoreheadfoot
	]{geometry}
	\input{../common.tex}
	\pagestyle{headings}
	\setboolean{lulu}{true}
% --------------------------------------------
% A4:
%	\documentclass[a4paper,11pt,twoside]{book}
%	\input{../co	mmon.tex}
%	\usepackage{a4wide}
% --------------------------------------------
    \graphicspath{{figures/} {../figures/}}
	\begin{document}
	% \renewcommand{\nnbb}[2]{} % Disable editorial comments
	\sloppy
\fi
%=================================================================
\chapter{Seaside by Example}
\label{cha:seaside}

%=================================================================

\ind{Seaside} is a framework for building web applications in Smalltalk.  It was originally developed by Avi Bryant \index{Bryant, Avi} in 2002; once mastered, Seaside makes web applications almost as easy to write as desktop applications.

Two of the better known applications built with Seaside are \ind{SqueakSource}\footnote{\url{http://SqueakSource.com}} and \ind{Dabble DB}\footnote{\url{http://DabbleDB.com}}.
Seaside is unusual in that it is thoroughly object-oriented: there are no XHTML templates, no complicated control flows through web pages, and no encoding of state in URLs. Instead, you just send messages to objects.  What a nice idea!

\section{Why do we need Seaside?}

Modern web applications try to interact with the user in the same way as desktop applications: they ask the user questions and the user responds, usually by filling in a form or clicking a button.
But the web works the other way around: the user's browser makes a request of the server, and the server responds with a new web page.
So \ind{web application development} frameworks have to cope with a host of problems, chief among them being the management of this ``inverted'' control flow. 
Because of this, many web applications try to forbid the use of the browser's ``back'' button due to the difficulty of keeping track of the state of a session. 
Expressing non-trivial control flows across multiple web pages is often cumbersome, and multiple control flows can be difficult or impossible to express.

% Seaside is a component-based framework that uses ``\ind{continuations}''\footnote{A \emph{continuation} represents ``the rest of the computation'' at any point in a computation. In Smalltalk, a continuation is just an object that captures the current state of the computation, and which can be resumed at any point.} to keep track of multiple points in the control flow of web applications. Continuations are managed automatically by Seaside, so web developers do not even have to be aware of the underlying machinery. It just works.

\index{Seaside!backtracking state}
\index{Seaside!transactions}
\index{Seaside!components}
Seaside is a component-based framework that makes web development easier in several ways.
First,  control flow can be expressed naturally using message sends.
Seaside keeps track of which web page corresponds to which point in the execution of the web application.
This means that the browser's ``back'' button works correctly.

Second, state is managed for you.
As the developer, you have the choice of enabling 
backtracking of state, so that navigation ``back'' in time will undo side-effects.
Alternatively, you can use the transaction support built into Seaside to prevent users from undoing permanent side-effects when they use the back button.
You do not have to encode state information in the URL\,---\,this too is managed automatically for you.

Third, web pages are built up from nested components, each of which can support its own, independent control flow.
There are no XHTML templates\,---\,instead valid XHTML is generated programmatically using a simple Smalltalk protocol.
Seaside supports Cascading Style Sheets (\ind{CSS}), so content and layout are cleanly separated.
\seeindex{Cascading Style Sheets}{CSS}

Finally, Seaside provides a convenient web-based development interface, making it easy to develop applications iteratively, debug applications interactively, and recompile and extend applications while the server is running.

%=================================================================
\section{Getting started}

The easiest way to get started is to download the ``Seaside \subind{Seaside}{One-Click Experience}'' from the Seaside \subind{Seaside}{web site}\footnote{\url{http://seaside.st}}.
This is a prepackaged version of Seaside 2.8 for \ind{Mac OSX}, \ind{Linux} and \ind{Windows}.
The same web site lists many pointers to additional resources, including documentation and tutorials.
Be warned, however, that Seaside has evolved considerably over the years, and not all available material refers to the latest version of Seaside.

% If you are feeling more adventurous, an alternative to the ``one-click'' image is to start with the latest \ind{\pharo web image}\footnote{\url{http://pharo-project.org/download}}, and install Seaside yourself by following the manual \subind{Seaside}{installation} instructions on the Seaside web site.

Seaside includes a web server; you can turn the server on, telling it to listen on port 8080, by evaluating \clsind{WAKom} \ct{startOn: 8080},
and you can turn it off again by evaluating \ct{WAKom stop}.
In the default installation, the default \subind{Seaside}{administrator login} is \lct{admin} and the default password is \lct{seaside}.
To change them, evaluate: \clsind{WADispatcherEditor} \ct{initialize}.
This will prompt you for a new name and password.


\begin{figure}[tbh]
\begin{center}
\includegraphics[width=\textwidth]{seasideStartup}
\caption{Starting up Seaside
}
\label{fig:seasideStartup}
\end{center}
\end{figure}

\dothis{Start the Seaside server and direct a web browser to \url{http://localhost:8080/seaside/}.}

\noindent
You should see a web page that looks like \figref{seasideStartup}.

\noindent
\dothis{Navigate to the \lct{examples{\go}counter} page. (\figref{counter})}


\begin{figure}[htb]
\begin{center}
\includegraphics[width=0.8\textwidth]{counter}
\caption{The counter.}
\label{fig:counter}
\end{center}
\end{figure}

\noindent
This page is a small Seaside application: it displays a \subind{Seaside}{counter} that can be incremented or decremented by clicking on the \link{++} and \link{--\,--} links.

\noindent
\dothis{Play with the counter by clicking on these links.
Use your browser's ``back'' button to go back to a previous state, and then click on \link{++} again.
Notice how the counter is correctly incremented with respect to the currently displayed state, rather than the state that the counter was in when you started using the ``back'' button.}

Notice the \subind{Seaside}{toolbar} at the bottom of the web page in \figref{seasideStartup}.
Seaside supports a notion of ``sessions'' to keep track of the state of the application for different users.
\button{New Session} will start a new session on the counter application.
\button{Configure} allows you to configure the settings of your application through a convenient web-interface.
(To close the \button{Configure} view, click on the \link{x} in the top right corner.)
\button{Toggle Halos} provides a way to explore the state of the application running on the Seaside server.
\button{Profiler} and \button{Memory} provide detailed information about the run-time performance of the application.
\button{XHTML} can be used to validate the generated web page, but works only  when the web page is publicly accessible from the Internet, because it uses the W3C validation service.
\index{Seaside!halos}

Seaside applications are built up from pluggable ``components''.
In fact, components are ordinary Smalltalk objects.
The only thing that is special about them is that they should be instances of classes that inherit from the Seaside framework class \ct{WAComponent}.
We can explore components and their classes from the \pharo image, or directly from the web interface using halos.

\begin{figure}[ht]
\begin{center}
\includegraphics[width=\textwidth]{counterHalos}
\caption{Halos}
\label{fig:counterHalos}
\end{center}
\end{figure}

\dothis{Select \button{Toggle Halos}. You should see a web page like \figref{counterHalos}.
At the top left the text \ct{WACounter} tells us the class of the Seaside component that implements the behavior of this web page. 
Next to this are three clickable icons. 
The first, with the pencil, activates a Seaside class browser on this class.
The second, with the magnifying glass, opens an object inspector on the currently active \ct{WACounter} instance. 
The third, with the coloured circles, displays the \ind{CSS} style sheet for this component. 
At the top right, the \link{R} and \link{S} let you toggle between the rendered and source views of the web page. 
Experiment with all of these links. 
Note that the \link{++} and \link{--} links are also active in the source view.
Contrast the nicely-formatted source view provided by the Halos with the unformatted source view offered by your browser.}

The Seaside class browser and object inspector can be very convenient when the server is running on another computer, especially when the server does not have a display, or if it is in remote place. 
However, when you are first developing a Seaside application, the server will be running locally, and it is easy to use the ordinary \pharo development tools in the server image.

\begin{figure}[ht]
\begin{center}
\includegraphics[width=0.7\textwidth]{haltingCounter}
\caption{Halting the counter}
\label{fig:haltingCounter}
\end{center}
\end{figure}

\dothis{Using the object inspector link in the web browser, open an inspector on the underlying Smalltalk counter object and evaluate \ct{self halt}.
The web page will stop loading.
Now switch to the Seaside image.
You should see a pre-debugger window (\figref{haltingCounter}) showing a \ct{WACounter} object executing a \ct{halt}.
Examine this execution in the debugger, and then \button{Proceed}.
Go back to the web browser and notice that the counter application is running again.}

Seaside components can be instantiated multiple times and in different contexts.

\begin{figure}[ht]
\begin{center}
\includegraphics[width=\textwidth]{multiCounterHalos}
\caption{Independent subcomponents}
\label{fig:multiCounterHalos}
\end{center}
\end{figure}

\dothis{Point your web browser to \url{http://localhost:8080/seaside/examples/multicounter}.
You will see an application built out of a number of independent instances of the counter component.
Increment and decrement several of the counters.
Verify that they behave correctly even if you use the ``back'' button.
Toggle the halos to see how the application is built out of nested components.
Use the Seaside class browser to view the implementation of \ct{WAMultiCounter}.
You should see three methods on the class side (\ct{canBeRoot}, \ct{description}, and \ct{initialize}) and three on the instance side (\ct{children}, \ct{initialize}, and \ct{renderContentOn:}).
Note that an application is simply a component that is willing to be at the root of the component containment hierarchy; this willingness is indicated by defining a class-side method \ct{canBeRoot}
to answer \ct{true}.}
\index{Seaside!multi-counter}

You can use the Seaside web interface to configure, copy or remove individual applications (\ie root-level components).  Try making the following configuration change.

\dothis{Point your web browser to \url{http://localhost:8080/seaside/config}.
Supply the login and password (\ct{admin} and \ct{seaside} by default).
Select \link{Configure} next to ``examples.''
Under the heading ``Add entry point'', enter the new name ``counter2'' for the type \emph{Application} and click on \button{Add} (see \figref{counter2}).
On the next screen, set the \emph{Root Component} to \clsind{WACounter}, then click \button{Save} and \button{Close}.
Now we have a new counter installed at \url{http://localhost:8080/seaside/examples/counter2}.
Use the same configuration interface to remove this entry point.
}
\index{Seaside!configuration}


\begin{figure}[ht]
\begin{center}
\includegraphics[width=0.8\textwidth]{counter2}
\caption{Configuring a new application}
\label{fig:counter2}
\end{center}
\end{figure}

Seaside operates in two modes: \emph{development} mode, which is what we have seen so far, and \emph{deployment} mode, in which the toolbar is not available.
\index{Seaside!deployment mode}
\index{Seaside!development mode}
You can put Seaside into deployment mode using either the configuration page (navigate to the entry for the application and click on the \link{Configure} link)
% \ab{How?  I couldn't find this}
or click the \button{Configure} button in the toolbar.
In either case, set the deployment mode to \emph{true}.
Note that this affects new sessions only.
You can also set the mode globally by evaluating
\clsind{WAGlobalConfiguration} \lct{setDeploymentMode}
or
\ct{WAGlobalConfiguration setDevelopmentMode}.
\index{Seaside!deployment mode}
\index{Seaside!development mode}

The configuration web page is just another Seaside application, so it too can be controlled from the configuration page.
If you remove the ``config'' application, you can get it back by evaluating
\clsind{WADispatcherEditor} \ct{initialize}.

%=================================================================
\section{Seaside components}
\label{sec:components}

%\ab{This section was too long\,---\,18 pages.  It also contained several self-references (``see section 1.3''). So I broke into smaller sections, by promoting some of the subsections and subsubsections.}

As we mentioned in the previous section, Seaside applications are built out of \emph{\subind{Seaside}{components}.}
Let's take a closer look at how Seaside works by implementing the \emph{Hello World} component.

Every Seaside component should inherit directly or indirectly from \clsind{WAComponent}, as shown in \figref{WACounter}.

\dothis{Define a subclass of \ct{WAComponent} called \ct{WAHelloWorld}.}

Components must know how to render themselves.
Usually this is done by implementing the method \mthind{WAPresenter}{renderContentOn:}, which gets as its argument an instance of \clsind{WAHtmlCanvas}, which knows how to render XHTML.
\index{Seaside!rendering}

\dothis{Implement the following method, and put it in a protocol called \prot{rendering}:}
\needlines{2}
\begin{code}{}
WAHelloWorld>>>renderContentOn: html
	html text: 'hello world'
\end{code}

\noindent
Now we must inform Seaside that this component is willing to be a standalone application. 

\dothis{Implement the following method on the class side of \ct{WAHelloWorld}.}

\begin{code}{}
WAHelloWorld class>>>canBeRoot
	^ true
\end{code}

\noindent
We are almost done!

\dothis{Point your web browser at \url{http://localhost:8080/seaside/config}, add a new entry point called ``hello'', and set its root component to be \ct{WAHelloWorld}.
Now point your browser to \url{http://localhost:8080/seaside/hello}.
That's it!  You should see a web page like \figref{WAHelloWorld}.}

\begin{figure}[htb]
\begin{center}
\includegraphics[width=\textwidth]{WAHelloWorld}
\caption{``Hello World'' in Seaside}
\label{fig:WAHelloWorld}
\end{center}
\end{figure}

%-----------------------------------------------------------------
\subsection{State backtracking and the ``Counter'' Application}
%{Simple and nested components}

The ``counter'' application is only slightly more complex than the ``hello world'' application.
\label{sec:backtracking}

\begin{figure}[ht]
\begin{center}
\includegraphics[width=\textwidth]{WACounter}
\caption{The \ct{WACounter} class, which implements the \emph{counter} application.  Methods with underlined names are on the class-side; those with plain-text names are on the instance side.}
\label{fig:WACounter}
\end{center}
\end{figure}

The class \clsind{WACounter} is a standalone application, so \ct{WACounter class} must answer \ct{true} to the  \mthind{WAComponent class}{canBeRoot} message.
It must also register itself as an application; this is done in its class-side \ct{initialize} method, as shown in \figref{WACounter}.

\ct{WACounter} defines two methods, \ct{increase} and \ct{decrease}, which will be triggered from the \link{++} and \link{--\,--} links on the web page.  
It also defines an instance variable \ct{count} to record the state of the counter.
However, we also want Seaside to synchronize the counter with the browser page:
when the user clicks on the browser's ``back'' button, we want seaside to ``backtrack'' the state of the \ct{WACounter} object.
Seaside includes a general mechanism for backtracking, but each application has to tell Seaside which parts of its state to track.  

A component enables backtracking by implementing the \ct{states} method on the instance side: 
% \ab{note that xspace messes up again, by inserting a space at the start of this line}
\ct{states} should answer an array containing all the objects to be tracked.
In this case, the \ct{WACounter} object adds itself to Seaside's table of backtrackable objects by returning \ct{Array with: self}.

\paragraph{\emph{Caveat.}}
There is a subtle but important point to watch for when declaring objects for backtracking.
Seaside tracks state by making a \emph{copy} of all the objects declared in the \ct{states} array.
It does this using a \clsind{WASnapshot} object; \ct{WASnapshot} is a subclass of \clsind{IdentityDictionary} that records the objects to be tracked as keys and shallow copies of their state as values.
If the state of an application is backtracked to a particular snapshot, the state of each object entered into the snapshot dictionary is overwritten by the copy saved in the snapshot.

Here is the point to watch out for:
In the case of \ct{WACounter}, you might think that the state to be tracked is a number\,---\,the value of the \ct{count} instance variable.
However, having the \ct{states} method answer \ct{Array with: count} won't work.  
This is because the object named by \ct{count} is an integer, and integers are immutable.
The \ct{increase} and \ct{decrease} methods don't change the state of the object \ct{0} into \ct{1} or the object \ct{3} into \ct{2}.
Instead, they make \ct{count} name a different integer: 
every time the count is incremented or decremented, the object named by \ct{count} is \emph{replaced} by another.
This is why \ct{WACounter>>>states} must return \ct{Array with: self}.
When the state of a \mbox{\ct{WACounter}} object is replaced by a previous state, the \emph{value} of each of the instance variable in the object is replaced by a previous value; this correctly replaces the current value of \ct{count} by a prior value. 
\index{Seaside!backtracking state}
\index{WAPresenter!states@\ct{states}}

\section{Rendering XHTML}

The purpose of a web application is to create, or ``render'', web pages.  As we mentioned in \secref{components}, each Seaside component is responsible for rendering itself.  
So, lets start our exploration of rendering by seeing how the counter component renders itself.

\subsection{Rendering the Counter}

The rendering of the counter is relatively straightforward; the code is shown in \figref{WACounter}.
The current value of the counter is displayed as an XHTML heading, and the increment and decrement operations are implemented as html anchors (that is, links) with callbacks to blocks that will send \ct{increase} and \ct{decrease} to the counter object.

We will have a closer look at the rendering protocol in a moment.
But before we do, let's have a quick look at the \subind{Seaside}{multi-counter}.

\subsection{From Counter to MultiCounter}

\ct{WAMultiCounter}, shown in \figref{WAMultiCounter} is also a standalone application, so it overrides \mthind{WAComponent class}{canBeRoot} to answer \ct{true}.
In addition, it is a \emph{composite} component, so Seaside requires it to declare its children by implementing a method \ct{children} that answers an array of all the components it contains.
It renders itself by rendering each of its subcomponents, separated by a horizontal rule.
Aside from instance and class-side initialization methods, there is nothing else to the multi-counter!

\begin{figure}[bht]
\begin{center}
\includegraphics[width=\textwidth]{WAMultiCounter}
\caption{WAMultiCounter}
\label{fig:WAMultiCounter}
\end{center}
\end{figure}

%-----------------------------------------------------------------
\subsection{More about Rendering XHTML}

As you can see from these examples, Seaside does not use templates to generate web pages.
Instead it generates XTHML programmatically.
The basic idea is that every Seaside component should override the method \mthind{WAPresenter}{renderContentOn:}; this message will be sent by the framework to each component that needs to be rendered.
This \ct{renderContentOn:} message will have argument that is an \seeindex{canvas}{html canvas} \emphind{html canvas} onto which the component should render itself.  By convention, the html canvas parameter is called \ct{html}.
An html canvas is analogous to the graphics canvas used by Morphic (and most other drawing frameworks) to abstract away from the device-dependent details of drawing.
 

Here are some of the most basic rendering methods:
\begin{code}{}
html text: 'hello world'.  "render a plain text string"
html html: '&ndash;'.     "render an XHTML incantation"
html render: 1.              "render any object"
\end{code}

The message \ct{render: anyObject} can be sent to an html canvas to render \ct{anyObject}; it is normally used to render subcomponents.  \lct{anyObject} will itself be sent the message \ct{renderContentOn:}
this is what happens in the multi-counter (see \figref{WAMultiCounter}).

\subsection{Using Brushes}
\label{sec:brushes}

A canvas provides a number of \emphind{brushes} that can be used to render (\ie ``paint'') content on the canvas.
There are brushes for every kind of XHTML element\,---\,paragraphs, tables, lists, and so on.
To see the full protocol of brushes and convenience methods, you should browse the class \clsind{WACanvas} and its subclasses.
The argument to \ct{renderContentOn:} is actually an instance of the subclass \clsind{WARenderCanvas}.

We have already seen the following brush used in the counter and multi-counter examples:
\needlines{2}
\begin{code}{}
html horizontalRule.
\end{code}

\begin{figure}[ht]
\begin{center}
\includegraphics[width=\textwidth]{RenderingDemo}
\caption{RenderingDemo}
\label{fig:RenderingDemo}
\end{center}
\end{figure}

In \figref{RenderingDemo} we can see the output of many of the basic brushes offered by Seaside.\footnote{The source code for \mthref{renderdemo} is in the package \ct{PBE-SeasideDemo} in the project \url{http://www.squeaksource.com/PharoByExample}.}
The root component \ct{SeasideDemo} simply renders its subcomponents, which are instances of \ct{SeasideHtmlDemo}, \ct{SeasideFormDemo}, \ct{SeasideEditCallDemo} and \ct{SeasideDialogDemo}, as shown in \mthref{renderdemo}.

\needspace{7ex}
\begin{method}[renderdemo]{\lct{SeasideDemo>>renderContentOn:}}
SeasideDemo>>>renderContentOn: html
	html heading: 'Rendering Demo'.
	html heading
		level: 2;
		with: 'Rendering basic HTML: '.
	html div
		class: 'subcomponent';
		with: htmlDemo.
	"render the remaining components ..."
\end{method}

\noindent
Recall that a root component must always declare its children, or Seaside will refuse to render them.
\begin{code}{}
SeasideDemo>>>children
	^ { htmlDemo . formDemo . editDemo . dialogDemo }
\end{code}

Notice that there are two different ways of instantiating the \ct{heading} brush.
The first way is to set the text directly by sending the message \ct{heading:}.
The second way is instantiate the brush by sending \ct{heading}, and then to send a cascade of messages to the brush to set its properties and render it.
Many of the available brushes can be used in these two ways.

\important{If you send a \ind{cascade} of messages to a brush including the message \mthind{WABrush}{with:}, then \ct{with:} should be the \emph{final} message.
\ct{with:}  both sets the content and renders the result.}


In \mthref{renderdemo}, the first heading is at level 1, since this is the default.
We explicitly set the level of the second heading to 2.
The subcomponent is rendered as an XHTML \emph{div} with the \ind{CSS} class ``subcomponent''.
(More on CSS in \secref{css}.)
Also note that the argument to the \ct{with:} keyword message need not be a literal string: it can be another component, or even\,---\,as in the next example\,---\,a block containing further rendering actions.

The \ct{SeasideHtmlDemo} component demonstrates many of the most basic brushes.
Most of the code should be self-explanatory.

\begin{code}{}
SeasideHtmlDemo>>>renderContentOn: html 
	self renderParagraphsOn: html.
	self renderListsAndTablesOn: html.
	self renderDivsAndSpansOn: html.
	self renderLinkWithCallbackOn: html
\end{code}

It is common practice to break up long rendering methods into many helper methods, as we have done here.

\important{Don't put all your rendering code into a single method. 
Split it into helper methods named using the pattern \ct{render*On:}.
All rendering methods go in the \prot{rendering} protocol.
Don't send \ct{renderContentOn:} from your own code, use \ct{render:} instead.}

Look at the following code.  
The first helper method, \ct{SeasideHtmlDemo>>>renderParagraphsOn:}, shows you how to generate XHTML paragraphs, plain and emphasized text, and images.
Note that in Seaside simple elements are rendered by specifying the text they contain directly, whereas complex elements are specified using blocks.
This is a simple convention to help you structure your rendering code.

\begin{code}{}
SeasideHtmlDemo>>>renderParagraphsOn: html 
	html paragraph: 'A plain text paragraph.'.
	html paragraph: [
		html
			text: 'A paragraph with plain text followed by a line break. ';
			break;
			emphasis: 'Emphasized text ';
			text: 'followed by a horizontal rule.';
			horizontalRule;
			text: 'An image URI: '.
		html image
			url: self squeakImageUrl;
			width: '50']
\end{code}

The next helper method, \ct{SeasideHtmlDemo>>>renderListsAndTablesOn:}, shows you how to generate lists and tables.
A table uses two levels of blocks to display each of its rows and the cells within the rows.

\begin{code}{}
SeasideHtmlDemo>>>renderListsAndTablesOn: html 
	html orderedList: [
		html listItem: 'An ordered list item'].
	html unorderedList: [
		html listItem: 'An unordered list item'].
	html table: [
		html tableRow: [
			html tableData: 'A table with one data cell.']]
\end{code}

The next example shows how we can specify CSS \emph{div}s and \emph{span}s with \emph{class} or \emph{id} attributes.
Of course, the messages \ct{class:} and \ct{id:} can also be sent to the other brushes, not just to \emph{div}s and \emph{span}s.
The method \ct{SeasideDemoWidget>>>style} defines how these XHTML elements should be displayed (see \secref{css}).

\begin{code}{}
SeasideHtmlDemo>>>renderDivsAndSpansOn: html 
	html div
		id: 'author';
		with: [
			html text: 'Raw text within a div with id ''author''. '.
			html span
				class: 'highlight';
				with: 'A span with class ''highlight''.']
\end{code}

Finally we see a simple example of a link, created by binding a simple \subind{Seaside}{callback} to an ``anchor'' (\ie a link).
Clicking on the link will cause the subsequent text to toggle between ``true'' and ``false'' by toggling the instance variable \ct{toggleValue}.

\needlines{3}
\begin{code}{}
SeasideHtmlDemo>>>renderLinkWithCallbackOn: html 
	html paragraph: [
		html text: 'An anchor with a local action: '.
		html span with: [
			html anchor
				callback: [toggleValue := toggleValue not];
				with: 'toggle boolean:'].
		html space.
		html span
			class: 'boolean';
			with: toggleValue ]
\end{code}

\important{Note that actions should appear only in callbacks.
The code executed while rendering should not change the state of the application!}

%-----------------------------------------------------------------
\subsection{Forms}

Forms are rendered just like the other examples that we have already seen.
Here is the code for the \ct{SeasideFormDemo} component in \figref{RenderingDemo}.
\index{Seaside!XHTML forms}

\begin{code}{}
SeasideFormDemo>>>renderContentOn: html
	| radioGroup |
	html heading: heading.
	html form: [
		html span: 'Heading: '.
		html textInput on: #heading of: self.
		html select
			list: self colors;
			on: #color of: self.
		radioGroup := html radioGroup.
		html text: 'Radio on:'.
		radioGroup radioButton
			selected: radioOn;
			callback: [radioOn := true].
		html text: 'off:'.
		radioGroup radioButton
			selected: radioOn not;
			callback: [radioOn := false].
		html checkbox on: #checked of: self.
		html submitButton
			text: 'done' ]
\end{code}{}

Since a form is a complex entity, it is rendered using a block.
Note that all the state changes happen in the callbacks, not as part of the rendering.

There is one Seaside feature used here that is worth special mention, namely the message \mthind{WAAnchorTag}{on:of:}.
In the example, this message is used to bind a text input field to the variable \ct{heading}.
Anchors and buttons also support this message.
The first argument is the name of an instance variable for which accessors have been defined;  the second argument is the object to which this instance variable belongs.
Both observer (\ct{heading}) and mutator (\ct{heading:}) accessor messages must be understood by the object, with the usual naming convention.
In the case here of a text input field, this saves us the trouble of having to define a callback that updates the field as well as having to bind the default contents of the html input field to the current value of the instance variable.
Using \ct{on: #heading of: self}, the \ct{heading} variable is updated automatically whenever the user updates the text input field.

The same message is used twice more in this example, to cause the selection of a colour on the html form to update the \ct{color} variable, and to bind the result of the checkbox to the \ct{checked} variable.
Many other examples can be found in the functional tests for Seaside.
Have a look at the category \scat{Seaside-Tests-Functional}, or just point your browser to \url{http://localhost:8080/seaside/tests/alltests}.
Select \menu{WAInputTest} and click on the \button{Restart} button to see most of the features of forms.

Don't forget, if you \button{Toggle Halos}, you can browse the source code of the examples directly using the Seaside class browser.

%-----------------------------------------------------------------
\section{CSS: Cascading style sheets}
\label{sec:css}

%\ab{I think that it just needs a few paragraphs telling the reader the key ideas behind CSS, and the new terminology that the CSS folks introduce, before going in to the details of how you define their "thingies".  Now I have forgotten what they call their "thingies" --- I know that there are effectively paragraph styles (divs) and character styles (spans), but I've forgotten what they call them.  So, I think that the text needs to tell the reader, for each thingie, (1) the CSS concept behind the thingie, (2) what it looks like in a CSS style sheet , (3) what it looks like in html, and (4) how to do it in Seaside.   Maybe (3) can be omitted, because it's not needed to use Seaside.}
% \on{I think we do most of that already.}

Cascading Style Sheets\footnote{\url{http://www.w3.org/Style/CSS/}}, or \ind{CSS} for short, have emerged as a standard way for web applications to separate style from content.
Seaside relies on CSS to avoid cluttering your rendering code with layout considerations.

You can set the CSS style sheet for your web components by defining the method \ct{style}, which should return a string containing the CSS rules for that component.
The styles of all the components displayed on a web page are joined together, so each component can have its own style.
A better approach can be to define an abstract class for your web application that defines a common style for all its subclasses.

Actually, for deployed applications, it is more common to define style sheets as external files.
This way the look and feel of the component is completely separate from its functionality.
(Have a look at \clsind{WAFileLibrary}, which provides a way to serve static files without the need for a standalone server.)

If you already are familiar with CSS, then that's all you need to know.
Otherwise, read on for a very brief introduction to CSS.

Instead of directly encoding display attributes in the paragraph and text elements of your web pages, with CSS you will define different classes of elements and place all display considerations in a separate style sheet.
Paragraph-like entities are called \emph{div}s and text-like entities are \emph{span}s.
You would then define symbolic names, like ``highlight'' (see example below) for text to be highlighted, and specify how highlighted text is to be displayed in your style sheet.

Basically a CSS style sheet consists of a set of rules that specify how to format given XHTML elements.
Each rule consists of two parts.
There is a \emph{selector} that specifies which XHTML elements the rule applies to, and there is a \emph{declaration} which sets a number of attributes for that element.

\begin{figure}[tb]
\begin{code}{}
SeasideDemoWidget>>>style
	^ '
body {
	font: 10pt Arial, Helvetica, sans-serif, Times New Roman;
}
h2 {
	font-size: 12pt;
	font-weight: normal;
	font-style: italic;
}
table { border-collapse: collapse; }
td {
	border: 2px solid #CCCCCC;
	padding: 4px;
}
#author {
	border: 1px solid black;
	padding: 2px;
	margin: 2px;
}
.subcomponent {
	border: 2px solid lightblue;
	padding: 2px;
	margin: 2px;
}
.highlight { background-color: yellow; }
.boolean { background-color: lightgrey; }
.field { background-color: lightgrey; }
'
\end{code}
\caption{\lct{SeasideDemoWidget} common style sheet.
\label{fig:democss}}
\end{figure}
\figref{democss} illustrates a simple style sheet for the rendering demo shown earlier in \figref{RenderingDemo}.
The first rule specifies a preference for the fonts to use for the \ct{body} of the web page.
The next few rules specify properties of second-level headings (\ct{h2}), tables (\ct{table}), and table data (\ct{td}).

The remaining rules have selectors that will match XHTML elements that have the given ``class'' or ``id'' attributes.
CSS selectors for class attributes start with a ``\ct{.}'' and those for id attributes with ``\ct{#}''.
The main difference between class and id attributes is that many elements may have the same class, but only one element may have a given id (\ie an \emph{identifier}). 
So, whereas a class attribute, such as \ct{highlight}, may occur multiple times on any page, an id must identify a \emph{unique} element on the page, such as a particular menu, the modified date, or author.
Note that a particular XHTML element may have multiple classes, in which case all the applicable display attributes will be applied in sequence.

% This style sheet expects at most one element to specify the \emph{author} of the web page.

Selector conditions may be combined, so the selector \ct{div.subcomponent} will only match an XHTML element if it is both a div \emph{and} it has a class attribute equal to ``subcomponent''.

It is also possible to specify nested elements, though this is seldom necessary.
For example, the selector ``\ct{p span}'' will match a span within a paragraph but not within a div.

There are numerous books and web sites to help you learn CSS.
For a dramatic demonstration of the power of CSS, we recommend you to have a look at the CSS Zen Garden\footnote{\url{http://www.csszengarden.com/}}, which shows how the same content can be rendered in radically different ways simply by changing the CSS style sheet.

%-----------------------------------------------------------------
\section{Managing control flow}

Seaside makes it particularly easy to design web applications with non-trivial control flow.
There are basically two mechanisms that you can use:

\begin{enumerate}
  \item A component can \emph{call} another component by sending \ct{caller call: callee}.
  The caller is temporarily replaced by the callee, until the callee returns control by sending \ct{answer:}.
  The caller is usually \ct{self}, but could also be any other currently visible component.

  \item A workflow can be be defined as a \emphsubind{Seaside}{task}.
  This is a special kind of component that subclasses \clsind{WATask} (instead of \clsind{WAComponent}). \label{sec:task}
  Instead of defining \ct{renderContentOn:}, it defines no content of its own, but rather defines a \ct{go} method that sends a series of \ct{call:} messages to activate various subcomponents in turn.
\end{enumerate}
\index{Seaside!control flow}

%-----------------------------------------------------------------
\subsection{Call and answer}

Call and answer are used to realize simple dialogues.

There is a trivial example of \ct{call:} and \ct{answer:} in the rendering demo of \figref{RenderingDemo}.
The component \ct{SeasideEditCallDemo} displays a text field and an \emph{edit} link.
The callback for the edit link calls a new instance of \ct{SeasideEditAnswerDemo} initialized to the value of the text field.
The callback also updates this text field to the result which is sent as an answer.

(We underline the \ct{call:} and \ct{answer:} sends to draw attention to them.)

\begin{code}{}
SeasideEditCallDemo>>>renderContentOn: html 
	html span
		class: 'field';
		with: self text.
	html space.
	html anchor
		callback: [self text: (self !\underline{call:}! (SeasideEditAnswerDemo new text: self text))];
		with: 'edit'
\end{code}{}

What is particularly elegant is that the code makes absolutely no reference to the new web page that must be created.
At run-time, a new page is created in which the \ct{SeasideEditCallDemo} component is replaced by a \ct{SeasideEditAnswerDemo} component; the parent component and the other peer components are untouched.

\important{\mthind{WAComponent}{call:} and \mthind{WAComponent}{answer:} should never be used while rendering.
They may safely be sent from within a \subind{Seaside}{callback}, or from within the \mthind{WATask}{go} method of a task.}

The \ct{SeasideEditAnswerDemo} component is also remarkably simple.
It just renders a form with a text field.
The submit button is bound to a callback that will answer the final value of the text field.

\begin{code}{}
SeasideEditAnswerDemo>>>renderContentOn: html
	html form: [
		html textInput
			on: #text of: self.
		html submitButton
			callback: [ self !\underline{answer:}! self text ];
			text: 'ok'.
		]
\end{code}{}

That's it.

Seaside takes care of the control flow and the correct rendering of all the components.
Interestingly, the ``back'' button of the browser will also work just fine (though side effects are not rolled back unless we take additional steps).

%-----------------------------------------------------------------
\subsection{Convenience methods}

Since certain call--answer dialogues are very common, Seaside provides some convenience methods to save you the trouble of writing components like \ct{SeasideEditAnswerDemo}.
The generated dialogues are shown in \figref{dialogs}.
We can see these convenience methods being used within \ct{SeasideDialogDemo>>>renderContentOn:}
\index{Seaside!convenience methods}

\begin{figure}[b]
\begin{center}
\includegraphics[width=\textwidth]{dialogs}
\caption{Some standard dialogs}
\label{fig:dialogs}
\end{center}
\end{figure}

The message \mthind{WAComponent}{request:} performs a call to a component that will let you edit a text field.
The component answers the edited string.
An optional label and default value may also be specified.

\needlines{3}
\begin{code}{}
SeasideDialogDemo>>>renderContentOn: html
	html anchor
		callback: [ self request: 'edit this' label: 'done' default: 'some text' ];
		with: 'self request:'.
...
\end{code}

The message \mthind{WAComponent}{inform:} calls a component that simply displays the argument message and waits for the user to click ``ok''.
The called component just returns \ct{self}.

\begin{code}{}
...
	html space.
	html anchor
		callback: [ self inform: 'yesBANG' ];
		with: 'self inform:'.
...
\end{code}

The message \mthind{WAComponent}{confirm:} asks a questions and waits for the user to select either ``Yes'' or ``No''.
The component answers a boolean, which can be used to perform further actions.

\begin{code}{}
...
	html space.
	html anchor
		callback: [
			(self confirm: 'Are you happy?')
				ifTrue: [ self inform: ':-)' ]
				ifFalse: [ self inform: ':-(' ]
			];
		with: 'self confirm:'.
\end{code}

A few further convenience methods, such as \mthind{WAComponent}{chooseFrom:caption:}, are defined in the \prot{convenience} protocol of \clsind{WAComponent}.

%-----------------------------------------------------------------
\subsection{Tasks}

A \subind{Seaside}{task} is a component that subclasses \clsind{WATask}.
It does not render anything itself, but simply calls other components in a control flow defined by implementing the method \mthind{WATask}{go}.

\clsind{WAConvenienceTest} is a simple example of a task defined in the category \scat{Seaside-Tests-Functional}.
To see its effect, just point your browser to \url{http://localhost:8080/seaside/tests/alltests}, select \menu{WAConvenienceTest} and click \button{Restart}.

\begin{code}{}
WAConvenienceTest>>>go
	[ self chooseCheese.
	  self confirmCheese ] whileFalse.
	self informCheese
\end{code}

This task calls in turn three components.
The first, generated by the convenience method \mthind{WAComponent}{chooseFrom: caption:}, is a \clsind{WAChoiceDialog} that asks the user to choose a cheese.

\begin{code}{}
WAConvenienceTest>>>chooseCheese
	cheese := self
		chooseFrom: #('Greyerzer' 'Tilsiter' 'Sbrinz')
		caption: 'What''s your favorite Cheese?'.
	cheese isNil ifTrue: [ self chooseCheese ]
\end{code}

% \alex{Is there a situation where cheese may be nil? Maybe if a browser authorizes an empty selection...}

The second is a \clsind{WAYesOrNoDialog} to confirm the choice (generated by the convenience method \mthind{WAComponent}{confirm:}).

\begin{code}{}
WAConvenienceTest>>>confirmCheese
	^self confirm: 'Is ', cheese,  ' your favorite cheese?'
\end{code}

Finally a \clsind{WAFormDialog} is called (via the convenience method \mthind{WAComponent}{inform:}).

\begin{code}{}
WAConvenienceTest>>>informCheese
	self inform: 'Your favorite cheese is ', cheese, '.'
\end{code}

The generated dialogues are shown in \figref{chooseCheese}.

\begin{figure}[ht]
\begin{center}
\includegraphics[width=0.8\textwidth]{chooseCheese}
\caption{A simple task}
\label{fig:chooseCheese}
\end{center}
\end{figure}

%-----------------------------------------------------------------
\subsection{Transactions}

We saw in \secref{backtracking} that Seaside can keep track of the correspondence between the state of components and individual web pages by having components register their state for backtracking:
all that a component need do is implement the method \ct{states} to answer an array of all the objects whose state must be tracked.

Sometimes, however, we do not want to backtrack state: instead we want to \emph{prevent} the user from accidentally undoing effects that should be permanent.
This is often referred to as ``the shopping cart problem''.
Once you have checked-out your shopping cart and paid for the items you have purchased, it should not be possible to go ``back'' with the browser and add more items to the shopping cart!

Seaside allows you to prevent this by defining a task within which certain actions are grouped together as \emph{transactions}.
You can backtrack within a transaction, but once a transaction is complete, you can no longer go back to it.
The corresponding pages are \emph{invalidated}, and any attempt to go back to them will cause Seaside to generate a warning and redirect the user to the most recent valid page.

\begin{figure}[ht]
\begin{center}
\includegraphics[width=\textwidth]{sushiStore}
\caption{The Sushi Store}
\label{fig:sushiStore}
\end{center}
\end{figure}

The Seaside \emphsubind{Seaside}{Sushi Store} is sample application that illustrates many of the features of Seaside, including transactions.
This application is bundled with your installation of Seaside, so you can try it out by pointing your browser at
\url{http://localhost:8080/seaside/examples/store}.\footnote{If you cannot find it in your image, there is a version of the sushi store available on SqueakSource from \url{http://www.squeaksource.com/SeasideExamples/}.}

The sushi store supports the following workflow:
\begin{enumerate}[itemsep=0pt]
  \item Visit the store.
  \item Browse or search for sushi.
  \item Add sushi to your shopping cart.
  \item Checkout.
  \item Verify your order.
  \item Enter shipping address.
  \item Verify shipping address.
  \item Enter payment information.
  \item Your fish is on its way!
\end{enumerate}

If you toggle the \subind{Seaside}{halos}, you will see that the top-level component of the sushi store is an instance of \clsind{WAStore}.
It does nothing but render the title bar, and then it renders \ct{task}, an instance of \clsind{WAStoreTask}.

\begin{code}{}
WAStore>>>renderContentOn: html
	"... render the title bar ..."
	html div id: 'body'; with: task
\end{code}

\clsind{WAStoreTask} captures this workflow sequence. At a couple of points it is critical that the user not be able to go back and change the submitted information.

\dothis{\,``Purchase'' some sushi and then use the ``back'' button to try to put more sushi into your cart.
You will get the message ``That page has expired.''}

Seaside lets the programmer say that a certain part of a workflow act like a transaction: once the transaction is complete, the user cannot go back and undo it.
You say this by sending \mthind{WAComponent}{isolate:} to a task with the transactional block as its argument.
We can see this in the sushi store workflow as follows:

\begin{code}{}
WAStoreTask>>>go
	| shipping billing creditCard |
	cart := WAStoreCart new.
	self isolate:
		[[self fillCart.
		self confirmContentsOfCart]
			whileFalse].

	self isolate:
		[shipping := self getShippingAddress.
		billing := (self useAsBillingAddress: shipping)
					ifFalse: [self getBillingAddress]
					ifTrue: [shipping].
		creditCard := self getPaymentInfo.
		self shipTo: shipping billTo: billing payWith: creditCard].

	self displayConfirmation.
\end{code}

Here we see quite clearly that there are two transactions.
The first fills the cart and closes the shopping phase.
(The helper methods \ct{fillCart} \etc take care of instantiating and calling the right subcomponents.)
Once you have confirmed the contents of the cart you cannot go back without starting a new session.
The second transaction completes the shipping and payment data.
You can navigate back and forth within the second transaction until you confirm payment.
However, once both transactions are complete, any attempt to navigate back will fail.

Transactions may also be nested.
A simple demonstration of this is found in the class \clsind{WANestedTransaction}.
The first \ct{isolate:} takes as argument a block that contains another, nested \ct{isolate:}

\begin{code}{}
WANestedTransaction>>>go
	self inform: 'Before parent txn'.
	self isolate:
			[self inform: 'Inside parent txn'.
			self isolate: [self inform: 'Inside child txn'].
			self inform: 'Outside child txn'].
	self inform: 'Outside parent txn'
\end{code}

\dothis{Go to \url{http://localhost:8080/seaside/tests/alltests}, select \menu{WATransactionTest} and click on \button{Restart}.
Try to navigate back and forth within the parent and child transaction by clicking the \button{back} button and then clicking \button{ok}.
Note that as soon as a transaction is complete, you can no longer go back inside the transaction without generating an error upon clicking \button{ok}.}

%=================================================================
\section{A complete tutorial example}

% ON: Should take about two hours

Let's see how we can build a complete Seaside application from scratch.\footnote{The exercise should take at most a couple of hours. If you prefer to just look at the completed source code, you can grab it from the SqueakSource project \url{http://www.squeaksource.com/PharoByExample}.
The package to load is \scat{PBE-SeasideRPN}. The tutorial that follows uses slightly different class names so that you can compare your implementation with ours.}
We will build a RPN (Reverse Polish Notation) calculator as a Seaside application that uses a simple stack machine as its underlying model.
Furthermore, the Seaside interface will let us toggle between two displays\,---\,one which just shows us the current value on top of the stack, and the other which shows us the complete state of the stack.
The calculator with the two display options is shown in \figref{stackMachine}.

\begin{figure}[ht]
\begin{center}
\includegraphics[width=0.8\textwidth]{stackMachine}
\caption{RPN calculator and its stack machine}
\label{fig:stackMachine}
\end{center}
\end{figure}

We begin by implementing the stack machine and its tests.

\dothis{Define a new class called \ct{MyStackMachine} with an instance variable \ct{contents} initialized to a new \ct{OrderedCollection}.}

\begin{code}{}
MyStackMachine>>>initialize
	super initialize.
	contents := OrderedCollection new.
\end{code}

The stack machine should provide operations to \ct{push:} and \ct{pop} values, view the \ct{top} of the stack, and perform various arithmetic operations to add, subtract, multiply and divide the top values on the stack.

\dothis{Write some tests for the stack operations and then implement these operations.
Here is a sample test:}

\needlines{4}
\begin{code}{}
MyStackMachineTest>>>testDiv
	stack
		push: 3;
		push: 4;
		div.
	self assert: stack size = 1.
	self assert: stack top = (4/3).
\end{code}

You might consider using some helper methods for the arithmetic operations to check that there are two numbers on the stack before doing anything, and raising an error if this precondition is not fulfilled.\footnote{It's a good idea to use \ct{Object>>>assert:} to specify the preconditions for an operation.
This method will raise an \ct{AssertionFailure} if the user tries to use the stack machine in an invalid state.}
If you do this, most  of your methods will just be one or two lines long.

You might also consider implementing \ct{MyStackMachine>>>printOn:} to make it easier to debug your stack machine implementation with the help of an object inspector.
(Hint: just delegate printing to the \ct{contents} variable.)
\index{Object!printOn:@\ct{printOn:}}

\dothis{Complete the \ct{MyStackMachine} by writing operations \ct{dup} (push a duplicate of the top value onto the stack), \ct{exch} (exchange the top two values), and \ct{rotUp} (rotate the entire stack contents up\,---\,the top value will move to the bottom).}

Now we have a simple stack machine implementation.
We can start to implement the Seaside RPN Calculator.

We will make use of 5 classes:
\begin{itemize}
  \item \ct{MyRPNWidget}\,---\,this should be an abstract class that defines the common CSS style sheet for the application, and other common behavior for the components of the RPN calculator.
  It is a subclass of \ct{WAComponent} and the direct superclass of the following four classes. 
  
    \item \ct{MyCalculator}\,---\,this is the root component.
  It should register the application (on the class side), it should instantiate and render its subcomponents, and it should register any state for backtracking.
  \item \ct{MyKeypad}\,---\,this displays the keys that we use to interact with the calculator.
  \item \ct{MyDisplay}\,---\,this component displays the top of the stack and provides a button to call another component to display the detailed view.
  \item \ct{MyDisplayStack}\,---\,this component shows the detailed view of the stack and provides a button to answer back.
  It is a subclass of \lct{MyDisplay}.
\end{itemize}

\dothis{Define \ct{MyRPNWidget} in the category \ct{MyCalculator}.
Define the common \ct{style} for the application.}

Here is a minimal CSS for the application.
You can make it more fancy if you like.
\begin{code}{}
MyRPNWidget>>>style
	^ 'table.keypad { float: left; }
td.key {
	border: 1px solid grey;
	background: lightgrey;
	padding: 4px;
	text-align: center;
}
table.stack { float: left; }
td.stackcell {
	border: 2px solid white;
	border-left-color: grey;
	border-right-color: grey;
	border-bottom-color: grey;
	padding: 4px;
	text-align: right;
}
td.small { font-size: 8pt; }'
\end{code}

\dothis{Define \ct{MyCalculator} to be a root component and register itself as an application (\ie implement \ct{canBeRoot} and \ct{initialize} on the class side).
Implement \ct{MyCalculator>>>renderContentOn:} to render something trivial (such as its name), and verify that the application runs in a browser.
}

\ct{MyCalculator} is responsible for instantiating \ct{MyStackMachine}, \ct{MyKeypad} and \ct{MyDisplay}.

\dothis{
Define \ct{MyKeypad} and \ct{MyDisplay} as subclasses of \lct{MyRPNWidget}.
All three components will need access to a common instance of the stack machine, so define the instance variable \ct{stackMachine} and an initialization method \ct{setMyStackMachine:} in the common parent, \ct{MyRPNWidget}.
Add instance variables \ct{keypad} and \ct{display} to \ct{MyCalculator} and initialize them in \ct{MyCalculator>>>initialize}.
(Don't forget to send \lct{super initialize}!)}

\dothis{
Pass the shared instance of the stack machine to the keypad and the display in the same initialize method.
Implement \ct{MyCalculator>>>renderContentOn:} to simply render in turn the keypad and the display.
To correctly display the subcomponents, you must implement \ct{MyCalculator>>>children} to return an array with the keypad and the display.
Implement placeholder rendering methods for the keypad and the display and verify that the calculator now displays its two subcomponents.
}

%\ab{Too long!}

Now we will change the implementation of the display to show the top value of the stack.

\dothis{
Use a table with class ``keypad'' containing a row with a single table data cell with class ``stackcell''.
Change the rendering method of the keypad to ensure that the number 0 is pushed on the stack in case it is empty.
(Define and use \ct{MyKeypad>>>ensureMyStackMachineNotEmpty}.)
Also make it display an empty table with class ``keypad''.
Now the calculator should display a single cell containing the value 0.
If you toggle the halos, you should see something like this:
}

\begin{figure}[ht]
\begin{center}
\includegraphics[width=0.7\textwidth]{firstStackDisplay}
\caption{Displaying the top of the stack}
\label{fig:firstStackDisplay}
\end{center}
\end{figure}

Now let's implement an interface to interact with the stack.

\dothis{
First define the following helper methods, which will make it easier to script the interface:
}

\needlines{3}
\begin{code}{}
MyKeypad>>>renderStackButton: text callback: aBlock colSpan: anInteger on: html 
	html tableData
		class: 'key';
		colSpan: anInteger;
		with: 
				[html anchor
					callback: aBlock;
					with: [html html: text]]
\end{code}


\begin{code}{}
MyKeypad>>>renderStackButton: text callback: aBlock on: html 
	self 
		renderStackButton: text
		callback: aBlock
		colSpan: 1
		on: html
\end{code}

We will use these two methods to define the buttons on the keypad with appropriate callbacks.
Certain buttons may span multiple columns, but the default is to occupy just one column.

\dothis{
Use the two helper methods to script the keypad as follows:
(Hint: start by getting the digit and ``Enter'' keys working, then the arithmetic operators.)
}

\needlines{4}
\begin{code}{}
MyKeypad>>>renderContentOn: html 
  self ensureStackMachineNotEmpty.
  html table
    class: 'keypad';
    with: [
      html tableRow: [
          self renderStackButton: '+' callback: [self stackOp: #add] on: html.
          self renderStackButton: '&ndash;' callback: [self stackOp: #min] on: html.
          self renderStackButton: '&times;' callback: [self stackOp: #mul] on: html.
          self renderStackButton: '&divide;' callback: [self stackOp: #div] on: html.
          self renderStackButton: '&plusmn;' callback: [self stackOp: #neg] on: html ].
        html tableRow: [
          self renderStackButton: '1' callback: [self type: '1'] on: html.
          self renderStackButton: '2' callback: [self type: '2'] on: html.
          self renderStackButton: '3' callback: [self type: '3'] on: html.
          self renderStackButton: 'Drop' callback: [self stackOp: #pop]
          	colSpan: 2 on: html ].
" and so on ... "
        html tableRow: [
          self renderStackButton: '0' callback: [self type: '0'] colSpan: 2 on: html.
          self renderStackButton: 'C' callback: [self stackClearTop] on: html.
          self renderStackButton: 'Enter'
          	callback: [self stackOp: #dup. self setClearMode]
			colSpan: 2 on: html ]]
\end{code}

Check that the keypad displays properly.
If you try to click on the keys, however, you will find that the calculator does not work yet ...

\dothis{
Implement \ct{MyKeypad>>>type:} to update the top of the stack by appending the typed digit.
You will need to convert the top value to a string, update it, and convert it back to an integer, something like this:
}
\begin{code}{}
MyKeypad>>>type: aString
	stackMachine push: (stackMachine pop asString, aString) asNumber.
\end{code}
Now when you click on the digit keys the display should be updated.
(Be sure that \ct{MyStackMachine>>>pop} returns the value popped, or this will not work!)


\dothis{Now we must implement \ct{MyKeypad>>>stackOp:}
Something like this will do the trick:}

\begin{code}{}
MyKeypad>>>stackOp: op
	[ stackMachine perform: op ] on: AssertionFailure do: [ ].
\end{code}

The point is that we are not sure that all operations will succeed, for example, addition will fail if we do not have two numbers on the stack.
For the moment we can just ignore such errors.
If we are feeling more ambitious later on, we can provide some user feedback in the error handler block.

\dothis{The first version of the calculator should be working now.
Try to enter some numbers by pressing the digit keys, hitting \menu{Enter} to push a copy of the current value, and entering \menu{+} to sum the top two values.}

You will notice that typing digits does not behave the way you might expect.
Actually the calculator should be aware of whether you are typing a \emph{new} number, or appending to an existing number.

\dothis{Adapt \ct{MyKeypad>>>type:} to behave differently depending on the current typing mode.
Introduce an instance variable \ct{mode} which takes on one of the three values \lct{typing} (when you are typing), \lct{push} (after you you have performed a calculator operation and typing should force the top value to be pushed), or \lct{clear} (after you have performed \menu{Enter} and the top value should be cleared before typing).
The new \ct{type:} method might look like this:
}

\begin{code}{}
MyKeypad>>>type: aString
	self inPushMode ifTrue: [
		stackMachine push: stackMachine top.
		self stackClearTop ].
	self inClearMode ifTrue: [ self stackClearTop ].
	stackMachine push: (stackMachine pop asString, aString) asNumber.
\end{code}

Typing might work better now, but it is still frustrating not to be able to see what is on the stack.

\dothis{
Define \ct{MyDisplayStack} as a subclass of \ct{MyDisplay}.
Add a button to the rendering method of \ct{MyDisplay} which will call a new instance of \ct{MyDisplayStack}.
You will need an html anchor that looks something like this:
}

\begin{code}{}
html anchor
	callback: [ self call: (MyDisplayStack new setMyStackMachine: stackMachine)];
	with: 'open'
\end{code}

The callback will cause the current instance of \ct{MyDisplay} to be temporarily replaced by a new instance of \ct{MyDisplayStack} whose job it is to display the complete stack.
When this component signals that it is done (\ie by sending \ct{self answer}), then control will return to the original instance of \ct{MyDisplay}.

\dothis{
Define the rendering method of \ct{MyDisplayStack} to display all of the values on the stack.
(You will either need to define an accessor for the stack machine's \ct{contents} or you can define \ct{MyStackMachine>>>do:} to iterate over the stack values.)
The stack display should also have a button labelled ``close'' whose callback will simply perform \ct{self answer}.
}

\begin{code}{}
html anchor
	callback: [ self answer];
	with: 'close'
\end{code}

Now you should be able to \emph{open} and \emph{close} the stack while you are using the calculator.

There is, however, one thing we have forgotten.
Try to perform some operations on the stack.
Now use the ``back'' button of your browser and try to perform some more stack operations.
(For example, \menu{open} the stack, type \menu{1}, \menu{Enter} twice and \menu {+}.
The stack should display ``2'' and ``1''.
Now hit the ``back'' button.
The stack now shows three times ``1'' again.
Now if you type \menu{+} the stack shows ``3''.
Backtracking is not yet working.

\dothis{
Implement \ct{MyCalculator>>>states} to return the contents of the stack machine.
Check that backtracking now works correctly!
}

Sit back and enjoy a tall glass of something cool!

%=================================================================
\section{A quick look at AJAX}

% Original text by Lukas Renggli

\ind{AJAX} (Asynchronous \ind{JavaScript} and \ind{XML}) is a technique to make web applications more interactive by exploiting JavaScript functionality on the client side.

Two well-known JavaScript libraries are \ind{Prototype} (\url{http://www.prototypejs.org}) and \ind{script.aculo.us} (\url{http://script.aculo.us}).
Prototype provides a framework to ease writing JavaScript.
script.aculo.us provides some additional features to support animations and drag-and-drop on top of Prototype.
Both frameworks are supported in Seaside through the package ``Scriptaculous''.

All ready-made images have the Scriptaculous package extensions already loaded.
The latest version is available from \url{http://www.squeaksource.com/Seaside}.
An online demo is available at \url{http://scriptaculous.seasidehosting.st}.
Alternatively, if you have a enabled image running, simply go to \url{http://localhost:8080/seaside/tests/scriptaculous}.

The Scriptaculous extensions follow the same approach as Seaside itself\,---\,simply configure Smalltalk objects to model your application, and the needed Javascript code will be generated for you.

Let us look at a simple example of how client-side Javascript support can make our RPN calculator behave more naturally.
Currently every keystroke to enter a digit generates a request to refresh the page.
We would like instead to handle editing of the display on the client-side by updating the display in the existing page.

\dothis{To address the display from JavaScript code we must first give it a unique id.
Update the calculator's rendering method as follows:\footnote{If you have not implemented the tutorial example yourself, you can simply load the complete example from \url{http://www.squeaksource.com/SqueakByExample} and apply the suggested changes to the classes \ct{RPN*} instead of \ct{My*}.}}

\begin{code}{}
MyCalculator>>>renderContentOn: html
	html div id: 'keypad'; with: keypad.
	html div id: 'display'; with: display.	
\end{code}
				
\dothis{To be able to re-render the display when a keyboard button is pressed, the keyboard needs to know the display component.
Add a \ct{display} instance variable to \ct{MyKeypad}, an initialize method \ct{MyKeypad>>>setDisplay:}, and call this from \ct{MyCalculator>>initialize}.
Now we are able to assign some JavaScript code to the buttons by updating \ct{MyKeypad>>>renderStackButton:callback:colSpan:on:} as follows:}

\begin{code}{}
MyKeypad>>>renderStackButton: text callback: aBlock colSpan: anInteger on: html 
	html tableData
		class: 'key';
		colSpan: anInteger;
		with: [
			html anchor
				callback: aBlock;
				onClick:				"handle Javascript event"
					(html updater
						id: 'display';
						callback: [ :r |
							aBlock value.
							r render: display ];
						return: false);
				with: [ html html: text ] ]
\end{code}

\mthind{WATagBrush}{onClick:} specifies a \ind{JavaScript} event handler.
\ct{html updater} returns an instance of \ct{SUUpdater}, a Smalltalk object representing the JavaScript Ajax.Updater object (\url{http://www.prototypejs.org/api/ajax/updater}).
This object performs an AJAX request and updates a container's contents based on the response text.
\ct{id:} tells the updater what XHTML DOM element to update, in this case the contents of the div element with the id 'display'.
\ct{callback:} specifies a block that is triggered when the user presses the button.
The block argument is a new renderer \ct{r}, which we can use to render the display component.
(Note: Even though html is still accessible, it is not valid anymore at the time this callback block is evaluated).
Before rendering the display component we evaluate \ct{aBlock} to perform the desired action.

\ct{return: false} tells the JavaScript engine to not trigger the original link callback, which would cause a full refresh.
We could instead remove the original anchor \ct{callback:}, but like this the calculator will still work even if JavaScript is disabled.

\dothis{Try the calculator again, and notice how a full page refresh is triggered every time you press a digit key. (The URL of the web page is updated at each keystroke.)}

Although we have implemented the client-side behavior, we have not yet activated it.
Now we will enable the Javascript event handling.

\dothis{
Click on the \link{Configure} link in the toolbar of the calculator.
Select ``Add Library:'' \ct{SULibrary}, click the \button{Add} button and \button{Close}.}

Instead of manually adding the library, you may also do it programmatically when you register the application:
\begin{code}{}
MyCalculator class>>>initialize
	(self registerAsApplication: 'rpn')
		addLibrary: SULibrary}}
\end{code}

\begin{figure}[ht]
\begin{center}
\includegraphics[width=\textwidth]{ajax-processing}
\caption{Seaside AJAX processing (simplified)}
\label{fig:ajax-processing}
\end{center}
\end{figure}

\dothis{Try the revised application.  Note that the feedback is much more natural. In particular, a new URL is not generated with each keystroke.}

You may well ask, \emph{yes, but how does this work?}
\figref{ajax-processing} shows how the RPN applications would both without and with AJAX.
Basically AJAX short-circuits the rendering to \emph{only} update the display component.
Javascript is responsible both for triggering the request and updating the corresponding DOM element.
Have a look at the generated source-code, especially the JavaScript code:

\begin{code}{}
new Ajax.Updater(
	'display',
	'http://localhost/seaside/RPN+Calculator',
	{'evalScripts': true,
	  'parameters': ['UNDERSCOREs=zcdqfonqwbeYzkza', 'UNDERSCOREk=jMORHtqr','9'].join('&')});
return false
\end{code}

For more advanced examples, have a further look at \url{http://localhost:8080/seaside/tests/scriptaculous}.

\paragraph{\emph{Hints.}}
In case of server side problems use the Smalltalk debugger.
In case of client side problems use FireFox (\url{http://www.mozilla.com}) with the JavaScript debugger FireBug (\url{http://www.getfirebug.com/}) plugin enabled.

%=================================================================
\section{Chapter summary}

\begin{itemize}
  \item The easiest way to get started is to download the ``Seaside One-Click Experience'' from \url{http://seaside.st}
  \item Turn the server on and off by evaluating \ct{WAKom startOn: 8080} and \ct{WAKom stop}.
  \item Reset the administrator login and password by evaluating \ct{WADispatcherEditor initialize}.
  \item \menu{Toggle Halos} to directly view application source code, run-time objects, CSS and XHTML.
  \item Send \ct{WAGlobalConfiguration setDeploymentMode} to hide the toolbar.
  \item Seaside web applications are composed of components, each of which is an instance of a subclass of \ct{WAComponent}.
  \item Only a root component may be registered as a component. It should implement \ct{canBeRoot} on the class side. Alternatively it may register itself as an application in its class-side \ct{initialize} method by sending \ct{self registerAsApplication:} \emph{application path}.
  If you override \ct{description} it is possible to return a descriptive application name that will be displayed in the configuration editor.
  \item To backtrack state, a component must implement the \ct{states} method to answer an array of objects whose state will be restored if the user clicks the browser's ``back'' button.
  \item A component renders itself by implementing \ct{renderContentOn:}.
  The argument to this method is an XHTML rendering \emph{canvas} (usually called \ct{html}).
  \item A component can render a subcomponent by sending \ct{self render:} \emph{subcomponent}.
  \item XHTML is generated programmatically by sending messages to \emph{brushes}. A brush is obtained by sending a message, such as \ct{paragraph} or \ct{div}, to the html canvas.
  \item If you send a cascade of messages to a brush that includes the message \ct{with:}, then \ct{with:} should be the last message sent.
  Thw \ct{with:} message sets the contents \emph{and} renders the result.
  \item Actions should appear only in callbacks.
You should not change the state of the application while you are rendering it.
  \item You can bind various form widgets and anchors to instance variables with accessors by sending the message \ct{on:} \emph{instance variable} \ct{of:} \emph{object} to the brush.
  \item You can define the CSS for a component hierarchy by defining the method \ct{style}, which should return a string containing the style sheet.
  (For deployed applications, it is more usual to refer to a style sheet located at a static URL.)
  \item Control flows can be programmed by sending \ct{x call: y}, in which case component \ct{x} will be replaced by \ct{y} until \ct{y} answers by sending \ct{answer:} with a result in a callback.
  The receiver of \ct{call:} is usually \ct{self}, but may in general be any visible component.
  \item A control flow can also be specified as a \emph{task}\,---\,a instance of a subclass of \ct{WATask}. It should implement the method \ct{go}, which should \ct{call:} a series of components in a workflow.
  \item Use \ct{WAComponents}'s convenience methods \ct{request:}, \ct{inform:}, \ct{confirm:} and \ct{chooseFrom:caption:} for basic interactions.
  \item To prevent the user from using the browser's ``back'' button to access a previous execution state of the web application, you can declare portions of the workflow to be a \emph{transaction} by enclosing them in an \ct{isolate:} block.
\end{itemize}
%-----------------------------------------------------------------

%=================================================================
\ifx\wholebook\relax\else 
   \bibliographystyle{jurabib}
   \nobibliography{scg}
   \end{document}
\fi
%=================================================================

% $Author$
% $Date$
% $Revision$

% HISTORY:
% 2008-07-12 - Stef first version

%=================================================================
\ifx\wholebook\relax\else
% --------------------------------------------
% Lulu:
	\documentclass[a4paper,10pt,twoside]{book}
	\usepackage[
		papersize={6.13in,9.21in},
		hmargin={.75in,.75in},
		vmargin={.75in,1in},
		ignoreheadfoot
	]{geometry}
	\input{../common.tex}
	\setboolean{lulu}{true}
% --------------------------------------------
% A4:
%	\documentclass[a4paper,11pt,twoside]{book}
%	\input{../common.tex}
%	\usepackage{a4wide}
% --------------------------------------------
    \graphicspath{{figures/} {../figures/}}
	\begin{document}
\fi
%=================================================================
%\renewcommand{\nnbb}[2]{} % Disable editorial comments
\sloppy
%=================================================================
\chapter{Installing }\label{cha:basic}

\sd{Harvested from \url{http://wiki.squeak.org/squeak/2665} with permission from Chris Muller. (To be rewritten with his help.)}

Magma is a fully supported, multi-user object database for Squeak 3.7, 3.8, 3.9, 3.10, and Croquet distributions. 

Magma provides transparent access to a large-scale shared persistent object model, 
supports multiple users concurrently via optimistic locking,  utilizes a simple transaction protocol, including nested transactions, supports collaborative program development via live class evolution, peer-to-peer model sharing and Monticello integration,  supports for large, indexed collections with robust querying, runs with pretty good performance and provides performance tuning mechanisms, 
learning basic usage takes just a few minutes,  is fault tolerant,  includes a small suite of tools, is written 100\% in intelligible, well-factored Smalltalk.
Magma  includes rigorous SUnit tests that utilize multiple images to simulate testing in a multi-user environment.

Magma is stable and usable within its limits, and is still under active development.

\section{Vocabulary point}

\subsection{What is meant by transparent access?}

Most object programs that interface to a database do so using an explicit interface, meaning at any point in the program where you "need some data," you write code to make a call to the database (perhaps through some kind of persistence framework).

By contrast, when you connect your Squeak image to a Magma repository, you think of your local image as "expanded" to include the objects in the repository. Magma client programs can and should be written as if they are operating in one large image on a computer with lots of memory. You don't ever need to "read some data" because you are already "in" the data. Your program is free to explore the persistent object model and, through transactions, make changes. Magma uses weak collections for its caching, so your program only consumes as much memory as the objects it references.

This level of transparency allows existing objects which have never been designed to reside in a database to reside in Magma. For example, no Morph knows anything about Magma, but Morphs may be stored, shared and collaborated via Magma.

Object databases allow you to design complex domain relationships without reservation or compromise. Domain code remains pure, and completely independent of storage. 

A number of commit-strategies allow Magma to be suitable for a variety of programs and users.

\subsection{About optimistic locking}

Locking of objects is done in multi-user systems to preserve integrity of changes; so that one persons changes do not accidently overwrite another.

\paragraph{Optimistic locking.}

With optimistic locking, you write your program under the assumption that any commit has a chance to fail on account of at least one the objects being committed was changed by someone else since you began the transaction.

Optimistic locking offers reduced contention and higher performance than pessimistic locking. It also avoids deadlocks.

\paragraph{Pessimistic locking.}
In pessimistic locking, your program must explicitly obtain a lock on one or more objects before making any changes. This prevents any other session from making changes to those objects, so you can be more assured that committing the transaction will succeed.

Once changes are complete, the objects are unlocked so that others may make changes to them.

Pessimistic locking increases contention because objects are tied up for longer periods. Locking objects also imposes more work on the database server. When there is a conflict, sessions have to "get in line" to wait for an object to become unlocked. These sessions may, themselves, already have other objects locked up while they're waiting. If any of these objects needed by a session are farther ahead in line then you will have a "deadlock," which can be difficult to manage. ok




\section{Getting started with Magma}

\subsection{Getting Started}

To use Magma, first download and install the code. Then you may decide in what mode would your magma be running and follow these steps:

\paragraph{If Running in Server/Client Mode}

\begin{itemize}
	\item create a repository
	\item start the server
	\item open a session
\end{itemize}

\paragraph{If Running in Single-user Mode}
\begin{itemize}
\item create a repository
\item open a session, in a different way(see: Single-user mode below)
\end{itemize}

\subsection{Creating a repository}

To create a repository, you must provide two things:
\begin{itemize}
\item a path to the directory magma may use to keep and maintain its files
\item the root object of the repository
\end{itemize}

Magma maintains a single directory on the filesystem for each repository. When creating or opening Magma repositories, you specify the fully-qualified directory in which it may create its various files.

You should also provide the root object of the repository. This is the root of your domain object in the domain is referenced.

The code, thus:

\begin{code}{}
	MagmaRepositoryController
		create: 'c:\myMagmaFolder'
		root: Dictionary new
\end{code}

\paragraph{Starting the server.}

Magma utilizes TCP/IP for its network communications. To enable multi-user access to a Magma repository, you may start the Magma server in its own image and inspect the following:

\begin{code}{}
MagmaServerConsole new
	open: 'c:\myMagmaFolder' ;
	processOn: 51001
\end{code}

Be sure to inspect this so you will have access to console commands, such as \ct{shutdown}.

\subsection{Open a MagmaSession}

Once the server is running, a \clsind{MagmaSession} can be connected from the same or another image to gain access and change objects in the repository. Connecting a session requires the following information:

\begin{itemize}
	\item a name to identify your session. When someone tries to overlay changes you've made to objects, they will be notified that those objects were changed by your session, identified by the name you provide.
	\item the IP address of the machine hosting the Magma repository, and the port it is listening on.
\end{itemize}

\begin{code}{}
	| mySession |
	mySession := MagmaSession
		hostAddress: #(192 168 1 13) asByteArray
		port: 51001.
	mySession connectAs: 'chris'
\end{code}
If you run this from a Workspace, be sure to inspect the result so you will be able to properly disconnect from the server.

Once connected, changes to the persistent model are committed through transactions. 

\begin{code}{}
mySession commit: 
	[ mySession root
		at: 'persons'
		put: (OrderedCollection with: (Person name: 'Paula')) ]
\end{code}

While your session is connected, it is recommended you keep up-to-date every 30 seconds or so changes made by other users will be reflected in the objects you visit. If you know there are very few users then this less important, but your session may eventually get booted if you wait too long. You can do this simply by aborting, even if you don't have a transaction:

\begin{code}{}
	mySession abort
\end{code}
This will allow you to see changes to the persistent model that were made by others.

You can access the root of the repository later and navigate from there.

\begin{code}{}
	mySession root at: 'persons'
\end{code}

To minimize concurrency it is recommended that transactions begin just before you make changes, and commit immediately after.

Once you're done using the session, you should disconnect it:
\begin{code}{}
	mySession disconnect
\end{code}

\paragraph{Single-user mode}

If you know will be operating in a single-user environment, starting a second image to run the server may not always be convenient. Thankfully, Magma supports a "direct-connect" single-user mode, where a single Magma session connects directly to the repository in one image.

To run in single-user mode, you do not need to use MagmaServerConsole. 

Instead, you must specify the name of the object file when opening a MagmaSession instead of the IP and port.

\begin{code}{}
	myMagmaSession := MagmaSession openLocal: 'c:\myMagmaFolder\myRepository.magma'.
	myMagmaSession connectAs: 'chris'
\end{code}

Additionally, when you're ready to disconnect your last session, you also should close the repository:

\begin{code}{}
	myMagmaSession disconnect; closeRepository
\end{code}

The repository can then be opened again in single or multi-user mode.


\subsection{Objects are persisted by reachability}

With object databases, there is no notion of, "inserting objects into the database". Instead, you merely attach new objects to persistent ones and commit. All directly or indirectly referenced objects from any persistent object are automatically detected and saved into the database.

%%%%%%%%%%%%%%%%%%%%%%%%%%%%%%%%%%%%%%%%%%%%%%%%%%%%%%%%%%%%%%%%%%%%%%%%
%%%%%%%%%%%%%%%%%%%%%%%%%%%%%%%%%%%%%%%%%%%%%%%%%%%%%%%%%%%%%%%%%%%%%%%%
\section{Designing for an ODBMS}
Now we discuss some programming guidelines and best practices.
Here are some advices

\begin{itemize}
	\item Web programming caution!
	Do not share objects between MagmaSession instances. Two or more separate MagmaSessions should never reference the same identical object intance, including proxies. Each session must always keep its own separate view of the persistent model, they should never be linked across sessions. The persistent objects are still identical within the repository, just not in the image across session views.

\item Try to keep transactions short.

Short transactions are essential for good server health and avoiding getting kicked off. That's right. Magma server will disconnect any sessions that accumulate too many "challengers". The default maximumNumberOfChallengers is 1000. That means if 1000 commits by other users are performed and you haven't refreshed, you're outta there.

\item Custom indexes.

Custom index types must inherit from MaIndexDefinition because all index definitions are part of the protocol between clients and server. This also means that the server must be started after the custom index was created and must be present in the servers image.

Since the server requires no other domain classes (though it won't hurt to have them), you may wish to define your custom index types in a separate package.

\end{itemize}

There are a number of design considerations when writing a Magma program such as the commit strategies, 

%%%%%%%%%%%%%%%%%%%%%%%%%%%%
\subsection{Magma Commit Strategies}
One important and interesting question is how to avoid to maintain a MagmaSession in our domain objects. How can we perform the necessary commits, while still keeping it decoupled from Magma? There are at least three approaches that can be used.



\paragraph{1) Program-controlled transactions:} You signal commit notifications in your domain. A controller catches these notifications and applies them to its session. When no controller is listening, the notifications are ignored, allowing testing of the domain with or without the database. While you do not have to reference a MagmaSession, it is not completely transparent because you do have to signal the notifications:

\begin{code}{}
	name: aString
		MagmaSessionRequest signalCommit: [ name := aString ]
\end{code}

This approach gives the program control over when commits are performed. It also allows the user to be free of any concern about remembering to "save"; everything that requires a transaction will be committed without them thinking about it.

One disadvantage is reduced concurrency-detection. The message \ct{signalCommit:} does the refresh, update and commit without the user ever seeing whether the changed value was already changed. For this reason, this sort of commmit strategy is useful for low-concurrency models, \ie where different analysts work on their "own" domain objects.


\paragraph{2) User-controlled transactions, with auto-begin:} With this approach, the user is always in a transaction. The program provides a "Save" or "Commit" button which the user occasionally presses as they modify the domain model. The program could also provide its own "Cancel" function (which would send \ct{abort}, followed by \ct{begin} to the MagmaSession) to provide a rudimentary undo function. To use auto-begin, your program does an initial \ct{begin} when connecting the session, then every commit should instead use \ct{commitAndBegin} instead of \ct{commit}.

Since Magma needs transactions to be short, this approach is most suitable for programs with very frequent domain changes; such as programs that support "data entry" activities. Performance is increased because the \ct{begin} step occurs right after the commit automatically with only one trip to the server.

This approach also allows complete transparency; \ie, the domain can be written with no awareness of any database. No commits or signals are needed in the code.

Concurrency detection is also much better; consider our name: setter example from above. This time, the code is, simply:

\begin{code}{}
	name: aString
		name := aString
\end{code}

This time, if this object has changed the user will get a commit-error when they commit and the object will update to the current value.

Users decide when to "save", therefore the program should be written so the user has control only when the domain is in a consistent state worthy of being committed. For example, in a banking application, when a user initiates a transfer, don't let them commit between the withdraw and the deposit, only after both.

This approach is less favorable for programs that do not update the domain model frequently. Because users are always in a transaction, their MagmaSessions will not receive auto-abort signals. It is therefore up to the users to keep their views refreshed by either committing or aborting. If they do not, and there are lots of changes happening to the domain by other users, Magma will eventually terminate the an inactive users session, bloated with pending changes, and the user will have to log in again.


\paragraph{3) User-controlled transactions, user-controlled begin:} In this approach, the program provides the user with both a "begin" as well as "commit" button somewhere, so they are in complete control of transactions. This is just like the previous option except, since the user is not always in a transaction, auto-abort signals will be received. This is in exchange for the additioonal clicking the user must do to explicitly \ct{begin} their transactions.


\paragraph{About refreshPersistentObjectsEvenWhenChangedOnlyByMe}

Within each of these strategies, another option further enhances their flexibility:
\begin{code}{}
	mySession refreshPersistentObjectsEvenWhenChangedOnlyByMe: true
\end{code}

Setting this option to true does reduce performance slightly, but is used to be "more-connected" to the repository. It causes your changes to be lost if they were changed by someone else.

The default value is false which, when using commit-strategy (3), allows users to make changes to an object, "test" them, and, only after that, \ct{begin} and immediately \ct{commit} a transaction. The \ct{begin} is non-destructive to their own changes which were not changed by another user. In this way it is a sort of "merge" with the users local model.


%%%%%%%%%%%%%%%%%%%%%%%%%%%%%%%%%%%%%%%%%%%%%%%%%%%%%
%%%%%%%%%%%%%%%%%%%%%%%%%%%%%%%%%%%%%%%%%%%%%%%%%%%%%
\subsection{About read strategies}

Magma provides programs with transparent access to an arbitrarily large, persistent object model. Memory consumption is minimized by way of the well-known "Proxy" design pattern. Parts of the object-model that are not being used by your program are "truncated" in your local image with a lightweight proxy object. Should the program happen to send a message to the proxy, the object it represents is materialized from the repository automatically. The proxy then "becomes" the real object.

There is a cost to this materialization process, so doing too many of them may cause performance issues. When a proxy is materialized, it is usually assumed that you will send a message to it that will require access to at least one of its instance variables. For this reason, it is generally desirable to bring back the proxy and several of the objects it references.

But bringing back more objects creates more work for the server, network and your client. For ideal performance, you want each materialization to bring back just the objects that will be needed, but not too many that won't be.

Magma's default ReadStrategy will read one level deep by default, which means the proxy object and the objects each of its instance variables refers to. As a programmer, you can change this default 0, 1, 2 or 3 (or maybe even 4, but I wouldn't go beyond that) by supplying a ReadStrategy.

minimumDepth, plus delta-depth

The minimumDepth is how far it reads for any objects, all objects, every time, all the time. Therefore, you want to keep it very small, probably 0 or 1. Then, you can use a ReadStrategy to specify how much deeper to read, below the minimumDepth, by class and instance-variable. So what this means exactly is, whenever the server is reading a graph and encounters a buffer for the specified class, it will be sure to also grab n-levels deeper, regardless of how deep it already is, for whatever variable name specified. 

\paragraph{Example.}
Let's say you are about to display a list of Employees and their department names in a list. The Employee class defines a 'name' attribute, which is a String, and a 'department' object, an instance of class Department. Departments also have a name. If the list was to include each employee name and their department name, the default ReadStrategy would retrieve each Employee with its Department as a proxy. So the list would populate slowly because each Department would be materialized one at a time. To improve this performance, you could specify the ReadStrategy to read one additional level on the 'department' object.

\begin{code}{}
	| employeeListReadStrategy |
	employeeListReadStrategy :=
		(MaReadStrategy minimumDepth: 1)
			forVariableNamed: 'department'
			onAny: Employee
			readToDepth: 1
	mySession readStrategy: employeeListReadStrategy. "returns instantly, no server call"
	"now enumerate the collection"
	myEmployees do: [ :each | ".. display name and department name in your list.." ]
\end{code}



%%%%%%%%%%%%%%%%%%%%%%%%%%%%%%%%%%%%%%%%%%%%%%%%%%%%%
%%%%%%%%%%%%%%%%%%%%%%%%%%%%%%%%%%%%%%%%%%%%%%%%%%%%%
\subsection{Optimizing Magma's Performance}

Efficiency was always a goal when building Magma. The work-load is heavy for the client but relatively light for the server, especially with short transactions allowing, theoretically, for consistent performance more client sessions connect.

If profiling your program reveals a lot of time spent in Magma, considering the following performance-sensitive guidelines may help.


\paragraph{Use ReadStrategies.} Read strategies can be used to optimize how many objects are accessed within a single call to the server.

\paragraph{Keep your commits medium-small.} Commits should be put in your program as close to the mutations to the persistent model as possible. Commits are serialized on the server, so large commits that take several seconds will most likely cause requests to queue in the server.

At the same time, you don't want commits to be so microscopic that you end up smothering the network with requests. For example, if building an OrderedCollection of 100 medium-sized objects, you should do those in one commit instead of 100 commits. However, if the objects are very large and completely non-persistent, you may want to do 100 commits.

\paragraph{Keep your cachedObjectCount as low as possible.}

With a connected Magma session, evaluate: \ct{mySession cachedObjectCount}
This number reprensents how many entries Magma has in its IdentityDictionaries. Magma tries to avoid the performance issues related to Squeak's IdentityDictionaries, but it can still slow down if you allow tens of thousands of objects to be cached in memory.

If you're not sure why your \ct{cachedObjectCount} is growing, you can use \ct{cachedObjectCountByClass} to see which ones are the most proliferate (they are sorted by most-occurrences at the top). If you see "UndefinedObject" near the top of the list, you need to send \ct{finalizeOids} to your session. This is because Squeak can be lazy about finalizing the entries in its \ct{WeakDictionary}'s.

As you traverse parts of the model, you should \ct{stubOut:} objects you no longer need. For example, after iterating a collection of large objects, \ct{stubOut:} the collection object if you no longer need them. \ct{MagmaSession>>stubOut:} chops off large branches of objects so the memory they consume can be reclaimed by the garbage collector.

But avoid too many calls to \ct{stubOut:}. For example, after you've enumerated the collection of large objects, \ct{stubOut:} the Collection object itself, not each object in the collection. This is due to unfortunate irony that \ct{stubOut:} requires use of one of Squeak's most inefficienct methods; \ct{Dictionary>>>removeKey:}. While fast in other Smalltalks, this method is VERY slow in Squeak but required for \ct{stubOut:}.

Finally, after you've stubbed out a large object, you may find it necessary to call "mySession finalizeOids". Unfortunately, Squeak's WeakIdentityKeyDictionary does not always remove finalized entries in a timely fashion, resulting in, once again, these very important Dictionaries slowing everything down.
 mySession finalizeOids

\paragraph{Other optimizations.}

Don't use MagmaSession>>refreshPersistentObjectsEvenWhenChangedOnlyByMe.
Use \ct{commitAndBegin} for bulk-load programs. Experiment and optimize your key and record sizes of your MagmaCollections. Avoid too many duplicate keys (\eg, don't index the word "the").


\paragraph{Know thy indexes.}
MagmaCollections have good read performance, but adding and removing objects is very slow. In theory, starting with an empty MagmaCollection, the rate-of-insertion will deteriorate a little bit before settling on a relatively fixed rate, IF you have a good key-dispersal.

If you put in a lot of duplicate keys, it will gradually get more costly to keep adding more of that key because a linear search for the "end" of that chain of keys is performed to find the point of insertion. So, for example, when you build a simple keyword index, consider eliminating prepositions such as "the" and "at".

Removing from MagmaCollections is even more expensive than insertions. Avoid using this operation for performance-intensive operations.











%=================================================================
\ifx\wholebook\relax\else\end{document}\fi
%=================================================================

%-----------------------------------------------------------------

%%% Local Variables:
%%% coding: utf-8
%%% mode: latex
%%% TeX-master: t
%%% TeX-PDF-mode: t
%%% ispell-local-dictionary: "english"
%%% End:
% $Author$
% $Date$
% $Revision$

% HISTORY:
% 2008-01-15 - Stef first draft based on Vassily Bykov's documentation
% 2008-08-06 - Alex revised
% 2008-11-25 - Oscar revised
% 2009-04-17 - Fabrizio Perin reviewed
% 2009-04-18 - Jorge Ressia reviewed
% 2009-07-15 - Oscar indexing
% 2011-09-11 - Migrated to PharoBox: svn checkout https://XXX@scm.gforge.inria.fr/svn/pharobooks/PharoByExampleTwo-Eng
% could change returns by answers 


%=================================================================
\ifx\wholebook\relax\else
% --------------------------------------------
% Lulu:
	\documentclass[a4paper,10pt,twoside]{book}
	\usepackage[
		papersize={6.13in,9.21in},
		hmargin={.75in,.75in},
		vmargin={.75in,1in},
		ignoreheadfoot
	]{geometry}
	\input{../common.tex}
	\pagestyle{headings}
	\setboolean{lulu}{true}
% --------------------------------------------
% A4:
%	\documentclass[a4paper,11pt,twoside]{book}
%	\input{../common.tex}
%	\usepackage{a4wide}
% --------------------------------------------
    \graphicspath{{figures/} {../figures/}}
	\begin{document}
\fi
%=================================================================
%\renewcommand{\nnbb}[2]{} % Disable editorial comments
\sloppy
%=================================================================
\chapter{Regular Expressions in \pharo}\chalabel{regex}

\chapterauthor{\authorsteph{}\\ \authoroscar{}}

\indexmain{Regular expressions|see{Regex}}
Regular expressions are widely used in many scripting languages such as \ind{Perl}, \ind{Python} and \ind{Ruby}.
They are useful to identify strings that match a certain pattern, to check that input conforms to an expected format, and to rewrite strings to new formats.
\pharo also supports regular expressions due to the \pkgind{Regex} package contributed by Vassili Bykov. \index{Bykov, Vassili}
Regex is installed by default in \pharo. If you are using an older image that does not include Regex the Regex package, you can install it yourself from \ind{\sqsrc}\footnote{\url{http://www.squeaksource.com/Regex.html}}.

A regular expression\footnote{\url{http://en.wikipedia.org/wiki/Regular_expression}} is a template that matches a set of strings.
For example, the regular expression \ct{'h.*o'} will match the strings \ct{'ho'}, \ct{'hiho'} and \ct{'hello'}, but it will not match \ct{'hi'} or \ct{'yo'}.
We can see this in \pharo as follows:
\mthindex{String}{matchesRegex:}
\begin{code}{@TEST}
'ho' matchesRegex: 'h.*o'     --> true
'hiho' matchesRegex: 'h.*o'  --> true
'hello' matchesRegex: 'h.*o' --> true
'hi' matchesRegex: 'h.*o'      --> false
'yo' matchesRegex: 'h.*o'     --> false
\end{code}

In this chapter we will start with a small tutorial example in which we will develop a couple of classes to generate a very simple site map for a web site.
We will use regular expressions
(i) to identify \ind{HTML} files,
(ii) to strip the full path name of a file down to just the file name,
(iii) to extract the title of each web page for the site map, and
(iv) to generate a relative path from the root directory of the web site to the HTML files it contains.
After we complete the tutorial example, we will provide a more complete description of the Regex package, based largely on Vassili Bykov's documentation provided in the package.\footnote{The original documentation can be found on the class side of \ct{RxParser}.}

%=================================================================
\section{Tutorial example\,---\,generating a site map}

% All the code is in the package PBE-Regex in http://www.squeaksource.com/SqueakByExample

Our job is to write a simple application that will generate a site map for a web site that we have stored locally on our hard drive.  The site map will contain links to each of the HTML files in the web site, using the title of the document as the text of the link. Furthermore, links will be indented to reflect the directory structure of the web site.

%-----------------------------------------------------------------
\subsection{Accessing the web directory}

\dothis{If you do not have a web site on your machine, copy a few HTML files to a local directory to serve as a test bed.}

We will develop two classes, \ct{WebDir} and \ct{WebPage}, to represent directories and web pages.  The idea is to create an instance of \ct{WebDir} which will point to the root directory containing our web site.  When we send it the message \ct{makeToc}, it will walk through the files and directories inside it to build up the site map.  It will then create a new file, called \ct{toc.html}, containing links to all the pages in the web site.

One thing we will have to watch out for: each \ct{WebDir} and \ct{WebPage} must remember the path to the root of the web site, so it can properly generate links relative to the root.

\dothis{Define the class \ct{WebDir} with instance variables \ct{webDir} and \ct{homePath}, and define the appropriate initialization method.
Also define class-side methods to prompt the user for the location of the web site on your computer, as follows:}

\begin{code}{}
WebDir>>setDir: dir home: path 
	webDir := dir.
	homePath := path

WebDir class>>onDir: dir
	^ self new setDir: dir home: dir pathName

WebDir class>>selectHome
	^ self onDir: FileList modalFolderSelector
\end{code}

The last method opens a browser to select the directory to open.
Now, if you inspect the result of \ct{WebDir selectHome}, you will be prompted for the directory containing your web pages, and you will be able to verify that \ct{webDir} and \ct{homePath} are properly initialized to the directory holding your web site and the full path name of this directory.

\begin{figure}[tbh]
\begin{center}
\includegraphics[width=0.8\textwidth]{aWebDir}
\caption{A WebDir instance}
\figlabel{aWebDir}
\end{center}
\end{figure}

It would be nice to be able to programmatically instantiate a \ct{WebDir}, so let's add another creation method.

\dothis{Add the following methods and try it out by inspecting the result of \mbox{\lct{WebDir onPath: '\emph{path to your web site}'}}.}

\begin{code}{}
WebDir class>>onPath: homePath
	^ self onPath: homePath home: homePath

WebDir class>>onPath: path home: homePath
	^ self new setDir: (FileDirectory on: path) home: homePath
\end{code}

%-----------------------------------------------------------------
\subsection{Pattern matching HTML files}

So far so good.
Now we would like to use regexes to find out which HTML files this web site contains.

If we browse the \ct{FileDirectory} class, we find that the method \ct{fileNames} will list all the files in a directory. We want to select just those with the file extension \ct{.html}. The regex that we need is \ct{'.*\.html'}. The first dot will match any character except a newline:

\begin{code}{@TEST}
'x' matchesRegex: '.' --> true
' ' matchesRegex: '.'  --> true
Character cr asString matchesRegex: '.' --> false
\end{code}

\index{Regex syntax!@\ct{*}}
The \ct{*} (known as the ``\ind{Kleene star}'', after Stephen Kleene, who invented it) is a regex operator that will match the preceding regex any number of times (including zero).

\mthindex{String}{matchesRegex:}
\begin{code}{@TEST}
'' matchesRegex: 'x*'     --> true
'x' matchesRegex: 'x*'   --> true
'xx' matchesRegex: 'x*' --> true
'y' matchesRegex: 'x*'   --> false
\end{code}

\index{Regex syntax!@\ct{.}}
Since the dot is a special character in regexes, if we want to literally match a dot, then we must escape it.

\begin{code}{@TEST}
'.' matchesRegex: '.'   --> true
'x' matchesRegex: '.'  --> true
'.' matchesRegex: '\.'  --> true
'x' matchesRegex: '\.' --> false
\end{code}

Now let's check our regex to find HTML files works as expected.

\begin{code}{@TEST}
'index.html' matchesRegex: '.*\.html' --> true
'foo.html' matchesRegex: '.*\.html'    --> true
'style.css' matchesRegex: '.*\.html'   --> false
'index.htm' matchesRegex: '.*\.html' --> false
\end{code}

Looks good. Now let's try it out in our application.

\dothis{Add the following method to \ct{WebDir} and try it out on your test web site.}

\begin{code}{}
WebDir>>htmlFiles
	^ webDir fileNames select: [ :each | each matchesRegex: '.*\.html' ]
\end{code}

If you send \ct{htmlFiles} to a \ct{WebDir} instance and \menu{print it}, you should see something like this:

\begin{code}{}
(WebDir onPath: '...') htmlFiles --> #('index.html' ...)
\end{code}

%-----------------------------------------------------------------
\subsection{Caching the regex}

Now, if you browse \mthind{String}{matchesRegex:}, you will discover that it is an extension method of \ct{String} that creates a fresh instance of \clsind{RxParser} every time it is sent.  That is fine for ad hoc queries, but if we are applying the same regex to every file in a web site, it is smarter to create just one instance of \ct{RxParser} and reuse it. Let's do that.

\dothis{Add a new instance variable \ct{htmlRegex} to \ct{WebDir} and initialize it by sending \ct{asRegex} to our regex string.  Modify \ct{WebDir>>htmlFiles} to use the same regex each time as follows:}

\begin{code}{}
WebDir>>initialize
	htmlRegex := '.*\.html' asRegex

WebDir>>htmlFiles
	^ webDir fileNames select: [ :each | htmlRegex matches: each ]
\end{code}

Now listing the HTML files should work just as it did before, except that we reuse the same regex object many times.

%-----------------------------------------------------------------
\subsection{Accessing web pages}

Accessing the details of individual web pages should be the responsibility of a separate class, so let's define it, and let the \ct{WebDir} class create the instances.

\dothis{Define a class \ct{WebPage} with instance variables \ct{path}, to identify the HTML file, and \ct{homePath}, to identify the root directory of the web site.  (We will need this to correctly generate links from the root of the web site to the files it contains.) Define an initialization method on the instance side and a creation method on the class side.}

\begin{code}{}
WebPage>>initializePath: filePath homePath: dirPath 
	path := filePath.
	homePath := dirPath

WebPage class>>on: filePath forHome: homePath
	^ self new initializePath: filePath homePath: homePath
\end{code}

A \ct{WebDir} instance should be able to return a list of all the web pages it contains.

\dothis{Add the following method to \ct{WebDir}, and inspect the return value to verify that it works correctly.}

\begin{code}{}
WebDir>>webPages
	^ self htmlFiles collect: 
		[ :each | WebPage 
			on: webDir pathName, '/', each
			forHome: homePath ]
\end{code}

You should see something like this:

\begin{code}{}
(WebDir onPath: '...') webPages --> an Array(a WebPage a WebPage ...)
\end{code}

%-----------------------------------------------------------------
\subsection{String substitutions}

That's not very informative, so let's use a regex to get the actual file name for each web page.
To do this, we want to strip away all the characters from the path name up to the last directory.
On a Unix file system directories end with a slash (\ct{/}), so we need to delete everything up to the last slash in the file path.

The \ct{String} extension method \mthind{String}{copyWithRegex:matchesReplacedWith:} does what we want:

\begin{code}{@TEST}
'hello' copyWithRegex: '[elo]+' matchesReplacedWith: 'i' --> 'hi'
\end{code}

\index{Regex syntax!@\ct{+}}
In this example the regex \ct{[elo]} matches any of the characters \ct{e}, \ct{l} or \ct{o}.
The operator \ct{+} is like the \ind{Kleene star}, but it matches exactly \emph{one} or more instances of the regex preceding it. Here it will match the entire substring \ct{'ello'} and replay it in a fresh string with the letter \ct{i}.

\dothis{Add the following method and verify that it works as expected.}

\begin{code}{}
WebPage>>fileName
	^ path copyWithRegex: '.*/' matchesReplacedWith: ''
\end{code}

Now you should see something like this on your test web site:

\begin{code}{}
(WebDir onPath: '...') webPages collect: [:each | each fileName ]
  --> #('index.html' ...)
\end{code}

%-----------------------------------------------------------------
\subsection{Extracting regex matches}

Our next task is to extract the title of each HTML page.

First we need a way to get at the contents of each page.  This is straightforward.

\dothis{Add the following method and try it out.}

\mthindex{FileStream}{oldFileOrNoneNamed:}
\begin{code}{}
WebPage>>contents
	^ (FileStream oldFileOrNoneNamed: path) contents
\end{code}

Actually, you might have problems if your web pages contain non-ascii characters, in which case you might be better off with the following code:

\clsindex{Latin1TextConverter}
\begin{code}{}
WebPage>>contents
	^ (FileStream oldFileOrNoneNamed: path)
		converter: Latin1TextConverter new;
		contents
\end{code}

You should now be able to see something like this:

\begin{code}{}
(WebDir onPath: '...') webPages first contents --> '<head>
<title>Home Page</title>
...
'
\end{code}

Now let's extract the title. In this case we are looking for the text that occurs \emph{between} the HTML tags \ct{<title>} and \ct{</title>}. 

\index{Regex syntax!@\ct{^}}
What we need is a way to extract \emph{part} of the match of a regular expression. Subexpressions of regexes are delimited by parentheses.  Consider the regex \ct{([CARETaeiou]+)([aeiou]+)}. It consists of two subexpressions, the first of which will match a sequence of one or more non-vowels, and the second of which will match one or more vowels. (The operator \ct{CARET} at the start of a bracketed set of characters negates the set.\footnote{NB: In \pharo the caret is also the return keyword, which we write as \ct{^}. To avoid confusion, we will write \ct{CARET} when we are using the caret within regular expressions to negate sets of characters, but you should not forget, they are actually the same thing.}) Now we will try to match a \emph{prefix} of the string \ct{'pharo'} and extract the submatches:

\mthindex{RxMatcher}{matchesPrefix:}
\mthindex{RxMatcher}{subexpression:}
\begin{code}{| re |}
re := '([CARETaeiou]+)([aeiou]+)' asRegex.
re matchesPrefix: 'pharo' --> true
re subexpression: 1         --> 'pha'
re subexpression: 2         --> 'ph'
re subexpression: 3         --> 'a'
\end{code}

After successfully matching a regex against a string, you can always send it the message \ct{subexpression: 1} to extract the entire match.  You can also send \lct{subexpression: $n$} where $n-1$ is the number of subexpressions in the regex. The regex above has two subexpressions, numbered $2$ and $3$.

We will use the same trick to extract the title from an HTML file.

\dothis{Define the following method:}

\mthindex{String}{asRegexIgnoringCase}
\begin{code}{}
WebPage>>title
	| re |
	re := '[\w\W]*<title>(.*)</title>' asRegexIgnoringCase.
	^ (re matchesPrefix: self contents)
		ifTrue: [ re subexpression: 2 ]
		ifFalse: [ '(', self fileName, ' -- untitled)' ]
\end{code}

There are a couple of subtle points to notice here.
First, HTML does not care whether tags are upper or lower case, so we must make our regex case insensitive by instantiating it with \ct{asRegexIgnoringCase}.

Second, since dot matches any character \emph{except a newline}, the regex \mbox{\lct{.*<title>(.*)</title>}} will not work as expected if multiple lines appear before the title.
The regex \ct{\w} matches any alphanumeric, and \ct{\W} will match any non-alphanumeric, so \ct{[\w\W]} will match any character \emph{including newlines}.
(If we expect titles to possible contain newlines, we should play the same trick with the subexpression.)

Now we can test our title extractor, and we should see something like this:

\begin{code}{}
(WebDir onPath: '...') webPages first title --> 'Home page'
\end{code}

%-----------------------------------------------------------------
\subsection{More string substitutions}

In order to generate our site map, we need to generate links to the individual web pages.
We can use the document title as the name of the link.  We just need to generate the right path to the web page from the root of the web site.
Luckily this is trivial\,---\,it is simple the full path to the web page minus the full path to the root directory of the web site.

We must only watch out for one thing.  Since the \ct{homePath} variable does not end in a \ct{/}, we must append one, so that relative path does not include a leading \ct{/}. Notice the difference between the following two results:

\mthindex{String}{copyWithRegex:matchesReplacedWith:}
\begin{code}{}
'/home/testweb/index.html' copyWithRegex: '/home/testweb' matchesReplacedWith: '' --> '/index.html'
'/home/testweb/index.html' copyWithRegex: '/home/testweb/' matchesReplacedWith: '' -->  'index.html'
\end{code}

The first result would give us an absolute path, which is probably not what we want.

\dothis{Define the following methods:}

\begin{code}{}
WebPage>>relativePath
	^ path 
		copyWithRegex: homePath , '/'
		matchesReplacedWith: ''

WebPage>>link
	^ '<a href="', self relativePath, '">', self title, '</a>'
\end{code}

You should now be able to see something like this:

\begin{code}{}
(WebDir onPath: '...') webPages first link --> '<a href="index.html">Home Page</a>'
\end{code}

%-----------------------------------------------------------------
\subsection{Generating the site map}

Actually, we are now done with the regular expressions we need to generate the site map.  We just need a few more methods to complete the application.

\dothis{If you want to see the site map generation, just add the following methods.}

If our web site has subdirectories, we need a way to access them:
\begin{code}{}
WebDir>>webDirs
	^ webDir directoryNames
		collect: [ :each | WebDir onPath: webDir pathName , '/' , each home: homePath ]
\end{code}

We need to generate HTML bullet lists containing links for each web page of a web directory.
Subdirectories should be indented in their own bullet list.
\begin{code}{}
WebDir>>printTocOn: aStream 
	self htmlFiles
		ifNotEmpty: [
			aStream nextPutAll: '<ul>'; cr.
			self webPages
				do: [:each | aStream nextPutAll: '<li>';
						 nextPutAll: each link;
						 nextPutAll: '</li>'; cr].
			self webDirs
				do: [:each | each printTocOn: aStream].
			aStream nextPutAll: '</ul>'; cr]
\end{code}

We create a file called ``toc.html'' in the root web directory and dump the site map there.
\begin{code}{}
WebDir>>tocFileName
	^ 'toc.html'

WebDir>>makeToc
	| tocStream |
	tocStream := webDir newFileNamed: self tocFileName.
	self printTocOn: tocStream.
	tocStream close.
\end{code}

Now we can generate a table of contents for an arbitrary web directory!
\begin{code}{}
WebDir selectHome makeToc
\end{code}

\begin{figure}[tbh]
\begin{center}
\includegraphics[width=\textwidth]{PBE-toc}
\caption{A small site map}
\figlabel{PBE-toc}
\end{center}
\end{figure}

%=================================================================
\section{Regex syntax}

\indexmain{Regex syntax}
We will now have a closer look at the syntax of regular expressions as supported by the Regex package.

The simplest regular expression is a single character.  It matches exactly that character. A sequence of characters matches a string with exactly the same sequence of characters:
\mthindex{String}{matchesRegex:}
\begin{code}{@TEST}
'a' matchesRegex: 'a'                  --> true
'foobar' matchesRegex: 'foobar'  --> true
'blorple' matchesRegex: 'foobar' --> false
\end{code}

Operators are applied to regular expressions to
produce more complex regular expressions. Sequencing (placing expressions one
after another) as an operator is, in a certain sense, ``invisible''\,---\,yet it is
arguably the most common.

\indexmain{Regex syntax!@\ct{*}}
We have already seen the \ind{Kleene star} (\ct{*}) and the \ct{+} operator.
A regular expression followed by an asterisk matches any number (including 0) of matches of the original expression. For example:
\begin{code}{@TEST}
'ab' matchesRegex: 'a*b'         --> true
'aaaaab' matchesRegex: 'a*b' --> true
'b' matchesRegex: 'a*b'           --> true
'aac' matchesRegex: 'a*b'	    --> false    "b does not match"
\end{code}

\index{Regex!operator precedence}
The Kleene star has higher precedence than sequencing. A star applies to the
shortest possible subexpression that precedes it. For example, \ct{ab*}
means \ct{a} followed by zero or more occurrences of \ct{b}, not ``zero or more
occurrences of \ct{ab}'':
\begin{code}{@TEST}
'abbb' matchesRegex: 'ab*' --> true
'abab' matchesRegex: 'ab*' --> false
\end{code}

\indexmain{Regex syntax!@\ct{()}}
To obtain a regex that matches ``zero or more occurrences of \ct{ab}'', we must enclose \ct{ab} in parentheses:
\begin{code}{@TEST}
'abab' matchesRegex: '(ab)*'   --> true
'abcab' matchesRegex: '(ab)*' --> false    "c spoils the fun"
\end{code}

\indexmain{Regex syntax!@\ct{+}}
\indexmain{Regex syntax!@\ct{?}}
Two other useful operators similar to \ct{*} are \ct{+} and \ct{?}.
\ct{+} matches one or more instances of the regex it modifies, and \ct{?} will match zero or one instance.
\begin{code}{@TEST}
'ac' matchesRegex: 'ab*c'	   --> true
'ac' matchesRegex: 'ab+c'	  --> false    "need at least one b"
'abbc' matchesRegex: 'ab+c' --> true
'abbc' matchesRegex: 'ab?c' --> false    "too many b's"
\end{code}

\indexmain{Regex syntax!escape}
As we have seen, the characters \ct{*}, \ct{+}, \ct{?}, \ct{(}, and \ct{)} have special meaning within regular expressions. If we need to match any of them literally, it should be escaped by preceding it with a backslash \ct{\}. Thus, backslash is also special character, and needs to be escaped for a literal match. The same holds for all further special characters we will see.
\begin{code}{@TEST}
'ab*' matchesRegex: 'ab*'  --> false    "star in the right string is special"
'ab*' matchesRegex: 'ab\*' --> true
'a\c' matchesRegex: 'a\\c'  --> true
\end{code}

\indexmain{Regex syntax!@\ct{|}}
The last operator is \ct{|}, which expresses choice between two subexpressions.
It matches a string if either of the two subexpressions matches the string.
It has the lowest precedence\,---\,even lower than sequencing. For example, \ct{ab*|ba*} means ``a followed by any number of b's, or b followed by any number of a's'':
\begin{code}{@TEST}
'abb' matchesRegex: 'ab*|ba*'   --> true
'baa' matchesRegex: 'ab*|ba*'	--> true
'baab' matchesRegex: 'ab*|ba*' --> false
\end{code}

A bit more complex example is the expression \ct{c(a|d)+r}, which matches the name of any of the Lisp-style car, cdr, caar, cadr, ... functions:
\begin{code}{@TEST}
'car' matchesRegex: 'c(a|d)+r'   --> true
'cdr' matchesRegex: 'c(a|d)+r'   --> true
'cadr' matchesRegex: 'c(a|d)+r' --> true
\end{code}

It is possible to write an expression that matches an empty string, for example the expression \ct{a|} matches an empty string.  However, it is an error to apply \ct{*}, \ct{+}, or \ct{?} to such an expression: \ct{(a|)*} is invalid.

\indexmain{Regex syntax!character set}
So far, we have used only characters as the \emph{smallest} components of regular expressions. There are other, more interesting, components. A character set is a string of characters enclosed in square brackets. It matches any single character if it appears between the brackets. For example, \ct{[01]} matches either \ct{0} or \ct{1}:
\begin{code}{@TEST}
'0' matchesRegex: '[01]'   --> true
'3' matchesRegex: '[01]'   --> false
'11' matchesRegex: '[01]' --> false  "a set matches only one character"
\end{code}

Using plus operator, we can build the following binary number recognizer:
\begin{code}{@TEST}
'10010100' matchesRegex: '[01]+' --> true
'10001210' matchesRegex: '[01]+' --> false
\end{code}

If the first character after the opening bracket is \ct{CARET}, the set is inverted: it matches any single character \emph{not} appearing between the brackets:
\begin{code}{@TEST}
'0' matchesRegex: '[CARET01]' --> false
'3' matchesRegex: '[CARET01]' --> true
\end{code}

\indexmain{Regex syntax!character range}
For convenience, a set may include ranges: pairs of characters separated by a hyphen (\ct{-}). This is equivalent to listing all characters in between: \ct{'[0-9]'} is the same as \ct{'[0123456789]'}.
Special characters within a set are \ct{CARET}, \ct{-}, and \ct{]}, which closes the set. Below are examples how to literally match them in a set:
\begin{code}{@TEST}
'CARET' matchesRegex: '[01CARET]'   --> true    "put the caret anywhere except the start"
'-' matchesRegex: '[01-]' --> true    "put the hyphen at the end"
']' matchesRegex: '[]01]'   --> true    "put the closing bracket at the start"
\end{code}

Thus, empty and universal sets cannot be specified.

%-----------------------------------------------------------------
\subsection{Character classes}
Regular expressions can also include the following backquote escapes to refer to popular classes of characters: \ct{\w} to match alphanumeric characters, \ct{\d} to match digits, and \ct{\s} to match whitespace.
Their upper-case variants, \ct{\W}, \ct{\D} and \ct{\S}, match the complementary characters (non-alphanumerics, non-digits and non-whitespace).
We can see a summary of the syntax seen so far in \tabref{regexsyntax}.

\indexmain{Regex syntax}
\begin{table}
\centering
	\begin{tabular}{ll}
		\toprule
		Syntax & What it represents \\
		\midrule
		\lct{a}				&	literal match of character \lct{a} \\
		\lct{.}				&	match any char (except newline) \\
		\lct{($\cdots$)}		&	group subexpression \\
		\lct{{\escape}}	&	escape following special character \\
		\midrule
		\lct{*}				&	Kleene star\,---\,match previous regex zero or more times \\
		\lct{+}				&	match previous regex one or more times \\
		\lct{?}				&	match previous regex zero times or once \\
		\lct{|}				&	match choice of left and right regex \\
		\midrule
		\lct{[abcd]}		&	match choice of characters \lct{abcd} \\
		\lct{[{\caret}abcd]}	&	match negated choice of characters \\
		\lct{[0-9]}		&	match range of characters \lct{0} to \lct{9} \\
		\midrule
		\lct{{\escape}w}			&	match alphanumeric \\
		\lct{{\escape}W}			&	match non-alphanumeric \\
		\lct{{\escape}d}			&	match digit \\
		\lct{{\escape}D}			&	match non-digit \\
		\lct{{\escape}s}			&	match space \\
		\lct{{\escape}S}			&	match non-space \\
		\bottomrule
	\end{tabular}
	\caption{Regex Syntax in a Nutshell\tablabel{regexsyntax}}
\end{table}


As mentioned in the introduction, regular expressions are especially useful for validating user input, and character classes turn out to be especially useful for defining such regexes.
For example, non-negative numbers can be matched with the regex \ct{\d+}:

\begin{code}{@TEST}
'42' matchesRegex: '\d+' --> true
'-1' matchesRegex: '\d+' --> false
\end{code}

Better yet, we might want to specify that non-zero numbers should not start with the digit 0:

\begin{code}{@TEST}
'0' matchesRegex: '0|([1-9]\d*)'     --> true
'1' matchesRegex: '0|([1-9]\d*)'     --> true
'42' matchesRegex: '0|([1-9]\d*)'   --> true
'099' matchesRegex: '0|([1-9]\d*)' --> false    "leading 0"
\end{code}

We can check for negative and positive numbers as well:

\begin{code}{@TEST}
'0' matchesRegex: '(0|((\+|-)?[1-9]\d*))'     --> true
'-1' matchesRegex: '(0|((\+|-)?[1-9]\d*))'   --> true
'42' matchesRegex: '(0|((\+|-)?[1-9]\d*))'   --> true
'+99' matchesRegex: '(0|((\+|-)?[1-9]\d*))' --> true
'-0' matchesRegex: '(0|((\+|-)?[1-9]\d*))'   --> false    "negative zero"
'01' matchesRegex: '(0|((\+|-)?[1-9]\d*))'   --> false    "leading zero"
\end{code}

Floating point numbers should require at least one digit after the dot:

\begin{code}{@TEST}
'0' matchesRegex: '(0|((\+|-)?[1-9]\d*))(\.\d+)?'      --> true
'0.9' matchesRegex: '(0|((\+|-)?[1-9]\d*))(\.\d+)?'   --> true
'3.14' matchesRegex: '(0|((\+|-)?[1-9]\d*))(\.\d+)?' --> true
'-42' matchesRegex: '(0|((\+|-)?[1-9]\d*))(\.\d+)?'  --> true
'2.' matchesRegex: '(0|((\+|-)?[1-9]\d*))(\.\d+)?'     --> false    "need digits after ."
\end{code}

%Checking if aString is a fixed-point number, with at least one digit is required after a dot:
%\begin{code}{}
%'' matchesRegex: '(\+|-)?\d+(\.\d+)?'
%The same, but allow notation like '123.':
%'' matchesRegex: '(\+|-)?\d+(\.\d*)?'
%\end{code}
%Recognizer for a string that might be a name: one word with first capital letter, no blanks, no digits.  More traditional:
%\begin{code}{}
%'' matchesRegex: '[A-Z][A-Za-z]*'
%more Smalltalkish:
%'' matchesRegex: ':isUppercase::isAlphabetic:*'
%\end{code}
%A date in format MMM DD, YYYY with any number of spaces in between, in XX century:
%\begin{code}{}
%'' matchesRegex: '(Jan|Feb|Mar|Apr|May|Jun|Jul|Aug|Sep|Oct|Nov|Dec)[ ]+(\d\d?)[ ]*,[ ]*19(\d\d)'
%\end{code}
%Note parentheses around some components of the expression above. As the Usage Section shows, they will allow us to obtain the actual strings that have matched them (\ie month name, day number, and year number).

For dessert, here is a recognizer for a general number format: anything like \ct{999}, or \ct{999.999}, or \ct{-999.999e+21}.
\begin{code}{@TEST}
'-999.999e+21' matchesRegex: '(\+|-)?\d+(\.\d*)?((e|E)(\+|-)?\d+)?' --> true
\end{code}

Character classes can also include the grep(1)-compatible elements listed in \tabref{charclasses}.

\indexmain{Regex syntax!character classes}
\begin{table}[htb]
\centering
	\begin{tabular}{lp{8cm}}
		\toprule
		Syntax & What it represents \\
		\midrule
\lct{[:alnum:]} & any alphanumeric \\
\lct{[:alpha:]} & any alphabetic character\\
\lct{[:cntrl:]} & any control character (ascii code is \lct{< 32})\\
\lct{[:digit:]} & any decimal digit\\
\lct{[:graph:]} & any graphical character (ascii code \lct{>= 32})\\
\lct{[:lower:]} & any lowercase character\\
\lct{[:print:]} & any printable character (here, the same as \lct{[:graph:]})\\
\lct{[:punct:]} & any punctuation character\\
\lct{[:space:]} & any whitespace character\\
\lct{[:upper:]} & any uppercase character\\
\lct{[:xdigit:]} & any hexadecimal character \\
		\bottomrule
	\end{tabular}
	\caption{Regex character classes\tablabel{charclasses}}
\end{table}

Note that these elements are components of the character classes, \ie they have to be enclosed in an extra set of square brackets to form a valid regular expression.  For example, a non-empty string of digits would be represented as \ct{[[:digit:]]+}. The above primitive expressions and operators are common to many implementations of regular expressions.

\begin{code}{@TEST}
'42' matchesRegex: '[[:digit:]]+' --> true
\end{code}

%-----------------------------------------------------------------
\subsection{Special character classes}
The next primitive expression is unique to this Smalltalk implementation. A sequence of characters between colons is treated as a unary selector which is supposed to be understood by characters. A character matches such an expression if it answers true to a message with that selector. This allows a more readable and efficient way of specifying character classes. For example, \ct{[0-9]} is equivalent to \ct{:isDigit:}, but the latter is more efficient. Analogously to character sets, character classes can be negated: \ct{:CARETisDigit:} matches a character that answers \ct{false} to \ct{isDigit}, and is therefore equivalent to \ct{[CARET0-9]}.

So far we have seen the following equivalent ways to write a regular expression that matches a non-empty string of digits: \ct{[0-9]+}, \ct{\d+}, \ct{[\d]+}, \ct{[[:digit:]]+}, \ct{:isDigit:+}.

\begin{code}{@TEST}
'42' matchesRegex: '[0-9]+'      --> true
'42' matchesRegex: '\d+'           --> true
'42' matchesRegex: '[\d]+'         --> true
'42' matchesRegex: '[[:digit:]]+' --> true
'42' matchesRegex: ':isDigit:+'  --> true
\end{code}

%-----------------------------------------------------------------
\subsection{Matching boundaries}
The last group of special primitive expressions is shown in \tabref{boundaries}, and is used to match boundaries of strings.

\indexmain{Regex syntax!matching string boundaries}
\begin{table}[htb]
\centering
	\begin{tabular}{lp{8cm}}
		\toprule
		Syntax & What it represents \\
		\midrule
		\lct{\caret} & match an empty string at the beginning of a line\\
		\lct{\$} & match an empty string at the end of a line\\
		\lct{{\escape}b} & match an empty string at a word boundary\\
		\lct{{\escape}B} & match an empty string not at a word boundary\\
		\lct{{\escape}<} & match an empty string at the beginning of a word\\
		\lct{{\escape}>} & match an empty string at the end of a word\\
		\bottomrule
	\end{tabular}
	\caption{Primitives to match string boundaries\tablabel{boundaries}}
\end{table}

\begin{code}{@TEST}
'hello world' matchesRegex: '.*\bw.*' --> true      "word boundary before w"
'hello world' matchesRegex: '.*\bo.*'  --> false    "no boundary before o"
\end{code}

%=================================================================
\section{Regex API}

Up to now we have focussed mainly on the syntax of regexes.  Now we will have a closer look at the different messages understood by strings and regexes.

%-----------------------------------------------------------------
\subsection{Matching prefixes and ignoring case}

So far most of our examples have used the \ct{String} extension method \ct{matchesRegex:}.

Strings also understand the following messages:
\mthind{String}{prefixMatchesRegex:}, \mthind{String}{matchesRegexIgnoringCase:} and
\mthind{String}{prefixMatchesRegexIgnoringCase:}.

The message \mthind{String}{prefixMatchesRegex:} is just like \mthind{String}{matchesRegex}, except that the whole receiver is not expected to match the regular expression passed as the argument; matching just a prefix of it is enough.
\begin{code}{@TEST}
'abacus' matchesRegex: '(a|b)+'                                --> false
'abacus' prefixMatchesRegex: '(a|b)+'                       --> true
'ABBA' matchesRegexIgnoringCase: '(a|b)+'            --> true
'Abacus' matchesRegexIgnoringCase: '(a|b)+'          --> false
'Abacus' prefixMatchesRegexIgnoringCase: '(a|b)+' --> true
\end{code}

%-----------------------------------------------------------------
\subsection{Enumeration interface}

Some applications need to access \emph{all} matches of a certain regular expression within a string.  The matches are accessible using a protocol modeled after the familiar \ct{Collection}-like enumeration protocol.

\mthind{String}{regex:matchesDo:} evaluates a one-argument \ct{aBlock} for every match of the regular expression within the receiver string.

\begin{code}{@TEST | list |}
list := OrderedCollection new.
'Jack meet Jill' regex: '\w+' matchesDo: [:word | list add: word].
list --> an OrderedCollection('Jack' 'meet' 'Jill')
\end{code}

\mthind{String}{regex:matchesCollect:} evaluates a one-argument \ct{aBlock} for every match of the regular expression within the receiver string. It then collects the results and answers them as a \clsind{SequenceableCollection}.

\begin{code}{@TEST}
'Jack meet Jill' regex: '\w+' matchesCollect: [:word | word size]                          --> an OrderedCollection(4 4 4)
\end{code}

\mthind{String}{allRegexMatches:} returns a collection of all matches (substrings of the receiver string) of the regular expression.

\begin{code}{@TEST}
'Jack and Jill went up the hill' allRegexMatches: '\w+'                                            --> an OrderedCollection('Jack' 'and' 'Jill' 'went' 'up' 'the' 'hill')
\end{code}

%-----------------------------------------------------------------
\subsection{Replacement and translation}

It is possible to replace all matches of a regular expression with a certain string using the message \mthind{String}{copyWithRegex:matchesReplacedWith:}.

\begin{code}{@TEST}
'Krazy hates Ignatz' copyWithRegex: '\<[[:lower:]]+\>' matchesReplacedWith: 'loves' --> 'Krazy loves Ignatz'
\end{code}

A more general substitution is match translation. This message evaluates a block passing it each match of the regular expression in the receiver string and answers a copy of the receiver with the block results spliced into it in place of the respective matches.

\begin{code}{@TEST}
'Krazy loves Ignatz' copyWithRegex: '\b[a-z]+\b' matchesTranslatedUsing: [:each | each asUppercase] --> 'Krazy LOVES Ignatz'
\end{code}

All messages of enumeration and replacement protocols perform a case-sensitive match.  Case-insensitive versions are not provided as part of a \ct{String} protocol.  Instead, they are accessible using the lower-level matching interface presented in the following question.
%-----------------------------------------------------------------
\subsection{Lower-level interface}

When you send the message \mthind{String}{matchesRegex:} to a string, the following happens:

\begin{enumerate}
\item A fresh instance of \clsind{RxParser} is created, and the regular expression string is passed to it, yielding the expression's syntax tree.
\item  The syntax tree is passed as an initialization parameter to an instance of \clsind{RxMatcher}. The instance sets up some data structure that will work as a recognizer for the regular expression described by the tree.
\item The original string is passed to the matcher, and the matcher checks for a match.
\end{enumerate}

%-----------------------------------------------------------------
\subsection{The Matcher}

If you repeatedly match a number of strings against the same regular expression using one of the messages defined in \clsind{String}, the regular expression string is parsed and a new matcher is created for every match.  You can avoid this overhead by building a matcher for the regular expression, and then reusing the matcher over and over again. You can, for example, create a matcher at a class or instance initialization stage, and store it in a variable for future use.
You can create a matcher using one of the following methods:

\begin{itemize}
\item You can send \mthind{String}{asRegex} or \mthind{String}{asRegexIgnoringCase} to the string.

\item You can directly invoke the \ct{RxMatcher} constructor methods
  \mthind{RxMatcher}{forString:} or
  \mthind{RxMatcher}{forString:ignoreCase:} (which is what the
  convenience methods above will do).
\mb{should be RxMatcher class in the mthind; I don't like 'invoke constructor methods'; better with: 'instantiate using one of its class methods}
%
%The \mthind{RxMatcher}{forString:} method is equivalent to \mthind{RxMatcher class}{forString: regexString ignoreCase: false}. A more convenient way is using one of the two matcher-created messages understood by \clsind{String}. 	 \cmind{RxMatcher}{regexString asRegex} is equivalent to \mthind{RxMatcher class}{forString: regexString}. 	 \ct{regexString asRegexIgnoringCase} is equivalent to \cmind{RxMatcher class}{forString: regexString ignoreCase: true}.

%\item Sending a \mthind{RxMatcher class}{forString:ignoreCase:} message to \clsind{RxMatcher} class, with the regular expression string and a Boolean indicating whether case is ignored as arguments.
\end{itemize}

Here we send \mthind{RxMatcher}{matchesIn:} to collect all the matches found in a string:

\begin{code}{@TEST | octal hex |}
octal := '8r[0-9A-F]+' asRegex.
octal matchesIn: '8r52 = 16r2A' --> an OrderedCollection('8r52')

hex := '16r[0-9A-F]+' asRegexIgnoringCase.
hex matchesIn: '8r52 = 16r2A'   --> an OrderedCollection('16r2A')

hex := RxMatcher forString: '16r[0-9A-Fa-f]+' ignoreCase: true.
hex matchesIn: '8r52 = 16r2A'   --> an OrderedCollection('16r2A')
\end{code}

%-----------------------------------------------------------------
\subsection{Matching}

A matcher understands these messages (all of them return \ct{true} to indicate successful match or search, and false otherwise):

\mthind{RxMatcher}{matches:} \ct{aString} -- true if the whole argument string (aString) matches.

\begin{code}{@TEST}
'\w+' asRegex matches: 'Krazy' --> true
\end{code}

\mthind{RxMatcher}{matchesPrefix:} \ct{aString} -- true if some prefix of the argument string (not necessarily the whole string) matches.

\begin{code}{@TEST}
'\w+' asRegex matchesPrefix: 'Ignatz hates Krazy' --> true
\end{code}

\mthind{RxMatcher}{search:} \ct{aString} -- Search the string for the first occurrence of a matching substring. (Note that the first two methods only try matching from  the very beginning of the string). Using the above example with a  matcher for \ct{a+}, this method would answer success given a string \ct{'baaa'}, while the previous two would fail.

\begin{code}{@TEST}
'\b[a-z]+\b' asRegex search: 'Ignatz hates Krazy' --> true    "finds 'hates'"
\end{code}


The matcher also stores the outcome of the last match attempt and can report it: \mthind{RxMatcher}{lastResult} answers a Boolean: the outcome of the most recent match attempt. If no matches were attempted, the answer is unspecified.

\begin{code}{@TEST | number |}
number := '\d+' asRegex.
number search: 'Ignatz throws 5 bricks'.
number lastResult --> true
\end{code}

\mthind{RxMatcher}{matchesStream:}, \mthind{RxMatcher}{matchesStreamPrefix:} and \mthind{RxMatcher}{searchStream:} are analogous to the above three messages, but takes streams as their argument.

\begin{code}{@TEST | ignatz |}
ignatz := ReadStream on: 'Ignatz throws bricks at Krazy'.
names := '\<[A-Z][a-z]+\>' asRegex.
names matchesStreamPrefix: ignatz --> true
\end{code}

%-----------------------------------------------------------------
\subsection{Subexpression matches}

After a successful match attempt, you can query which part of the original string has matched which part of the regex. A subexpression is a parenthesized part of a regular expression, or the whole expression. When a regular expression is compiled, its subexpressions are assigned indices starting from 1, depth-first, left-to-right.

\index{Regex!subexpression matches}
For example, the regex \ct{((\d+)\s*(\w+))} has four subexpressions, including itself.
\begin{code}{}
1:    ((\d+)\s*(\w+))    "the complete expression"
2:    (\d+)\s*(\w+)       "top parenthesized subexpression"
3:    \d+                      "first leaf subexpression"
4:    \w+                     "second leaf subexpression"
\end{code}

The highest valid index is equal to 1 plus the number of matching parentheses.  (So, 1 is always a valid index, even if there are no parenthesized subexpressions.)

After a successful match, the matcher can report what part of the original string matched what subexpression. It understands these messages:

\mthind{RxMatcher}{subexpressionCount} answers the total number of subexpressions: the highest value that can be used as a subexpression index with this matcher. This value 	is available immediately after initialization and never changes.

\mthind{RxMatcher}{subexpression:} takes a valid index as its argument, and may be sent only after a successful match attempt. The method answers a substring of the original string the corresponding subexpression has matched to.

\mthind{RxMatcher}{subBeginning:} and \mthind{RxMatcher}{subEnd:} answer the positions within the argument string or stream where the given subexpression match has started and ended, respectively. 
% This facility provides a convenient way of extracting parts of input strings of complex format.

\begin{code}{@TEST | items |}
items := '((\d+)\s*(\w+))' asRegex.
items search: 'Ignatz throws 1 brick at Krazy'.
items subexpressionCount --> 4
items subexpression: 1      --> '1 brick'    "complete expression"
items subexpression: 2      --> '1 brick'    "top subexpression"
items subexpression: 3      --> '1'             "first leaf subexpression"
items subexpression: 4      --> 'brick'       "second leaf subexpression"
items subBeginning: 3       --> 14
items subEnd: 3                 --> 15
items subBeginning: 4       --> 16
items subEnd: 4                 --> 21
\end{code}

As a more elaborate example, the following piece of code uses a \ct{MMM DD, YYYY} date format recognizer to convert a date to a three-element array with year, month, and day strings:

\begin{code}{@TEST | date result |}
date := '(Jan|Feb|Mar|Apr|May|Jun|Jul|Aug|Sep|Oct|Nov|Dec)\s+(\d\d?)\s*,\s*19(\d\d)' asRegex.
result := (date matches: 'Aug 6, 1996')
       ifTrue: [{ (date subexpression: 4) .
				(date subexpression: 2) .
				(date subexpression: 3) } ]
        ifFalse: ['no match'].
result --> #('96' 'Aug' '6')
\end{code}

%-----------------------------------------------------------------
\subsection{Enumeration and Replacement}

The \ct{String} enumeration and replacement protocols that we saw earlier in this section are actually implemented by the matcher.
\lct{RxMatcher} implements the following methods for iterating over matches within strings:
\mthind{RxMatcher}{matchesIn:},
\mthind{RxMatcher}{matchesIn:do:},
\mthind{RxMatcher}{matchesIn:collect:},
\mthind{RxMatcher}{copy:replacingMatchesWith:} and
\mthind{RxMatcher}{copy:translatingMatchesUsing:}.

% It seems that tests spanning multiple lines are not handled -- strange, since
% this must have worked, once upon a time ... omn 2011-05-10
\begin{code}{@TEST | seuss aWords |}
seuss := 'The cat in the hat is back'.
aWords := '\<([^aeiou]|[a])+\>' asRegex.    "match words with 'a' in them"
aWords matchesIn: seuss
    --> an OrderedCollection('cat' 'hat' 'back')
aWords matchesIn: seuss collect: [:each | each asUppercase ]
    --> an OrderedCollection('CAT' 'HAT' 'BACK')
aWords copy: seuss replacingMatchesWith: 'grinch'
    --> 'The grinch in the grinch is grinch'
aWords copy: seuss translatingMatchesUsing: [ :each | each asUppercase ]
    --> 'The CAT in the HAT is BACK'
\end{code}

There are also the following methods for iterating over matches within streams:
\mthind{RxMatcher}{matchesOnStream:},
\mthind{RxMatcher}{matchesOnStream:do:},
\mthind{RxMatcher}{matchesOnStream:collect:},
\mthind{RxMatcher}{copyStream:to:replacingMatchesWith:} and
\mthind{RxMatcher}{copyStream:to:translatingMatchesUsing:}.

\begin{code}{@TEST | in out numMatch |}
in := ReadStream on: '12 drummers, 11 pipers, 10 lords, 9 ladies, etc.'.
out := WriteStream on: ''.
numMatch := '\<\d+\>' asRegex.
numMatch
  copyStream: in
  to: out
  translatingMatchesUsing: [:each | each asNumber asFloat asString ].
out close; contents --> '12.0 drummers, 11.0 pipers, 10.0 lords, 9.0 ladies, etc.'
\end{code}


%-----------------------------------------------------------------
\subsection{Error Handling}

Several exceptions may be raised by \ct{RxParser} when building regexes.  The exceptions have the common parent \ct{RegexError}.  You may use the usual Smalltalk exception handling mechanism to catch and handle them.

\begin{itemize}

\item \clsind{RegexSyntaxError} is raised if a syntax error is detected while parsing a regex

\item \clsind{RegexCompilationError} is raised if an error is detected while building a matcher

\item \clsind{RegexMatchingError} is raised if an error occurs while matching (for example, if a bad selector was specified using \ct{':<selector>:'} syntax, or because of the matcher's internal error)

%\item If a syntax error is detected while parsing expression, \cmind{RxParser}{signalSyntaxException:} is raised/signaled;

%\item If an error is detected while building a matcher, \cmind{RxParser}{signalCompilationException:} is raised/signaled;

%\item If an error is detected while matching (for example, if a bad selector was specified using \ct{':<selector>:'} syntax, or because of the matcher's internal error), \cmind{RxParser}{signalMatchException:} is raised.
\end{itemize}

%The parent class of these three exception is \clsind{RegexError}. Since any of the three signals can be raised within a call to \mthind{matchesRegex:}, it is handy if you want to catch them all.  For example:

\begin{code}{@TEST}
['+' asRegex] on: RegexError do: [:ex | ^ ex printString ]                                        --> 'RegexSyntaxError:  nullable closure'
\end{code}
%=================================================================
\section{Implementation Notes by Vassili Bykov}
% Edited by ON

\index{Bykov, Vassili}
\paragraph{What to look at first.}
In 90\% of the cases, the method \cmind{String}{matchesRegex:}  is all you need to access the package.

\clsind{RxParser} accepts a string or a stream of characters with a regular expression, and produces a syntax tree corresponding to the expression. The tree is made of instances of \clsind{Rxs*} classes.

\clsind{RxMatcher}  accepts a syntax tree of a regular expression built by the parser and compiles it into a matcher: a structure made of instances of \ct{Rxm*} classes. The \clsind{RxMatcher} instance can test whether a string or a positionable stream of characters matches the original regular expression, or it can search a string or a stream for substrings matching the expression. After a match is found, the matcher can report a specific string that matched the whole expression, or any parenthesized subexpression of it. All other classes support the same functionality and are used by \clsind{RxParser}, \clsind{RxMatcher}, or both.

\paragraph{Caveats} The matcher is similar in spirit, but \emph{not} in design
%--let alone the code--
to Henry Spencer's original regular expression implementation in C.  The focus is on simplicity, not on efficiency. I didn't optimize or profile anything.
%  I may in future\,---\,or I may not: I do this in my spare time and I don't promise anything. 
The matcher passes H. Spencer's test suite (see ``test suite'' protocol), with quite a few extra tests added, so chances are good there are not too many bugs.  But watch out anyway.

\paragraph{Acknowledgments}
Since the first release of the matcher, thanks to the input from several fellow Smalltalkers, I became convinced a native Smalltalk regular expression matcher was worth the effort to keep it alive. For the advice and encouragement that made this release possible, I want to thank: Felix Hack, Eliot Miranda, Robb Shecter, David N. Smith, Francis Wolinski and anyone whom I haven't yet met or heard from, but who agrees this has not been a complete waste of time.
% (Coding the same ''the hard way'' is an exercise to a curious reader).

%=================================================================
\section{Chapter Summary}

Regular expressions are an essential tool for manipulating strings in a trivial way.
% whenever one has to deal with strings in a non-trivial way.
This chapter presented the Regex package for \pharo. The essential points of this chapter are:

\begin{itemize}
\item For simple matching, just send \ct{matchesRegex:} to a string
\item When performance matters, send \ct{asRegex} to the string representing the regex, and reuse the resulting matcher for multiple matches
\item Subexpression of a matching regex may be easily retrieved to an arbitrary depth
\item A matching regex can also replace or translate subexpressions in a new copy of the string matched
\item An enumeration interface is provided to access all matches of a certain regular expression
\item Regexes work with streams as well as with strings.
\end{itemize}


%=============================================================
\ifx\wholebook\relax\else
   \bibliographystyle{jurabib}
   \nobibliography{scg}
   \end{document}
\fi
%=============================================================


%-----------------------------------------------------------------
%%% Local Variables:
%%% coding: utf-8
%%% mode: latex
%%% TeX-master: t
%%% TeX-PDF-mode: t
%%% ispell-local-dictionary: "english"
%%% End:

% $Author$
% $Date$
% $Revision$

% HISTORY:
% 2008-01-19 - Alex first draft
% 2008-03-31 - David Roethlisberger reviewed and extended

%=================================================================

%A mail in the mailing list:
%just to inform you, and in particular Juraj, that the MetagraphBuilder
%is now more dynamic. It's just a very small changes, but it allows
%packages to modify the metagraph with a method addition.

%For example, DynamicProtocols adds a class extension to
%MetagraphBuilder which does:

%populateDynamicProtocols
%  | protocols |
%  protocols := OBMetaNode named: 'DynamicProtocols'.
%  class childAt: #dynamicProtocols put: protocols.
%  metaclass childAt: #dynamicProtocols put: protocols.	
%  protocols childAt: #methods put: method.

%For your work Juraj, you can easily do:

%populateTraitFilter
%  root childAt: #usedTraits labeled: 'traits' put: class;



\ifx\wholebook\relax\else
% --------------------------------------------
% Lulu:
	\documentclass[a4paper,10pt,twoside]{book}
	\usepackage[
		papersize={6in,9in},
		hmargin={.75in,.75in},
		vmargin={.75in,1in},
		ignoreheadfoot
	]{geometry}
	\input{../common.tex}
	\pagestyle{headings}
	\setboolean{lulu}{true}
% --------------------------------------------
% A4:
%	\documentclass[a4paper,11pt,twoside]{book}
%	\input{../common.tex}
%	\usepackage{a4wide}
% --------------------------------------------
    \graphicspath{{figures/} {../figures/}}
	\begin{document}
	% \renewcommand{\nnbb}[2]{} % Disable editorial comments
	\sloppy
\fi

%\newcommand{\figlabel}[1]{\label{fig:#1}}
%\newcommand{\seclabel}[1]{\label{sec:#1}}


%\usepackage{graphicx}
%\usepackage{alltt}
%\usepackage{xspace}
%\usepackage{moreverb}
%\usepackage[pdftex,colorlinks=true,pdfstartview=FitV,linkcolor=black,citecolor= black,urlcolor=blue]{hyperref}
%\usepackage{stmaryrd}
%%%%%%%%%%%%%%%%%%%%%%%%%%%%%%%%%%%%%%%%%%%%%%%%%%%%%%%%%%%%%
%%Related to the formalization
%\usepackage{amssymb}
%\usepackage{amsmath}


%\newcommand{\paragraph}[1]{\noindent\textbf{#1.}}
%\newcommand{\etal}{\emph{et al.}}



%\inputs{macros}

%\usepackage{theorem}
%\theoremstyle{plain}\theorembodyfont{\rmfamily}
%\newtheorem{definition}{Definition}
%\newtheorem{proposition}{Proposition}
%\newtheorem{theorem}{Theorem}
%\newtheorem{lemme}{Lemme}
%\newenvironment{proof}{{\bf Proof.}}{$\square$}
%%%%%%%%%%%%%%%%%%%%%%%%%%%%%%%%%%%%%%%%%%%%%%%%%%%%%%%%%%%%
%\newif\ifpdf
%\ifx\pdfoutput\undefined
%\pdffalse
%\else
%\pdfoutput=1
%\pdftrue
%\fi
%\ifpdf
%\DeclareGraphicsExtensions{.pdf, .jpg, .tif}
%\else
%\DeclareGraphicsExtensions{.eps, .jpg}
%\fi
%\graphicspath{{figures/}}
%%%%%%%%%%%%%%%%%%%%%%%%%%%%%%%%%%%%%%%%%%%%%%%%%%%%%%%%%%%%
%Useful Comment
%\newcommand{\co}[1]{\textsf{#1}\xspace}
%\newcommand{\infe}{$<$}
%\newcommand{\supe}{$\rightarrow$\xspace}
%\newcommand{\ret}{$\uparrow$\xspace}
%\newcommand{\sep}{$\gg$\xspace}
\newcommand{\pipe}{$\mid$}
%%%%%%%%%%%%%%%%%%%%%%%%%%%%%%%%%%%%%%%%%%%%%%%%%%%%%%%%%%%%
%Comments
\newcommand\fix[1]{\nb{FIX}{#1}}
\newcommand\cp[1]{\nb{CP}{#1}}
\newcommand\rw[1]{\nb{RW}{#1}}
%%%%%%%%%%%%%%%%%%%%%%%%%%%%%%%%%%%%%%%%%%%%%%%%%%%%%%%%%%%%
%\def\figref#1{Figure \cite{#1}
%\newcommand{\secref}[1]{Section~\ref{sec:#1}}
%\newcommand{\seclabel}[1]{\label{sec:#1}}
%\newcommand{\figlabel}[1]{\label{fig:#1}}
%\newcommand{\figref}[1]{Figure~\ref{fig:#1}}
%% \newcommand{\tabref}[1]{Table~\ref{#1}}
%\newcommand{\defref}[1]{Definition~\ref{def:#1}}
%\renewcommand{\paragraph}[1]{\par\noindent{\textbf{#1.}}}
%\newcommand{\figScale}{0.4}
%\newcommand{\colWidth}{8.5cm}
%\newcommand{\ie}{\emph{i.e.},\xspace}
%\newcommand{\eg}{\emph{e.g.},\xspace}
%\newcommand{\cf}{cf.\xspace}
%\newcommand{\ct}[1]{\textsf{#1}}

%\newenvironment{code}
%        {\begin{alltt}\sffamily}
%        {\end{alltt}}

%\newenvironment{tcode}
%{\footnotesize\sf\begin{tabbing}
%xxxx\=xxxx\=xxxx\=xxxx\=xxxx\=xxxx\=xxxx\=xxxx\=xxxx\=xxxx\=xxxx\=xxxx\=\kill}
%{\end{tabbing}\sf\normalsize}       
    
%\newcommand{\twocolumnpic}[3]{
%   \begin{figure*}[!ht]
%   \begin{center}
%   \includegraphics[scale=\defaultScale]{#1}
%   \caption{#2}
%   \label{#3}
%   \end{center}
%   \end{figure*}}
%%%%%%%%%%%%%%%%%%%%%%%%%%%%%%%%%%%%%%%%%%%%%%%%%%%%%%%%%%%%
%\usepackage{ifthen}
%\usepackage{amssymb}
%\newboolean{showcomments}
%\setboolean{showcomments}{true}
%\ifthenelse{\boolean{showcomments}}
%  {\newcommand{\nb}[2]{
%% \fbox{\bfseries\sffamily#1}
%        \fbox{\bfseries\sffamily\scriptsize#1}
%    {\sf\small$\blacktriangleright$\emph{#2}$\blacktriangleleft$}
%    % \marginpar{\fbox{\bfseries\sffamily#1}}
%   }
%   \newcommand{\cvsversion}{\emph{\scriptsize$-$Id$-$}}
%  }
%  {\newcommand{\nb}[2]{}
%   \newcommand{\cvsversion}{}
%  }
%\newcommand{\here}{\nb{***}{CONTINUE HERE}}
%% \newcommand{\st}[1]{\textsf{\small #1}}
%\newcommand{\st}[1]{\textsf{#1}}
%\newcommand{\sitem}{\vspace{-5 pt}\item}

%
\newcommand{\ob}{OmniBrowser\xspace}
\newcommand{\obf}{OmniBrowser framework\xspace}
\newcommand{\applflab}{ApplFLab\xspace}

%%%%%%%%%%%%%%%%%%%%%%%%%%%%%%%%%%%%%%%%%%%%%%%%%%%%%%%%%%%%

\chapter{Creating Browsers with OmniBrowser}


\noindent

%Smalltalk is not only an object-oriented programming language; it is also known for its extensive integrated development environment supporting interactive and dynamic programming. While the default tools are adequate for browsing the code and developing applications, it is often cumbersome to extend the environment to support new language constructs or to build additional tools supporting new ways of navigating and presenting source code. 
In this chapter, we present \ob, a browser framework that supports the definition of browsers based on explicit metamodels. In \obf, a browser is a graphical list-oriented tool to navigate and edit any arbitrary domain. The most common representative of this category of tools is the Smalltalk system browser, which is used to navigate and edit Smalltalk source code.
In \ob, a browser is described by a domain model and a metagraph which specifies how the domain space may be navigated through. Widgets such as list menus and text panels are used to display information gathered from a particular path in the metagraph. Although widgets are programmatically composed by the framework, \ob allows for interaction with the end user.

In the following, we show how to build new browsers from predefined parts and how to easily describe new tools. Three exemplary browsers, a file browser, a remake of the ubiquitous Smalltalk system browser, and a coverage browser, will illustrate how to define sophisticated browsers for various domains.

%%%%%%%%%%%%%%%%%%%%%%%%%%%%%%%%%%%%%%%%%%%%%%%%%%
%\section{Introduction}\label{sec:introduction}

%context
%Smalltalk is an object-oriented language featuring a complete development environment supporting interactive and dynamic programming \cite{Gold83a,Gold84a}. While the default environment already supports advanced ways of navigating source code and fluid development since the eighties, new browsers have been developed over the years: the \emph{Refactoring Browser} \cite{Fowl99a,Robe96a,Robe97a} which was the first system browser supporting refactoring, the \emph{StarBrowser} \cite{Wuyt04a} which supports smart groups, a browser for incremental development supporting visual feedback of undefined methods \cite{Scha04c} and the \emph{Whiskers} browser that shows multiple methods at the same time maximizing the screen space. Strong\-Talk, a more exotic Smalltalk version featuring optional typing, offered a glyph based browsing environment. 


%problem
%The problem when building all of these browsers is that they are always rebuilt from scratch because there hardly exists any domain models or frameworks for building such development tools. In fact, the current browsers in most Smalltalk environments are hard to extend for two reasons: (a) they are monolythic applications that are not really meant to be included elsewhere, and (b) the navigation and interaction of the end-user with the browsers is typically hardcoded in the browser UI elements, and is therefore hard to change or extend.

%some solutions exist, but only partially
%Note that some Smalltalk environments allow one to embed applications within each-other. VisualWorks for example has a notion of \emph{subcanvases} which can be used to that end. This helps to reduce the problem (a) in the previous paragraph, but not problem (b) of the hardcoding of the  the navigation and interaction in the browser UI elements. Other browsers are designed with a certain amount of customizability in mind, and are therefore easier to extend, but even those lack explicit descriptions of the navigation.
%
%As was already reported by Steyaert \emph{et al.}~\cite{Stey96a}, we conclude that current visual application builders and application frameworks do not live up to their expectations of rapid application development or non-programming-expert application development. They fall short when compared to component-oriented development environments in which applications are built with components that have a strong affinity with the problem domain (\ie being domain-specific). 



%\rw{I do not agree with this part. If you want to say that the situation in Squeak is not good, then say so. Generalizing it to other Smalltalk (e.g. Dolphin, VW, ...) is stretcing it much too far}. \ab{I haven't checked with VW (I cannot run it on my machine), but I would be really surprised if the situation is not similar than in Squeak. Is there an easy way to extend the refactoring browser with a new pane for example?}\sd{IN VW you can easily add a tab after this is not a easy composition}
%Still these browsers acts as standalone application and it is difficult to extend them or compose their functionality to produce even richer environments. In the Squeak Smalltalk environment, for example, the default environment browsers and tools are inflexible and changing them requires nearly patching the system. The main reason is that the domain of the browser \ie the navigation and the interaction with the end-user are not explicit but hardcoded into the browser UI elements and such as cannot be easily extended or customizable. 

%Application builders such as the ones present in VisualWorks or Dolphin Smalltalk already improve the situation by offering an interpretation of the widgets elements via window spec like declaration interpretation. However as reported by Steayert \etal \cite{Stey96a}, current visual application builders and application frameworks do not live up to their expectations of rapid application development or non-programming-expert application development. They fall short when compared to component-oriented development environments in which applications are built with components that have a strong affinity with the problem domain (\ie being domain-specific). 

%solution: ob
%This chapter presents \ob, a framework to define and compose new browsers. In \obf, a browser is a graphical list-oriented tool to navigate and edit an arbitrary domain. The most common representative of this category of tools is the Smalltalk system browser, which is used to navigate and edit Smalltalk source code.
%In \obf, a browser is described by a domain model and a metagraph which specifies how the domain space is navigated through. Widgets such as list menus and text panels are used to display information gathered from a particular path in the metagraph. Although widgets are programmatically composed, the \obf framework supports their interaction.

%The contributions of this article are: the description of a metadriven framework to build system browsers and the application of the framework to build some tools. 
%In \secref{problem} we describe difficulties and challenges to define states and flow between those states for a graphical user interface. In \secref{omnibrowser} we present the key entities of \obf. In \secref{codebrowser} we present the \ob-based system browser and in \secref{coverageBrowser} we describe the coverage code browser. In \secref{discussion} we discuss about properties of the \obf. In \secref{relatedwork} we provide an overview of related work. In \secref{conclusion} we conclude by summarizing the presented work.


%:%%%%%%%%%%%%%%%%%%%%%%%%%%%%%%%%%%%%%%%%%%%%%
\section{Representing State of a User Interface} \label{sec:problem}
%In this section we stress some of the problems encountered when building complex tools such as an advanced code editor. 

The state of a graphical user interface (GUI) is defined as a collection of the states of the widgets making up the interface. The state of a widget refers to the state the widget is in. It may be modified whenever an end-user performs an action on this widget such as clicking a button or selecting an entry in a menu. Therefore, a GUI has a high number of different states. Asserting the validity for each of these states is crucial to avoid broken or inconsistent interfaces.

Given the potential high number of different states of a GUI, asserting the validity of a GUI is a challenging task. Let's illustrate this situation with the Smalltalk system browser, a graphical tool to edit and navigate into Smalltalk source code.

\begin{figure}[!ht]
\begin{center}
\includegraphics[scale=0.65]{miniStBrowser}
\caption{The traditional Smalltalk System Browser roughly depicted.} \figlabel{miniStBrowser}
\end{center}
\end{figure}


\figref{miniStBrowser} depicts the different widgets of a traditional Smalltalk class system browser (see \figref{ob} for a real picture). Without entering into details, A, B, C and D are lists that show class categories (groups of classes), classes, method protocols (groups of methods) and methods. E is a radio button composed of three choices and F is a text pane. 

%After having selected a class category in the pane A, selecting a class in pane B renders the definition of this class in pane F. After having selected a class (pane B) and a method category (pane C), pane D shows a list of method names. Clicking a method name makes the definition of the selected method appear in the text pane (pane F). 

Pane A lists the categories in the system. Selecting a category in this list, makes the classes in that category appear in pane B. Selecting a class results in the protocols for that class being shown in pane C, and selecting a protocol lists the method names in pane D. Switch E controls whether the class or the metaclass is being edited, and therefore whether the protocols and methods shown are instance level or class level methods. Pane F is a text pane that gives feedback on whatever is selected in the top panes, always displaying the most specific information possible. For example, when a user has selected a method in a protocol in a class in a certain category, pane F shows the definition of that method (and not the definition of the class of that method). 
%\sd{this part is a bit boring what I would be nice is  to have is a complex example of interaction.}\rw{Yes, but we need to give the basic first :-( }

The description of how the browser works shows a number of navigation invariants that need to be kept when implementing the browser. For example, the selections goes from left to right: it is not possible to have methods listed in pane D with pane C being empty.

Invariants such as the one given above need to be implemented and checked when building a browser. So we are dealing with writing an application that deals with a potentially very big number of states in which only certain transitions between states need to be allowed (the ones that correspond to navigations the user of the browser). Whenever a user clicks on widgets that make up the GUI of the browser, the state of one or more widgets is changed, and possibly new navigation possibilities open up (being able to select a method name, for example). To deal with the fact that a widget can be in an inconsistent state, developers often rely on guards: the method performing an action in reaction to a user action always checks whether the state is actually correct or not nil. 

In addition the state management is often spread over the UI elements. This leads to code with complex and often error-prone logic. In addition it makes tool elements difficult to extend and reuse in different context.

The main problem when building a browser is representing the mapping from the intended navigation model to the domain model and widgets. 
%Even though graphical frameworks like MVC~\cite{Tryg79a,Tryg03a} and Coral~\cite{Szek88a} offer ways to modularize the model and the graphical user interface, they do not provide means (i) to preserve consistency of the interface by restricting unexpected state transition to happen and (ii) to keep the widgets synchronized with each other~\cite{Kras88a}.
In the next section, we describe \ob, a framework to design browsers where the domain model is distinct from the navigation space. The latter is being described by a metagraph. The state of a browser is defined by a path in this metagraph.

%:%%%%%%%%%%%%%%%%%%%%%%%%%%%%%%%%%%%%%%%%%%%%%
\section{Graph and Metagraph of a Browser} \label{sec:omnibrowser}

The domain of the \obf is \emph{browsers}, applications with a graphical user interface that are used to navigate a graph of domain elements. When instantiating  the \obf to create a browser for a particular domain, the domain elements need to be specified, as well as the desired navigation paths between them.

The \obf is structured around (i) an explicit domain model and (ii) a metagraph, a state machine, that specifies the navigation in and interaction with the domain model. The user interface is constructed by the framework, and uses a layout similar to the Smalltalk System Browser, with two horizontal parts. The top part is a column-based section where the navigation is done. The bottom half is a text pane.

%\secref{overview} explains the major classes that make up the \obf. \secref{fileBrowserExample} shows a concrete instantiation to build a file browser. \secref{corebehavior} goes in some more detail and describes the core behavior of the framework. Section~\ref{widgets} explains how the widgets are glued together.

\subsection{Overview of the \obf}\label{sec:overview}
The major classes that make up the \obf are presented in \figref{core}, and explained briefly in the rest of this section. %After an example, Section~\ref{sec:corebehavior} discusses the core behavior of the classes in more detail.

\begin{figure}[!ht]
\begin{center}
\includegraphics[scale=0.59]{CoreOnly}
\caption{Core of the \obf. } \figlabel{core}
\end{center}
\end{figure}

\paragraph{Browser.} A \emph{browser} is a graphical tool to navigate and edit a domain space. This domain has to be described in terms of a directed cyclic graph (DCG). It is cyclic because for example file systems or structural meta models of programming language (\ie packages, classes, methods...) contain cycles, and we need to be able to model those. The domain graph has to have an entry point, its root. The path from this root to a particular node corresponds to a state of the browser defined by a particular combination of user actions (such as menu selections or button presses).
The navigation of this domain graph is specified in a \emph{metagraph}, a state machine describing the states and their possible transitions.

\paragraph{Node.} A \emph{node} is a wrapper for a domain object, and has two responsibilities: rendering the domain object, and returning domain nodes. Note that how the domain graph can be navigated is implemented in the \emph{metagraph}.
%\rw{it is linked to? it points to? -- cannot find a clear and easy way to say what is needed here} \ab{it generates nodes actually...}

\paragraph{Metagraph.} A browser's \emph{metagraph} defines the way a user traverses the graph of domain objects. A metagraph is composed of metanodes and metaedges. A metanode identifies a state in which the browser may be. A metanode may reference a filter (described below)
%\footnote{In a former version of \ob~\cite{Berg07c}, actors where used to defines actions and where attached to a metanode. In the current version of \ob actors have been replaced by commands.}.
The metanode does not have the knowledge of the domain nodes, however each node is associated to a metanode. Transitions between meta\-nodes are defined by metaedges. When a metaedge is traversed (\ie result of pressing a button or selecting an entry list), sibling nodes are created from a given node by invoking a method that has the name of the metaedge.

A \emph{metanode} has the ability to be auto selected with the method \ct{MetaNode>>>autoSelect: aMetaNode}. When a particular child for auto selection is designated, the first node produced by following its metaedge will be selected.


\paragraph{Command.} A \emph{Command} enables interaction and manipulation of the domain graph. Commands may be available through menus and buttons in the browser. They therefore have the ability to render themselves in a user interfaces and are responsible for handling exceptions that may occur when triggered. 

Commands are defined in a non-invasive way: adding and removing commands is done without any method redefinition of the core framework. This enables a smooth gathering of commands independently realized.

A command is defined by subclassing \ct{OBCommand}, then redefining its four main methods with the desired behavior and finally defining a method on the browser class whose name begins with \ct{cmd}. This method has to return a command class. An example is provided in the following subsection.

\paragraph{Filter.} 
The metagraph describes a state machine. When the browser is in a state in which more than one transition are available, the user decides which transition to follow. To allow that to happen \ob displays the possible transitions  to the user. From all the possible transitions, \obf fetches all the nodes that represent the states the user could arrive at by following those transitions and list them in the next column. Note that the transition is not actually made yet, and the definition pane is still displaying the current definition. Once a click is made, the transition  actually happens, the definition pane is updated (and perhaps other panes such as button bars), and \ob gathers the next round of possible transitions.

%\ab{This was not introduced before, it is not clear to me that people will get this example: for example \ct{\#allCategory} (which will lead to the allMethodCategory state) and\ct{\#categories} (which will lead to the methodCategories state)}

A filter provides a strategy for filtering out some of the nodes from the display. If a node is the starting point of several edges, a filter may be needed to filter out all but one edge to determine which path has to be taken in the metagraph. 

\paragraph{Definition.} While navigating in the domain space, information about the selected node is displayed in a dedicated textual panel. If edition of the text is expected by the browser user, then a definition is necessary to handle modification and commitment (\ie an \emph{accept} in the Smalltalk terminology). A definition is produced by a node. 


\subsection{Building a File Browser}\label{sec:fileBrowserExample}

To illustrate how the \obf is instantiated, we describe the implementation of a simple file browser supporting the navigation in directories and files.

\begin{figure}[!ht]
\begin{center}
\includegraphics[scale=0.54]{filebrowser}
\caption{A minimal file browser based on OmniBrowser.} \label{fig:filebrowser}
\end{center}
\end{figure}

\figref{filebrowser} shows the file browser in action. A browser is opened by evaluating \ct{FileBrowser open} in a workspace. The navigation columns in the case of a file browser are used to navigate through directories, where every column lists the contents of the directory selected in its left column, similar to the \emph{Column View} of the Finder in the Mac OS-X operating system.
Note that we can have an infinite numbers of panes navigating through the file system. The horizontal scrollbar lets the user browse the directory structure. A text panel below the columns displays additional properties of the currently selected directory or file and provides means to manipulate these properties.

\paragraph{Metagraph Definition.} 
A filesystem encompasses basically two kind of entities, files and directories. To model the navigation of a filesystem we thus need two metanodes in the metagraph, \ct{Directory} and \ct{File}. Within any directory of a filesystem, we can again find files and other directories, hence there are two kind of transitions outgoing from a directory metanode, \ct{files} and \ct{directories}. When opening the filesystem browser, we launch it for a given directory, \eg the root directory of the filesystem. Thus the metagraph's root metanode represents a directory. \figref{graphfs}, right, shows this metagraph describing a filesystem.

To concretely implement this filesystem metagraph we define a class \ct{OBFileBrowser} as a subclass of \ct{OBBrowser} and write the method \ct{defaultMetaNode} on the class side. This method first defines the two metanodes  \ct{Directory} and \ct{File} and specifies second the two transitions leaving directory and going to the metanodes \ct{Directory} and \ct{File}, respectively. These transitions are implemented as children of the metanode \ct{Directory} and are called \ct{directories} and \ct{files}, respectively. \ct{defaultMetaNode} finally answers the root metanode, in our case \ct{Directory}.

\begin{code}{}
OBFileBrowser class>>defaultMetaNode
     "returns the directory metanode that acts as the root metanode"
 
     | directory file |
     directory := OBMetaNode named: 'Directory'.
     file := OBMetaNode named: 'File'.
     directory 
          childAt: #directories put: directory;
          childAt: #files put: file.
     ^ directory
\end{code}

When one of the two \ct{#directories} and \ct{#files} metaedges is traversed, the name of this metaedge is used as a message name sent to the metanode's node.

As soon as we have defined the metagraph, we can model the domain with node classes. For every metanode in the metagraph we also need a concrete node class in our model, in this case we need two node classes, one representing a directory, the other a file. As the root metanode in the graph represents a directory, the concrete node in the model has to be a concrete directory node, eg. representing the root directory of the filesystem. This default root node is answered by the class-side method \ct{defaultRootNode} of \ct{OBFileBrowser}: 
                    
\begin{code}{}    
OBFileBrowser class>>defaultRootNode
     ^OBDirectoryNode new path: '/'
\end{code}

The next step consists of modeling the domain objects, \ie nodes.

\paragraph{Node definitions.} Nodes wrap objects of the browsed domain. First the class \ct{OBFileNode}, a subclass of \ct{OBNode}, has to be defined. Instances of this class will represent concrete files. A file node is identified by a full path name, stored in a variable. A directory is another entity in our model that contains directories and files. A directory can be simply modeled as a special kind of file. The only difference between a file and a directory node is that for a directory the path variable points to a directory, not to a file. 

\begin{code}{}
OBNode subclass: #OBFileNode
     instanceVariableNames: 'path'
     classVariableNames: ''
     poolDictionaries: ''
     category: 'OBExample-FileBrowser'

OBFileNode subclass: #OBDirectoryNode
     instanceVariableNames: ''
     classVariableNames: ''
     poolDictionaries: ''
     category: 'OBExample-FileBrowser'
\end{code}

The name of the node is simply the name of the file selected:

\begin{code}{}
OBFileNode>>name
     ^ (self path subStrings: '/') last
\end{code}

The variable \ct{path} has to be accessed:

\begin{code}{}
OBFileNode>>path
     ^ path

OBFileNode>>path: aString
     path := aString
\end{code}

A text containing information about the selected file is returned by the method \ct{text}:

\begin{code}{}
OBFileNode>>text
     ^ 'First 1000 characters: ', String cr,
        ((FileStream readOnlyFileNamed: path) converter: Latin1TextConverter new; 
              next: 1000) asString
\end{code}

The methods \ct{files} and \ct{directories} are defined on the class \ct{OBDirectoryNode}.

\begin{code}{}
OBDirectoryNode>>directories
     | dir | 
     dir := FileDirectory on: path.
     ^ dir directoryNames collect: [:each | 
                                  OBDirectoryNode new path: (dir fullNameFor: each)]

OBDirectoryNode>>files
     | dir | 
     dir := FileDirectory on:  path.
     ^ dir fileNames collect: [:each | 
                            OBFileNode new path: (dir fullNameFor: each)]
\end{code}

The implementation shows the two responsibilities of a node: rendering itself (implemented in the \ct{text} method), and calculating the nodes reachable from a node (in the \ct{directories} and \ct{files} methods). As there is no further navigation leaving a file node, such a node does not have to define navigation methods such as \ct{directories} or \ct{files}.

\begin{figure}[!ht]
\begin{center}
\includegraphics[scale=0.55]{metagraph-fs.pdf}
\caption{A filesystem as a graph (a) and its corresponding metagraph (b).} \figlabel{graphfs}
\end{center}
\end{figure}

To visually distinguish files from directories when browsing a directory with our file browser, we can add an icon to each element in the list. To illustrate this, we will denote directories with a small folder icon. 

The first step is to integrate the icon itself into a Squeak image. In the class \ct{OBMorphicIcons} you see some pre-defined icons stored in methods such as \ct{arrowUp}. To import an icon stored as an image (\eg as a GIF file), you can use this code:

\begin{code}{}
| image stream |
image := ColorForm fromFileNamed: '/path/to/icon.gif'.
stream := WriteStream with: String new.
image storeOn: stream.
stream contents.
\end{code}

Inspect this whole code listing. In the inspector you see the definition of the color form for the icon. You can now install the content of this \ct{ByteString} as a method in the method protocol \ct{icons} of \ct{OBMorphicIcons} in a method called \ct{folder}. Make sure that you do not return the string, but the code within the string, so that if the method gets invoked a color form for the folder icon is returned. For example, a flag icon is defined as:

\begin{code}{}
OBMorphicIcons>>flag
	^ ((ColorForm
		extent: 12@12
		depth: 8
		fromArray: #( 437918234 437918234 437918234 436470535 101584139 387389210 436404481 17105924 303634202 436666638 ...
\end{code}

In the second step you can take this icon and display it in the columns for every directory. To achieve this, simply add a method \ct{icon} to the class \ct{OBDirectoryNode}:

\begin{code}{}
OBDirectoryNode >>icon
	^#folder
\end{code}

The method \ct{icon} gets executed for every element that is added to a column. If it answers a symbol, then the method of \ct{OBMorphicIcons} with the same name is executed, answering the icon as a color form to be added on the left of the list element, \ie the directory name.

At this stage, we can open a file browser by evaluating \ct{OBFileBrowser open} in a workspace. To allow users to perform actions on a selected file, we add commands to the browser. Note that you will need to open a new browser to see these command in effect. They are implemented with subclasses of \ct{OBCommand}:

\begin{code}{}
OBCommand subclass: #OBRemoveFileCommand
     instanceVariableNames: ''
     classVariableNames: ''
     poolDictionaries: ''
     category: 'OBExample-FileBrowser'
\end{code}


The functionality of this command is basically implemented in four methods: 

\begin{itemize}
\item \ct{isActive} - test condition to determine if this command is active in the current column for the currently selected node 
\item \ct{keystroke} - a letter used to trigger this command with the keyboard
\item \ct{label} - the string denoting this command in the command menu
\item \ct{execute} - holds the functionality to be triggered if the user executes this command
\end{itemize}

When these methods get executed, the command already knows the column from which it gets triggered (stored in the instance variable \ct{requestor}) and the target node for which the action has to be exectuted (stored in the instance variable \ct{target}). With this information available we can implement these four methods as follows:

\begin{code}{}
OBRemoveFileCommand>>isActive
     "only active for files"
     ^ (target isKindOf: OBFileNode) and: [requestor isSelected: target]

OBRemoveFileCommand>>keystroke
     ^ $d

OBRemoveFileCommand>>label
     ^ 'remove file'

OBRemoveFileCommand>>execute
     FileDirectory deleteFilePath: target path
\end{code}

To integrate this command the class \ct{OBFileBrowser} has to be extended with a method whose name needs to start with \ct{'cmd'}:

\begin{code}{}
OBFileBrowser>>cmdRemoveFile
	^OBRemoveFileCommand
\end{code}

Open a new browser, then right click on a selected file and you will get a menu that contains this command. Currently, the list of files is not refreshed when files are removed. Refreshing can for instance be done by announcing a \ct{nodeDeleted} announcement in the \ct{execute} method. This can be achieved by inserting the expression \ct{target announce: (OBNodeDeleted node: self)}. Since this is a common operation, an helper is provided for that purpose: simply send the \ct{signalDeletion} message to \ct{target}.

%:===============
\subsection{Core Behavior of the Framework}\label{sec:corebehavior}

The core of the \ob framework is composed of 8 classes (\figref{core}). We denote the Smalltalk metaclass hierarchy by a dashed arrow. 

\begin{figure}[!ht]
\begin{center}
\includegraphics[scale=0.52]{Core}
\caption{Core of the \obf and its extension for the file browser.} \figlabel{coreextend}
\end{center}
\end{figure}

The metaclass of the class \ct{OBBrowser} is \ct{OBBrowser class}. It defines two abstract methods \ct{defaultMetaNode} and \ct{defaultRootNode}. These methods are abstract, they therefore need to be overridden in subclasses. These methods are called when a browser is instantiated. The methods  \ct{defaultMetaNode} and \ct{defaultRootNode} return the root metanode and the root domain node, respectively. A browser is opened by sending the message \ct{open} to an instance of the class \ct{OBBrowser}.

The navigation graph is built with instances of the class \ct{OBMetaNode}. Transitions are built by sending the message \ct{childAt: selector put: metanode} to a \ct{MetaNode} instance. This has the effect to create a metaedge named \ct{selector} leading away the metanode receiver of the message and \ct{metanode}.

At runtime, the graph traversal is triggered by user actions (\eg pressing a button or selecting a list entry) which send the metaedge's name to the node that is currently selected. The rendering of a node is performed by invoking on the domain node the selector stored in the variable \ct{displaySelector} in the metanode.

The class \ct{OBCommand} is instantiated by the framework and the set of commands for a browser is discovered (through the Smalltalk reflection API) when a browser is instantiated. All methods starting with the \ct{cmd} prefix are considered as commands. Each of this method should return the \emph{class} of the command (and not an instance of it).

The class \ct{OBNode} represents an element of the domain graph. Each node has a name. This name is used when lists of nodes are displayed in the navigation columns of the browser. When a node is selected in a list, information related to this node needs to be displayed in the bottom text pane. 
When the node is not supposed to be edited, the message \ct{text} is sent to it, returning a string  displayed in the bottom pane. When it is editable, the message \ct{definition} is sent and it is expected to return an instance of a subclass of \ct{OBDefinition}. Note that the nodes do not need to be configured to be editable or not. When they implement a method \ct{definition}, this will be used and the node will be editable. If that method is not present, then the method \ct{text} is used. 

When the browser is in a state where several transitions are available, it displays the navigation possibilities to the user. From all the possible transitions, \obf fetches all the nodes that represent the states the user could arrive at by following those transitions and lists them in the next column. Once a selection is made, the transition  actually happens, the pane definition is updated and the process repeats.

As explained before, a filter or modal filter can be used to select only a number of outgoing edges when not all of them need to be shown to the user. This is useful for instance to display the instance side, comments, or class side of a particular class in the classic standard system browser (cf. \secref{codebrowser}). Class \ct{OBFilter} is responsible for filtering nodes in the graph. The method \ct{nodesForParent:} computes a transition in the domain metagraph. This method returns a list of nodes obtained from a given node passed as argument. The class \ct{OBFilter} is subclassed into \ct{OBModalFilter}, a handy filter that represents transitions in the metagraph that can be traversed by using a radio button in the GUI.


%:%%%%%%%%%%%%%%%%%%%%%%%%%%%%%%%%%%%%%%%%%%%%%



\subsection{Glueing Widgets with the Metagraph}\label{widgets}
From the programmer point of view, creating a new browser implies defining a domain model (set of nodes like \ct{FileNode} and \ct{DirectoryNode}), a metagraph intended to steer the navigation and a set of commands to define interaction and actions with domain elements. The graphical user interface of a browser is automatically generated by the \obf. The GUI generated by \obf is contained in one window, and it is composed of 4 kinds of widgets (lists, radio buttons, menus and text panes).

%The layout of a browser can be redefined and use other widgets then the ones described above, but those are then not used by the metagraph. For instance, the \obf-based system browser uses a toolbar widget that allows a user to launch other kind of browsers like the variable and hierarchy browsers. We will not describe how to use other widgets, as this is outside the scope of this paper.\\

%The metagraph has a well-defined flow that is presented below.\rw{What does this mean? Graphs do not flow?}

\paragraph{Lists.} Navigation in \obf is rendered with a set of lists and triggered by selecting one entry in a list. Lists displayed in a browser are ordered and are displayed from left to right. Traversing a new metanode, by selecting a node in a list \textit{A}, triggers the construction of a set of nodes intended to fill a list \textit{B}. List \textit{B} follows list \textit{A}.

The root of a metagraph corresponds to the left-most list. The number of lists displayed is equal to the depth of the metagraph. The depth of the system browser metagraph (\figref{obmetagraph}) is 4, therefore the system browser has 4 lists (\figref{ob}). Because the metagraph of a filesystem may contain cycles (\ie a directory may contain directories, as shown in \figref{graphfs}), the number of lists in the browser increases for each directory selected in the right-most list. Therefore a horizontal scrollbar is used to keep the width of the browser constant, yet displaying a potentially infinite number of lists in the top half.

\paragraph{Radio buttons.} A modal filter in the metagraph is represented in the GUI by a radio button. Each edge leading away from the filter is represented as a button in the radio button. Only one button can be selected at a time in the radio button, and the associated choice is used to determine the outgoing edges. For example, the second list in the system browser contains the three buttons \ct{instance}, \ct{?} and \ct{class} as shown the transition from the environment to the three metanodes class, class comment and metaclass in \figref{ob}.

\paragraph{Menus.} A menu can be displayed for each list widget of a browser. Typically such a menu displays a list of actions that can be executed by the user. These actions enable interaction with the domain model, however they do not allow further navigation in the metagraph.

\begin{figure}[!ht]
\begin{center}
\includegraphics[scale=0.65]{menu.pdf}
\caption{Example of menu in the \obf system browser.} \figlabel{menu}
\end{center}
\end{figure}

\figref{menu} shows an example of a menu offering actions related to a class. These correspond to the list of commands defined in the class \ct{OBCodeBrowser}.

\paragraph{Definition pane.} When a node is selected in a list, information related to this node is displayed in a text pane. Committing a change in the definition pane sends the message \ct{accept: newText notifying: aController} to the definition shown in this pane. A browser contains only one text pane.
%\sd{can we have multiple of them? I guess not so we should say it.} 
%\sd{how to the notification of events relates to the widgets, I see how it is done but may be we should say it}

% - The toolbar

% - How it can be extended with a package/class button.

%:%%%%%%%%%%%%%%%%%%%%%%%%%%%%%%%%%%%%%%%%%%%%%

\section{The \ob-based System Browser} \seclabel{codebrowser}

In this section we show how the framework is used to implement the traditional class system browser. 

\subsection{The Smalltalk System Browser}\seclabel{systemBrowser}
The system browser is probably the most important tool offered by the Squeak programming environment. It enables code navigation and code editing. \figref{ob} shows the graphical user interface of this browser, and how it appears to the Smalltalk programmer. 


\begin{figure}[!ht]
\begin{center}
\includegraphics[scale=0.50]{obbrowser.pdf}
\caption{\ob based Smalltalk system browser.} \figlabel{ob}
\end{center}
\end{figure}

This browser just replicates the traditional four panes system browser discussed in~\secref{problem}.
The system browser is mainly composed of four lists (upper part) and a panel (lower part). From left to right, the lists represent (i) class categories, (ii) classes contained in the selected class category, (iii) method categories defined in the selected class to which the \ct{-- all --} category is added, and (iv) the list of methods defined in the selected method category. On \figref{ob}, the class named \ct{Class}, which belongs to the class category \ct{Kernel-Classes} is selected. \ct{Class} has three methods categories, plus the \ct{-- all --} one. The method \ct{templateForSubclassOf:category} contained in the \ct{instance creation} method category is selected.

The lower part of the system browser contains a large textual panel displaying information about the current selection in the lists. Selecting a class category triggers the display of a class template intended to be filled out to create a new class in the system. If a class is selected, then this panel shows the definition of this class. If a method is selected, then the definition of this method is displayed. The text contained in the panel can be edited. The effect of this is to create a new class, a new methods, or changing the definition of a class (\eg adding a new variable, changing the superclass) or redefining a method.

In the upper part, the class list contains three buttons (titled \ct{instance}, \ct{?} and \ct{class}) to let one switch between different ``views'' on a class: the class definition, its comment and the definition of its metaclass. Just above the definition panel, there is a toolbar intended to open more specific browsers like a hierarchy browser or a variable access browser.

The \ct{-- all --} method category gets automatically selected when no other method category is selected. This is specified in the \ct{OBMetagraphBuilder>>>populateClassNode} method by invoking \ct{autoSelect: aMetanode}.
%\alex{This method does not exist in the last version of Squeak-dev I have. OB is evolving fast} \dr{i guess this was just a mistake in the paper, the method is called populateClassNode, not populateClass}

%:===============

\subsection{System Browser Internals}
The \ob-based implementation of the Squeak system browser is composed of 17 classes (2 classes for the browser, 3 classes for the definitions of classes, methods and organization, 10 classes defining nodes and 2 utility classes with abstractions to help link the browser and the system). \figref{obInternal} shows the classes in \obf that need to be subclassed to produce the system browser. Note that the two utility classes are not represented on the picture.

% Squeak-dev3.10-7143: Number of Classes = 102, number of methods: 675

% Number of classes = 36
% (Smalltalk allClasses select: [:c| c category beginsWith: 'OB-Standard']) size

% Number of methods = 299
% (Smalltalk allClasses select: [:c| c category beginsWith: 'OB-Standard']) inject: 0 into: [:sum :el| sum + el methodDictionary values size]

% Number of methods related to the system browser = 220
%(#(CodeBrowser SystemBrowser ClassActor CategoryActor OrganizationDefinition MethodDefinition ClassDefinition CodeNode ClassCommentNode ClassAwareNode EnvironmentNode ClassNode MethodCategoryNode MethodNode ClassCategoryNode MetaclassNode AllMethodCategoryNode) collect: [:cname| Smalltalk at: ('OB', cname) asSymbol]) inject: 0 into: [:sum :el| sum + el methodDictionary size]

\begin{figure}[!ht]
\begin{center}
\includegraphics[scale=0.55]{obInternal.pdf}
\caption{Extension of the \obf to define the system browser.} \figlabel{obInternal}
\end{center}
\end{figure}


Compared to the default implementation of the Squeak System Browser this is less code and better factored. In addition other code-browsers can freely reuse these parts.

\begin{figure}[!ht]
\begin{center}
\includegraphics[scale=0.55]{ob-graph.pdf}
\caption{Metagraph of the system browser.} \figlabel{obmetagraph}
\end{center}
\end{figure}

\figref{obmetagraph} depicts the metagraph of the system browser. The metanode \ct{environment} contains information about class categories. The filter is used to select what has to be displayed from the selected class (\ie the class definition, its comment or the metaclass definition). A class and a metaclass have a list of method categories, including the \ct{-- all --} method category that shows a list of all methods.

As in the file browser example, we implement a method \ct{defaultMetaNode} on the class side of the browser class, \ie \ct{OBSystemBrowser}, returning the root metanode of the metagraph. This method reads:

\begin{code}{}
OBSystemBrowser class>>defaultMetaNode
	| env classCategory |
	env := OBMetaNode named: 'Environment'.
	classCategory := OBMetaNode named: 'ClassCategory'.
	env childAt: #categories put: classCategory.
	classCategory ancestrySelector: #isDescendantOfClassCat:.
	self buildMetagraphOn: classCategory.
	^env
\end{code}

There is a dedicated utility class called \ct{OBMetagraphBuilder} to create the complex metagraph of the system browser. The method \ct{defaultMetaNode} outsources most parts of the metagraph building to this class. \ct{OBMetagraphBuilder} implements its functionality in several small methods, \ie for every metanode of the metagraph there is a method holding all code to create this metanode and the outgoing edges, hence it is easily possible to adapt the metagraph by providing a dedicated subclass overriding the appropriate methods to change the right metanodes. 

The root node of the domain graph is answered by the method \ct{defaultRootNode}. For the system browser, the root node is the environment node:

\begin{code}{}
OBSystemBrowser class>>defaultRootNode
	^OBEnvironmentNode forImage
\end{code}


\paragraph{Ancestry mechanism.} As shown in \figref{obInternal} there is a number of different nodes that are required to implement the system browser, such as class node, metaclass node, method node, method protocol node, class comment node, etc. We do not want to cover all these nodes in detail. Instead we report on an important feature of \obf to locate specific nodes in a large domain graph: the ancestry mechanism. 

When a target node has to be selected, we start from the root node and traverse the tree down to the target node, remembering all nodes we pass during the traversal. Starting from the root node, we test for all children whether a child is an ancestry of the target node or not. If so, we go one level deeper and test the same for all children of this child, and so on, until we reach the target node. Every metanode, which basically models one level in the domain graph or tree, knows the ancestry selector to be used on this level. For a class node, the ancestrySelector is called \ct{isDescendantOfClass:}. If we search for a class node in the domain tree, we test for every class node if the class to be found is a descendant of that class, \ie if it is the same class as we search for. On the class category level, the ancestry selector is called \ct{descendantOfClassCat:}, expecting a class category as a parameter. For every class category, we test whether the target node is a descendant of the passed class category or not. 

This method \ct{descendantOfClassCat:} is implemented as follows for a node having a class associated (\eg a class node or a method node):

\begin{code}{}
OBClassAwareNode>>isDescendantOfClassCat: aClassCategoryNode
    ^(self theNonMetaClass environment organization 
		listAtCategoryNamed: aClassCategoryNode name)
			includes: self theNonMetaClassName
\end{code}

To define which metanode, \ie which level in the tree, uses which ancestry selector, we just pass this selector when building the metagraph, using the method \ct{ancestrySelector: aSymbol} of \ct{OBMetaNode}.
With these kind of methods, it is possible to locate any node in the domain tree to \eg jump to it. This is for instance used when opening a browser for a certain node, \eg by using the \ct{OBSystemBrowser} class-side method \ct{openOn: aClass selector: aSymbol}.

\paragraph{Filtering of nodes.}

In the metagraph we can also define several filters for a metanode, used to filter and otherwise manipulate the nodes represented by this metanode before they get displayed in columns \ab{Is this the same kind of filter we have previously seen?}. For the class category metanode, for instance, there are two filters defined: a class sort filter and the modal filter used to select one of the three outgoing metaedges instance, comment or class. 

Let's have a look at these two filters, starting with the class sort filter implemented in class \ct{OBClassSortFilter}. Its responsibility is to sort and indent all classes of a class category according to their position in a class hierarchy. If a class category for instance contains two distinct class hierarchies, \eg class C inherits from B, and B and D inherit from A, and E has two subclasses F and G, then the class sort filter sorts and indents these classes as shown in \figref{classSortFilter}.

\begin{figure}[!ht]
\begin{center}
\includegraphics[scale=1]{classSortFilter.pdf}
\caption{How OBClassSortFilter sorts and indents two distinct class hierarchies in one class category.} \figlabel{classSortFilter}
\end{center}
\end{figure}

When a metanode is asked for its children nodes (in method \ct{childrenForNode: aNode}) it asks its associated filters to answer the nodes by invoking their \ct{nodesFrom: aCollection forNode: aNode} method. In the case of the class sort filter, \ct{aNode} refers to the class category node and \ct{aCollection} holds all class nodes this class category node returns when the message \ct{classes} is sent to it. The class sort filter can now sort the passed class nodes and indent them appropriately in the method \ct{OBClassSortFilter >> nodesFrom:forNode:}.

The other filter defined for a class category metanode, \ct{OBModalFilter}, has a different task: It selects one edge of the three outgoing edges from the class category metanode, \ie instance, comment or class.  The user of the system browser can select using the switch in the class column (widget E in \figref{miniStBrowser}) whether he wants to see the instance-, the class-side or the comment of the selected class. \ct{OBModalFilter} remembers the selection of the user. Dependent on this selection, it answers the corresponding metaedge to be traversed, \eg the comment metaedge. This is done in the method \ct{edgesFrom: aCollection forNode: aNode}. The metanode, \ie the class category metanode, passes all available metaedges to this method, along with the currently selected class node, and the modal filter answers just the metaedge selected by the user. Other filters than a modal filter, such as the class sort filter, typically just return all edges passed to them.

There are two other important tasks performed by filters besides filtering edges and nodes: Manipulating the name of a node to be displayed and defining an icon shown along with a node in the column. The former is handled in the method \ct{displayString: aString forParent: pNode child:}, the latter in \ct{icon: aSymbol forNode: aNode}. Before a node's name gets displayed, all defined filters can manipulate the display of its name, \eg emphasize it in bold. Note that the filter also has access to the parent of a node to be displayed, not the current node alone. There are also filters enriching a node with an icon before display, the \ct{OBInheritanceFilter} for instance adds arrow up, down icons to methods, if a method overrides a method with the same name from a super class or is overridden in subclasses.\ab{how do you do this?}

A metanode can have arbitrarily many filters, resulting in a chain of filters. However, if several filters do the same kind of task, \eg adding an icon to a node, the last added filter providing this functionality will finally be responsible to define the icon which the node gets. Hence the order in which the filters get added to the metanode is relevant. 

%\paragraph{Indentation} The second list in the upper part of \figref{ob} shows an indentation of class hierarchy. It is a very convenient way of representing a class hierarchy contained in a class category. Although this indentation does not involve any dedicated feature in the model, it is still interesting to see how the Omnibrowser framework is open to visual improvements.

%The class \ct{ClassAwareNode}

%\rw{Do not like  this paragraph. It reads like: 'look how brilliant it is', when what I want is a Tree view. Moreover, the structure of the paragraph is upside down, it does not say what its point is, etc. I propose to remove it}

%The class \ct{ClassAwareNode} is an (indirect) subclass of \ct{Node}. It defines a variable \ct{superior} that refers to the node related to the superclass. When a class category is selected, the collection of class nodes that belongs to this class category is computed by the corresponding metanodes (\ie \ct{Class}, \ct{ClassComment} or \ct{Metaclass} according to the view selected in the browser). This collection is then sorted according to the class inheritance hierarchy. The selector \ct{\#indentedName} is the value of the variable \ct{displaySelector} in the three class-related metanodes (cf. class \ct{MetaNode} in \figref{core}). This method is then called to render a node in the browser. It is simply defined as:

%\begin{code}
%OBClassAwareNode\sep{}indentedName
%     ^ self indent, self name

%OBClassAwareNode\sep{}indent
%     \pipe size indent \pipe
%     size := 0.
%     self superiorsDo: [:ea \pipe size := size + 1].
%     indent := Text new: size * 2.
%     indent atAllPut: $ .
%     ^ indent
%\end{code}

%The method \ct{indent} computes how deep a class is in the superiors, which corresponds to the class hierarchy, then it builds a white character text, appended before the name of the class (cf. method \ct{indentedName}).

\paragraph{Widgets notification.} Widgets like menu lists and text panels interact with each other by triggering events and receiving notifications. Each browser has a dispatcher (referenced by the variable \ct{dispatcher} in the class \ct{Browser}) to conduct events passing between widgets of a browser. The vocabulary of events is the following one:
\begin{itemize}
\item \ct{refresh} is emitted when a complete refresh of the browser is necessary. For instance, if a change happens in the system, this event is triggered to trigger a complete redraw.

\item \ct{nodeSelected} is emitted when a list entry is selected with a mouse click. 

\item \ct{nodeDeleted} is emitted when a list entry has been removed, \eg by executing a remove command. 

\item \ct{nodeChanged} is emitted when the node that is currently displayed changes. This typically occurs when a filter button related to the class is selected. For example, if a class is displayed, pressing the button \ct{instance}, \ct{class} or \ct{comment} triggers this event.

\item \ct{okToChangeNode} is emitted to prevent losing some text edition while changing the content of a text panel if this was modified without being validated. This happens when a user writes the definition of a method, without accepting (\ie compiling) it, and then selects another method.
\end{itemize}

Each graphical widget composing a browser is a listener and can emit events. Creation and registration of widgets as listeners and event emitters is completely transparent to the end user.

%:AB ==> Done \sd{ok this is emitted but I do not know who is listening or registering this event I guess it but we should say it.}


\paragraph{State of the browser.} Contrary to the original Squeak system browser where each widget state is contained in a dedicated variable, the state of a \obf-based browser is defined as a path in the metagraph starting from the root metanode. Each metanode taking part of this path is associated to a domain node. This preserves the synchronization between different graphical widgets of a browser.

%\sd{would be nice to add that taken from colin email: but I do not completely understand it. Alex do you get it? ----
%you wrote in your nodes: New nodes can be created by subclassing \ct{Node}. When a \textit{find class} operation is performed, the method \ct{ancestrySelector} can be overridden to specify the ancestry relationship. 
%-- colin wrote ---
%The notion of ancestry refers to domain objects. It's mainly useful for jumping directly to a node that's deep in the graph - browsing a particular method, for example. To do this, one has to be able to select the correct class category, class, method category and method. To do this, we start at the root node, and walk down the graph, scanning the edges of the metagraph and choosing those that lead to ancestors of the target node, until the target node is reached.
%}

%:===============

\section{The Coverage Browser}\label{sec:coverageBrowser}

The coverage browser is an extension to the system browser to show the coverage of code by unit tests.
It extends the system browser in two ways.
First of all it appends the percentage of elements covered by tests to the elements in the lists making up the browser. Secondly it adds a fifth pane that lists the unit tests that test a selected method. 
A screenshot is shown in \figref{cb-screenshot}.
It shows us that 39\% of the class \ct{UUID} is covered by tests, and that the method \ct{initialize} is covered a 100\% by the tests shown in the right-most pane. One of these tests is \ct{testCreation}.

\begin{figure}[!ht]
\begin{center}
\includegraphics[scale=0.44]{coverageBrowserScreenshot.pdf}
\caption{Screenshot of the coverage browser.} \figlabel{cb-screenshot}
\end{center}
\end{figure}

\begin{figure}[!ht]
\begin{center}
\includegraphics[scale=0.55]{coverageBrowser.pdf}
\caption{Extension of Omnibrowser and system browser to define the coverage browser.} \figlabel{cb}
\end{center}
\end{figure}

The coverage browser is composed of 11 classes (one class for the browser, five commands and five nodes). \figref{cb} illustrates how classes in \ob and in the system browser are extended to define this new browser. The metagraph is depicted in \figref{cb-graph} and is identical to the system browser except with a new \ct{Method Coverage} metanode. The depth of the graph, which is 5, is reflected in the number of list panes the browser is composed of.

\begin{figure}[!ht]
\begin{center}
\includegraphics[scale=0.45]{cb-graph.pdf}
\caption{Metagraph for the coverage browser.} \figlabel{cb-graph}
\end{center}
\end{figure}


%This browser is the "control center" of the coverage tool (probably be named Christo in the future ?). It let the user create "coverage sets". A coverage set is a kind of persistent configuration that knows everything about how to collect coverage data and on which elements. An element might be a package, a method or any other OBNode. The elements can be added by drag & dropping OBNodes from any other Omnibrowser supporting drag & drop. The browser has a button to browse the selection using the coveage browser. Besides several operations on sets and elements the browser enables limited browsing capabilities (sources, tests, obsolete coverage) and has control over the coverage cache.

%The purpose of this browser it to provide a user interface that makes the handling of the coverage tool simple and intuitive - without the need of knowing the technology or how it need to be used/applied.

% number of methods in the coverage browser = 83 
%(#(CoverageActor CoverageElementActor CoverageEnvironmentActor CoverageMethodNodeActor CoverageSetActor CoverageBrowser CoverageMethodNode CoverageNode CoverageElementNode CoverageEnvironmentNode CoverageSetNode) collect: [:cname| Smalltalk at: ('OB', cname) asSymbol]) inject: 0 into: [:sum :el| sum + el methodDictionary size] 

%:%%%%%%%%%%%%%%%%%%%%%%%%%%%%%%%%%%%%%%%%%%%%%%%%%

\section{Evaluation and Discussions} \seclabel{discussion}

Several other browsers such as a browser specifically supporting new language constructs such as Traits have been developed using \obf demonstrating that the framework is mature and extensible. Figure~\ref{browsers} shows some browsers that are based on \obf.
 We now discuss the strengths and limitations of the \obf.

%\paragraph{Explicit state transitions} As metagraph are statically defined and each metaedges describes an action the user can perform on a browser, states a browser can be in are explicit and fully described. 


\begin{figure}
\begin{center}
\includegraphics[width=8cm]{BrowserFamily}
\caption{Some code browsers developed using \obf.}\label{browsers}
\end{center}
\end{figure}


\subsection{Strengths}

\paragraph{Ease of use.}
As any good framework, extending it following the framework intention makes it easy to specify advanced browsers. The fact that the browser navigation is explicitly defined in one place lets the programmer easily understand and control the tool navigation and user interaction. The programmer does not have the burden to explicitly create and glue together the UI widgets and their specific layout. To add additional custom widgets in a concrete browser, the developer can simply define a class implementing this widget and add an object of this class to the list of widgets used during the creation of the browser. This list is defined on the class-side of \ct{OBBrowser} in the method \ct{panels}. Still the programmer focuses on the key domain of the browser: its navigation and the interaction with the user. 

\paragraph{Explicit state transitions.} Maintaining coherence among different widgets and keeping them synchronized is a non-trival issue that, while well supported by GUI frameworks,  is often not well used. For instance, in the original Squeak browser, methods are scattered with checks for nil or 0 values. For instance, the method \ct{classComment: aText notifying: aPluggableTextMorph}, which is called by the text pane (F widget) to assign a new comment to the selected class (B widget), is:

\begin{code}
Browser\sep{}classComment: aText notifying: aPluggableTextMorph 
    theClass := self selectedClassOrMetaClass.
    theClass
        ifNotNil: [ ... ]
\end{code}

The code above copes with the fact that when pressing on the class comment button, there is no warranty that a class is selected. In a good UI design, the comment class button should have been disabled, however there are still checks done whether a class is selected or not. Among the 438 accessible methods in the non \ob-based Squeak class \ct{Browser}, 63 of them invoke \ct{ifNil:} to test whether a list is selected or not and 62 of them send the message \ct{ifNotNil:}. Those are not isolated Smalltalk examples. The code that describes some GUI present in the JHotDraw framework also contains the pattern checking for a nil value of variables that may reference graphical widgets. 

Such a situation does not occur in \obf, as metagraphs are declaratively defined, and each metaedge describes an action the user can perform on a browser, states a browser can be in are explicit and fully described.

\paragraph{Separation of domain and navigation.} The domain model and its navigation are fully separated: a metanode does not and cannot have a reference to the domain node currently selected and displayed. Therefore both can be reused independently.

\subsection{Limitations} 

\paragraph{Hardcoded flow.} As any framework, \obf constraints the space of its own extension. \obf does not support well the definition of navigation not following the left  to right list construction (the result of the selection creates a new pane to the right of the current one and  the text pane is displayed). For example, building a browser such as Whiskers that displays multiple methods at the same time would require to deeply change the text pane state to keep the status of the currently edited methods. 

%\paragraph{Currently selected item} The \obf does not easily support the building of advanced browsing facilities such as the one of the VisualWorks standard browser. In VisualWorks, it is possible to select a package, then select one class of this package and as third step see the inheritance hierarchy of this class within the context of the previously selected package. The problem is that conceptually the selected item is not part of the state representation. It is possible using UI events passing among the widgets to implement \dr{what?} \ab{I do not remember what I wanted to say here.}

%\subsection{Discussions}
%Alternate approaches to build browsers exist such as using VisualWorks ValueModels and application model.  It would be possible to represent the state of the browser as a model or application model and use the implicit dependency mechanism and the propagation to represent the state change and user navigation in the browser. Such an approach is still to be implemented but is close to the idea of reflective application builder discussed in next section.

%@@ To continue

%\paragraph{Separation of domain and navigation} The domain model and the navigation into the space domain are fully separated: a metanode does not and cannot have a reference to the domain node currently selected and displayed.

%The questions of having to define nodes or reusing the Smalltalk underlying metamodel is interesting. Using class extensions it should possible to avoid to have to wrap domain entities into nodes. Still creating a new node that could merge two kinds of entities such as classes and packages is important to build advanced navigation as the VisualWorks package/class selection described above.

%



%\begin{code}
%contents
%	"Depending on the current selection, different information is retrieved.
%	Answer a string description of that information. This information is the
%	method of the currently selected class and message."

%	| comment theClass latestCompiledMethod |
%	latestCompiledMethod _ currentCompiledMethod.
%	currentCompiledMethod _ nil.

%	editSelection == #none ifTrue: [^ ''].
%	editSelection == #editSystemCategories 
%		ifTrue: [^ systemOrganizer printString].
%	editSelection == #newClass 
%		ifTrue: [^ (theClass _ self selectedClass)
%			ifNil:
%				[Class template: self selectedSystemCategoryName]
%			ifNotNil:
%				[Class templateForSubclassOf: theClass category: self selectedSystemCategoryName]].
%	editSelection == #editClass 
%		ifTrue:
%			[^ self classDefinitionText ].
%	editSelection == #editComment 
%		ifTrue:
%			[(theClass _ self selectedClass) ifNil: [^ ''].
%			comment _ theClass comment.
%			currentCompiledMethod _ theClass organization commentRemoteStr.
%			^ comment size = 0
%				ifTrue: ['This class has not yet been commented.']
%				ifFalse: [comment]].
%	editSelection == #hierarchy 
%		ifTrue: [^ self selectedClassOrMetaClass printHierarchy].
%	editSelection == #editMessageCategories 
%		ifTrue: [^ self classOrMetaClassOrganizer printString].
%	editSelection == #newMessage
%		ifTrue:
%			[^ (theClass _ self selectedClassOrMetaClass) 
%				ifNil: ['']
%				ifNotNil: [theClass sourceCodeTemplate]].
%	editSelection == #editMessage
%		ifTrue:
%			[self showingByteCodes ifTrue: [^ self selectedBytecodes].
%			currentCompiledMethod _ latestCompiledMethod.
%			^ self selectedMessage].

%	self error: 'Browser internal error: unknown edit selection.'
%\end{code}

%:%%%%%%%%%%%%%%%%%%%%%%%%%%%%%%%%%%%%%%%%%%%%%%%%%

%\section{Related Work} \seclabel{relatedwork}

%\paragraph{MVC} The Model-View-Controller~\cite{Kras88a,Tryg03a,Tryg79a} promotes a distinction between three important roles (namely data, output and interaction) that should be reflected in the design of a user interface framework. Those roles were reflected in three abstract superclasses: \ct{Model}, \ct{View}, \ct{Controller}. Still for system browsers, developers consider the model as the entities of the domain and do not have explicit or meta entities describing the navigation within the domain model. Note also that a controller in MVC captures the interaction of users with a widget,and passes this information to the model. The level of abstraction, however, is lower than what is offered by  \emph{Command} in the \obf, which is not programmed in terms of a widget but in terms of the domain entities.\\

%\paragraph{HotDraw}
%The state transitions between the possible tools in HotDraw~\cite{John92a} 
%are driven by an explicit state machine and follow an explicit transition structure. There is a graphical editor (constructed with HotDraw itself) to construct the view and edit the state machine. The goal of the state machine is similar to the goal of the metagraph in the \obf: to make navigation explicit.  In HotDraw, however, the events to go from one state to another are taken from a limited set of possible actions such as mouse over. \\

%\paragraph{HyperCard} Conceptually, a HyperCard~\cite{Good87a} application is a stack of cards. Each card contains some information and links to other cards in the same or other stacks. The information on the cards is shown using text and graphics. The links to other cards are presented as buttons, typically completed with an icon representing the destination card. A user of HyperCard browses the cards of a stack using the link button. Only one card of a stack is displayed at a time. Clicking a link button results in the display of the destination card. When a stack has not only information to be displayed, but also has to exhibit an active behavior, the stack designer has to develop cards by means of a scripting level, on which programming in the dedicated language HyperTalk is supported. Still there is not as such a metagraph describing the navigation of a domain graph.\\


%\paragraph{Constrained graphical objects} Coral (Constraint-based Object-oriented Relations And Language)~\cite{Szek88a} is a user interface toolkit supporting constraint between graphical objects. Examples of constraints are lines between two graphical objects that stay connected when those objects move and a graphical chess piece restricted to some legal moves. \sd{so this is not really related}


%\paragraph{\applflab} Steyaert \textit{et al.} defined the notion of reflective application buil\-der~\cite{Stey96b} with as explicit goal to be able to construct and reuse (parame\-trizable) user interface components. \applflab was used to construct several domain specific user interfaces, including browsers in development environments \cite{Wuyt96a}.\\

%\applflab structures a software program using four distinct kinds of components: 
%\begin{itemize}
%\item a \textit{user interface component} controls the display and the user interaction of a particular piece of information, supplied by the domain model. Note that this component is parametrized by the domain model, and therefore can be reused across different domains.

%\item an \textit{application model} manages the global behavior of group of interface components. It is responsible for the user interface logic and controls user interface. A same application model can be reused on different domain models and a domain model can have several application models in parallel.

%\item a \textit{domain model} models the overall functionality of the problem domain and maintains user interface independent constraints.

%\item a set of \textit{aspects} is needed to separate the domain model from the user interface component.
%\end{itemize}

%Interaction between these four components is based on emitting events and being notified. There are three kinds of event: \textit{display}, \textit{notify} and \textit{control}.

%The advantage of \applflab lies in its notion of parametrized user interface component. A user interface component consists of a GUI description, and parameters to link the component to the domain or to specify other information when it is used in an application. The components are plugged together to form applications. One could for example build a list component, and parametrize it with categories, classes, protocols and selectors to get the four top elements that make a System Browser (as shown in \secref{systemBrowser}). Combine it with a Text component and the System Browser is complete. 

%While both \applflab and the \ob make it easy to build browsers, there are some differences. The \ob is a domain specific approach for building browsers, while \applflab is general. So when using \applflab to build browser, browser specific components need to be built first, for example to get the left-to-right selection behavior that is built-in with \ob. \applflab also had a steeper learning curve, since building a good reusable component (be it a visual one or a regular one) remains fairly difficult. On the other hand, \ob offers more built-in behavior which makes it easier to use but also forces certain behavior that might not always be wanted.\\

%\paragraph{ThingLab} Freeman-Benson and Maloney~\cite{Free89a} wrote ThingLab II, an object-oriented constraint system for direct manipulation user interface implemented in Smalltalk-80. In ThingLab II, user-manipulable entities are collections of objects know as \textit{Things}. ThingLab II provides a large number of primitive Things equivalent to the operations and data structures provided in any high-level language: numerical operations, points, strings, bitmaps, conversion, etc. 

%A thing is constructed from things objects and constraint objects. Higher-level things can be built out of the lower-level ones. Constraints are either satisfied or they are not satisfied, and they are simple declarative declarations that do not hold state. 
%Browser navigation can be expressed by constraints between the different elements that composed a browser. But there is no explicit distinction between the domain and its navigation. 


%:%%%%%%%%%%%%%%%%%%%%%%%%%%%%%%%%%%%%%%%%%%%%%%%%%


\section{Conclusion} \seclabel{conclusion}

% [RoelSpace :-) ]
%Just to Remember for some future version:
%    Meta node -> Navigation Node
%    Meta edge -> Navigation Edge
%    Node -> Domain Node

%Browsers are build by describing the model in term of domain nodes, the navigation with a metagraph and a set of actors to enable interaction between the browser user with the domain browsed. Graphical widgets and their layout are fully abstracted by the Omnibrowser framework, enabling a graphical tool to navigate and edit a domain model without writing any graphical description.

%context
Smalltalk is known for its advanced development environment, featuring advanced browsers that let developers navigate and change code relatively easily.


%problem
Building browsers, however, is a daunting task. The main problem is that every navigation action performed by a user in a widget changes the state of that (and possibly other) widgets. Given the high number of possible navigation actions, the complexity of managing the navigation by managing the states of the browser is a very complex task.
This can be seen in most current browser implementations, which are complex and hard to extend because the navigation is implicitly encoded in the management of the state of the widgets.
 

%solution
To make it easier to build and extend browsers, \ob is a framework for building browsers that is based on modeling user navigation through an explicit graph.
In this framework, browsers are built by modeling the domain with \emph{nodes}, expressing the navigation with a \emph{metagraph} and describing the interaction between the browser and the domain through \emph{commands}. 
The framework uses these descriptions to construct a graphical application. The top half of the application uses lists that allow the user to navigate the described domain. The bottom half of the window is used to visualize and edit nodes selected in the top half.

%validation
The framework is implemented in Squeak Smalltalk and called \ob.
This Chapter shows three concrete instantiations of the framework: a file browser to navigate a file system, a reimplementation of the ubiquitous Smalltalk System Browser, and a code coverage browser.
Of course, there are more instantiations available than we have not discussed in this chapter.
The validation shows that the goals of the frameworks are met. Building the System Browser with the \obf shows that the code is much simpler. The Code Coverage browser shows that it is easy to extend an existing browser.

%%extensions, open up the paper again
%For future work we plan to enhance the \obf with the ability to have multiple text panes be part of a browser.
%We also plan to extend the framework to support more and richer widgets (such as toolbars and flaps).
%Last but not least we want to investigate how we can extend the metagraph to look at other ways of navigating it.\\

%=================================================================
%\section{Chapter Summary}
%\begin{itemize}

%\item 

%\item 

%\item 

%\item 

%\item 

%%\item Monticello allows a package version different from the local version to be merged, creating a new branch and changing the local version of your package.
%%\on{confusing sentence --- not sure what you want to say}
%%\ab{ditto}
%\end{itemize}

%=============================================================
\ifx\wholebook\relax\else
   \bibliographystyle{jurabib}
   \nobibliography{scg}
   \end{document}
\fi
%=============================================================


\part{Language}
% $Author$
% $Date$
% $Revision$

% HISTORY:
% 2008-05-14 - Alex started chapter
% 2008-06-11 - Alex ported text from Vassili Bykov
% 2008-08-22 - Stef added part 1
% 2008-11-26 - Alex completed translation from French article
% 2008-11-29 - Damien Pollet fixes
% 2008-12-13 - Oscar revised
% 2009-03-18 - Stef extended
% 2009-06-17 - Oscar migrated to Pharo
% 2009-07-16 - Oscar indexing
% 2009-07-16 - Lukas commenting
% 2009-08-12 - Oscar fixed Lukas' comments
% 2009-08-29 to 09-02 - Andrew brought code up to date with current Pharo, clarified
% 2009-09-07 - Alexandre revised
% 2009-10-21 - Hernan Wilkinson comments
% 2010-03-02 - posted as draft chapter for PBE2 on web site
% 2010-03-03 - Henrik Sperre Johansen spotted an error. Alexandre fixed it
% 2010-03-11 - Various corrections from Max Leske - fixed by Oscar
% 2011-09-11 - Migrated to PharoBox: svn checkout https://XXX@scm.gforge.inria.fr/svn/pharobooks/PharoByExampleTwo-Eng

% Todo for stef	Should add a note on 	on: fork:

%=================================================================
\ifx\wholebook\relax\else
% --------------------------------------------
% Lulu:
	\documentclass[a4paper,10pt,twoside]{book}
	\usepackage[
		papersize={6.13in,9.21in},
		hmargin={.75in,.75in},
		vmargin={.75in,1in},
		ignoreheadfoot
	]{geometry}
	\input{../common.tex}
	\pagestyle{headings}
	\setboolean{lulu}{true}
% --------------------------------------------
% A4:
%	\documentclass[a4paper,11pt,twoside]{book}
%	\input{../common.tex}
%	\usepackage{a4wide}
% --------------------------------------------
    \graphicspath{{figures/} {../figures/}}
	\begin{document}
	\renewcommand{\nnbb}[2]{} % Disable editorial comments
	\sloppy
\fi

%=================================================================
\chapter{Handling exceptions}
\chapterauthor{\authorsteph{}}

All applications have to deal with exceptional situations.
Arithmetic errors may occur (such as division by zero), unexpected situations may arise (file not found), or resources may be exhausted (network down, disk full, \etc).
The old-fashioned solution is to have operations that fail return a special \emph{error code}; this means that client code must check the return value of each operation, and take special action to handle errors.

Modern programming languages, including \st, instead offer a dedicated exception-handling mechanism that greatly simplifies the way in which exceptional situations are signaled and handled.
Before the development of the \ind{ANSI \st{}} standard in 1996, several  exception handling mechanisms existed, largely incompatible with each other. \pharo's exception handling follows the ANSI standard, with some embellishments; we present it in this chapter from a user perspective.

The basic idea behind exception handling is that 
client code does not clutter the main logic flow with checks for error codes, but specifies instead an \emph{exception handler} to ``catch'' exceptions.
When something goes wrong, instead of returning an error code, the method that detects the exceptional situation interrupts the main flow of execution by  \emph{signaling} an exception.
This does two things: it captures essential information about the context in which the exception occurred, and transfers control to the \ind{exception handler}, written by the client, which decides what to do about it.
The ``essential information about the context'' is saved in an \ct{Exception} object; 
various classes of \clsind{Exception} are specified to cover the varied exceptional situations that may arise.

\pharo's exception-handling mechanism is particularly expressive and flexible, covering a wide range of possibilities. Exception handlers can be used to \emph{ensure} that certain actions take place even if something goes wrong, or to take action only if something goes wrong.
Like everything in \st, exceptions are objects, and respond to a variety of messages.
When an exception is caught by a handler, there are many possible responses: the  handler can specify an alternative action to perform; it can ask the exception object to \emph{resume} the interrupted operation; it can \emph{retry} the operation; it can \emph{pass} the exception to another handler; or it can \emph{reraise} a completely different exception.

With the help of a series of examples, we shall explore all of these possibilities, and we shall also take a brief look into the internal mechanics of exceptions and exception handlers.
However, before we do that, we need to stop and think for a moment about the consequence of adding exceptions into a language: \emph{we can no longer be sure that a message send will give us an answer}.  
In other words, once we have exceptions, any message send has the potential not to return to the sender: it may fail.

%=================================================================
\section{Ensuring execution}

The \mthind{BlockClosure}{ensure:} message can be sent to a block to make sure that, even if the block fails (\eg raises an exception) the argument block will still be executed:
\begin{code}{}
!\emph{anyBlock}! ensure: !\emph{ensuredBlock}!    "ensuredBlock will run even if anyBlock fails"
\end{code}

Consider the following example, which creates an image file from a screenshot taken by the user:

\mthindex{FileStream}{newFileNamed:}
\mthindex{GIFReadWriter}{nextPutImage:}
\needlines{4}
%\begin{code}{}
%| writer |
%[	writer := GIFReadWriter on: (FileStream newFileNamed: 'Pharo.gif').
%	writer nextPutImage: (Form fromUser)
%]	ensure: [ writer ifNotNil: [ writer close ] ]
%\end{code}

%\lr{The typical pattern is like this:}

\begin{code}{}
| writer |
writer := GIFReadWriter on: (FileStream newFileNamed: 'Pharo.gif').
[ writer nextPutImage: (Form fromUser) ]
	ensure: [ writer close ]
\end{code}

%\lr{This has the exact same semantics as the other implementation, but is much simpler and does not require the nil test. Lint also suggests this variation if you write the above one.}

\noindent
This code ensures that the \ct{writer} file handle will be closed, even if an error occurs in \ct{Form fromUser} or while writing to the file.

Here is how it works in more detail.
The \ct{nextPutImage:} method of the class \ct{GIFReadWriter} converts a form (\ie an instance of the class \ct{Form}, representing a bitmap image) into a GIF image. This method writes into a stream which has been opened on a file. The \ct{nextPutImage:} method does not close the stream it is writing to, therefore we should be sure to close the stream even if a problem arises while writing. This is achieved by sending the message \ct{ensure:} to the block that does the writing. In case \ct{nextPutImage:} fails, control will flow into the block passed to \ct{ensure:}.  If it does \emph{not} fail, the ensured block will still be executed.  So, in either case, we can be sure that \ct{writer} is closed.

Here is another use of \ct{ensure:}, in class \ct{Cursor}:

%\lr{That code is in \ct{Cursor} not in \ct{Sensor}.}

\needlines{7}
\begin{code}{}
Cursor>>>showWhile: aBlock 
	"While evaluating the argument, aBlock,
	make the receiver be the cursor shape."
	| oldcursor |
	oldcursor := Sensor currentCursor.
	self show.
	^aBlock ensure: [ oldcursor show ]
\end{code}

The argument \ct{[ oldcursor show ]} is evaluated whether or not  \ct{aBlock} signals an exception. Note that the result of \ct{ensure:} is the value of the receiver, not that of the argument.

\begin{code}{@TEST}
[ 1 ] ensure: [ 0 ] --> 1    "not 0"
\end{code}

%=================================================================
\section{Handling non-local returns}

The message \mthind{BlockClosure}{ifCurtailed:} is typically used for ``cleaning'' actions. It is similar to \ct{ensure:}, but instead of ensuring that its argument block is evaluated even if the receiver terminates abnormally, \ct{ifCurtailed:} does so \emph{only} if the receiver fails or returns.

In the following example, the receiver of \ct{ifCurtailed:} performs an early return, so the following statement is never reached.
In \st, this is referred to as a \emph{non-local return}.
Nevertheless the argument block will be executed.
\needlines{4}
\begin{code}{}
[^ 10] ifCurtailed: [Transcript show: 'We see this'].
Transcript show: 'But not this'.
\end{code}

In the following example, we can see clearly that the argument to \ct{ifCurtailed:} is evaluated only when the receiver terminates abnormally.
\clsindex{Error}
\begin{code}{}
[Error signal] ifCurtailed: [Transcript show: 'Abandoned'; cr].
Transcript show: 'Proceeded'; cr.
\end{code}

\dothis{Open a transcript and evaluate the code above in a workspace.
When the pre-debugger windows opens, first try selecting \button{Proceed} and then \button{Abandon}. Note that the argument to \ct{ifCurtailed:} is evaluated only when the receiver terminates abnormally. What happens when you select \button{Debug}?}

Here are some  examples of \lct{ifCurtailed:} usage: the text of the \ct{Transcript show:} describes the situation:

\begin{code}{}
[^ 10] ifCurtailed: [Transcript show: 'This is displayed'; cr] 

[10] ifCurtailed: [Transcript show: 'This is not displayed'; cr] 

[1 / 0] ifCurtailed: [Transcript show: 'This is displayed after selecting Abandon in the debugger'; cr]
\end{code}

Although in \pharo \ct{ifCurtailed:} and \ct{ensure:} are implemented using a a marker primitive (described at the end of the chapter), 
%\lr{That's not true, primitive 198 and 199 are no-ops and just act as a markers. Today the same could be done using method annotations, but that was not possible when the exceptions were implemented. The methods \ct{isHandlerContext} and \ct{isUnwindContext} just check for the presence of these marker primitives and are called by \ct{findNextHandlerContextStarting} and \ct{findNextUnwindContextUpTo:} to find the contexts. The two latter methods are primitives for efficiency reasons, but apart from that the complete exception handling is implemented at the Smalltalk level without primitive support.}\sd{exactly, but this is not wrong: ensure: is still a primitive, we talk about that at the end of the chapter}, 
in principle \ct{ifCurtailed:} could be implemented using \ct{ensure:} as follows:

\begin{code}{}
ifCurtailed: curtailBlock
	| result curtailed |
	curtailed := true.
	[	result := self value.
		curtailed := false
	]	ensure: [ curtailed ifTrue: [ curtailBlock value ] ].
	^ result
\end{code}

In a similar fashion, \ct{ensure:} could be implemented using \ct{ifCurtailed:} as follows:

\begin{code}{}
ensure: ensureBlock
	| result |
	result := self ifCurtailed: ensureBlock.
	"If we reach this point, then the receiver has not been curtailed,
	so ensureBlock still needs to be evaluated"
	ensureBlock value.
	^ result
\end{code}

Both \ct{ensure:} and \ct{ifCurtailed:} are very useful for making sure that important ``cleanup'' code is executed, but are not by themselves sufficient for handling all exceptional situations.
Now let's look at a more general mechanism for handling exceptions.

%=================================================================
\section{Exception handlers}

\index{Exception handling}
The general mechanism is provided by the message \mthind{BlockClosure}{on:do:}. It looks like this:
\begin{code}{}
!\emph{aBlock}! on: !\emph{exceptionClass}! do: !\emph{handlerAction}!
\end{code}
\noindent
\lct{\emph{aBlock}} is the code that detects an abnormal situation and signals an exception; it is called the \emph{protected block}.   
\lct{\emph{handlerAction}} is the block that is evaluated if an exception is signaled; it is called the \emph{exception handler}.
\ct{exceptionClass} defines the class of exceptions that \ct{handlerAction} will be asked to handle.

%\ab{This is inappropriate, because in the ANSI standard the first argument to \ct{on:do:} is a selector, not a class.}  

%\ind{ANSI \st} defines this message as follows:
%\begin{quote}
%``The receiver is evaluated such that if during its evaluation an exception corresponding to selector is signaled then action will be evaluated. The result of evaluating the receiver is returned.''
%\end{quote}

The beauty of this mechanism lies in the fact that the protected block can be written in  a straightforward way, \emph{without regard to any possible errors}. A single exception handler is responsible for taking care of anything that may go wrong.

Consider the following example, where we want to copy the contents of one file to another.
Although several file-related things could go wrong, with exception handling we simply write a straight-line method, and define a single exception handler for the whole transaction: 
\clsindex{FileStream}
\clsindex{FileStreamException}
\clsindex{FileDirectory}
\mthindex{FileDirectory}{oldFileNamed:}
\mthindex{FileDirectory}{newFileNamed:}
%\begin{code}{| source destination fromStream toStream |}
%source := 'log.txt'.
%destination := 'log-backup.txt'.
%[	fromStream := FileDirectory default oldFileNamed: source.
%	toStream := FileDirectory default newFileNamed: destination.
%	toStream nextPutAll: fromStream contents
%]
%	on: FileStreamException
%	do: [ :ex | UIManager default inform: 'Copy failed -- ', ex description ].
%fromStream ifNotNil: [fromStream close].
%toStream ifNotNil: [toStream close].
%\end{code}

%\lr{That code is not correct. If any other exception than \ct{FileStreamException} happens the files are not properly closed. Following the pattern of the initial example, the code has to be written like this:}

\needlines{7}
\begin{code}{| source destination fromStream toStream |}
source := 'log.txt'.
destination := 'log-backup.txt'.
[	fromStream := FileDirectory default oldFileNamed: source.
	[	toStream := FileDirectory default newFileNamed: destination.
		[ toStream nextPutAll: fromStream contents ]
			ensure: [ toStream close ] ]
		ensure: [ fromStream close ] ]
	on: FileStreamException
	do: [ :ex | UIManager default inform: 'Copy failed -- ', ex description ].
\end{code}

If any exception concerning \ct{FileStreams} is raised, the handler block (the block after \ct{do:}) is executed with the exception object as its argument.
Our handler code alerts the user that the copy has failed, and delegates to the exception object \ct{ex} the task of providing details about the error.
Note the two nested uses of \ct{ensure:} to make sure that the two file streams are closed, whether or not an exception occurs.

It is important to understand  that the block that is the receiver of the message \ct{on:do:} defines the scope of the exception handler. This handler will be used only if the receiver (\ie the protected block) has not completed. Once completed, the exception handler will not be used. Moreover, a handler is associated exclusively with the kind of exception specified as the first argument to \ct{on:do:}. 
Thus, in the previous example, only a \clsind{FileStreamException} (or a more specific variant thereof) can be handled.

%=================================================================
\section{Error codes --- don't  do this!}

Without exceptions, one (bad) way to handle a method that may fail to produce an expected result is to introduce explicit error codes as possible return values. In fact, in languages like C, code is littered with checks for such error codes, which often obscure the main application logic.
Error codes are also fragile in the face of evolution: if new error codes are added, then all clients must be adapted to take the new codes into account. By using exceptions instead of error codes, the programmer is freed from the task of explicitly checking each return value, and the program logic stays uncluttered.
Moreover, because exceptions are classes, as new exceptional situations are discovered, they can be subclassed; old clients will still work, although they may provide less-specific exception handling than newer clients.

If \st did not provide exception-handling support, then the tiny example we saw in the previous section would be written something like this, using error codes:

\begin{code}{| source destination fromStream toStream contents result success failure |}
"Pseudo-code -- luckily Smalltalk does not work like this. Without the 
benefit of exception handling we must check error codes for each operation."
source := 'log.txt'.
destination := 'log-backup.txt'.
success := 1. "define two constants, our error codes"
failure := 0.
fromStream := FileDirectory default oldFileNamed: source.
fromStream ifNil: [
	UIManager default inform: 'Copy failed -- could not open', source.
	^ failure "terminate this block with error code" ].
toStream := FileDirectory default newFileNamed: destination.
toStream ifNil: [
	fromStream close.
	UIManager default inform: 'Copy failed -- could not open', destination.
	^ failure ].
contents := fromStream contents.
contents ifNil: [
	fromStream close.
	toStream close.
	UIManager default inform: 'Copy failed -- source file has no contents'.
	^ failure ].
result := toStream nextPutAll: contents.
result ifFalse: [
	fromStream close.
	toStream close.
	UIManager default inform: 'Copy failed -- could not write to ', destination.
	^ failure ].
fromStream close.
toStream close.
^ success.
\end{code}
\noindent
What a mess!
Without exception handling, we must explicitly check the result of each operation before proceeding to the next.
Not only must we check error codes at each point that something might go wrong, but we must also be prepared to cleanup any operations performed up to that point and abort the rest of the code.

%=================================================================
\section{Specifying which Exceptions will be Handled}


\begin{figure}[t]\centering
        \includegraphics[width=.5\linewidth]{SimpleHierarchy}
        \caption{A small part of the \pharo exception hierarchy.\figlabel{hierarchy}}
\end{figure}


In \st, exceptions are, of course, objects. 
In \pharo{}, an exception is an instance of an exception class which is part of a hierarchy of exception classes.
For example, 
because the exceptions \clsind{FileDoesNotExistException}, \clsind{FileExistsException} and \clsind{CannotDeleteFileException} are special kinds of \clsind{FileStreamException}, they are represented as subclasses of \ct{FileStreamException}, as shown in \figref{hierarchy}.
This notion of ``specialization'' lets us associate an exception handler with a more or less general exceptional situation.
So, we can write:
\begin{code}{}
[ ... ] on: Error do: [ ... ]
[ ... ] on: FileStreamException do: [ ... ]
[ ... ] on: FileDoesNotExistException do: [ ... ]
\end{code}


The class \ct{FileStreamException} adds information to class \ct{Exception} to characterize the specific abnormal situation it describes. Specifically, \ct{FileStreamException} defines the \ct{fileName} instance variable, which contains the name of the file that signaled the exception. The root of the exception class hierarchy is \ct{Exception}, which is a direct subclass of \ct{Object}.

%In some versions of \pharo, \ct{Exception} has 10 direct subclasses and 103 indirect subclasses!
%\begin{code}{} % NB: Fragile -- should not be a test
%Exception subclasses size     --> 10
%Exception subclasses 	        --> {Error . IllegalResumeAttempt . Notification . ProgressInitiationException . Abort . UnhandledError . TestFailure . Halt . MCNoChangesException . RefactoringWarning}
%Exception allSubclasses size --> 103
%\end{code}
%\ab{I removed this because I added a new subsection that discusses the hierarchy more completely.}

Two key messages are involved in exception handling: \ct{on:do:}, which, as we have already seen, is sent to blocks to set an exception handler, and \ct{signal}, which is sent to subclasses of \ct{Exception} to signal that an exception has occurred.

%=================================================================
\section{Signaling an exception}

To signal an exception\footnote{Synonyms are to ``raise'' or to ``throw'' an exception. Since the vital message is called \lct{signal}, we use that terminology exclusively in this chapter.}, you only need to create an instance of the exception class, and to send it the message \ct{signal}, or \ct{signal:} with a textual description. The class \ct{Exception class} provides a convenience method \ct{signal}, which creates and signals an exception. So, here are two equivalent ways to signal a \clsind{ZeroDivide} exception:
\needlines{2}
\begin{code}{}
	ZeroDivide new signal.
	ZeroDivide signal.    "class-side convenience method does the same as above"
\end{code}

You may wonder why it is necessary to create an instance of an exception in order to signal it, rather than having the exception class itself take on this responsibility. Creating an instance is important because it encapsulates information about the context in which the exception was signaled. We can therefore have many exception instances, each describing the context of a different exception.

When an exception is signaled, the exception handling mechanism searches in the execution stack for an exception handler associated with the class of the signaled exception. When a handler is encountered (\ie the message \ct{on:do:} is on the stack),
the implementation checks that the \ct{exceptionClass} is a superclass of the signaled exception, and then executes the \ct{handlerAction} with the exception as its sole argument. We will see shortly some of the ways in which the handler can use the exception object.

When signaling an exception, it is possible to provide information specific to the situation just encountered, as illustrated in the code below. 
For example, if the file to be opened does not exist, the name of the non-existent file can be recorded in the exception object:

\mthindex{StandardFileStream class}{oldFileNamed:}
\begin{code}{}
StandardFileStream class>>>oldFileNamed: fileName
	"Open an existing file with the given name for reading and writing. If the name has no directory part, then default directory will be assumed. If the file does not exist, an exception will be signaled. If the file exists, its prior contents may be modified or replaced, but the file will not be truncated on close."
	| fullName |
	fullName := self fullName: fileName.
	^(self isAFileNamed: fullName)
		ifTrue: [self new open: fullName forWrite: true]
		ifFalse: ["File does not exist..."
			(FileDoesNotExistException new fileName: fullName) signal]
\end{code}

The exception handler may make use of this information to recover from the abnormal situation. The argument \ct{ex} in an exception handler \ct{[:ex | ...]} will be an instance of \ct{FileDoesNotExistException} or of one of its subclasses. Here the exception is queried for the filename of the missing file by sending it the message \ct{fileName}.

\begin{code}{}
| result |
result := [(StandardFileStream oldFileNamed: 'error42.log') contentsOfEntireFile]
	on: FileDoesNotExistException
	do: [:ex | ex fileName , ' not available'].
Transcript show: result; cr
\end{code}

Every exception has a default description that is used by the development tools to report exceptional situations in a clear and comprehensible manner. To make the description available, all exception objects respond to the message \ct{description}. Moreover, the default description can be changed by sending the message \lct{messageText:  \emph{aDescription}}, or by signaling the exception using \lct{signal: \emph{aDescription}}.

Another example of signaling occurs in the \ct{doesNotUnderstand:} mechanism, a pillar of the reflective capabilities of \st. Whenever an object is sent a message that it does not understand, the VM will (eventually) send it the message \ct{doesNotUnderstand:} with an argument representing the offending message. The default implementation of \ct{doesNotUnderstand:}, defined in class \ct{Object}, simply signals a \clsind{MessageNotUnderstood} exception, causing a debugger to be opened at that point in the execution.

The \ct{doesNotUnderstand:} method illustrates the way in which exception-specific information, such as the receiver and the message that is not understood, can be stored in the exception, and thus made available to the debugger.
\mthindex{Object}{doesNotUnderstand:}

\needlines{4}
\begin{code}{}
Object>>>doesNotUnderstand: aMessage 
	 "Handle the fact that there was an attempt to send the given message to the receiver but the receiver does not understand this message (typically sent from the machine when a message is sent to the receiver and no method is defined for that selector)."
	MessageNotUnderstood new 
		message: aMessage;
		receiver: self;
		signal.
	^ aMessage sentTo: self.
\end{code}

That completes our description of how exceptions are used.  The remainder of this chapter discusses how exceptions are implemented, and adds some details that are relevant only if you define your own exceptions.

%=================================================================


\section{How breakpoints are Implemented}

As we discussed in the Debugger chapter of \emph{Pharo By Example}, 
the usual way of setting a \ind{breakpoint} within a \st{} method is to insert the message-send \ct{self halt} into the code. The method \mthind{Object}{halt}, implemented in \ct{Object}, uses exceptions to open a debugger at the location of the breakpoint; it is defined as follows:

\needlines{6}
\begin{code}{}
Object>>>halt
	"This is the typical message to use for inserting breakpoints during 
	debugging. It behaves like halt:, but does not call on halt: in order to 
	avoid putting this message on the stack. Halt is especially useful when 
	the breakpoint message is an arbitrary one."
	Halt signal
\end{code}

\index{exception!resumable}
\clsind{Halt} is a direct subclass of \clsind{Exception}. A \ct{Halt} exception is \emph{resumable}, which means that it is possible to continue execution after a \ct{Halt} is signaled. 

\ct{Halt} overrides the \mthind{Halt}{defaultAction} method, which specifies the action to perform if the exception is not caught (\ie there is no exception handler for \ct{Halt} anywhere on the execution stack):

\begin{code}{}
Halt>>>defaultAction
	"No one has handled this error, but now give them a chance to decide
	how to debug it.  If no one handles this then open debugger
	(see UnhandedError-defaultAction)"
	UnhandledError signalForException: self
\end{code}

This code signals a new exception, \ct{UnhandledError}, that conveys the idea that no handler is present. The \ct{defaultAction} of \ct {UnhandledError} is to open a debugger:

\mthindex{UnhandledError}{defaultAction}
\begin{code}{}
UnhandledError>>>defaultAction
	"The current computation is terminated. The cause of the error should be logged or reported to the user. If the program is operating in an interactive debugging environment the computation should be suspended and the debugger activated."
	^ ToolSet debugError: exception.
\end{code}

\noindent
A few messages later, the debugger opens:

\mthindex{StandardToolSet}{debug:context:label:contents:fullView:}
\begin{code}{}
StandardToolSet>>>debug: aProcess context: aContext label: aString contents: contents fullView: aBool
	^ Debugger openOn: aProcess context: aContext label: aString contents: contents fullView: aBool
\end{code}

%=================================================================
\section{How handlers are found}

\index{exception!handler}
\index{activation context}
We will now take a look at how exception handlers are found and fetched from the execution stack when an exception is signaled. 
However, before we do this, we need to understand how the control flow of a program is internally represented in the virtual machine.

At each point in the execution of a program, the execution stack of the program is represented as a list of activation contexts. Each activation context represents a method invocation and contains all the information needed for its execution, namely its receiver, its arguments, and its local variables. It also contains a reference to the context that triggered its creation, \ie the activation context associated with the method execution that sent the message that created this context. In \pharo, the class \ct{MethodContext} models this information. 
The references between activation contexts link them into a chain: this chain of activation contexts \emph{is} \st's execution stack.

Actually, there are two kinds of activation context in \pharo: \ct{methodContext}s and \ct{blockContext}s: the latter are used to represent the execution of blocks.  They have a common superclass \ct{ContextPart}.  We will ignore this detail for now.

Suppose that we attempt to open a \ct{FileStream} on a non-existent file from a \ct{doIt}.
A \ct{FileDoesNotExistException} will be signaled, and the execution stack will contain \ct{MethodContext}s for \ct{doIt}, \ct{oldFileNamed:}, and \ct{signal}, as shown in \figref{stack}.

\begin{figure}[bth]\centering
        \includegraphics[width=\linewidth]{Stack}
        \caption{A \pharo execution stack.\figlabel{stack}}
\end{figure}

Since everything is an object in \st, we would expect method contexts to be objects.
However, some \st implementations use the native \ind{C} execution stack of the \ind{virtual machine} to avoid creating objects all the time.
The current \pharo virtual machine does actually use full Smalltalk objects all the time;  for speed, it recycles old method context objects rather than creating a new one for each message-send.

\index{BlockClosure!on:do:}
When we send \lct{\emph{aBlock} on: \emph{ExceptionClass} do: \emph{actionHandler}}, we intend to associate an exception handler (\lct{\emph{actionHandler}}) with a given class of exceptions (\lct{\emph{ExceptionClass}}) for the activation context of the protected block \lct{\emph{aBlock}}.
This information is used to identify and execute \lct{\emph{actionHandler}} whenever an exception of an appropriate class is signaled; \lct{\emph{actionHandler}} can be found by traversing the stack starting from the top (the most recent message-send) and working down to the context that sent the \ct{on:do:} message.

If there is no exception handler on the stack, the message \ct{defaultAction} will be sent either by \cmind{ContextPart}{handleSignal:} or by \cmind{UndefinedObject}{handleSignal:}. The latter is associated with the bottom of the stack, and is defined as follows:

\begin{code}{}
UndefinedObject>>>handleSignal: exception
	"When no more handler (on:do:) context is left in the sender chain, this gets called.  Return from signal with default action."
	^ exception resumeUnchecked: exception defaultAction
\end{code}

The message \ct{handleSignal:} is sent by \cmind{Exception}{signal}. 

When an exception $E$ is signaled, the system identifies and fetches the corresponding exception handler by searching down the stack as follows:

\begin{enumerate}

\item Look in the current activation context for a handler, and test if that handler \ct{canHandleSignal:} $E$.

\item If no handler is found and the stack is not empty, go down the stack and return to step 1.

\item If no handler is found and the stack is empty, then send \ct{defaultAction} to $E$. The default implementation in the \ct{Error} class leads to the opening of a debugger.

\item If the handler is found, send it \ct{value:} $E$.

\end{enumerate}

\paragraph{Nested Exceptions.}
Exception handlers are outside of their own scope.  This means that if an exception is signaled from within an exception handler\,---\,what we call a nested exception\,---\,a \emph{separate} handler must be set to catch the nested exception.

Here is an example where one \ct{on:do:} message is the receiver of another one; the second will catch errors signaled by the handler of the first:
\begin{code}{@TEST | result |}
result := [[ Error signal: 'error 1' ]
	on: Exception
	do: [ Error signal: 'error 2' ]]
		on: Exception
		do: [:ex | ex description ].
result --> 'Error: error 2'
\end{code}

Without the second handler, the nested exception will not be caught, and the debugger will be invoked.

An alternative would be to specify the second handler within the first one:
\needlines{5}
\begin{code}{@TEST | result |}
result := [ Error signal: 'error 1' ]
	on: Exception
	do: [[ Error signal: 'error 2' ]
		on: Exception
		do: [:ex | ex description ]].
result --> 'Error: error 2'
\end{code}

%=================================================================
\section{Handling exceptions}

\index{exception!handling}
When an exception is signaled, the handler has several choices about how to handle it.
In particular, it may:
\begin{itemize}
\item[(i)] \emph{abandon} the execution of the protected block, by simply specifying an alternative result;
\item[(ii)] \emph{return} an alternative result for the protected block, by sending \lct{return: \emph{aValue}} to the exception object;
\item[(iii)] \emph{retry} the protected block, by sending \ct{retry}, or try a different block by sending \ct{retryUsing:};
\item[(iv)] \emph{resume} the protected block at the failure point, by sending \ct{resume} or \ct{resume:};
\item[(v)] \emph{pass} the caught exception to the enclosing handler, by sending \ct{pass}; or
\item[(vi)] \emph{resignal} a different exception, by sending \ct{resignalAs:} to the exception.
% \lr{I would call this \emph{resignal}, because this is not the same as signaling a new exception using \ct{Exception signal} from within a handler block.}
\end{itemize}

We will briefly look at the first three possibilities, and then we will take a closer look at the remaining ones.

%-----------------------------------------------------------------
\subsection{Abandon the protected block}

The first possibility is to abandon the execution of the protected block, as follows:
\needlines{7}
\begin{code}{@TEST |answer|}
answer := [ |result|
	result := 6 * 7.
	Error signal.
	result 	"This part is never evaluated"
]	on: Error
	do: [ :ex | 3 + 4 ].
answer --> 7
\end{code}

The handler takes over from the point where the error is signaled, and any code following in the original block is not evaluated.

%-----------------------------------------------------------------
\subsection{Return a value with \ct{return:}}
A block returns the value of the last statement in the block, regardless of whether the block is protected or not. However, there are some situations where the result needs to be returned by the handler block. The message \lct{return: \emph{aValue}} sent to an exception has the effect of returning \lct{\emph{aValue}} as the value of the protected block:

\begin{code}{@TEST |result|}
result := [Error signal]
	on: Error
	do: [ :ex | ex return: 3 + 4 ].
result --> 7
\end{code}

The \ind{ANSI standard} is not clear regarding the difference between using \ct{do: [:ex | 100 ]} and \ct{do: [:ex | ex return: 100]} to return a value. We suggest that you use \mthind{Exception}{return:} since it is more intention-revealing, even if these two expressions are equivalent in \pharo.

A variant of \ct{return:} is the message \ct{return}, which returns \ct{nil}. 

Note that, in any case, control will \emph{not} return to the protected block, but will be passed on up to the enclosing context.

\begin{code}{@TEST}
6 * ([Error signal] on: Error do: [ :ex | ex return: 3 + 4 ]) --> 42
\end{code}

%-----------------------------------------------------------------
\subsection{Retry a computation with \ct{retry} and \ct{retryUsing:}}

\index{exception!retrying}
Sometimes we may want to change the circumstances that led to the exception and retry the protected block. This is done by sending \mthind{Exception}{retry} or \mthind{Exception}{retryUsing:} to the exception object. It is important to be sure that the conditions that caused the exception have been changed before retrying the protected block, or else an infinite  loop will result:
\begin{code}{}
[Error signal] on: Error do: [:ex | ex retry]    "will loop endlessly"
\end{code}

Here is a better example.
The protected block is re-evaluated within a modified environment where \ct{theMeaningOfLife} is properly initialized:
\begin{code}{@TEST | result theMeaningOfLife |}
result := [ theMeaningOfLife * 7 ]    "error -- theMeaningOfLife is nil"
	on: Error
	do: [:ex | theMeaningOfLife := 6. ex retry ].
result --> 42
\end{code}

The message \ct{retryUsing: aNewBlock} enables the protected block to be replaced by \ct{aNewBlock}. This new block is executed and is protected with the same handler as the original block.

\begin{code}{@TEST | x result |}
x := 0.
result := [ x/x ]    "fails for x=0"
	on: Error
	do: [:ex |
		x := x + 1.
		ex retryUsing: [1/((x-1)*(x-2))]    "fails for x=1 and x=2"
	].
result --> (1/2)    "succeeds when x=3"
\end{code}

The following code loops endlessly:
\begin{code}{}
[1 / 0] on: ArithmeticError do: [:ex | ex retryUsing: [ 1 / 0 ]]
\end{code}
whereas this will signal an \ct{Error}: 
\begin{code}{}
[1 / 0] on: ArithmeticError do: [:ex | ex retryUsing: [ Error signal ]]
\end{code}

As another example, recall the file handling code we saw earlier, in which we printed a message to the Transcript when a file is not found. Instead, we could prompt for the file as follows:
\begin{code}{}
[(StandardFileStream oldFileNamed: 'error42.log') contentsOfEntireFile]
	on: FileDoesNotExistException
	do: [:ex | ex retryUsing: [FileList modalFileSelector contentsOfEntireFile] ]
\end{code}

%=================================================================
\section{Resuming execution}

\index{exception!resuming execution}
A method that signals an exception that \ct{isResumable} can be resumed at the place immediately following the signal. An exception handler may therefore perform some action, and then resume the execution flow. This behavior is achieved by sending \mthind{Exception}{resume:} to the exception in the handler.
The argument is the value to be used in place of the expression that signaled the exception.
In the following example we signal and catch \ct{MyResumableTestError}, which is defined in the Tests-Exceptions category:

\begin{code}{}
result := [ | log |
	log := OrderedCollection new.
	log addLast: 1.
	log addLast: MyResumableTestError signal. 
	log addLast: 2.
	log addLast: MyResumableTestError signal.
	log addLast: 3.
	log ] 
		on: MyResumableTestError 
		do: [ :ex |  ex resume: 0 ].
result --> an OrderedCollection(1 0 2 0 3)
\end{code}
Here we can clearly see that the value of \ct{MyResumableTestError signal} is the value of the argument to the \ct{resume:} message.

The message \ct{resume} is equivalent to \ct{resume: nil}.

The usefulness of resuming an exception is illustrated by the class \lct{Installer}, which implements an automatic package loading mechanism. When installing packages, warnings may be signaled. Warnings should not be considered fatal errors, so the installer should simply ignore the warning and continue installing. \lr{Maybe better use something from Pharo-Core like \ct{UndeclaredVariableWarning} or \ct{TimedOut}.}
\ab{St\'{e}ph and I failed to find a better example.  If you have one, go ahead and replace this one.}

\begin{code}{}
Installer>>>installQuietly: packageNameCollectionOrDetectBlock 
	self package: packageNameCollectionOrDetectBlock. 
	 [ self install ] on: Warning do: [ :ex | ex resume ]. 
\end{code}
%	 [ self install ] on: Warning do: [ :ex | ex resume: true ]. 
% The code actually says resume: true, but Keith Hodges confirms that this is not necessary

Another situation where resumption is useful is when you want to ask the user what to do.  For example, suppose that we were to define a class \ct{ResumableLoader} with the following method:
\begin{code}{}
ResumableLoader>>>readOptionsFrom: aStream 
	| option |
	[aStream atEnd]
		whileFalse: [option := self parseOption: aStream.
			"nil if invalid"
			option isNil
				ifTrue: [InvalidOption signal]
				ifFalse: [self addOption: option]].
	aStream close
\end{code}
\noindent
If an invalid option is encountered, we signal an \ct{InvalidOption} exception.
The context that sends \ct{readOptionsFrom:} can set up a suitable handler:

\begin{code}{}
ResumableLoader>>>readConfiguration
	| stream |
	stream := self optionStream.
	[self readOptionsFrom: stream]
		on: InvalidOption
		do: [:ex | (UIManager default confirm: 'Invalid option line. Continue loading?')
				ifTrue: [ex resume]
				ifFalse: [ex return]].
	stream close
\end{code}

\noindent
Depending on user input, the handler in \ct{readConfiguration} might \lct{return} \lct{nil}, or it might \ct{resume} the exception, causing the \ct{signal} message send in \ct{readOptionsFrom:} to return and the parsing of the options stream to continue.

Note that \ct{InvalidOption} must be resumable; it suffices to define it as a subclass of \ct{Exception}.

%=================================================================
\subsection{Example: Deprecation}

\index{deprecation (pattern)}
\emph{Deprecation} offers a case study of a mechanism built using resumable exceptions.
Deprecation is a software re-engineering pattern that allows us to mark a method as being ``deprecated'', meaning that it may disappear in a future release and should not be used by new code.
In \pharo, a method can be marked as deprecated as follows:

\mthindex{Object}{deprecated:}
\begin{code}{}
Utilities class>>>convertCRtoLF: fileName
	"Convert the given file to LF line endings. Put the result in a file with the extention '.lf'"

	self deprecated: 'Use ''FileStream convertCRtoLF: fileName'' instead.' 
		on: '10 July 2009' in: #Pharo1.0 .
	FileStream convertCRtoLF: fileName
\end{code}

When the message \ct{convertCRtoLF:} is sent, if the \ct{raiseDeprecationWarnings} preference is  \ct{true}, then a pop-up window is displayed with a notification and the programmer may resume the application execution; this is shown in \figref{deprecation}.


\begin{figure}[ht]\centering
        \includegraphics[width=\linewidth]{Deprecation}
        \caption{Sending a deprecated message.\figlabel{deprecation}}
\end{figure}

Deprecation is implemented in \pharo in just a few steps.
First, we define \clsind{Deprecation} as a subclass of \clsind{Warning}.
It should have some instance variables to contain information about the deprecation: in 
\pharo{} these are \ct{methodReference}, \ct{explanationString}, \ct{deprecationDate} and \ct{versionString}; we therefore need to define an instance-side initialization method for these variables, and a class-side instance creation method that sends the corresponding message.

When we define a new exception class, we should consider overriding \ct{isResumable}, \ct{description}, and \ct{defaultAction}.
In this case the inherited implementations of the first two methods are fine:

\begin{itemize}
\item \ct{isResumable} is inherited from \clsind{Exception}, and answers \ct{true};
\item \ct{description} is inherited from \ct{Exception}, and answers an adequate textual description.
\end{itemize}

However, it is necessary to override the implementation of \lct{defaultAction}, because we want that to depend on some preferences.  Here is \pharo's implementation:
\begin{code}{}
Deprecation>>>defaultAction
	Log ifNotNil: [:log| log add: self].
	Preferences showDeprecationWarnings ifTrue:
		[Transcript nextPutAll: explanationString; cr; flush].
	Preferences raiseDeprecatedWarnings ifTrue:
		[super defaultAction]
\end{code}

The first preference simply causes a warning message to be written on the \ct{Transcript}.  The second preference asks for an exception to be signaled, which is accomplished by \super-sending \ct{defaultAction}.

We also need to implement some convenience methods in \ct{Object}, like this one:
\needlines{8}
\begin{code}{}
Object>>>deprecated: anExplanationString on: date in: version
	(Deprecation
		method: thisContext sender method
		explanation: anExplanationString
		on: date
		in: version) signal
\end{code}

%=================================================================
\section{Passing exceptions on}

To illustrate the remaining possibilities for handling exceptions, we will look at how to implement a generalization of the \ct{perform:} method.
If we send \lct{perform: \emph{aSymbol}} to an object, this will cause the message named \lct{\emph{aSymbol}} to be sent  to that object:
\begin{code}{@TEST}
5 perform: #factorial --> 120    "same as: 5 factorial"
\end{code}

Several variants of this method exist. For example:
\begin{code}{@TEST}
1 perform: #+ withArguments: #(2) --> 3    "same as: 1 + 2"
\end{code}
These \ct{perform:}-like methods are very useful for accessing an interface dynamically, since the messages to be sent can be determined at run-time.

One message that is missing is one that will send a cascade of unary messages to a given receiver. A simple and naive implementation is:
\begin{code}{}
Object>>>performAll: selectorCollection
	selectorCollection do: [:each | self perform: each]    "aborts on first error"
\end{code}

This method could be used as follows:
\begin{code}{}
Morph new performAll: #( #activate #beTransparent #beUnsticky)
\end{code}

However, there is a complication. There might be a selector in the collection that the object does not understand (such as \ct{#activate}). We would like to ignore such selectors and continue sending the remaining messages. The following implementation seems to be reasonable:

\needlines{4}
\begin{code}{}
Object>>>performAll: selectorCollection 
	selectorCollection do: [:each |
		[self perform: each]
			on: MessageNotUnderstood
			do: [:ex | ex return]]    "also ignores internal errors"
\end{code}

On closer examination we notice another problem. This handler will not only catch and ignore messages not understood by the original receiver, but also any messages sent but not understood in methods for messages that \emph{are} understood! This will hide programming errors in those methods, which is not our intent.
To fix this, we need our handler to analyze the exception to see if it was indeed caused by the attempt to perform the current selector.
Here is the correct implementation.
\begin{method}[objectPerformAll]{Object>>performAll:}
Object>>>performAll: selectorCollection 
	selectorCollection do: [:each | 
		[self perform: each] 
			on: MessageNotUnderstood 
			do: [:ex | (ex receiver == self and: [ex message selector == each]) 
				ifTrue: [ex return] 
				ifFalse: [ex pass]]]    "pass internal errors on"
\end{method}

This has the effect of passing on \clsind{MessageNotUnderstood} errors to the surrounding context when they are not part of the list of messages we are performing. The \ct{pass} message will pass the exception on to the next applicable handler in the execution stack.

If there is no next handler on the stack, the \ct{defaultAction} message is sent to the exception instance. The \ct{pass} action does not modify the sender chain in any way\,---\,but the handler that control is passed to may do so. Like the other messages discussed in this section, \ct{pass} is special\,---\,it never returns to the sender.

The goal of this section has been to demonstrate the power of exceptions.
It should be clear that while you can do almost anything with exceptions, the code
that results is not always easy to understand.   
There is often a simpler way to get he same effect without exceptions; see \mthref{simplerObjectPerfromAll} on page \pageref{mth:simplerObjectPerfromAll} for a better way to implement \ct{performAll:}.

%=================================================================
\section{Resending exceptions}

\index{exception!resending}
Now suppose that in our \ct{performAll:} example we no longer want to ignore selectors not understood by the receiver, but instead we want to consider an occurrence of such a selector as an error. However, we want it to be signaled as an application-specific exception, let's say \ct{InvalidAction}, rather than the generic \ct{MessageNotUnderstood}. In other words, we want the ability to ``resignal'' a signaled exception as a different one.

It might seem that the solution would simply be to signal the new exception in the handler block. The handler block in our implementation of \ct{performAll:} would be:

\mthindex{Exception}{pass}
\begin{code}{}
[:ex | (ex receiver == self and: [ex message selector == each])
	ifTrue: [InvalidAction signal]    "signals from the wrong context"
	ifFalse: [ex pass]]
\end{code}

A closer look reveals a subtle problem with this solution, however. Our original intent was to replace the occurrence of \ct{MessageNotUnderstood} with \ct{InvalidAction}. This replacement should have the same effect as if \lct{InvalidAction} were signaled at the same place in the program as the original \ct{MessageNotUnderstood} exception. Our solution signals \ct{InvalidAction} in a different location. The difference in locations may well lead to a difference in the applicable handlers.

To solve this problem, resignaling an exception is a special action handled by the system. For this purpose, the system provides the message \lct{resignalAs:}. The correct implementation of a handler block in our \ct{performAll:} example would be:

\begin{code}{}
 [:ex |  (ex receiver == self and: [ex message selector == each])
	ifTrue: [ex resignalAs: InvalidAction]    "resignals from original context"
	ifFalse: [ex pass]]
\end{code}

%=================================================================
\section{Comparing \lct{outer} with \lct{pass}}

The method \mthind{Exception}{outer} is very similar to \ct{pass}. Sending \ct{outer} to an exception also evaluates the enclosing handler action. The only difference is that if the outer handler resumes the exception, then control will be returned to the point where \ct{outer} was sent, not the original point where the exception was signaled:

\begin{code}{@TEST | passResume |}
passResume := [[ Warning signal . 1 ]    "resume to here"
	on: Warning
	do: [ :ex | ex pass . 2 ]]
		on: Warning
		do: [ :ex | ex resume ].
passResume --> 1    "resumes to original signal point"
\end{code}

\needlines{6}
\begin{code}{@TEST | outerResume |}
outerResume := [[ Warning signal . 1 ]
	on: Warning
	do: [ :ex | ex outer . 2 ]]    "resume to here"
		on: Warning
		do: [ :ex | ex resume ].
outerResume --> 2    "resumes to where outer was sent"
\end{code}

%=================================================================
\section{Catching sets of exceptions}

So far we have always used \ct{on:do:} to catch just a single class of exception. The handler will only be invoked if the exception signaled is a sub-instance of the specified exception class.
However, we can imagine situations where we might like to catch multiple classes of exceptions. This is easy to do:

\begin{code}{@TEST | result |}
result := [ Warning signal . 1/0 ]
	on: Warning, ZeroDivide
	do: [:ex | ex resume: 1 ].
result --> 1
\end{code}

If you are wondering how this works, just have a look at the implementation of \ct{Exception class>>>,}

\begin{code}{}
Exception class>>>, anotherException
	"Create an exception set."

	^ExceptionSet new
		add: self;
		add: anotherException;
		yourself
\end{code}

The rest of the magic occurs in the class \clsind{ExceptionSet}, which has a surprisingly trivial implementation.

\begin{code}{}
Object subclass: #ExceptionSet
	instanceVariableNames: 'exceptions'
	classVariableNames: ''
	poolDictionaries: ''
	category: 'Exceptions-Kernel'

ExceptionSet>>>initialize
	super initialize.
	exceptions := OrderedCollection new

ExceptionSet>>>, anException
	self add: anException.
	^self

ExceptionSet>>>add: anException
	exceptions add: anException

ExceptionSet>>>handles: anException
	exceptions do: [:ex | (ex handles: anException) ifTrue: [^true]].
	^false
\end{code}

\noindent
\ab{Is there some reason that \ct{ExceptionSet>>>handles:} isn't implemented using \ct{anySatisfy:}?}

%=========================================================
\section{How exceptions are implemented}

%\on{As an explanation, this part really does not work. I would prefer to leave it out.}
%\sd{no this is really important I did a new pass on it}

Let's have a look at how exceptions are implemented at the Virtual Machine level.

%\on{need new intro}

%\ugh{Up to now, we have presented the use of exceptions in \st without really saying a word about their implementation.
%at the execution engine level (Virtual Machine). 
%Normally you do not need to know how handlers are really looked up at execution time to use exceptions.
%Therefore in the first reading you can skip this section. Now if you are curious and really want to know how this is implemented at the Virtual Machine level, this section is for you. 
%This section will uncover the internal of the Pharo exception mechanism.
%The mechanism is quite simple, making it worth to know how it operates.
%This section is a wonderful example on how Pharo can reveal information just by browsing its source code.}


\paragraph{Storing Handlers.}
First we need to understand how the exception class and its associated handler  are stored and how this information is found at run-time. 
Let's look at the definition of the central method \mthind{BlockClosure}{on:do:} defined on the class \ct{BlockClosure}. % which is the class representing the block-closure. 

\mthindex{BlockClosure}{on:do:}
\needlines{6}
\begin{code}{}
BlockClosure>>>on: exception do: handlerAction 
	"Evaluate the receiver in the scope of an exception handler." 
	| handlerActive | 
	<primitive: 199> 
	handlerActive := true. 
	^self value 
\end{code}

This code tells us two things: First, this method is implemented as a primitive, which means that  a primitive operation of the virtual machine is executed when this method is invoked.  VM primitives don't normally return: successful execution of a primitive terminates the method that contains the \lct{<primitive: \emph{n}\,>} instruction, answering the result of the primitive.
So, the \st code that follows the primitive serves two purposes: it documents what the primitive does, and is available to be executed if the primitive should fail. 
Here we see that \ct{on:do:} simply sets the temporary variable \ct{handlerActive} to true, and then evaluates the receiver (which is, of course, a block).  

This is surprisingly simple, but somewhat puzzling.  Where are the arguments of the \ct{on:do:} method stored?  Let's look at the definition of the class \ct{MethodContext}, whose instances make up the execution stack: 

\begin{code}{}
ContextPart variableSubclass: #MethodContext
	instanceVariableNames: 'method closureOrNil receiver'
	classVariableNames: ''
	poolDictionaries: ''
	category: 'Kernel-Methods'
\end{code}

There is no instance variable here to store the exception class or the handler, nor is there any place  in the superclass to store them. 
However, note that \ct{MethodContext} is defined as a \ct{variableSubclass}.
This means that in addition to the named instance variables, objects of this class have some numbered slots.  In fact, every \ct{MethodContext} has a numbered slot for each  argument of the method whose invocation it represents.  There are also additional numbered slots for the temporary variables of the method.

To verify this, you can evaluate the following piece of code:

\begin{code}{}
| exception handler | 
[exception := thisContext sender at: 1. 
 handler := thisContext sender at: 2. 
 1 / 0] 
on: Error 
do: [:ex| ]. 
{ exception . handler } explore
\end{code}

The last line explores a 2-element array that contains the exception class and the exception handler. 

\paragraph{Finding Handlers.}
Now that we know where the information is stored, let's have a look at how it is found at runtime. 

We might think that the primitive 199 is complex to write. 
But it too is trivial, because primitive 199 \emph{always} fails!
Because the primitive always fails, the \st{} body of \ct{on:do:} is always executed.
However, the presence of the \ct{<primitive: 199>} bytecode marks the executing context in a unique way. 

The source code of the primitive is found in \ct{Interpreter>>>primitiveMarkHandlerMethod} in the \pkgind{VMMaker} \sqsrc project: 

\begin{code}{}
primitiveMarkHandlerMethod
     "Primitive. Mark the method for exception handling. The primitive must fail after
     marking the context so that the regular code is run."
     
     self inline: false.
    ^self primitiveFail
\end{code}
\hyphenation{Method-Context}
So now we know that when the method \ct{on:do:} is executed, the \ct{MethodContext} that makes up the stack frame is tagged and the handler and exception class 
are stored there. 

Now, if an exception is signaled further up the stack, the method \ct{signal} can search the stack to find the appropriate handler:

\mthindex{Exception}{signal}
\needlines{5}
\begin{code}{}
Exception>>>signal
	"Ask ContextHandlers in the sender chain to handle this signal.
	The default is to execute and return my defaultAction."

	signalContext := thisContext contextTag.
	^ thisContext nextHandlerContext handleSignal: self
\end{code}

\mthindex{Exception}{nextHandlerContext}
\begin{code}{}
ContextPart>>>nextHandlerContext

	^ self sender findNextHandlerContextStarting
\end{code}

The method \ct{findNextHandlerContextStarting} is implemented as a primitive (number 197); its body describes what it does. It looks to see
if the stack frame is a context created by the execution of the method \ct{on:do:} (it just looks to see if the primitive number is 199). If this is the case it answers with that context. 

\mthindex{MethodContext}{findNextHandlerContextStarting}
\begin{code}{}
ContextPart>>>findNextHandlerContextStarting 
	"Return the next handler marked context, returning nil if there 
	is none. Search starts with self and proceeds up to nil." 
	| ctx |	
	<primitive: 197> 
	ctx := self. 
	[  ctx isHandlerContext ifTrue: [^ctx]. 
	   (ctx := ctx sender) == nil ] whileFalse. 
	^nil 
\end{code}

\begin{code}{}
MethodContext>>>isHandlerContext 
	"is this context for method that is marked?" 
	^method primitive = 199 
\end{code}

Since the method context supplied by \ct{findNextHandlerContextStarting} contains all the exception-handling information, it can be examined to see if the exception class is suitable for handling the current exception.
If so, the associated handler can be executed; if not, the look-up can continue further. 
This is all implemented in the \ct{handleSignal:} method.

\begin{code}{}
ContextPart>>>handleSignal: exception
	"Sent to handler (on:do:) contexts only.  If my exception class (first arg) handles exception then execute my handle block (second arg), otherwise forward this message to the next handler context.  If none left, execute exception's defaultAction (see nil>>handleSignal:)."

	| val |
	(((self tempAt: 1) handles: exception) and: [self tempAt: 3]) ifFalse: [
		^ self nextHandlerContext handleSignal: exception].

	exception privHandlerContext: self contextTag.
	self tempAt: 3 put: false.  "disable self while executing handle block"
	val := [(self tempAt: 2) valueWithPossibleArgs: {exception}]
		ensure: [self tempAt: 3 put: true].
	self return: val.  "return from self if not otherwise directed in handle block"
\end{code}

Notice how this method uses \ct{tempAt: 1} to access the exception class, and ask if it handles the exception.   What about  \ct{tempAt: 3}?  That is the temporary variable \ct{handlerActive} of the \ct{on:do:} method.   Checking that \ct{handlerActive} is \ct{true} and then setting it to \ct{false} ensures that a handler will not be asked to handle an exception that it signals itself. 
The \ct{return:} message sent as the final action of \ct{handleSignal} is responsible for ``unwinding'' the execution stack by removing the stack frames above \self{}.

The full story is only slightly more complicated because there are actually two classes of objects that make up the stack, \ct{MethodContext}s, which we have already discussed, and \ct{BlockContext}s, which represent the execution of blocks.  \ct{ContextPart} is their common superclass.

So, to summarize, the \ct{signal} method, with minimal assistance from the virtual machine, finds the context that correspond to an \ct{on:do:} message with an appropriate exception class.   Because the execution stack is made up of Context objects that may be manipulated just like any other object, the stack can be shortened at any time.  This is a superb example of flexibility of \st.

%=========================================================
\section{Other kinds of Exception}

The class \ct{Exception} in \pharo{} has ten direct subclasses, as shown in \figref{wholeHierarchy}.
The first thing that we notice from this figure is that the Exception hierarchy is a bit of a mess; you can expect to see some of the details change as \pharo{} is improved.

\begin{figure}[ht]\centering
        \includegraphics[width=.95\linewidth]{ExceptionSubclasses}
        \caption{The whole \pharo exception hierarchy.\figlabel{wholeHierarchy}}
\end{figure}

The second thing that we notice is that there are two large sub-hierarchies: \ct{Error} and \ct{Notification}. 
Errors tell us that the program has fallen into some kind of abnormal situation. 
In contrast, Notifications tell us that an event has occurred, but without the assumption that it is abnormal. 
So, if a \ct{Notification} is not handled, the program will continue to execute. 
An important subclass of \ct{Notification} is \ct{Warning};  warnings are used to notify other parts of the system, or the user, of abnormal but non-lethal behavior.
%
%Graphical user interfaces make great use of notifications. In \pharo, \ct{ProgressNotification}, another subclass of \ct{Notification}, is used to move the progress bar forward when a long-running task is being executed. \ab{I wanted to put in a reference to the methods that do this, but couldn't find them.  Can someone else put this in, please?} 
%\ab{This appears not to be true, so I commented out the whole paragraph}

The property of being resumable is largely orthogonal to the location of an exception in the hierarchy.   In general, \ct{Error}s are not resumable, but 10 of its subclasses \emph{are} resumable.  For example, \ct{MessageNotUnderstood} is a subclass of \ct{Error}, but it is resumable.  \ct{TestFailure}s are not resumable, but, as you would expect, \ct{ResumableTestFailure}s are. 

Resumability is controlled by the private \ct{Exception} method \mthind{Exception}{isResumable}. 
For example:
\begin{code}{@TEST}
Exception new isResumable --> true
Error new isResumable --> false
Notification new isResumable --> true
Halt new isResumable --> true
MessageNotUnderstood new isResumable --> true
\end{code}

As it turns out, roughly 2/3 of all exceptions are resumable:
\begin{code}{}
Exception allSubclasses size --> 103
(Exception allSubclasses select: [:each | each new isResumable]) size --> 66
\end{code}
If you declare a new subclass of exceptions, you should look in its protocol for the \ct{isResumable} method, and override it as appropriate to the semantics of your exception.

In some situations, it will never makes sense to resume an exception.
In such a case you should signal a non-resumable subclass\,---\,either an existing one or one of your own creation.
In other situations, it will always be OK to resume an exception, without the handler having to do anything.
In fact, this gives us another way of characterizing a notification:
a \ct{Notification} is a resumable \ct{Exception} that can be safely resumed without first modifying the state of the system.
More often, it will be safe to resume an exception only if the state of the system is first modified in some way.
So, if you signal a resumable exception, you should be very clear about what you expect an exception handler to do before it resumes the exception.

\section{When not to use Exceptions}
Just because \pharo{} has exception handling, you should not conclude that it is always appropriate to use it.  
Recall that in the introduction to this chapter, we said that exception handling is for \emph{exceptional} situations.  
So, the first rule for using exceptions is \emph{not} to use them for situations that \emph{can reasonably be expected to occur} in a normal execution. 

Of course, if you are writing a library, what is normal depends on the context in which your library is used.
To make this concrete, let's look at \ct{Dictionary} as an example:
\lct{\emph{aDictionary} at: \emph{aKey}} will signal an \ct{Error} if \lct{\emph{aKey}} is not present.
But you should not write a handler for this error!
If the logic of your application is such that there is some possibility that the key will not be in the dictionary, then you should instead use \lct{at: \emph{aKey} ifAbsent: [\emph{remedial action}]}.
In fact, \ct{Dictionary>>>at:} is implemented using \ct{Dictionary>>>at:ifAbsent:}.
\lct{\emph{aCollection} detect: \emph{aPredicateBlock}} is similar: if there is any possibility that the predicate might not be satisfied, you should use \lct{\emph{aCollection} detect: \emph{aPredicateBlock} ifNone: [\emph{remedial action}]}. 

When you write methods that signal exceptions, you should consider whether you should also provide an alternative method that takes a remedial block as an additional argument, and evaluates it if the normal action cannot be completed.  
Although this technique can be used in any programming language that support closures, because \st{} uses closures for \emph{all} its control structures, it is a particularly natural one to use in \st{}.

Another way of avoiding exception handling is to test the precondition of the exception before sending the message that may signal it.  For example, in \mthref{objectPerformAll}, we sent a message to an object using \ct{perform:}, and handled the \ct{MessageNotUnderstood} error that might ensue.  A much simpler alternative is to check to see if the message is understood before executing the \ct{perform:}

\needlines{5}
\begin{method}[simplerObjectPerfromAll]{Object>>performAll: revisited}
performAll: selectorCollection
	selectorCollection
		do: [:each | (self respondsTo: each)
				ifTrue: [self perform: each]]
\end{method}

The primary objection to \mthref{simplerObjectPerfromAll} is efficiency.  The implementation of \ct{respondsTo: s} has to lookup \ct{s} in the target's method dictionary  to find out if \ct{s} will be understood.  If the answer is yes, then \ct{perform:} will look it up again.  Moreover, the first lookup is implemented in \st, not in the virtual machine.  If this code is in a performance-critical loop, this might be an issue.  However, if the collection of messages comes from a user interaction, the speed of \ct{performAll:} will not be a problem.

%=================================================================
\section{Chapter Summary}

In this chapter we saw how to use exceptions to signal and handle abnormal situations arising in our code.

\begin{itemize}
\item Don't use exceptions as a control-flow mechanism.  Reserve them for notifications and for \emph{abnormal} situations.  Consider providing methods that take blocks as arguments as an alternative to signaling exceptions.

\item Use \lct{\emph{protectedBlock} ensure: \emph{actionBlock}} to ensure that \lct{\emph{actionBlock}} will be performed even if \lct{\emph{protectedBlock}} terminates abnormally.

\item Use \lct{\emph{protectedBlock} ifCurtailed: \emph{actionBlock}} to ensure that \lct{\emph{actionBlock}} will be performed \emph{only} if \lct{\emph{protectedBlock}} terminates abnormally.

\item Exceptions are objects. Exception classes form a hierarchy with the class \ct{Exception} at the root of the hierarchy.

\item Use \lct{\emph{protectedBlock} on: \emph{ExceptionClass} do: \emph{handlerBlock}} to catch exceptions that are instances of \lct{\emph{ExceptionClass}} (or any of its subclasses). The \lct{\emph{handlerBlock}} should take an exception instance as its sole argument.

\item Exceptions are signaled by sending one of the messages \lct{signal} or \lct{signal:}. \ct{signal:} takes a descriptive string as its argument. The description of an exception can be obtained by sending it the message \ct{description}.

\item You can set a breakpoint in your code by inserting the message-send \ct{self halt}. This signals a resumable \ct{Halt} exception, which, by default, will open a debugger at the point where the breakpoint occurs.

\item When an exception is signaled, the runtime system will search up the execution stack, looking for a handler for that specific class of exception. If none is found, the \ct{defaultAction} for that exception will be performed (\ie in most cases the debugger will be opened).

\item An exception handler may terminate the protected block by sending \ct{return:} to the signaled exception; the value of the protected block will be the argument supplied to \ct{return:}. 

\item An exception handler may retry a protected block by sending \ct{retry} to the signaled exception. The handler remains in effect.

\item An exception handler may specify a new block to try by sending \ct{retryUsing:} to the signaled exception, with the new block as its argument. Here, too, the handler remains in effect.

\item Notifications are subclass of Exception with the property that they can be safely resumed without the handler having to take any specific action.

\end{itemize}

\paragraph{Acknowledgments.}  We gratefully acknowledge Vassili Bykov for the raw material he provided. We also thank Paolo Bonzini, the main developer of GNU \st, for the \st implementations of \ct{ensure:} and \ct{ifCurtailed:}.
\index{Bonzini, Paolo}
\index{GNU \st}

%=========================================================
\ifx\wholebook\relax\else
   \bibliographystyle{jurabib}
   \nobibliography{scg}
   \end{document}
\fi
%=========================================================


% % $Author$
% $Date$
% $Revision$

% HISTORY:
% 2007-05-12 - Stef started chapter

%=================================================================
\ifx\wholebook\relax\else
% --------------------------------------------
% Lulu:
	\documentclass[a4paper,10pt,twoside]{book}
	\usepackage[
		papersize={6in,9in},
		hmargin={.75in,.75in},
		vmargin={.75in,1in},
		ignoreheadfoot
	]{geometry}
	\input{../common.tex}
	\setboolean{lulu}{true}
% --------------------------------------------
% A4:
%	\documentclass[a4paper,11pt,twoside]{book}
%	\input{../common.tex}
%	\usepackage{a4wide}
% --------------------------------------------
    \graphicspath{{figures/} {../figures/}}
	\begin{document}
\fi
%=================================================================
%\renewcommand{\nnbb}[2]{} % Disable editorial comments
\sloppy
%=================================================================
\chapter{Concurrency}\label{cha:basic}

Squeak as any Smalltalk is a sequential language in the sense that at one point in time there is only 
one computation is carried on. However, Smalltalk has the ability to run programs concurrently by interleaving their 
executions. The idea behind Smalltalk was to propose a complete OS and as such a Smalltalk run-time offers the possibility to execute different threads that are scheduled by the Smalltalk thread scheduler. 

Smalltalk's concurrency is \emph{collaborative} and \emph{preemptive}. It is preemptive in the sense that a process with higher priority can interrupt the current thread running. It is collaborative in the sense that thread of same priority should collaborate, the current thread should explicit release the control to give a chance to the other processes of the same priority can get executed by the scheduler. 

In this chapter we present how processes are created and their lifetime. We will show how the process scheduler manages the system. We will present one basic abstraction proposed by Squeak:  Semaphore and the critical section. 

The subsection chapter will present the other abstractions offered by Squeak: Monitor, Delay...

Note in the current version of Squeak, the Transcript is not thread-safe. 

Need to have a thread safe transcript.

%=================================================================
\section{Processes}


\begin{figure}
\includegraphics[width=10cm]{ProcessState}
\end{figure}


In Smalltalk, threads are called process, instance of the class \clsindmain{Process}.


% For all intents and purposes, \clsindmain{Object} is the root of the inheritance hierarchy. Actually, in Squeak the true root of the hierarchy is \clsind{ProtoObject}, which is used to define minimal entities that masquerade as objects, but we can ignore this point for the time being.
% % (more on this later in the chapter on reflection).
% 
% \ct{Object} can be found in the \scatind{Kernel-Objects} category. Astonishingly, there are some 400 methods to be found here (including extensions).  In other words, every class that you define will automatically provide these 400 methods, whether you know what they do or not. Note that some of the methods should be removed and new versions of Squeak may remove some of the superfluous methods. 
% 
% \sd{I do not like to quote something that can change and that people can find simply in the image but let us keep it for now.}
% The class comment for the \ct{Object} states:
% 
% \ct{Object>>>printOn:} is very likely one of the methods that you will most frequently override. This method takes as its argument a \clsind{Stream} on which a \clsind{String} representation of the object will be written. The default implementation simply write the class name preceded by ``\ct{a}'' or ``\ct{an}''. \ct{Object>>>printString} returns the \ct{String} that is written:
% 
% For example, the class \clsind{Browser} does not redefine the method \ct{printOn:} and sending the message printString to an instance executes the methods defined in \ct{Object}. 
% \begin{code}{@TEST}
% Browser new printString --> 'a Browser'
% \end{code}
% 
% The class \ct{TTCFont} shows an example of \mthind{TTCFont}{printOn:}
% specialization. It prints the name of the class followed by the family
% name, the size and the subfamily name of the font as shown by the code
% below which prints an instance of the class.
% 
% \begin{method}[zork]{printOn: redefinition.}
% TTCFont>>>printOn: aStream
%         aStream nextPutAll: 'TTCFont(';
% 		nextPutAll: self familyName; space;
% 		print: self pointSize; space;
% 		nextPutAll: self subfamilyName;
% 		nextPut: $)
% \end{method}\ignoredollar$
% 
% % \begin{code}{@TEST}
% \begin{code}{} % ON: THIS IS FRAGILE -- breaks in Pharo
% TTCFont allInstances anyOne printString --> 'TTCFont(BitstreamVeraSans 6 Bold)'
% \end{code}
% 
% \begin{method}{Hash must be reimplemented for complex numbers}
% Complex>>>hash
%     "Hash is reimplemented because = is implemented."
%     ^ real hash bitXor: imaginary hash.
% \end{method}

\section{Protocol}


Squeak comme tous les Smalltalks offre la possibilité d'exécuter différents programmes en parallele. 
La concurrence en Smalltalk est préemptive et collaborative. Préemptive car les processus de priorité superieure 
prennent précédence sur ceux de priorité inferieure. Collaborative car pourqu'un processus ait une chance de s'executer alors qu'un processus de meme priorité est deja entrain de s'executer, celui-ci doit explicitement relacher le contrôle. Dans cet article nous montrons les élements de base et dans un prochain article nous montrerons les abstractions essentielles telleque Monitor, Delay.

Les processus
Pour les besoins de la programmation concurrente, Smalltalk offre une panoplie de classes qui permettent la manipulation de processus légers (threads) instances de la classe Process [sharp97]. Un processus est caractérisé par un bloc (instance de BlockContext) qui décrit les opérations qu'il doit exécuter et une priorité d'exécution.

La gestion des processus est à la charge de l'instance unique de la class ProcessorScheduler. Cet objet référencé par la variable globale Processor - donne accès à l'ensemble des processus et permet ainsi de les retrouver et de les manipuler. En particulier, il permet de définir le processus à activer en fonction des priorités et de l'ordre de réveil.

Priorités et préemption
Comme les processus Smalltalk sont préemptifs, ceux à la priorité la plus importante (i.e. la plus proche de 100) sont ceux qui ont le plus de chance d'être exécutés. Le processus ayant la plus grande priorité est exécuté jusqu'à sa terminaison ou sa mise en sommeil.

Tous les autres processus sont mis en attente. Dans le cas où plusieurs processus
ont la même priorité, alors l'ordre de réveil est utilisé. Le premier à être réveillé sera exécuté jusqu'à sa terminaison ou mise en sommeil. Les autres processus de même priorité ou de priorité inférieure seront mis en attente.
A titre d'exemple, supposons que nous disposions d'un objet compte bancaire référencé par la variable \verb!monCompte! utilisée dans les trois processus correspondant aux trois lignes suivantes~:

@@change l'exemple@@

[monCompte depot: 50] forkAt: 30.
[monCompte solde] forkAt: 40.
[monCompte retrait: 200] forkAt: 40.


Dans chacune de ces trois lignes de code, un bloc donne les traitements réalisés par le processus. Le processus est ensuite créé et réveillé en envoyant le message forkAt: au bloc. L'argument de ce message correspond à la
priorité du processus.

Une fois la première ligne évaluée, le processus de dépôt est lancé. Cependant, il est rapidement mis en attente car la ligne suivante lance le processus de consultation de solde à une priorité plus élevée. La troisième ligne lance un processus de retrait à la même priorité que pour la consultation. De ce fait, il reste en attente. Lorsque la
consultation est complètement terminée, le processus de retrait est activé car il a la priorité la plus forte. Ce n'est que lorsque tous les processus de priorité supérieure à 30 sont terminés que notre processus de dépôt est activé.

En plus de ces règles, le programmeur Smalltalk peut créer des processus de même niveau de priorité qui ne se bloquent pas les uns les autres. Pour ce faire, chacun des processus doit régulièrement demander au gestionnaire de processus Processor d'activer les éventuels autres processus de même priorité mis en attente.
Cette demande se fait à l'aide du message yield (Processor yield).

Mise en sommeil
Une autre solution consiste en une mise en sommeil. Pour ce faire, Smalltalk intègre la classe Delay qui permet de suspendre le processus actif pendant un laps de temps. Dans l'exemple suivant, le processus exécute une boucle de consultations de solde espacées de 30 secondes.

[ | attente |
    attente := Delay forSeconds: 30.
    100 timesRepeat: [
             attente wait.  "Mise en sommeil"
             monCompte solde.]
] forkAt: 40.


Nous faisons l'hypothèse que monCompte est une variable qui référence un compte bancaire. Le processus commence par creer une instance de Delay qui correspond à l'intervalle d'attente entre deux consultations de solde. Il réalise ensuite 100 fois (message timesRepeat: envoyé à l'entier 100) une consultation précédée par une attente.

\section{Synchronisation}
Afin de gérer les accès concurrents aux ressources, Smalltalk dispose d'objets verrous. Ce sont des instances de la classe Semaphore qui permettent la synchronisation des sections critiques du code. Pour ce faire, tous les processus qui partagent une même ressource doivent envoyer au sémaphore le message wait avant l'accès et le message signal après l'accès. En effet, l'exécution du processus qui envoie un wait est suspendue. Lorsque le sémaphore reçoit un signal, il réveille le plus ancien processus (i.e. le premier à avoir envoyé wait).

CompteBancaireCourant subclass: #CompteSynchronise
   instanceVariableNames: `verrou'!\\
   classVariableNames: `'!\\
04&\tab \verb!poolDictionaraies: `'!\\
05&\tab \verb!category: `Comptes Bancaires'!\\
\hline
06&{\bf initialize}\\
07&\tab   \verb!super initialize.!\\
08&\tab   \verb!verrou := Semaphore forMutualExclusion!\\
\hline
09&{\bf depot\verb!:! montant}\\
10&\tab   \verb!verrou critical: [super depot: montant]!\\
\hline
11&{\bf retrait\verb!:! montant}\\
12&\tab    \verb!verrou critical: [super retrait: montant]!\\
\hline
13&{\bf solde}\\
14&\tab   \verb!verrou critical: [^super solde]!\\
\hline
\end{tabular}
}
    \caption{Définition de la classe {\tt CompteSynchronise}}
     \label{fig:compteSynchronise}
   \end{center}
\end{figure}

Par exemple, si nous souhaitons disposer de comptes courants dont les dépôts, retraits et consultation de solde sont synchronisés, nous définissons la sous-classe de CompteBancaireCourant comme suit~:

 La sous-classe CompteSynchronise ajoute une variable d'instance
verrou! qui représente un sémaphore (ligne~2, figure~\ref{fig:compteSynchronise})

L'initialisation de la variable d'instance verrou est opérée dans la méthode d'initialisation initialize (lignes~6 à~8, figure~\ref{fig:compteSynchronise}). Cette méthode crée un sémaphore à l'aide le message forMutualExclusion qui (par opposition au new) retourne un sémaphore qui ne suspend pas le premier processus qui envoie un message wait. Le but étant de ne suspendre les processus qu'en cas d'accès concurrents, un tel sémaphore est un sémaphore normal qui a reçu un signal juste après sa création.

Les méthodes depot: (lignes~9 et~10, figure~), retrait: (lignes~11 et~12, figure ) et solde (lignes~13 et~14, figure) sont redéfinies pour synchroniser les accès.
A noter qu'au lieu d'utiliser un couple wait/signal, nous utilisons le message critical: qui réalise le même traitement. L'avantage de ce dernier est d'envoyer toujours le signal et donc la libération de la ressource même en cas d'exception.


Afin de simplifier les échanges de données (des objets) entre processus, Smalltalk introduit des files synchronisées, instances de la classe SharedQueue. Plusieurs processus producteurs peuvent ainsi injecter des données dans la file, pendant que des processus consommateurs les retirent. Le recours aux sémaphores est inutile, puisque la file assure la synchronisation. En particulier, elle suspend les processus consommateurs lorsqu'elle est vide et les réveille quand des producteurs injectent des données.

Ce comportement simple peut bien sûr être étendu. Par exemple, il est possible d'enrichir le comportement FIFO ({\em first in first out}) avec des priorités \cite{sharp97}.
Chaque objet injecté dans la file est marqué d'une priorité.
Les objets les plus prioritaires sont les premiers à être extraits de la file.


Liens 
[BLUEB] Smalltalk-80: The Language and its Implementation : http://stephane.ducasse.free.fr/FreeBooks/BlueBook/ 

Le site officiel : http://www.squeak.org/
Le Wiki de la communauté française : http://community.ofset.org/wiki/Squeak
Le groupe des utilisateurs européens de Smalltalk (European Smalltalk User Group). L'adhésion est gratuite : http://www.esug.org/
Des livres gratuits en ligne sur Smalltalk et Squeak : http://stephane.ducasse.free.fr/FreeBooks.html
Un livre sur Squeak en français: http://stephane.ducasse.free.fr/Books.html : Squeak,  X. Briffault et S. Ducasse, Eyrolles, 2002. 

\section{semaphore}
	Boolean vs integer
	
\section{Critical Section}

\section{Scheduler}
How does it work?

\section{Delay}

\chapter{Avanced Abstractions}

\section{Monitor -- Broken}

\section{SharedQueue}

\section{Mutex - Reentrant Semaphore}
RecursionLock in VW


%=============================================================
\ifx\wholebook\relax\else
   \bibliographystyle{jurabib}
   \nobibliography{scg}
   \end{document}
\fi
%=============================================================

%-----------------------------------------------------------------

%%% Local Variables:
%%% coding: utf-8
%%% mode: latex
%%% TeX-master: t
%%% TeX-PDF-mode: t
%%% ispell-local-dictionary: "english"
%%% End: % ON: seems to be broken for now
% $Author$
% $Date$
% $Revision$

% HISTORY:
% Chapter started by Stef (2008-07-26)
% 2011-09-11 - Migrated to PharoBox: svn checkout https://XXX@scm.gforge.inria.fr/svn/pharobooks/PharoByExampleTwo-Eng

%=================================================================
\ifx\wholebook\relax\else
% --------------------------------------------
% Lulu:
	\documentclass[a4paper,10pt,twoside]{book}
	\usepackage[
		papersize={6.13in,9.21in},
		hmargin={.75in,.75in},
		vmargin={.75in,1in},
		ignoreheadfoot
	]{geometry}
	\input{../common.tex}
	\setboolean{lulu}{true}
% --------------------------------------------
% A4:
%	\documentclass[a4paper,11pt,twoside]{book}
%	\input{../common.tex}
%	\usepackage{a4wide}
% --------------------------------------------
    \graphicspath{{figures/} {figures/}}
	\begin{document}
\fi
%=================================================================
%\renewcommand{\nnbb}[2]{} % Disable editorial comments
\sloppy
%=================================================================
\chapter{Announcements: an Object Notification Framework}\chalabel{announcements}

It is often necessary to get notified when an action has been performed. This can be simply done by having an instance variable pointing to the objects to be notified. However such solution is not optimal because of the coupling it introduces between the objects. In this Chapter, we present 
Announcements, a framework to notify objects based on a registration mechanism and first class announcements. Indeed contrary to the traditional Smalltalk dependency mechanism which broadcast simple symbol, Announcement notification are full objects. Announcement can be the basis to Observer Design pattern.


\section{A word about object coupling}
\sd{should rewrite to get it more about objects and not components}
Here is an interesting question that often comes up often when writing
components. It is one that we faced when embedding our components. How
do the components communicate with each other in a way that doesn't
bind them together explicitly? That is, how does a child component
send a message to its parent component without explicitly knowing who
the parent is? Designing a component to refer to its parent is just a
part of the solution since the interfaces of different parents may be
different which would prevent the component from being reused in
different contexts.

There is a solution based on explicit dependencies also called the
change/update mechanism. Since early versions of Smalltalk, a
dependency mechanism based on a change/update protocol is available
and it is the foundation of the MVC framework itself. A component
registers its interest in some event and that event triggers a
notification. This is the basis of the Observer design pattern. 


\section{Announcements}
Announcement developed originally by Vassili Bykov is a new framework that offers notifications with full objects.  While the original dependency framework relied on symbols for the event registration and notification, Announcements promotes an object-oriented solution. Events are plain objects.

The main idea behind the framework is to set up announcers, define or
reuse some announcements, let clients register interest in events and
signal events. An announcement is an object representing an occurrence of a
specific event. It is the place to define all the information related
to the event occurrence. An announcer is responsible for registering
interested clients and announcing events, see Figure~\ref{fig:announcementFlow}.

\begin{figure}[ht]\centering
	\includegraphics[width=8cm]{AnnouncementFlow2}
	\caption{Announcement flow}
	\label{fig:announcementFlow}
\end{figure} 

\subsection{API}

Any object interested in an announcement registers its interest
using the method \mthind{on:do:}{on: anAnnouncementClass do: aBlock}
or \mthind{on: anAnnouncementClass send: aSelector to: anObject}{on:
  anAnnouncementClass send: aSelector to: anObject}.  The messages
\ct{on:do:} and \ct{on:send:to:} are strictly equivalent to the
messages \mthind{subscribe:do:}{subscribe: anAnnouncementClass do:
  aValuable}  and \mthind{subscribe:send:to:}{subscribe: anAnnouncementClass send:
  aSelector to: anObject}. You can also ask an announcer to
\ct{unsubcribe:} an object.


\begin{code}{Registering to an announcement}
	anAnnouncer on: MyAnnouncement do: [:announcement | announcement doSomething ]
\end{code}


\begin{code}{Emitting an announcement}
	anAnnouncer announce: (MyAnnouncement new)
\end{code}

\begin{method}{Unsubscribing from the announcer}
    anAnnouncer unsubscribe: self
\end{method}








%=====
%BEN


\section{Weak Announcements}

Even if you do not bind the emitter with the receiver, you create a dependency between to the announcer by registering yourself. And because announcers keep a reference to subscribers, this dependency prevent your registered objects to be garbage collected.

To solve this problem, weak announcements have been implemented. A weak announcement acts like regular announcements but keeps only a weak reference to subscribers, allowing them to be garbage collected.

\subsection{How to subscribe to weak announcement}

This protocol is really simple, because it's almost the same than for regular announcements.

When you use to do

\begin{code}{}
self session announcer on: RemoveChild send: #removeChild: to: self
\end{code}

you just do

\begin{code}{}
self session announcer !\textbf{weak}! on: RemoveChild send: #removeChild: to: self
\end{code}

The major difference is that you can't use the method \textbf{when:send:} anymore. Indeed if a weak announcement store a block, it will also store the block context and due to that, keep a strong reference to the subscriber.

\subsection{System Announcements}

We saw that you can create your own announcements to communicate between objects, but there are already existing announcements. All subclasses of \ct{SystemAnnouncement} are managed by the system to inform that it changed.

%\paragraph{SystemCategoryAddedAnnouncement}
%This announcement is emitted when a category is added to the system, by example using \mbox{\ct{SystemOrganizer>>#addCategory:}}
%
%\paragraph{SystemCategoryRemovedAnnouncement}
%This announcement is emitted when a category is removed to the system, by example using \mbox{\ct{SystemOrganizer>>#removeCategory:}}
%
%\paragraph{SystemCategoryRenamedAnnouncement}
%This announcement is emitted when a category is renamed to the system, by example using \mbox{\ct{SystemOrganizer>>#renameCategory:toBe:}}
%
%\paragraph{SystemClassAddedAnnouncement}
%This announcement is emitted when a class or a trait is added to the system, by example using \mbox{\ct{Trait>>#named:}} or \mbox{\ct{Class>>#subclass:}}
%
%\paragraph{SystemClassCommentedAnnouncement}
%This announcement is emitted twice when a class or a trait is commented, by example using \mbox{\ct{Trait>>#comment:}} or \mbox{\ct{Class>>#comment:}}
%
%\paragraph{SystemClassRecategorizedAnnouncement}
%This announcement is emitted when a class or a trait is set, by example using \mbox{\ct{Trait>>#category:}} or \mbox{\ct{Class>>#category:}}
%
%\paragraph{SystemClassRemovedAnnouncement}
%This announcement is emitted when a class or a trait is deleted from the system, by example using \mbox{\ct{Trait>>#removeFromSystem}} or \mbox{\ct{Class>>#removeFromSystem}}
%
%\paragraph{SystemClassRenamedAnnouncement}
%This announcement is emitted when a class or a trait is renamed, by example using \mbox{\ct{Trait>>#rename:}} or \mbox{\ct{Class>>#rename:}}
%
%\paragraph{SystemClassReorganizedAnnouncement}
%This announcement is emitted when a class or a trait's protocol is changed, by example using \mbox{\ct{Trait>>#removeCategory:}} or \mbox{\ct{Class>>#removeCategory:}}


\begin{code}{SystemAnnouncement hierarchy}
SystemAnnouncement
			SystemCategoryAddedAnnouncement
			SystemCategoryRemovedAnnouncement
			SystemCategoryRenamedAnnouncement
			SystemClassAddedAnnouncement
			SystemClassCommentedAnnouncement
			SystemClassRecategorizedAnnouncement
			SystemClassRemovedAnnouncement
			SystemClassRenamedAnnouncement
			SystemClassReorganizedAnnouncement
			SystemMethodAddedAnnouncement
			SystemMethodModifiedAnnouncement
			SystemMethodRecategorizedAnnouncement
			SystemMethodRemovedAnnouncement
			SystemProtocolAddedAnnouncement
			SystemProtocolRemovedAnnouncement
			SystemUnknownAnnouncement
\end{code}

\section{Discussions}
Announcements are like exceptions because they notify about events, but they are different in the nature of relationship between the announcer and the handler. Exceptions are for communication along the sender stack chain. There is no explicit connection between the announcer and the handler apart from the fact that the handler calls the announcer, directly or indirectly. Announcements are for communications across the object network, with a connection between the announcer and the handler pre-established by subscribing.

%=====


%=============================================================
\ifx\wholebook\relax\else
   \bibliographystyle{jurabib}
   \nobibliography{scg}
   \end{document}
\fi
%=============================================================


\subsection{Example in Seaside} 

\sd{I would remove this example -  I know I wrote it but....}Here is an example taken from Ramon Leon's
very good Smalltalk blog. This example shows how we can use
announcements to manage the communication between a parent component
and its children as for example in the context of a menu and its menu
items.

\begin{method}{Defining an announcer.}
MySession>>announcer
    ^ announcer ifNil: [announcer := Announcer new]
\end{method}

Then subclass \clsind{Announcement} for any interesting thing that
might happen. The announcement subclass is the place to add any extra
information about the specific announcement such as a context, the
objects involved \etc This is why announcement objects are both more
powerful and simpler than using symbols.

\begin{classdef}{Defining a specific Announcement.}
Announcement subclass: #RemoveChild
    instanceVariableNames: 'child'

RemoveChild class>>child: aChild
    ^self new
        child: aChild;
        yourself

RemoveChild>>child: anChild
    child := anChild

RemoveChild>>child
    ^child
\end{classdef}

Any component interested in this announcement registers its interest
using the method \mthind{on:do:}{on: anAnnouncementClass do: aBlock}
or \mthind{on: anAnnouncementClass send: aSelector to: anObject}{on:
  anAnnouncementClass send: aSelector to: anObject}.  The messages
\ct{on:do:} and \ct{on:send:to:} are strictly equivalent to the
messages \mthind{subscribe:do:}{subscribe: anAnnouncementClass do:
  aValuable}  and
\mthind{subscribe:send:to:}{subscribe: anAnnouncementClass send:
  aSelector to: anObject}. You can also ask an announcer to
\ct{unsubcribe:} an object.

In the following example, when a parent component is created, it
expresses interest in the \clsind{RemoveChild} event and specifies the
action that it will perform when such an event happens.

\begin{method}{}
Parent>>initialize
    super initialize.
    !\textbf{self session announcer on: RemoveChild do: [:it | self removeChild: it child] }!

Parent>>removeChild: aChild
    self children remove: aChild
\end{method}

And any component that wants to fire this event simply announces it by
sending in an instance of that custom announcement object.

\begin{method}{Announcing an event.}
Child>>removeMe
    self session announcer announce: (RemoveChild child: self)
\end{method}

Note that depending on where you place the announcer, you can even
have different sessions sending events to each other, or different
applications.

%=====
%BEN

Finally, when you have done with announcement, you may want to unsubscribe your object from the announcer.

\begin{method}{Unsubscribing from the announcer}
Child>>unsubscribe
    self session announcer unsubscribe: self
\end{method}

%=====

Announcements are not always the best way to establish communication
between components.  On one hand, announcements let you create loosely
coupled components and thus maximize reusability.  On the other hand,
they introduce additional complexity when you may be able solve your
communication problem with a simple message send.



%-----------------------------------------------------------------

%%% Local Variables:
%%% coding: utf-8
%%% mode: latex
%%% TeX-master: t
%%% TeX-PDF-mode: t
%%% ispell-local-dictionary: "english"
%%% End:
% $Author$
% $Date$
% $Revision$

% HISTORY: [see also Metaprogramming2.tex]
% 2007-05-22 - Damien Pollet started (translation from French article by ...?)
% 2008-01-15 - Alex added text
% 2008-12-15 - Oscar revised
% 2009-03-24 - Stef started new chapter (acttalk ... see separate file)
% 2009-06-01 - Oscar started to revise again and add new material

%=================================================================
\ifx\wholebook\relax\else
% --------------------------------------------
% Lulu:
	\documentclass[a4paper,10pt,twoside]{book}
	\usepackage[
		papersize={6in,9in},
		hmargin={.75in,.75in},
		vmargin={.75in,1in},
		ignoreheadfoot
	]{geometry}
	\input{../common.tex}
	\pagestyle{headings}
	\setboolean{lulu}{true}
% --------------------------------------------
% A4:
%	\documentclass[a4paper,11pt,twoside]{book}
%	\input{../common.tex}
%	\usepackage{a4wide}
% --------------------------------------------
    \graphicspath{{figures/} {../figures/}}
	\begin{document}
	% \renewcommand{\nnbb}[2]{} % Disable editorial comments
	\sloppy
\fi
%=================================================================
\chapter{Introspection and metaprogramming}\label{cha:metaprog}

\on{I am revising this chapter}

\lr{Hi Oscar,
I think that would be valuable information to add to the meta-programming chapter. Otherwise beginners will run into troubles when they try to do some reflective tricks on these messages.
Selectors that are never sent:\\
	class yourself \\
Selectors that have a special shortcut behavior. Depending on the receiver they are sometimes not looked up but directly invoke the primitive:\\
	+ - < > <= >= = ~= * / $\backslash$ ==
	@ bitShift: // bitAnd: bitOr:
	at: at:put: size
	next nextPut: atEnd
	blockCopy: value value: do: new new: x y \\
Selectors that are never send, because inlined by the compiler and transformed to comparison and jump bytecodes:\\
	ifTrue: ifFalse: ifTrue:ifFalse: ifFalse:ifTrue:
	and: or:
	whileFalse: whileTrue: whileFalse whileTrue
	to:do: to:by:do:
	caseOf: caseOf:otherwise:
	ifNil: ifNotNil:  ifNil:ifNotNil: ifNotNil:ifNil:\\
Interesting is that attempts to send these messages to non boolean objects can be intercepted and execution can be resumed with a valid boolean value by overriding \#mustBeBoolean in the receiver or by catching the NonBooleanReceiver exception.\\
Cheers,
Lukas
}

\st is a reflective programming language.
In a nutshell, this means that programs are able to reflect on their own execution.
A program that manipulates other programs (or even itself) is a \emph{metaprogram}.
For a programming language to be reflective, it should support both \emph{introspection} and \emph{intercession}.
Introspection is the ability to \emph{examine} the data structures that define the language, such as objects, classes, methods and the execution stack.
Intercession is the ability to \emph{modify} these structures, in other words to change the language semantics and the behavior of a program from within the program itself.

In this chapter we will explore many practical examples illustrating how \st supports introspection and metaprogramming.

%:======================================
\section{Introspection}

Using the inspector, you can look at an object, change the values of its instance variables, and even send messages to it.

\dothis{Evaluate the following code in a workspace:}
\begin{code}{| w |}
w := Workspace new.
w openLabel: 'My Workspace'.
w inspect
\end{code}

This will open a second workspace and an inspector.
The inspector shows the internal state of this new workspace, listing its instance variables in the left part (\ct!dependents!, \ct!contents!, \ct!bindings!...) and the value of the selected instance variable in the right part.
The \ct!contents! instance variable represents whatever the workspace is displaying in its text area, so if you select it, the right part will show an empty string.

\begin{figure}[ht]\centering
	\includegraphics[width=\linewidth]{workspaceInspector}
	\caption{Inspecting a \ct!Workspace!.\label{fig:workspaceInspector}}
\end{figure}

\dothis{Now type \ct!'hello'! in place of that empty string, then \emph{accept} it.}
The value of the \ct!contents! variable will change, but the workspace window will not notice it, so it does not redisplay itself.
To trigger the window refresh, evaluate \ct!self contentsChanged! in the lower part of the inspector.

%-----------------------------------------------------------------
\subsection{Accessing instance variables}

How does the inspector work?
In \st, all instance variables are protected.
In theory, it is impossible to access them if the class doesn't define any accessor.
In practice, the inspector can access instance variables without needing accessors, because it uses the reflective abilities of \st.
In \st, classes define instance variables either by name or by numeric indices.
The inspector uses methods defined by the \ct!Object! class to access them: \lct{instVarAt: \emph{index}} and \lct{instVarNamed: \emph{aString}} can be used to get the value of the instance variable at position \lct{\emph{index}} or identified by \lct{\emph{aString}}, respectively; to assign new values to these instance variables, it uses \ct!instVarAt:put:! and \ct!instVarNamed:put:!.

For instance, you can change the value of the \ct!w! instance variable of the first workspace by evaluating:
\begin{code}{}
w instVarNamed: 'contents' put: 'howdy!'; contentsChanged
\end{code}

\emph{Caveat:} Although these methods are useful for building development tools, using them to develop conventional applications is a bad idea: these reflective methods break the encapsulation boundary of your objects and can therefore make your code much harder to understand and maintain.

Both \ct!instVarAt:! and \ct!instVarAt:put:! are primitive methods, meaning that they are implemented as primitive operations of the \pharo virtual machine.
If you consult the code of these methods, you will see the special syntax \ct!<primitive: xx>! where \ct!xx! is an integer.

\needlines{5}
\begin{code}{}
Object>>>instVarAt: index 
	"Primitive. Answer a fixed variable in an object. ..."
	!\textbf{<primitive: 73>}!
	"Access beyond fixed variables."
	^self basicAt: index - self class instSize		
\end{code}

Typically, the code after the primitive invocation is not executed.
It is executed only if the primitive fails. In this specific case, if we try to access a variable that does not exist, then the code following the primitive will be tried.
This also allows the debugger to be started on primitive methods.
Although it is possible to modify the code of primitive methods, beware that this is a quite risky business for the stability of your \pharo system.

%There is a limited number of primitive operations in \pharo. They constitute the basic actions in \st: basic operations on integers, accessing input/output peripherals (disk, screen, network...).
%Some of these primitives also accelerate system functions.
%It is also possible to define custom primitives and to link them to plugins written in C to benefit from external libraries or from the speed of a native implementation.

\begin{figure}[ht]\centering
	\includegraphics[width=0.8\linewidth]{allInstanceVariables}
	\caption{Displaying all instance variables of a \ct!Workspace!.\label{fig:allInstanceVariables}}
\end{figure}

\figref{allInstanceVariables} shows how to display the values of the instance variables of an arbitrary instance (\ct!w!) of class \ct!Workspace!.
The method \ct!allInstVarNames! returns all the names of the instance variables of a given class.

In the same spirit, it is possible to gather instances that have specific properties.
%For instance, to get all instances of class \ct!Browser! whose instance variable \ct!systemOrganizer! is not \ct!nil!, try this expression:
For instance, to get all instances of class \ct!SketchMorph! whose instance variable \ct!owner! is set to the world morph (\ie images currently displayed), try this expression:
\begin{code}{}
SketchMorph allInstances select: [:c | (c instVarNamed: 'owner') isWorldMorph]
\end{code}
% Browser allInstances select: [:c | (c instVarNamed: 'systemOrganizer') notNil]

%-----------------------------------------------------------------
\subsection{Iterating over instance variables}

Let us consider the message \ct!instanceVariableValues!, which returns a collection of all values of instance variables defined by this class, excluding the inherited instance variables.
For instance:
\begin{code}{@TEST}
(1@2) instanceVariableValues --> an OrderedCollection(1 2)
\end{code}

The method is implemented in \ct{Object} as follows:
\needlines{9}
\begin{code}{}
Object>>>instanceVariableValues
	"Answer a collection whose elements are the values of those instance variables of the receiver which were added by the receiver's class."	
	| c |
	c := OrderedCollection new.
	self class superclass instSize + 1
		to: self class instSize
		do: [ :i | c add: (self instVarAt: i)].
	^ c
\end{code}

This method iterates over the indices of instance variables that the class defines, starting just after the last index used by the superclasses.
(The method \ct!instSize! returns the number of all named instance variables that a class defines.)

%-----------------------------------------------------------------
\subsection{Querying classes and interfaces}

The development tools in \pharo (code browser, debugger, inspector...) all use the reflective features we have seen so far.

Here are a few other messages that might be useful to build development tools:

\lct{isKindOf: \emph{aClass}} returns true if the receiver is instance of \lct{\emph{aClass}} or of one of its superclasses.

For instance:
\begin{code}{@TEST}
1.5 class                     --> Float
1.5 isKindOf: Number --> true
1.5 isKindOf: Integer   --> false
\end{code}

\lct{respondsTo: \emph{aSymbol}} returns true if the receiver has a method whose selector is \lct{\emph{aSymbol}}.

For instance:
\needlines{3}
\begin{code}{@TEST}
1.5 respondsTo: #floor      --> true    "since Number implements floor"
1.5 floor                            --> 1
Exception respondsTo: #, --> true    "exception classes can be grouped"
\end{code}

\emph{Caveat:} Although these features are especially useful for defining development tools, they are normally not appropriate for typical applications.
Asking an object for its class, or querying it to discover which messages it understands, are typical signs of design problems, since they violate the principle of encapsulation.
Development tools, however, are not normal applications, since their domain is that of software itself. As such these tools have a right to dig deep into the internal details of code.

There also exist mechanisms for introspecting on various parts of the run-time system, such as  the process scheduler, the memory manager and so on. For now we will focus on navigating through objects, classes and methods, and we will look more closely at rest of the runtime system in an other chapter.

%-----------------------------------------------------------------
\subsection{Code metrics}

Let's see how we can use \st's introspection features to quickly extract some code metrics.
Code metrics measure some aspect of program code such as depth of the inheritance hierarchy, the number of direct or indeirect subclasses, the number of methods or of instance variables in each class, or the number of locally defined methods or instance variables.
Here are a few metrics for the class \ct!Morph!, which is the superclass of all graphical objects in \pharo, revealing that it is a huge class, and that it is at the root of a huge hierarchy. Maybe it needs some refactoring!

\begin{code}{}
Morph allSuperclasses size.  -->       2 "inheritance depth"
Morph allSelectors size.        --> 1378 "number of methods"
Morph allInstVarNames size. -->      6 "number of instance variables"
Morph selectors size.             -->  998 "number of new methods"
Morph instVarNames size.     -->      6 "number of new variables"
Morph subclasses size.          -->    45 "direct subclasses"
Morph allSubclasses size.      -->  326 "total subclasses"
\end{code}

One of the most interesting metrics in the domain of object-oriented languages is the number of methods that extend methods inherited from the superclass.
This informs us about the relation between the class and its superclasses.

%:======================================
\section{Browsing code}

In \st, everything is an object. In particular, classes are objects that provide useful features for navigating through their instances.
Most of the messages we will look at now are implemented in \ct{Behavior}, so they are understood by all classes.

As we saw previously, you can obtain an instance of a given class by sending it the message \ct!#someInstance!.

\begin{code}{@TEST} % Possibly fragile!
Point someInstance --> 0@0
\end{code}

You can also gather all the instances with \ct!#allInstances!, or the number of alive instances in memory with \ct!#instanceCount!.

\begin{code}{} % Cannot test this
ByteString allInstances     --> #('collection' 'position'  ...)
ByteString instanceCount --> 104565
\end{code}

These features can be very useful when debugging an application, because you can ask a class to enumerate those of its methods exhibiting specific properties.
\begin{itemize}
\item \ct!whichSelectorsAccess:! returns the list of all selectors of methods that read the instance variable named as the argument
\item \ct!whichSelectorsStoreInto:! returns the selectors of methods that modify the value of an instance variable
\item \ct!whichSelectorsReferTo:! returns the selectors of methods that send a given message
\item \ct!crossReference! associates each message with the set of methods that send it.
\end{itemize}

%\ct!an IdentitySet(#rotateBy:about: #translateBy: #isInsideCircle:with:with: #sideOf: #nearestPointAlongLineFrom:to: #normalized #eightNeighbors #dist: #hash #rotateBy:centerAt: #theta #grid: #fourNeighbors #dotProduct: #scaleFrom:to: #normal #onLineFrom:to:within: #+ #degrees #interpolateTo:at:)!.

\begin{code}{} % TOO FRAGILE TO TEST
Point whichSelectorsAccess: 'x'    --> an IdentitySet(#'\\' #= #scaleBy: ...)
Point whichSelectorsStoreInto: 'x' --> an IdentitySet(#setX:setY: ...)
Point whichSelectorsReferTo: #+  --> an IdentitySet(#rotateBy:about: ...)
Point crossReference --> an Array(
		an Array('*' an IdentitySet(#rotateBy:about: ...))
		an Array('+' an IdentitySet(#rotateBy:about: ...))
		...)
\end{code}

The following messages take inheritance into account:
\begin{itemize}
\item \ct{whichClassIncludesSelector:} returns the superclass that implements the given message
\item \ct{unreferencedInstanceVariables} returns the list of instance variables that are neither used in the receiver class nor any of its subclasses
\end{itemize}

\begin{code}{@TEST}
Rectangle whichClassIncludesSelector: #inspect --> Object
Rectangle unreferencedInstanceVariables            --> #()
\end{code}

\ct{SystemNavigation} is a facade that supports various useful methods for querying and browsing the source code of the system.
\ct!SystemNavigation default! returns an instance you can use to navigate the system.
For example:

\begin{code}{@TEST}
SystemNavigation default allClassesImplementing: #yourself --> {Object}
\end{code}

%\ct!SystemNavigation default allClassesImplementing: #+! returns all classes that implement the message \ct!#+!.
%	In this case we get \ct!an Array(Number Fraction Float Integer SmallInteger LargePositiveInteger ScaledDecimal DateAndTime Duration Timespan Player Color Collection WordArray FloatArray KedamaFloatArray String Interval AbstractSound MixedSound Point Voice Complex TraitDescription TraitComposition TraitTransformation TComposingDescription)!.
%	As you can see, it is possible to add integers, fractions, strings or intervals.

The following messages should also be self-explanatory:

\begin{code}{}
SystemNavigation default allSentMessages size          --> 24930
SystemNavigation default allUnsentMessages size      --> 6431
SystemNavigation default allUnimplementedCalls size --> 270
\end{code}

Note that messages implemented but not sent are not necessarily useless, since they may be sent implicitly (\eg using \ct{perform:}).
Messages sent but not implemented, however, are more problematic, because the methods sending these messages will fail at runtime.
They may be a sign of unfinished implementation, obsolete APIs, or missing libraries.

\ct!SystemNavigation default allCallsOn: #Point! returns all messages sent explicitly to \ct!Point! as a receiver.

All these features are integrated in the programming environment of \pharo, in particular in the code browsers.
As you should already be aware, there are convenient keyboard shortcuts for browsing all i\underline{m}plementors (\short{m}) and se\underline{n}ders (\short{n}) of a given message.
What is perhaps not so well known is that there are many such pre-packaged queries implemented as methods of the \ct{SystemNavigation} class in the \prot{browsing} protocol.
For example, you can programmatically browse all implementors of the message \ct{ifTrue:} by evaluating:
\begin{code}{}
SystemNavigation default browseAllImplementorsOf: #ifTrue:
\end{code}

\begin{figure}[ht]\centering
	\includegraphics[width=\linewidth]{implementors}
	\caption{Browse all implementations of \ct!\#ifTrue:!.\label{fig:implementors}}
\end{figure}

Particularly useful are the methods \ct{browseAllSelect:} and \ct{browseMethodsWithSourceString:}.  Here are two different ways to browse all methods in the system that perform super sends (the first way is rather brute force; the second way is better and eliminates some false positives):
\begin{code}{}
SystemNavigation default browseMethodsWithSourceString: 'super'.
SystemNavigation default browseAllSelect: [:method | method sendsToSuper ].
\end{code}

%:======================================
\section{Classes, method dictionaries and methods}

Since classes are objects, we can inspect or explore them just like any other object.

\dothis{Evaluate \ct{Point explore}.}

In \figref{CompiledMethod}, the explorer shows the structure of class \ct!Point!.
You can see that the class stores its methods in a dictionary, indexing them by their selector.
The selector \ct{#*} points to the decompiled bytecode of \ct!Point>>>*!.

\begin{figure}[ht]\centering
	\includegraphics[width=.5\linewidth]{CompiledMethod}
	\caption{Explorer class \ct!Point! and the bytecode of its \ct!\#*! method.\label{fig:CompiledMethod}}
\end{figure}

% ON: THIS DISCUSSION DOES NOT BELONG HERE
%\st is really a completely open and transparent system, where all the code is available for reading.
%You can study the whole system as you wish, because there is no hidden code.
%The primitive part of the system, \ie which is not written in \st, is reduced to its simplest expression.
%Among \st dialects, \pharo has the peculiarity that even its virtual machine is written in \st, so it is possible to run and debug it inside a \pharo image!
%Though this actually works, it is quite slow, and for actual use the virtual machine is translated to C and compiled to a native executable.

Let us consider the relationship between classes and methods.
In \figref{MethodsAsObjects} we see that classes and metaclasses have the common superclass \ct{Behavior}. This is where \ct{new} is defined, amongst other key methods for classes.
Every class has a method dictionary, which maps method selectors to compiled methods.
Each compiled method knows the class in which it is installed.
In \figref{CompiledMethod} we can even see that this is stored in an association in \ct{literal5}.

\begin{figure}[ht]\centering
	\includegraphics[width=0.8\linewidth]{MethodsAsObjects}
	\caption{Classes, method dictionaries and compiled methods\label{fig:MethodsAsObjects}}
\end{figure}

We can exploit the relationships between classes and methods, to pose queries about the system.
For example, to discover which methods are newly introduced in a given class, \ie do not override superclass methods, we can navigate from the class to the method dictionary as follows:
\begin{code}{}
[:aClass| aClass methodDict keys select: [:aMethod |
  (aClass superclass canUnderstand: aMethod) not ]] value: SmallInteger
  --> an IdentitySet(#threeDigitName #printStringBase:nDigits: ...)
\end{code}

A compiled method does not simply store the bytecode of a method.
It is also an object the provides numerous useful methods for querying the system.
One such method is \ct{isAbstract} (which tells if the method sends \ct{subclassResponsibility}).
We can use it to identify all the abstract methods of an abstract class
\begin{code}{}
[:aClass| aClass methodDict keys select: [:aMethod |
  (aClass>>aMethod) isAbstract ]] value: Number
  --> an IdentitySet(#storeOn:base: #printOn:base: #+ #- #* #/ ...)
\end{code}
Note that this code sends the \ct{>>} message to a class to obtain the compiled method for a given selector.

As a slightly more complex example, we can browse 

To browse the super-sends within a given hierarchy, for example within the Collections hierarchy, we can pose a more sophisticated query:
\begin{code}{}
class := Collection.
SystemNavigation default
  browseMessageList: (class withAllSubclasses gather: [:each |
    each methodDict associations
      select: [:assoc | assoc value sendsToSuper]
      thenCollect: [:assoc | MethodReference class: each selector: assoc key]])
  name: 'Supersends of ' , class name , ' and its subclasses'
\end{code}
Note how we navigate from classes to method dictionaries to compiled methods to identify the methods we are interested in.
A \ct{MethodReference} is a lightweight proxy for a compiled method that is used by many tools.

%:======================================
\section{Browsing environments}

Although \ct{SystemNavigation} offers some useful ways to programmatically query and browse system code, there is a better way.  The Refactoring Browser, which is integrated into \pharo, provides both interactive and programmatic ways to pose complex queries.

Suppose we are interested to discover which methods in the \ct{Collection} hierarchy send a message to \super which is different from the method's selector.
This is normally considered to be a bad code smell, since such a \super-send should normally be replaced by a \self-send. (Think about it --- you only \emph{need} \super to extend a method you are overriding; all other inherited methods can be accessed by sending to \self!)

The refactoring browser provides us with an elegant way to restrict our query to just the classes and methods we are interested in.

\dothis{Open a system browser on the class \ct{Collection}.
\actclick on the class name and select \menu{refactoring scope>subclasses with}.
This will open a new Browser Environment on just the \ct{Collection} hierarchy.
Within this restricted scope select \menu{refactoring scope>super-sends} to open a new environment with all methods that perform super-sends within the \ct{Collectuon} hierarchy.
Now \click on any method and select \menu{refactor>code critics}.
Navigate to \menu{Lint checks>Possible bugs>Sends different super message} and \actclick to select \menu{browse}.}

In \figref{sendDifferentSuper} we can see that 19 such methods have been found within the \ct{Collection} hierarchy, including \ct{Collection>>>printNameOn:}, which sends \ct{super printOn:}.
\begin{figure}[ht]\centering
	\includegraphics[width=\linewidth]{sendDifferentSuper}
	\caption{Finding methods that send a different super message.\label{fig:sendDifferentSuper}}
\end{figure}

Browser environments can also be created programmatically.
Here, for example, we create a new \ct{BrowserEnvironment} for \ct{Collection} and its subclasses, select the super-sending methods, and open the resulting environment.
\needlines{4}
\begin{code}{}
((BrowserEnvironment new forClasses: (Collection withAllSubclasses))
	selectMethods: [:method | method sendsToSuper])
	label: 'Collection methods sending super';
	open.
\end{code}{}

Note how this is considerably more compact than the earlier, equivalent example using \ct{SystemNavigation}.

Finally, we can find just those methods that send a different super message programmatically as follows:
\begin{code}{}
((BrowserEnvironment new forClasses: (Collection withAllSubclasses))
	selectMethods: [:method | 
		method sendsToSuper
		and: [(method parseTree superMessages includes: method selector) not]])
	label: 'Collection methods sending different super';
	open
\end{code}
Here we ask each compiled method for its (Refactoring Browser) parse tree, in order to find out whether the super messages differ from the method's selector.
Look at the \prot{querying} protocol of the class \ct{RBProgramNode} to see some the things we can ask of parse trees.

%:======================================
\section{Accessing the run-time context}

We have seen how \st's reflective capabilities let us query and explore objects, classes and methods.  But what about the run-time environment?

%-----------------------------------------------------------------
\subsection{Method Contexts}

In fact, the run-time context of an executing method is in the virtual machine --- it is not in the image at all.
On the other hand, the debugger obviously has access to this information, and we can happily explore the run-time context, just like any other object.
How is this possible?

Actually, there is nothing magical about the debugger.
The secret is the pseudo-variable \ct{thisContext}, which we have encountered only in passing before.
Whenever \ct{thisContext} is referred to in a running method, the entire run-time context of that method is reified and made available as \ct{MethodContext} objects to the image.

We can easily experiment with this mechanism ourselves.

\dothis{Change the definition of \ct{Integer>>>factorial} by inserting the underlined expression as shown below:}

\begin{code}{}
Integer>>>factorial
	"Answer the factorial of the receiver."
	self = 0 ifTrue: [!\underline{thisContext explore. self halt.}! ^ 1].
	self > 0 ifTrue: [^ self * (self - 1) factorial].
	self error: 'Not valid for negative integers'
\end{code}

\dothis{Now evaluate \ct{3 factorial} in a workspace. You should see both a pre-debugger window and an explorer, as shown in \figref{exploringThisContext}.}

\begin{figure}[ht]\centering
	\includegraphics[width=\linewidth]{exploringThisContext}
	\caption{Exploring \lct{thisContext}.\label{fig:exploringThisContext}}
\end{figure}

If you now browse the class of the explored object (\ie by evaluating \ct{self browse} in the bottom pane of the explorer) you will discover that it is an instance of the class \ct{MethodContext}, as is each \ct{sender} in the chain. All of these objects have been created dynamically in the image by the \st virtual machine at the point where \ct{thisContext} was referred to in the \ct{factorial} method.

\ct{thisContext} is not intended to be used for day-to-day programming, but it is essential for implementing tools like debuggers, and for accessing information about the call stack.
You can evaluate the following expression to discover which methods make use of \ct{thisContext}:

\begin{code}{}
SystemNavigation default browseMethodsWithSourceString: 'thisContext'
\end{code}

As it turns out, one of the most common applications is to discover the sender of a message.
Here is a typical application:

\begin{code}{}
Object>>>subclassResponsibility
	"This message sets up a framework for the behavior of the class' subclasses.
	Announce that the subclass should have implemented this message."

	self error: 'My subclass should have overridden ', thisContext sender selector printString
\end{code}

By convention, methods in \st that send \ct{self subclassResponsibility} are considered to be abstract.  But how does \ct{Object>>>subclassResponsibility} provide a useful error message indicating which abstract method has been invoked?  Very simply, by asking \ct{thisContext} for the sender.

%-----------------------------------------------------------------
\subsection{Intelligent breakpoints}

The \st way to set a breakpoint is to evaluate \ct{self halt} at an interesting point in a method.  This will cause \ct{thisContext} to be reified, and a debugger window will open at the breakpoint.
Unfortunately this poses problems for methods that are intensively used in the system.

Suppose, for instance, that we want to explore the execution of \ct{OrderedCollection>>>add:}.
Setting a breakpoint in this method is problematic.

\dothis{Take a \emph{fresh} image and set the following breakpoint:}
\needlines{3}
\begin{code}{}
OrderedCollection>>>add: newObject
	!\underline{self halt.}!
	^self addLast: newObject
\end{code}

Notice how your image immediately freezes!  We do not even get a debugger window.
The problem is clear once we understand that (i) \ct{OrderedCollection>>>add:} is used by many parts of the system, so the breakpoint is triggered very soon after we accept the change, but (ii) the debugger itself sends \ct{add:} to an instance of \ct{OrderedCollection}, preventing the debugger from opening.
What we need is a way to \emph{conditionally halt} only if we are in a context of interest.
This is exactly what \ct{Object>>haltIf:} offers.

Suppose now that we only want to halt if \ct{add:} is sent from, say, the context of \ct{OrderedCollectionTest>>>testAdd}.

\dothis{Fire up a fresh image again, and set the following breakpoint:}
\begin{code}{}
OrderedCollection>>>add: newObject
	!\underline{self haltIf: \#testAdd.}!
	^self addLast: newObject
\end{code}

This time the image does not freeze. Try running the \ct{OrderedCollectionTest}.
(You can find it in the \cat{CollectionsTests-Sequenceable} category.)

How does this work?  Let's have a look at \ct{Object>>>haltIf:}:

%\on{talk about squeak here?} In the 3.8 version of \squeak introduced a new kind of breakpoints.
%Those are quite useful and are implemented by the \ct!Object>>>haltIf:! method.
%This breakpoint only halts program execution if the method it's placed in was called by the one whose name was passed as an argument.
%This is very useful to halt a method only during the execution of a test and not every time it is called.
%For instance, the expression \ct!self haltIf: #testFoo! will only halt execution of its containing method if this method was called, directly or indirectly, from \ct!testFoo!.
%The definition of \ct!haltIf:! is quite simple and fits in five lines; it uses the pseudo-variable \ct!thisContext!, which refers to the execution stack, represented as an object (\ct!thisContext! is one of the six keywords in \st, with \self, \super, \nil, \ct!true!, and \ct!false!).

\begin{code}{}
Object>>>haltIf: condition
	| cntxt |
	condition isSymbol ifTrue: [
		"only halt if a method with selector symbol is in callchain"
		cntxt := thisContext.
		[cntxt sender isNil] whileFalse: [
			cntxt := cntxt sender. 
			(cntxt selector = condition) ifTrue: [Halt signal]. ].
		^self.
	].
	...
\end{code}

Starting from \ct!thisContext!, \ct!haltIf:! goes up through the execution stack, checking if the name of the calling method is the same as the one passed as parameter.
If this is the case, then it raises an exception which, by default, summons the debugger.

It is also possible to supply a boolean or a boolean block as an argument to \ct{haltIf:}, but these cases are straightforward and do not make use of \ct{thisContext}.

%This example shows the power of \st, which allows to define powerful features and tools from within the language itself.

%The pseudo-variable \ct!thisContext! is especially used in the \pharo debugger.
%It contains an instance of the \clsind{MethodContext} class.
%This instance contains information about the method that is being executed, the stack pointer (\ct!stackpc!), and the program pointer (\ct!pc!).

%In the following example, we halted \pharo's evaluation loop by pressing \short{.} and inspecting the current context.

%\begin{figure}[ht]\centering
%	\includegraphics[width=\linewidth]{MethodContext}
%	\caption{Inspecting the execution context of the evaluation loop in \pharo.\label{fig:MethodContext}}
%\end{figure}

%Via \ct!thisContext!, the Seaside web framework also accesses the execution stack, but it modifies it on the fly to easily implement reusable components over HTTP.

%:======================================
\section{Intercepting doesNotUnderstand}

So far we have used the reflective features of \st mainly to query and explore objects, classes, methods and the run-time stack. Now we will look at how to use our knowledge of the \st system structure to modify behaviour at run-time.

As we have seen in \charef{model}, when an object receives a message, it first looks in the method dictionary of its class for a corresponding method to respond to the message.
If no such method exists, it will continue looking up the class hierarchy, until it reaches \ct{Object}. If still no method is found for that message, the object will send itself the message \ct{doesNotUnderstand:} with the message selector as its argument.
The process then starts all over again, until \ct{Object>>>doesNotUnderstand:} is found, and the debugger is launched.

But what if \ct{doesNotUnderstand:} is overridden by one of the subclasses of \ct{Object} in the lookup path?
As it turns out, this is a convenient way of realizing certain kinds of very dynamic behaviour. An object that does not understand a message can, by overriding \ct{doesNotUnderstand:}, fall back to an alternative strategy for responding to that message.

Two very common applications of this technique are (1) to implement lightweight proxies for objects, and (2) to dynamically compile or load missing code.
In the first case, an 



%:==>HERE<==
\on{HERE}


\on{should be a trait}
\begin{code}{}
doesNotUnderstand: aMessage
	| messageName |
	messageName := aMessage selector asString.
	(self class instVarNames includes: messageName)
		ifTrue: [
			self class compile: messageName, String cr, ' ^ ', messageName.
			^ aMessage sendTo: self ].
	super doesNotUnderstand: aMessage
\end{code}



\begin{figure}[ht]\centering
	\includegraphics[width=\linewidth]{DynamicAccessors}
	\caption{Dynamically creating accessors.\label{fig:DynamicAccessors}}
\end{figure}


%:======================================
\section{Controlling messages}

\on{
- override doesNotUnderstand\\
--- trap msgs sent to an object (with Proxy and with msgNotUnderstood)\\
- anonymous classes (see slides)\\
- method wrappers
}

\on{Using doesNotUnderstand to automatically create accessors for instance variables; package as a reusable trait!}

\on{Object perform:}

\on{Object primitiveChangeClassTo: become: and becomeForward: (see slides with minimal object example)}

%:======================================
\section{Methods as first class entities}

Compared to other reflective programming languages, \pharo is a \st implementation that brings reflection a step further. Whereas it handles \emph{methods as first class entities} that can be modified and replaced on the fly, it also allows for \emph{method execution reification}, exposing dynamic information related to method execution as objects.


\on{
- Methods as Objects; compiled methods; wrappers; spy\\
--- method dictionaries and compiled methods -- compiling, invoking \& removing methods\\
--- method wrappers: coverage; memoization\\
--- coverage checking of a class w. method wrappers\\
--- automatic memoization with method wrappers\\
}

\on{;""Q How do I check which methods of my package are not covered by tests?""
;""A"" Load the *http://www.squeaksource.com/ObjectsAsMethodsWrap.html* SqueakSource project and run:}

\begin{code}{}
category := 'SCGPier'.
w := (ObjectAsOneTimeMethodWrapper installOnClassCategory: category).

tr := TestRunner new.
ToolBuilder open: tr.
[tr
	categoryAt: (tr categoryList indexOf: 'SCGPier') put: true;
	selectAllClasses;
	runAll.]
ensure: [[w do: [:each| each uninstall ]] valueUnpreemptively].

((w select: [:each | each executed not ])
	collect: [:each | each wrappedClass name, '>>', each selector name ]) explore.
\end{code}


%-------------------------------------------------------------------------
\subsection{Reification of Methods} 

In \pharo, both classes and methods are represented by first class objects that can be accessed and even changed at runtime. 

\emph{Compiled methods} are objects in \pharo that describe the executable part of methods. A compiled method contains a pointer to the source code and the code executed by the virtual machine, the bytecode.

Compiled methods are stored in a method dictionary, in which keys are method names and values corresponding compiled methods. A method dictionary may be accessed from a given class by sending the message \ct{methodDict} to this class. For instance, the expression \ct{ExampleClass methodDict} returns the dictionary of methods for the class \ct{ExampleClass}.

Method dictionaries provide an interface for adding or replacing methods. When a new method is added from a development tool, such as the system browser, in essence, the system adds a new entry in the method dictionary of the class in which the method has been compiled for:

\begin{code}{}
ExampleClass methodDict at: #myMethod put: aCompiledMethod
\end{code}

A compiled method (\ct{aCompiledMethod}) is added to the method dictionary of the class \ct{ExampleClass} under the name \ct{myMethod}. The effect is the addition of a new method.

Having compiled methods as plain objects is a characteristic shared by most \st implementations. As we shall see in the coming section, one particularity of \pharo is to not restrict values contained in a method dictionary to be instances of \ct{CompiledMethod}.

%-------------------------------------------------------------------------
\subsection{Reification of method execution}

When a message named \emph{meth} is sent to an object, the class of receiver looks up the method to use to handle the message. If this class does have a method, it asks its superclass, and so on, up the inheritance chain. Once the method named \emph{meth} found in one of the superclass (\ie its method dictionary contains an entry for \emph{meth}), then the corresponding value associated with \emph{meth} is fetched from the method dictionary by the virtual machine (VM).

Once fetch, this method gets interpreted by the VM. At that stage, one of two cases happens: either this method is a \emph{compiled method} as we have seen previously, or it is a plain object that is not a compiled method. If this value is a compiled method (an instance of the class \ct{CompiledMethod}), then the VM directly interprets the bytecode. If this value is \emph{not} a compiled method, then the VM performs a \emph{reification of the method execution}.

This method execution reification is performed by the VM when sending the message \ct{run: methodName with: listOfArguments in: receiver} to this object method. In such a case, \ct{methodName} contains the name of the method currently invoked, \ct{listOfArguments} contains the list of arguments provided, and \ct{receiver} a reference to the receiver to the message.

For instance, let assume two classes, \ct{C} and \ct{Wrapper}:

\begin{code}{}
Object subclass: #C
	instanceVariableNames: ''
	classVariableNames: ''
	poolDictionaries: ''
	category: 'Example'.
 
Object subclass: #Wrapper
	instanceVariableNames: ''
	classVariableNames: ''
	poolDictionaries: ''
	category: 'Example'.
\end{code}

The class \ct{Wrapper} contains a method named \ct{run:with:in:} that takes three arguments, a name (\ct{methodName}), an array of objects (\ct{listOfArguments}), and an object reference (\ct{receiver}):

\begin{code}{}
Wrapper>>>run: methodName with: listOfArguments in: receiver
	"We first display some info on the standard output stream"
	Transcript show: 'Method ', methodName,
		' arguments: ', listOfArguments printString,
		' receiver: ', receiver printString.

	"Then we return the first element of the provided list"
	^ listOfArguments first
\end{code}

Prior to returning the first elements of \ct{listOfArguments}, some information is written to the standard output stream.

%The set of methods defined by a class is contained in a method dictionary, which can be accessed by sending the message \ct{methodDict} to a class. 
As we have seen in the previous section, a method \ct{foo:} is added to the class \ct{C} by altering its method dictionary:

\begin{code}{}
"We create an instance of Wrapper"
w := Wrapper new.

"We create an instance of C"
c := C new.

"We add a new method named foo: to C"
C methodDict at: #foo: put: w.
\end{code}

The last line above adds an entry named \ct{foo:} to the method dictionary of \ct{C}. Instances of \ct{C} therefore understand messages named \ct{foo:}. Because of the method execution reification mechanism, evaluating \ct{c foo: 10} is in fact evaluated by the VM as:

\begin{code}{}
w run: #foo: with: #(10) in: c
\end{code}

The result is \ct{10}, and displays the following on the standard output stream:

\begin{code}{}
Method foo: arguments: #(10) receiver: a C
\end{code}

Reification of method execution is an expressive mechanism that enables a whole range of dynamic meta operations. As an example, 

%-----------------------------------------------------------------
\subsection{Example with Spy}

As an illustration of section given above, this section presents \emph{Spy}, a small application that infers relationship between classes at runtime. Spy is available on SqueakSource\footnote{\url{http://www.squeaksource.com/Spy}}. 

A spy is installed on any class, with an \ct{installOn:} method and is removed by sending \ct{removeSpy} to a spied class. Spy is used as follows: 
\begin{itemize}
\item \ct{Spy installOn: aClass} -- installs a spy on a class \ct{aClass};
\item \ct{aClass stat} -- returns a textual description of the interaction of \ct{aClass};
\item \ct{aClass removeSpy} -- removes the spy from the class.
\end{itemize}

The idea is the following: installing spy will replace all compiled method of the class with instance of the class \ct{Spy}. Each instance of these instance knows the compiled method it has replaced. At runtime, each spy evaluates its compiled methods and uses introspection to extract dynamic information.

\ct{Spy} is a subclass of \ct{Object}
\begin{code}{}
Object subclass: #Spy
	instanceVariableNames: 'originalMethod nbOfcalls callingClasses selector'
	classVariableNames: ''
	poolDictionaries: ''
	category: 'Spy'
\end{code}

Few accessors and an initialization are necessary:
\begin{code}{}
Spy>>>callingClasses
	^ callingClasses ifNil: [callingClasses := OrderedCollection new]

Spy>>>originalMethod
	^ originalMethod

Spy>>>originalMethod: aCompiledMethod
	originalMethod := aCompiledMethod 

Spy>>>initialize
	super initialize.
	nbOfcalls := 0.
\end{code}

The variable \ct{callingClasses} will contains the classes that the spied class interact with. The compiled method the spy will replace is referenced by \ct{originalMethod}.

The main method is \ct{run:with:in:}:
\begin{code}{}
Spy>>>run: methodName with: listOfArguments in: receiver
	| callingClass |
	callingClass := thisContext sender receiver class.
	(self callingClasses includes: callingClass)
		ifFalse: [self callingClasses add: callingClass].
	nbOfcalls := nbOfcalls + 1.
		
	^ originalMethod valueWithReceiver: receiver arguments: listOfArguments 
\end{code}

The pseudo variable \ct{topContext} represents the top frame of the run-time stack. The frame below the top one is the result of \ct{thisContext sender}. The class of the receiver of this frame is obtained by sending the message \ct{receiver} and \ct{class}. This class is then inserted into \ct{callingClasses}. The compiled method is then evaluated on the receiver of the original message (and not the spy!) with the same arguments.

The \pharo interpreter may have to remove all references to this method form its method look cache. The following has to be defined as well:

\begin{code}{}
Spy>>>flushCache
	<primitive: 116>
\end{code}

Note that each object intended to replace a compiled method in a method dictionary has to understand this method.

Each spy is able to generate a little summary:

\begin{code}{}
Spy>>>summary
	^ selector printString, ' has been called: ', nbOfcalls asString, ' by ', self callingClasses asString
\end{code}

The \ct{installOn:} method will replace each compiled method with an instance of \ct{Spy}. Note that this method has to be defined on the class side:

\begin{code}{}
Spy class>>>installOn: aClass
	| dict cm |
	dict := aClass methodDictionary.
	dict keys do:
		[:k|
			cm := dict at: k.
			cm isCompiledMethod 
			 	ifTrue: [dict at: k put: (self new originalMethod: cm; selector: k; yourself)]]
			
\end{code}

The \ct{Spy} class is fully defined. The \ct{Behavior} class needs to be extended with few methods:

\begin{code}{}
Behavior>>>removeSpy
	| v |
	self methodDictionary keys do:
		[:k|
			v := self methodDict at: k. 
			(v class == Spy)
					ifTrue: [self methodDict at: k put: v originalMethod]]

Behavior>>>stat
	| ans |
	ans := ''.
	self methodDict values do: 
		[:spy|
				(spy isKindOf: Spy)
					ifTrue: [ans := ans, spy summary, String cr]].
	^ ans
\end{code}

The Spy application is now fully defined. As an example of usage, a spy may be installed on the class \ct{Stream} with \ct{Spy installOn: Stream}. Then print the result of \ct{Spy stat}. You should have something that looks like:

\begin{code}{}
 '#next:put: has been called: 0 by an OrderedCollection()
...
#basicNextPut: has been called: 71 by an OrderedCollection(Latin1TextConverter)
...
#nextPutAll: has been called: 27 by an OrderedCollection(WriteStream LimitedWriteStream)
...
'
\end{code}

\ct{Stream} may be freed from its spy with \ct{Stream removeSpy}.


%:======================================
\section{Chapter summary}

It is natural to use the reflective features in \st.
Indeed, the whole development environment and its code browsers are built using the introspective interfaces of objects and classes.
We want to stress that it is really interesting to be able to access and modify the objects that represent the program; it makes it unneccessary to devise alternative representations such as abstract syntax trees to build development tools (like Eclipse).
To fix Java~1.0's lack in this respect, Java~1.2 added the ``reflextive'' API ---which is actually only introspective.
Moreover, if the classes and other objects underlying the program execution offer introspective interfaces, then all program representations stay synchronized with the code at all times.
It is thus not needed to keep these representations up-to-date anymore.



%=================================================================
\ifx\wholebook\relax\else\end{document}\fi
%=================================================================



1.1 Introspection
- inspecting objects
- accessing instance variables
- query all instances with given properties
- iterating instance variables
- querying classes; subclasses; methods understood
- browsing code; querying method relationships
- code metrics

1.2 Browsing Code
- SystemNavigation

1.3 Classes, method Dictionaries and methods
- classes, method dictionaries and compiled methods
- inspecting classes

1.4 Browsing Environments
- refactoring browser scoped environments
- code smells

1.5 Accessing the run-time context
- thisContext and Method Contexts
- haltIf:



---

* perform:; valueWithReceiver:
* compile:
* MethodContext


Controlling messages
- trap with Proxy and override doesNotUnderstand (uses become:) ???
- instead use example with object that auto-creates accessors on demand (using a trait?)
- anonymous classes (uses compile: and primitiveChangeClassTo:) ???
- spy (form a method wrapper)
- method wrappers (simplified interface)

Method wrappers
- run:with:in:
- coverage
- memoization
- collect direct senders; class collaborations
- pre and postconditions
Other stuff:
- Object primitiveChangeClassTo: become: and becomeForward: (see tests and slides with minimal object example)

- PointerFinder?

%=================================================================

% \printglossary
\bibliographystyle{jurabib}
\bibliography{scg}

\printindex

\end{document}

%=================================================================
%:Planned for later releases

% The following chapters are planned for later releases.



%:Compiler
% $Author$
% $Date$

% HISTORY:
% 2007-07-18 - Matthieu started chapter (one paragraph only)

%=================================================================
\ifx\wholebook\relax\else
% --------------------------------------------
% Lulu:
	\documentclass[a4paper,10pt,twoside]{book}
	\usepackage[
		papersize={6.13in,9.21in},
		hmargin={.75in,.75in},
		vmargin={.75in,1in},
		ignoreheadfoot
	]{geometry}
	\input{../common.tex}
	\pagestyle{headings}
	\setboolean{lulu}{true}
% --------------------------------------------
% A4:
%	\documentclass[a4paper,11pt,twoside]{book}
%	\input{../common.tex}
%	\usepackage{a4wide}
% --------------------------------------------
    \graphicspath{{figures/} {../figures/}}
	\begin{document}
	\renewcommand{\nnbb}[2]{} % Disable editorial comments
	\sloppy
\fi
%=================================================================
\chapter{The old and the new compiler}
\chalabel{compiler}

% ONE PARAGRAPH BY MSUEN:
On part of the metalevel of squeal is the compiler. 
You probably notice that each time you add a method you compiled it and it is add to the class.
It make the way of programming in Smalltalk different from most of the language.
In Java you compiled the all class and the virtual machine take care of adding you class.
In Pharo your class is already inside the image and behavior are added bit by bit inside the image.
Each time you add a method you request the compiler.
The compiler is responsible of transforming Smalltalk into object oriented bytecode. 
So the bytecode can be understand by the virtual machine


%=============================================================
\ifx\wholebook\relax\else
   \bibliographystyle{jurabib}
   \nobibliography{scg}
   \end{document}
\fi
%=============================================================

%=================================================================
%%% Local Variables:
%%% coding: utf-8
%%% mode: latex
%%% TeX-master: t
%%% TeX-PDF-mode: t
%%% ispell-local-dictionary: "english"
%%% End:


%=================================================================

\chapter{Inheritance}

\on{or: The darker side of inheritance ?}

Include use and misuse of Inheritance.

\on{should be about idioms, not design!}

Prefer abstract superclasses.

Include the stuff that never gets talked about, like weak subclasses and variable subclasses.

Why it's impossible to subclass SmallInteger.

\ab{programming idioms involving inheritance.  singleton, initialization, set... methods for instance initialization, self class and self species rather than using explicit class names.  More as they appear.  }

\chapter{Exceptions}

Stef sez: see the test cases

%	For exceptions
%	--------------
%	On Oct 26, 2006, at 9:23, stephane ducasse wrote:

%	BTW oscar if you want to have fun look at the test for the exception in Squeak.
%	They are excellent. I was planning to use them to illustrate exception
%	(and understand myself deeply) in a separate chapter.

%	I mean

%		on:do:
%		
%	Look at the classes in the category Exceptions-Tests.
%	You have all the possible scenario on what we can do with exception in smalltalk
%	and this is impressive.

%	ExceptionTests  uses ExceptionTester to check all the possible combinations and
%	semantical points. I just read everything once just to learn.
%	Now I would like to remember it and I guess that writing it down would be the best
%	(No time for that so far)

%	Stef
\chapter{Concurrency}
Processes and friends.

\chapter{Dependency Mechanisms}
Stef sez: we need on overview on publish/subscribe and the dependency mechanisms in squeak.

\chapter{Common Errors}
The last chapter of Smalltalk with Style is a good source.

\chapter{Leftover stuff}
Time.

\section{XML processing :)}

\section{How make an image self executable withSqueak}
We save an image with our main application window already opened, and unwanted things disabled (halos, alt-., menus, etc). Additionally, we have a class-side \#startUp: method to initialize what's necessary on startup, like re-reading config files etc.

The image is started automatically when double-clicking  "plopp.exe", or rather a short-cut in the program menu placed there by the installer. This exe is nothing more than a renamed Squeak.exe, with a different icon patched in using a resource editor. There is a VM config file "plopp.ini", which most importantly contains the path to the image to load (see http://minnow.cc.gatech.edu/squeak/3274).

The exe along with the needed plugins, the image and other data files is installed in the "programs" directory by a regular Windows installer (InstallShield).

It's really not too different from other applications, which nowadays also consist of an executable and several additional files. That the actual application logic in Squeak's case is not contained in the executable but in a "data file" is of no concern to the end user.

Note that this is an FAQ item (http://minnow.cc.gatech.edu/squeak/3563). Feel free to put my explanation on the Swiki, I haven't checked carefully if a similar answer is already there.

- Bert -
%=================================================================

% \printglossary
\bibliographystyle{jurabib}
\bibliography{scg}

\printindex

\end{document}

%:===> END OF ADDITIONAL MATERIAL <===
% \end{document} % NB: The actual end is further up
%=================================================================
%%% Local Variables:
%%% coding: utf-8
%%% mode: latex
%%% TeX-master: t
%%% TeX-PDF-mode: t
%%% ispell-local-dictionary: "english"
%%% End:
