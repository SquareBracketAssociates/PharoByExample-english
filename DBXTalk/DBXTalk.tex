% $Author: gpolito $
% $Date: 2011-07-09 17:14:33 -0300 (Sat, 09 Jul 2011) $
% $Revision: 1 $

%=================================================================
\ifx\wholebook\relax\else
% --------------------------------------------
% Lulu:
	\documentclass[a4paper,10pt,twoside]{book}
	\usepackage[
		papersize={6.13in,9.21in},
		hmargin={.75in,.75in},
		vmargin={.75in,1in},
		ignoreheadfoot
	]{geometry}
	\input{../common.tex}
	\setboolean{lulu}{true}
% --------------------------------------------
% A4:
%	\documentclass[a4paper,11pt,twoside]{book}
%	\input{../common.tex}
%	\usepackage{a4wide}
% --------------------------------------------
    \graphicspath{{figures/} {../figures/}}
	\begin{document}
\fi
%=================================================================
%\renewmessage{\nnbb}[2]{} % Disable editorial comments
\sloppy

%=================================================================
%\renewcommand{\nnbb}[2]{#2} % Disable editorial comments

\chapter{DBXTalk}
\chapterauthor{\authorguille{}}

DBXTalk is a database driver that allows interaction with major relational database engines such as
Oracle and MSSQL, apart from those which are open source, like PostgreSQL and MySQL.

Moreover, this driver is integrated with GLORP, enabling a complete and open-source
solution to relational database access.  To do this, DBXTalk uses a library called OpenDBX.
\section{DBXTalk Driver Architecture}
The DBXTalk Driver relies on several components in order to connect to different relational databases:
\begin{itemize}
\item The OpenDBXDriver package talks to the OpenDBX library and handles the engines differences.
\item OpenDBX is a C library which stands as an adapter between the different database engines 
and our Pharo image, and provides a common interface to interact with through FFI.
\item We will need the corresponding client database library that OpenDBX will talk to.
\end{itemize}
\includegraphics[width=\linewidth]{dbx_architecture.png}


\section{Installing DBXTalk OpenDBX Driver}

In Order to install DBXTalk library, we need to install the previously detailed components.

\subsection*{Install OpenDBX Driver}

As we already introduced, an important part of DBXTalk architecture is the OpenDBX Driver, which allows us
to communicate with different relational database engines with a common approach.  OpenDBX is an open source library,
licensed as LGPL.

It's installation instructions for different engines and operative systems can be found on \url{http://www.linuxnetworks.de/doc/index.php/OpenDBX}.

You can also participate in the Issue Tracker and mailing list of this proyect to ask questions and contribute.

Issue Tracker: \url{http://bugs.linuxnetworks.de/index.php?project=3}
Mailing List: \url{https://lists.sourceforge.net/lists/listinfo/libopendbx-devel}

\subsection*{Install Smalltalk OpenDBXDriver}

The OpenDBXDriver written in Smalltalk is also needed in order to use the DBXTalk suite.  The easiest way to download it
is using it's metacello configuration, ensuring all it's dependencies to be loaded, such as FFI.  This can be performed executing
in a workspace the following script:

\begin{code}
Gofer it
	squeaksource: 'DBXTalk';
	package: 'ConfigurationOfOpenDBXDriver';
	load.
	
(((Smalltalk at: #ConfigurationOfOpenDBXDriver) perform: #project) perform: #version: with: #stable) load
\end{code}

\subsection*{Ensuring everything was ok}

The OpenDBXDriver package comes along with lots of tests cases you can use to test the your installation.  But before running
 the tests, you may want to configure the database connection settings in order to match your actual configuration.
To do that, just go to the corresponsing DBX*Engine*Facility class in the OpenDBXDriverTests package, and modify the
createConnection method to suite your needs.

For example, if I want to configure the tests to run in my MySql database, I should go to DBXMySQLFacility>>createConnection, to see the following:

\begin{code}
createConnection
    self connectionSettings: (DBXConnectionSettings
			    host: 'localhost'
			    port: '3306'
			    database: 'sodbxtest'
			    userName: 'sodbxtest'
			    userPassword: 'sodbxtest').
    self platform: DBXMySQLBackend new.
\end{code}

There I can change the host, port, database, username and password to connecto to my own database.

\section{Getting down to business}

Now you have already installed your database driver, you are ready to use it and build applications with it!  But you need to know the basics in order to get the mayor leagues.  In the following section you will be introduced in creating connections, execute SQL statements and handle transactions.

%=========================================================
\ifx\wholebook\relax\else
   \bibliographystyle{jurabib}
   \nobibliography{scg}
   \end{document}
\fi
%=========================================================



%%% Local Variables: 
%%% coding: utf-8
%%% mode: latex
%%% TeX-master: Lint.tex
%%% TeX-PDF-mode: t
%%% End:

