% $Author: polito $
% $Date: 2011-07-09 17:14:33 -0300 (Sat, 09 Jul 2011) $
% $Revision: 1 $

%=================================================================
\ifx\wholebook\relax\else
% --------------------------------------------
% Lulu:
	\documentclass[a4paper,10pt,twoside]{book}
	\usepackage[
		papersize={6.13in,9.21in},
		hmargin={.75in,.75in},
		vmargin={.75in,1in},
		ignoreheadfoot
	]{geometry}
	\input{../common.tex}
	\setboolean{lulu}{true}
% --------------------------------------------
% A4:
%	\documentclass[a4paper,11pt,twoside]{book}
%	\input{../common.tex}
%	\usepackage{a4wide}
% --------------------------------------------
    \graphicspath{{figures/} {../figures/}}
	\begin{document}
\fi
%=================================================================
%\renewmessage{\nnbb}[2]{} % Disable editorial comments
\sloppy

%=================================================================
%\renewcommand{\nnbb}[2]{#2} % Disable editorial comments

\chapter{DBXTalk}
DBXTalk is a database driver that allows interaction with major relational database engines such as
Oracle and MSSQL, apart from those which are open source, like PostgreSQL and MySQL.

Moreover, this driver is integrated with GLORP, enabling a complete and open-source
solution to relational database access.  To do this, DBXTalk uses a library called OpenDBX.
\section{DBXTalk Driver Architecture}
The DBXTalk Driver relies on several components in order to connect to different relational databases:
\begin{itemize}
\item The OpenDBXDriver package talks to the OpenDBX library and handles the engines differences.
\item OpenDBX is a C library which stands as an adapter between the different database engines 
and our Pharo image, and provides a common interface to interact with through FFI.
\item We will need the corresponding client database library that OpenDBX will talk to.
\end{itemize}
\includegraphics[width=\linewidth]{dbx_architecture.png}
\section{Installing DBXTalk}

This section will introduce the installation of the components needed to run DBXTalk in MS Windows, Unix and MacOS.


%=========================================================
\ifx\wholebook\relax\else
   \bibliographystyle{jurabib}
   \nobibliography{scg}
   \end{document}
\fi
%=========================================================



%%% Local Variables: 
%%% coding: utf-8
%%% mode: latex
%%% TeX-master: Lint.tex
%%% TeX-PDF-mode: t
%%% End:

