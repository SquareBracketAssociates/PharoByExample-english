% $Author$
% $Date$
% $Revision$

% HISTORY:
% Chapter started by Stef (2008-07-26)
% 2011-09-11 - Migrated to PharoBox: svn checkout https://XXX@scm.gforge.inria.fr/svn/pharobooks/PharoByExampleTwo-Eng

%=================================================================
\ifx\wholebook\relax\else
% --------------------------------------------
% Lulu:
	\documentclass[a4paper,10pt,twoside]{book}
	\usepackage[
		papersize={6.13in,9.21in},
		hmargin={.75in,.75in},
		vmargin={.75in,1in},
		ignoreheadfoot
	]{geometry}
	\input{../common.tex}
	\setboolean{lulu}{true}
% --------------------------------------------
% A4:
%	\documentclass[a4paper,11pt,twoside]{book}
%	\input{../common.tex}
%	\usepackage{a4wide}
% --------------------------------------------
    \graphicspath{{figures/} {figures/}}
	\begin{document}
\fi
%=================================================================
%\renewcommand{\nnbb}[2]{} % Disable editorial comments
\sloppy
%=================================================================
\chapter{Announcements: an Object Notification Framework}\chalabel{announcements}

It is often necessary to get notified when an action has been performed. This can be simply done by having an instance variable pointing to the objects to be notified. However such solution is not optimal because of the coupling it introduces between the objects. In this Chapter, we present 
Announcements, a framework to notify objects based on a registration mechanism and first class announcements. Indeed contrary to the traditional Smalltalk dependency mechanism which broadcast simple symbol, Announcement notification are full objects. Announcement can be the basis to Observer Design pattern.


\section{A word about object coupling}
\sd{should rewrite to get it more about objects and not components}
Here is an interesting question that often comes up often when writing
components. It is one that we faced when embedding our components. How
do the components communicate with each other in a way that doesn't
bind them together explicitly? That is, how does a child component
send a message to its parent component without explicitly knowing who
the parent is? Designing a component to refer to its parent is just a
part of the solution since the interfaces of different parents may be
different which would prevent the component from being reused in
different contexts.

There is a solution based on explicit dependencies also called the
change/update mechanism. Since early versions of Smalltalk, a
dependency mechanism based on a change/update protocol is available
and it is the foundation of the MVC framework itself. A component
registers its interest in some event and that event triggers a
notification. This is the basis of the Observer design pattern. 


\section{Announcements}
Announcement developed originally by Vassili Bykov is a new framework that offers notifications with full objects.  While the original dependency framework relied on symbols for the event registration and notification, Announcements promotes an object-oriented solution. Events are plain objects.

The main idea behind the framework is to set up announcers, define or
reuse some announcements, let clients register interest in events and
signal events. An announcement is an object representing an occurrence of a
specific event. It is the place to define all the information related
to the event occurrence. An announcer is responsible for registering
interested clients and announcing events, see Figure~\ref{fig:announcementFlow}.

\begin{figure}[ht]\centering
	\includegraphics[width=8cm]{AnnouncementFlow2}
	\caption{Announcement flow}
	\label{fig:announcementFlow}
\end{figure} 

\subsection{API}

Any object interested in an announcement registers its interest
using the method \mthind{on:do:}{on: anAnnouncementClass do: aBlock}
or \mthind{on: anAnnouncementClass send: aSelector to: anObject}{on:
  anAnnouncementClass send: aSelector to: anObject}.  The messages
\ct{on:do:} and \ct{on:send:to:} are strictly equivalent to the
messages \mthind{subscribe:do:}{subscribe: anAnnouncementClass do:
  aValuable}  and \mthind{subscribe:send:to:}{subscribe: anAnnouncementClass send:
  aSelector to: anObject}. You can also ask an announcer to
\ct{unsubcribe:} an object.


\begin{code}{Registering to an announcement}
	anAnnouncer on: MyAnnouncement do: [:announcement | announcement doSomething ]
\end{code}


\begin{code}{Emitting an announcement}
	anAnnouncer announce: (MyAnnouncement new)
\end{code}

\begin{method}{Unsubscribing from the announcer}
    anAnnouncer unsubscribe: self
\end{method}








%=====
%BEN


\section{Weak Announcements}

Even if you do not bind the emitter with the receiver, you create a dependency between to the announcer by registering yourself. And because announcers keep a reference to subscribers, this dependency prevent your registered objects to be garbage collected.

To solve this problem, weak announcements have been implemented. A weak announcement acts like regular announcements but keeps only a weak reference to subscribers, allowing them to be garbage collected.

\subsection{How to subscribe to weak announcement}

This protocol is really simple, because it's almost the same than for regular announcements.

When you use to do

\begin{code}{}
self session announcer on: RemoveChild send: #removeChild: to: self
\end{code}

you just do

\begin{code}{}
self session announcer !\textbf{weak}! on: RemoveChild send: #removeChild: to: self
\end{code}

The major difference is that you can't use the method \textbf{when:send:} anymore. Indeed if a weak announcement store a block, it will also store the block context and due to that, keep a strong reference to the subscriber.

\subsection{System Announcements}

We saw that you can create your own announcements to communicate between objects, but there are already existing announcements. All subclasses of \ct{SystemAnnouncement} are managed by the system to inform that it changed.

%\paragraph{SystemCategoryAddedAnnouncement}
%This announcement is emitted when a category is added to the system, by example using \mbox{\ct{SystemOrganizer>>#addCategory:}}
%
%\paragraph{SystemCategoryRemovedAnnouncement}
%This announcement is emitted when a category is removed to the system, by example using \mbox{\ct{SystemOrganizer>>#removeCategory:}}
%
%\paragraph{SystemCategoryRenamedAnnouncement}
%This announcement is emitted when a category is renamed to the system, by example using \mbox{\ct{SystemOrganizer>>#renameCategory:toBe:}}
%
%\paragraph{SystemClassAddedAnnouncement}
%This announcement is emitted when a class or a trait is added to the system, by example using \mbox{\ct{Trait>>#named:}} or \mbox{\ct{Class>>#subclass:}}
%
%\paragraph{SystemClassCommentedAnnouncement}
%This announcement is emitted twice when a class or a trait is commented, by example using \mbox{\ct{Trait>>#comment:}} or \mbox{\ct{Class>>#comment:}}
%
%\paragraph{SystemClassRecategorizedAnnouncement}
%This announcement is emitted when a class or a trait is set, by example using \mbox{\ct{Trait>>#category:}} or \mbox{\ct{Class>>#category:}}
%
%\paragraph{SystemClassRemovedAnnouncement}
%This announcement is emitted when a class or a trait is deleted from the system, by example using \mbox{\ct{Trait>>#removeFromSystem}} or \mbox{\ct{Class>>#removeFromSystem}}
%
%\paragraph{SystemClassRenamedAnnouncement}
%This announcement is emitted when a class or a trait is renamed, by example using \mbox{\ct{Trait>>#rename:}} or \mbox{\ct{Class>>#rename:}}
%
%\paragraph{SystemClassReorganizedAnnouncement}
%This announcement is emitted when a class or a trait's protocol is changed, by example using \mbox{\ct{Trait>>#removeCategory:}} or \mbox{\ct{Class>>#removeCategory:}}


\begin{code}{SystemAnnouncement hierarchy}
SystemAnnouncement
			SystemCategoryAddedAnnouncement
			SystemCategoryRemovedAnnouncement
			SystemCategoryRenamedAnnouncement
			SystemClassAddedAnnouncement
			SystemClassCommentedAnnouncement
			SystemClassRecategorizedAnnouncement
			SystemClassRemovedAnnouncement
			SystemClassRenamedAnnouncement
			SystemClassReorganizedAnnouncement
			SystemMethodAddedAnnouncement
			SystemMethodModifiedAnnouncement
			SystemMethodRecategorizedAnnouncement
			SystemMethodRemovedAnnouncement
			SystemProtocolAddedAnnouncement
			SystemProtocolRemovedAnnouncement
			SystemUnknownAnnouncement
\end{code}

\section{Discussions}
Announcements are like exceptions because they notify about events, but they are different in the nature of relationship between the announcer and the handler. Exceptions are for communication along the sender stack chain. There is no explicit connection between the announcer and the handler apart from the fact that the handler calls the announcer, directly or indirectly. Announcements are for communications across the object network, with a connection between the announcer and the handler pre-established by subscribing.

%=====


%=============================================================
\ifx\wholebook\relax\else
   \bibliographystyle{jurabib}
   \nobibliography{scg}
   \end{document}
\fi
%=============================================================


\subsection{Example in Seaside} 

\sd{I would remove this example -  I know I wrote it but....}Here is an example taken from Ramon Leon's
very good Smalltalk blog. This example shows how we can use
announcements to manage the communication between a parent component
and its children as for example in the context of a menu and its menu
items.

\begin{method}{Defining an announcer.}
MySession>>announcer
    ^ announcer ifNil: [announcer := Announcer new]
\end{method}

Then subclass \clsind{Announcement} for any interesting thing that
might happen. The announcement subclass is the place to add any extra
information about the specific announcement such as a context, the
objects involved \etc This is why announcement objects are both more
powerful and simpler than using symbols.

\begin{classdef}{Defining a specific Announcement.}
Announcement subclass: #RemoveChild
    instanceVariableNames: 'child'

RemoveChild class>>child: aChild
    ^self new
        child: aChild;
        yourself

RemoveChild>>child: anChild
    child := anChild

RemoveChild>>child
    ^child
\end{classdef}

Any component interested in this announcement registers its interest
using the method \mthind{on:do:}{on: anAnnouncementClass do: aBlock}
or \mthind{on: anAnnouncementClass send: aSelector to: anObject}{on:
  anAnnouncementClass send: aSelector to: anObject}.  The messages
\ct{on:do:} and \ct{on:send:to:} are strictly equivalent to the
messages \mthind{subscribe:do:}{subscribe: anAnnouncementClass do:
  aValuable}  and
\mthind{subscribe:send:to:}{subscribe: anAnnouncementClass send:
  aSelector to: anObject}. You can also ask an announcer to
\ct{unsubcribe:} an object.

In the following example, when a parent component is created, it
expresses interest in the \clsind{RemoveChild} event and specifies the
action that it will perform when such an event happens.

\begin{method}{}
Parent>>initialize
    super initialize.
    !\textbf{self session announcer on: RemoveChild do: [:it | self removeChild: it child] }!

Parent>>removeChild: aChild
    self children remove: aChild
\end{method}

And any component that wants to fire this event simply announces it by
sending in an instance of that custom announcement object.

\begin{method}{Announcing an event.}
Child>>removeMe
    self session announcer announce: (RemoveChild child: self)
\end{method}

Note that depending on where you place the announcer, you can even
have different sessions sending events to each other, or different
applications.

%=====
%BEN

Finally, when you have done with announcement, you may want to unsubscribe your object from the announcer.

\begin{method}{Unsubscribing from the announcer}
Child>>unsubscribe
    self session announcer unsubscribe: self
\end{method}

%=====

Announcements are not always the best way to establish communication
between components.  On one hand, announcements let you create loosely
coupled components and thus maximize reusability.  On the other hand,
they introduce additional complexity when you may be able solve your
communication problem with a simple message send.



%-----------------------------------------------------------------

%%% Local Variables:
%%% coding: utf-8
%%% mode: latex
%%% TeX-master: t
%%% TeX-PDF-mode: t
%%% ispell-local-dictionary: "english"
%%% End: