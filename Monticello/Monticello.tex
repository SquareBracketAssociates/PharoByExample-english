% $Author $
% $Date$
% $Revision$

% HISTORY:
% 2007-10-29 - Alex first draft
% 2007-12-18 - Andrew review pass
% 2008-05-30 - Stef review pass
% 2009-04-22 - Oscar revised
% 2009-04-23 - Alex revised (noted Squeak vs Pharo)
% 2009-06-17 - Oscar migrated to Pharo
% 2009-07-15 - Oscar indexing
% 2011-09-11 - Migrated to PharoBox: svn checkout https://XXX@scm.gforge.inria.fr/svn/pharobooks/PharoByExampleTwo-Eng



%=================================================================
\ifx\wholebook\relax\else
% --------------------------------------------
% Lulu:
	\documentclass[a4paper,10pt,twoside]{book}
	\usepackage[
		papersize={6.13in,9.21in},
		hmargin={.75in,.75in},
		vmargin={.75in,1in},
		ignoreheadfoot
	]{geometry}
	% $Author$
% $Date$
% $Revision$

% HISTORY:
% 2006-10-31 - Oscar code macros
% ...

%=============================================================
% NB: documentclass must be set in main document.
% Allows book to be generated in multiple formats.
%=============================================================
%:Packages
\usepackage[T1]{fontenc}  %%%%%% really important to get the code directly in the text!
\usepackage{lmodern}
%\usepackage[scaled=0.85]{bookmanx} % needs another scale factor if used with \renewcommand{\sfdefault}{cmbr}
\usepackage{palatino}
\usepackage[scaled=0.85]{helvet}
\usepackage[protrusion,expansion=false]{microtype}
\usepackage{graphicx}
\usepackage{theorem}
\usepackage[english]{babel}
% ON: pdfsync breaks the use of p{width} for tabular columns!
\ifdefined\usepdfsync\usepackage{pdfsync}\fi % Requires texlive 2007
%=============================================================
%:More packages
%Stef should check which ones are used!
%\usepackage{picinpar}
%\usepackage{layout}
%\usepackage{color}
%\usepackage{enum}
%\usepackage{a4wide}
% \usepackage{fancyhdr}
\usepackage{ifthen}
\usepackage{float}
\usepackage{longtable}
\usepackage{makeidx}
\usepackage[nottoc]{tocbibind}
\usepackage{multicol}
\usepackage{booktabs}	% book-style tables
\usepackage{topcapt}	% enables \topcaption
\usepackage{multirow}
\usepackage{tabularx}
%\usepackage[bottom]{footmisc}
\usepackage{xspace}
\usepackage{alltt}
\usepackage{amssymb,textcomp}
\usepackage[usenames,dvipsnames]{color}
%\usepackage{colortbl}
\usepackage[hang]{subfigure}\makeatletter\def\p@subfigure{\thefigure\,}\makeatother
\usepackage{rotating}
\usepackage{enumitem}	% apb: allows more control over tags in enumerations
\usepackage{verbatim}     % for comment environment
\usepackage{varioref}	% for page references that work
\labelformat{footnote}{\thechapter--#1} % to distinguish citations from jurabib
\usepackage{needspace}
\usepackage{isodateo} % enable \isodate
\usepackage[newparttoc]{titlesec}
\usepackage{titletoc}
\usepackage{wrapfig}
\usepackage[
	super,
	citefull=first,
	authorformat={allreversed,and},
	titleformat={commasep,italic}
]{jurabib} % citations as footnotes
\usepackage[
	colorlinks=true,
	linkcolor=black,
	urlcolor=black,
	citecolor=black
]{hyperref}   % should come last
%=============================================================
%:PDF version
\pdfminorversion=3 % Set PDF to 1.3 for Lulu
%=============================================================
%:URL style
\makeatletter
\def\url@leostyle{%
  \@ifundefined{selectfont}{\def\UrlFont{\sf}}{\def\UrlFont{\sffamily}}}
\makeatother
% Now actually use the newly defined style.
\urlstyle{leo}
%=============================================================
%:Booleans
\newboolean{lulu}
\setboolean{lulu}{false}
\newcommand{\ifluluelse}[2]{\ifthenelse{\boolean{lulu}}{#1}{#2}}
%=============================================================
%:Names
\newcommand{\SUnit}{SUnit\xspace}
\newcommand{\sunit}{SUnit\xspace}
\newcommand{\xUnit}{$x$Unit\xspace}
\newcommand{\JUnit}{JUnit\xspace}
\newcommand{\st}{Smalltalk\xspace}
\newcommand{\pharo}{Pharo\xspace} % Use this, not \Pharo
%\newcommand{\sqmap}{SqueakMap\xspace}
\newcommand{\squeak}{Squeak\xspace} % use this, not \Squeak or \sq
\newcommand{\sqsrc}{SqueakSource\xspace}
\newcommand{\sbe}{\url{http://SqueakByExample.org}\xspace}
\newcommand{\pharoweb}{\url{http://pharo-project.org}\xspace}
\newcommand{\pbe}{\url{http://PharoByExample.org}\xspace}
\newcommand{\sba}{\url{http://SquareBracketAssociates.org}\xspace}
\newcommand{\bam}{\lct{Bounc\-ing\-Atoms\-Morph}\xspace}
%=============================================================
%:Markup macros for proof-reading
\usepackage[normalem]{ulem} % for \sout
\usepackage{xcolor}
\newcommand{\ra}{$\rightarrow$}
\newcommand{\ugh}[1]{\textcolor{red}{\uwave{#1}}} % please rephrase
\newcommand{\ins}[1]{\textcolor{blue}{\uline{#1}}} % please insert
\newcommand{\del}[1]{\textcolor{red}{\sout{#1}}} % please delete
\newcommand{\chg}[2]{\textcolor{red}{\sout{#1}}{\ra}\textcolor{blue}{\uline{#2}}} % please change
%=============================================================
%:Editorial comment macros
%\newcommand{\nnbb}[2]{
%    % \fbox{\bfseries\sffamily\scriptsize#1}
%    \fcolorbox{gray}{yellow}{\bfseries\sffamily\scriptsize#1}
%    {\sf\small$\blacktriangleright$\textit{#2}$\blacktriangleleft$}
%   }
\newcommand{\yellowbox}[1]{\fcolorbox{gray}{yellow}{\bfseries\sffamily\scriptsize#1}}
\newcommand{\triangles}[1]{{\sf\small$\blacktriangleright$\textit{#1}$\blacktriangleleft$}}
\newcommand{\nnbb}[2]{\yellowbox{#1} \triangles{#2}}
\newcommand{\fix}{\yellowbox{FIX!}}
\newcommand{\here}{\yellowbox{CONTINUE HERE!}}
% editor macros
\newcommand{\apl}[1]{\nnbb{Alain}{#1}} % Alain
\newcommand{\ab}[1]{\nnbb{Andrew}{#1}} % Black
\newcommand{\sd}[1]{\nnbb{St\'{e}f}{#1}} % Ducasse
\newcommand{\dc}[1]{\nnbb{DamienC}{#1}} % Ducasse
\newcommand{\md}[1]{\nnbb{Marcus}{#1}} % Denker
\newcommand{\on}[1]{\nnbb{Oscar}{#1}} % Nierstrasz
\newcommand{\damien}[1]{\nnbb{Damien}{#1}} % Pollet
\newcommand{\lr}[1]{\nnbb{Lukas}{#1}} % Renggli
\newcommand{\orla}[1]{\nnbb{Orla}{#1}} % Greevy
\newcommand{\alex}[1]{\nnbb{Alex}{#1}} % Bergel
\newcommand{\alx}[1]{\nnbb{Alex}{#1}} % Bergel
\newcommand{\dr}[1]{\nnbb{David}{#1}} % Roethlisberger
\newcommand{\ja}[1]{\nnbb{Jannik}{#1}} % Laval
\newcommand{\jr}[1]{\nnbb{Jorge}{#1}} % Ressia
\newcommand{\jb}[1]{\nnbb{JB}{#1}} % JB
\newcommand{\fp}[1]{\nnbb{Fabrizio}{#1}} % Perin
\newcommand{\michael}[1]{\nnbb{Michael}{#1}} % Davies
\newcommand{\ew}[1]{\nnbb{Erwann}{#1}} % Wernli
\newcommand{\mb}[1]{\nnbb{Martial}{#1}} % Boniou
\newcommand{\hw}[1]{\nnbb{Hernan}{#1}} % Wilkinson
\newcommand{\ben}[1]{\nnbb{Benjamin}{#1}} % Benjamin Van Ryseghem
\newcommand{\mmp}[1]{\nnbb{Mariano}{#1}} % Mariano Martinez Peck
%=============================================================
%:Abbreviation macros
\newcommand{\ie}{\emph{i.e.},\xspace}
\newcommand{\eg}{\emph{e.g.},\xspace}
\newcommand{\etc}{etc.\xspace}
%=============================================================
%:Cross reference macros
\newcommand{\charef}[1]{Chapter~\ref{cha:#1}\xspace}
\newcommand{\secref}[1]{Section~\ref{sec:#1}\xspace}
\newcommand{\figref}[1]{Figure~\ref{fig:#1}\xspace}
\newcommand{\Figref}[1]{Figure~\ref{fig:#1}\xspace}
\newcommand{\appref}[1]{Appendix~\ref{app:#1}\xspace}
\newcommand{\tabref}[1]{Table~\ref{tab:#1}\xspace}
\newcommand{\faqref}[1]{FAQ~\ref{faq:#1}, p.~\pageref{faq:#1}\xspace}
% APB: I removed trailing \xspace commands from these macros because
% \xspace mostly doesn't work.  If you want a space after your
% references, type one!
% ON: xspace has always worked just fine for me!  Please leave them in.
%
\newcommand{\ruleref}[1]{\ref{rule:#1}\xspace}
%
\newcommand{\egref}[1]{example~\ref{eg:#1}\xspace}
\newcommand{\Egref}[1]{Example~\ref{eg:#1}\xspace}
%
\newcommand{\scrref}[1]{script~\ref{scr:#1}\xspace}
\newcommand{\Scrref}[1]{Script~\ref{scr:#1}\xspace}
\newcommand{\tscrref}[1]{the script~\ref{scr:#1}\xspace}
\newcommand{\Tscrref}[1]{The script~\ref{scr:#1}\xspace}
%
\newcommand{\mthref}[1]{method~\ref{mth:#1}\xspace}
\newcommand{\mthsref}[1]{methods~\ref{mth:#1}\xspace}
\newcommand{\Mthref}[1]{Method~\ref{mth:#1}\xspace}
\newcommand{\tmthref}[1]{the method~\ref{mth:#1}\xspace}
\newcommand{\Tmthref}[1]{The method~\ref{mth:#1}\xspace}
%
\newcommand{\clsref}[1]{class~\ref{cls:#1}\xspace}
\newcommand{\tclsref}[1]{the class~\ref{cls:#1}\xspace}
\newcommand{\Tclsref}[1]{The class~\ref{cls:#1}\xspace}

\newcommand{\chalabel}[1]{\label{cha:#1}}
\newcommand{\seclabel}[1]{\label{sec:#1}}
\newcommand{\figlabel}[1]{\label{fig:#1}}
\newcommand{\tablabel}[1]{\label{tab:#1}}
\newcommand{\rulelabel}[1]{\label{rule:#1}}
\newcommand{\eglabel}[1]{\label{eg:#1}}
\newcommand{\scrlabel}[1]{\label{scr:#1}}
\newcommand{\mthlabel}[1]{\label{mth:#1}}
\newcommand{\clslabel}[1]{\label{cls:#1}}
\newcommand{\faqlabel}[1]{\label{faq:#1}}
%=============================================================
%:Menu item macro
% for menu items, so we can change our minds on how to print them! (apb)
\definecolor{lightgray}{gray}{0.89}
\newcommand{\menu}[1]{{%
	\setlength{\fboxsep}{0pt}%
	\colorbox{lightgray}{{{\upshape\sffamily\strut \,#1\,}}}}}
\newcommand{\link}[1]{{%
	\fontfamily{lmr}\selectfont
 	\upshape{\sffamily \underline{#1}}}}
% For submenu items:
\newcommand{\go}{\,$\triangleright$\,}
% \newcommand{\go}{\,$\blacktriangleright$\,}
% For keyboard shortcuts:
%\newcommand{\short}[1]{\mbox{$\langle${\sc CMD}$\rangle$-#1}\xspace}
\newcommand{\short}[1]{\mbox{{\sc cmd}\hspace{0.08em}--\hspace{0.09em}#1}\xspace}
% For buttons:
\newcommand{\button}[1]{{%
	\setlength{\fboxsep}{0pt}%
	\fbox{{\upshape\sffamily\strut \,#1\,}}}}
% NB: The button macro does not work within captions -- incompatible with xcolor package :-(
\newcommand{\toolsflap}{\textit{Tools} flap\xspace}
%=============================================================
%:Mouse clicks
\newcommand{\click}{click\xspace} % RED
\newcommand{\actclick}{action-click\xspace} % YELLOW
\newcommand{\metaclick}{meta-click\xspace} % BLUE
\newcommand{\Click}{Click\xspace} % RED
\newcommand{\Actclick}{Action-click\xspace} % YELLOW
\newcommand{\Metaclick}{Meta-click\xspace} % BLUE
%=============================================================
%:ToSh macros
\newboolean{tosh}
\setboolean{tosh}{false}
\newcommand{\iftoshelse}[2]{\ifthenelse{\boolean{tosh}}{#1}{#2}}
%=============================================================
%:ToSh colors
%\newcommand{\highlightcolor}{\color{blue!65}}
%\newcommand{\boxcolor}{\color{gray!25}}
\newcommand{\highlight}[1]{\textcolor{blue!65}{#1}}
%\newcommand{\codecolor}{\color{blue!65}}
%%\setlength{\fboxrule}{2pt}
%\newcommand{\asPict}[1]{%
%	{\Large\highlight{#1}}}
%=============================================================
%:Reader cues (do this)
%
% Indicate something the reader should try out.
% \newcommand{\dothisicon}{\raisebox{-.5ex}{\includegraphics[width=1.4em]{squeak-logo}}}
\iftoshelse{
	\usepackage{marginnote}
		\renewcommand*{\marginfont}{\footnotesize}
	\newcommand{\vartriangleout}{\ifthenelse{\isodd{\thepage}}{\vartriangleright}{\vartriangleleft}}
	\newcommand{\dothisicon}{\fcolorbox{blue!65}{white}{\highlight{$\vartriangleout$}}}
	\newcommand{\dothis}[1]{%
		\noindent\par\noindent
		{\reversemarginpar
			\marginnote{\fcolorbox{blue!65}{white}{\highlight{$\vartriangleout$}}}}
		%\MarginLabel{do this}
		\noindent\emph{#1}
		\nopagebreak}
}{
	\newcommand{\dothisicon}{\raisebox{-.5ex}{\includegraphics[height=1.2em]{pharo}}}
	\newcommand{\dothis}[1]{%
		\medskip
		\noindent\dothisicon
		\ifx#1\empty\else\quad\emph{#1}\fi
		\par\smallskip\nopagebreak}
}
%===> NEW VERSION <===
% NB: To use this in an individual chapter, you must set:
%\graphicspath{{figures/} {../figures/}}
% at the head of the chapter.  Don't forget the final /
%=============================================================
%:Reader hints (hint)
%
% Indicates a non-obvious consequence 
\newcommand{\hint}[1]{\vspace{1ex}\noindent\fbox{\textsc{Hint}} \emph{#1}}
%=================================================================
% graphics for Morphic handles
\newcommand{\grabHandle}{\raisebox{-0.2ex}{\includegraphics[width=1em]{blackHandle}}}
\newcommand{\moveHandle}{\raisebox{-0.2ex}{\includegraphics[width=1em]{moveHandle}}}
\newcommand{\debugHandle}{\raisebox{-0.2ex}{\includegraphics[width=1em]{debugHandle}}}
%=============================================================
%:Highlighting Important stuff (doublebox)
%
% From Seaside book ...
\newsavebox{\SavedText}
\newlength{\InnerBoxRule}\setlength{\InnerBoxRule}{.75\fboxrule}
\newlength{\OuterBoxRule}\setlength{\OuterBoxRule}{1.5\fboxrule}
\newlength{\BoxSeparation}\setlength{\BoxSeparation}{1.5\fboxrule}
\addtolength{\BoxSeparation}{.5pt}
\newlength{\SaveBoxSep}\setlength{\SaveBoxSep}{2\fboxsep}
%
\newenvironment{doublebox}{\begin{lrbox}{\SavedText}
    \begin{minipage}{.75\textwidth}}
    {\end{minipage}\end{lrbox}\begin{center}
    \setlength{\fboxsep}{\BoxSeparation}\setlength{\fboxrule}{\OuterBoxRule}
    \fbox{\setlength{\fboxsep}{\SaveBoxSep}\setlength{\fboxrule}{\InnerBoxRule}%
      \fbox{\usebox{\SavedText}}}
  \end{center}}
% Use this:
\newcommand{\important}[1]{\begin{doublebox}#1\end{doublebox}}
%=============================================================
%:Section depth
\setcounter{secnumdepth}{2}
%% for this to happen start the file with
%\ifx\wholebook\relax\else
%% $Author$
% $Date$
% $Revision$

% HISTORY:
% 2006-10-31 - Oscar code macros
% ...

%=============================================================
% NB: documentclass must be set in main document.
% Allows book to be generated in multiple formats.
%=============================================================
%:Packages
\usepackage[T1]{fontenc}  %%%%%% really important to get the code directly in the text!
\usepackage{lmodern}
%\usepackage[scaled=0.85]{bookmanx} % needs another scale factor if used with \renewcommand{\sfdefault}{cmbr}
\usepackage{palatino}
\usepackage[scaled=0.85]{helvet}
\usepackage[protrusion,expansion=false]{microtype}
\usepackage{graphicx}
\usepackage{theorem}
\usepackage[english]{babel}
% ON: pdfsync breaks the use of p{width} for tabular columns!
\ifdefined\usepdfsync\usepackage{pdfsync}\fi % Requires texlive 2007
%=============================================================
%:More packages
%Stef should check which ones are used!
%\usepackage{picinpar}
%\usepackage{layout}
%\usepackage{color}
%\usepackage{enum}
%\usepackage{a4wide}
% \usepackage{fancyhdr}
\usepackage{ifthen}
\usepackage{float}
\usepackage{longtable}
\usepackage{makeidx}
\usepackage[nottoc]{tocbibind}
\usepackage{multicol}
\usepackage{booktabs}	% book-style tables
\usepackage{topcapt}	% enables \topcaption
\usepackage{multirow}
\usepackage{tabularx}
%\usepackage[bottom]{footmisc}
\usepackage{xspace}
\usepackage{alltt}
\usepackage{amssymb,textcomp}
\usepackage[usenames,dvipsnames]{color}
%\usepackage{colortbl}
\usepackage[hang]{subfigure}\makeatletter\def\p@subfigure{\thefigure\,}\makeatother
\usepackage{rotating}
\usepackage{enumitem}	% apb: allows more control over tags in enumerations
\usepackage{verbatim}     % for comment environment
\usepackage{varioref}	% for page references that work
\labelformat{footnote}{\thechapter--#1} % to distinguish citations from jurabib
\usepackage{needspace}
\usepackage{isodateo} % enable \isodate
\usepackage[newparttoc]{titlesec}
\usepackage{titletoc}
\usepackage{wrapfig}
\usepackage[
	super,
	citefull=first,
	authorformat={allreversed,and},
	titleformat={commasep,italic}
]{jurabib} % citations as footnotes
\usepackage[
	colorlinks=true,
	linkcolor=black,
	urlcolor=black,
	citecolor=black
]{hyperref}   % should come last
%=============================================================
%:PDF version
\pdfminorversion=3 % Set PDF to 1.3 for Lulu
%=============================================================
%:URL style
\makeatletter
\def\url@leostyle{%
  \@ifundefined{selectfont}{\def\UrlFont{\sf}}{\def\UrlFont{\sffamily}}}
\makeatother
% Now actually use the newly defined style.
\urlstyle{leo}
%=============================================================
%:Booleans
\newboolean{lulu}
\setboolean{lulu}{false}
\newcommand{\ifluluelse}[2]{\ifthenelse{\boolean{lulu}}{#1}{#2}}
%=============================================================
%:Names
\newcommand{\SUnit}{SUnit\xspace}
\newcommand{\sunit}{SUnit\xspace}
\newcommand{\xUnit}{$x$Unit\xspace}
\newcommand{\JUnit}{JUnit\xspace}
\newcommand{\st}{Smalltalk\xspace}
\newcommand{\pharo}{Pharo\xspace} % Use this, not \Pharo
%\newcommand{\sqmap}{SqueakMap\xspace}
\newcommand{\squeak}{Squeak\xspace} % use this, not \Squeak or \sq
\newcommand{\sqsrc}{SqueakSource\xspace}
\newcommand{\sbe}{\url{http://SqueakByExample.org}\xspace}
\newcommand{\pharoweb}{\url{http://pharo-project.org}\xspace}
\newcommand{\pbe}{\url{http://PharoByExample.org}\xspace}
\newcommand{\sba}{\url{http://SquareBracketAssociates.org}\xspace}
\newcommand{\bam}{\lct{Bounc\-ing\-Atoms\-Morph}\xspace}
%=============================================================
%:Markup macros for proof-reading
\usepackage[normalem]{ulem} % for \sout
\usepackage{xcolor}
\newcommand{\ra}{$\rightarrow$}
\newcommand{\ugh}[1]{\textcolor{red}{\uwave{#1}}} % please rephrase
\newcommand{\ins}[1]{\textcolor{blue}{\uline{#1}}} % please insert
\newcommand{\del}[1]{\textcolor{red}{\sout{#1}}} % please delete
\newcommand{\chg}[2]{\textcolor{red}{\sout{#1}}{\ra}\textcolor{blue}{\uline{#2}}} % please change
%=============================================================
%:Editorial comment macros
%\newcommand{\nnbb}[2]{
%    % \fbox{\bfseries\sffamily\scriptsize#1}
%    \fcolorbox{gray}{yellow}{\bfseries\sffamily\scriptsize#1}
%    {\sf\small$\blacktriangleright$\textit{#2}$\blacktriangleleft$}
%   }
\newcommand{\yellowbox}[1]{\fcolorbox{gray}{yellow}{\bfseries\sffamily\scriptsize#1}}
\newcommand{\triangles}[1]{{\sf\small$\blacktriangleright$\textit{#1}$\blacktriangleleft$}}
\newcommand{\nnbb}[2]{\yellowbox{#1} \triangles{#2}}
\newcommand{\fix}{\yellowbox{FIX!}}
\newcommand{\here}{\yellowbox{CONTINUE HERE!}}
% editor macros
\newcommand{\apl}[1]{\nnbb{Alain}{#1}} % Alain
\newcommand{\ab}[1]{\nnbb{Andrew}{#1}} % Black
\newcommand{\sd}[1]{\nnbb{St\'{e}f}{#1}} % Ducasse
\newcommand{\dc}[1]{\nnbb{DamienC}{#1}} % Ducasse
\newcommand{\md}[1]{\nnbb{Marcus}{#1}} % Denker
\newcommand{\on}[1]{\nnbb{Oscar}{#1}} % Nierstrasz
\newcommand{\damien}[1]{\nnbb{Damien}{#1}} % Pollet
\newcommand{\lr}[1]{\nnbb{Lukas}{#1}} % Renggli
\newcommand{\orla}[1]{\nnbb{Orla}{#1}} % Greevy
\newcommand{\alex}[1]{\nnbb{Alex}{#1}} % Bergel
\newcommand{\alx}[1]{\nnbb{Alex}{#1}} % Bergel
\newcommand{\dr}[1]{\nnbb{David}{#1}} % Roethlisberger
\newcommand{\ja}[1]{\nnbb{Jannik}{#1}} % Laval
\newcommand{\jr}[1]{\nnbb{Jorge}{#1}} % Ressia
\newcommand{\jb}[1]{\nnbb{JB}{#1}} % JB
\newcommand{\fp}[1]{\nnbb{Fabrizio}{#1}} % Perin
\newcommand{\michael}[1]{\nnbb{Michael}{#1}} % Davies
\newcommand{\ew}[1]{\nnbb{Erwann}{#1}} % Wernli
\newcommand{\mb}[1]{\nnbb{Martial}{#1}} % Boniou
\newcommand{\hw}[1]{\nnbb{Hernan}{#1}} % Wilkinson
\newcommand{\ben}[1]{\nnbb{Benjamin}{#1}} % Benjamin Van Ryseghem
\newcommand{\mmp}[1]{\nnbb{Mariano}{#1}} % Mariano Martinez Peck
%=============================================================
%:Abbreviation macros
\newcommand{\ie}{\emph{i.e.},\xspace}
\newcommand{\eg}{\emph{e.g.},\xspace}
\newcommand{\etc}{etc.\xspace}
%=============================================================
%:Cross reference macros
\newcommand{\charef}[1]{Chapter~\ref{cha:#1}\xspace}
\newcommand{\secref}[1]{Section~\ref{sec:#1}\xspace}
\newcommand{\figref}[1]{Figure~\ref{fig:#1}\xspace}
\newcommand{\Figref}[1]{Figure~\ref{fig:#1}\xspace}
\newcommand{\appref}[1]{Appendix~\ref{app:#1}\xspace}
\newcommand{\tabref}[1]{Table~\ref{tab:#1}\xspace}
\newcommand{\faqref}[1]{FAQ~\ref{faq:#1}, p.~\pageref{faq:#1}\xspace}
% APB: I removed trailing \xspace commands from these macros because
% \xspace mostly doesn't work.  If you want a space after your
% references, type one!
% ON: xspace has always worked just fine for me!  Please leave them in.
%
\newcommand{\ruleref}[1]{\ref{rule:#1}\xspace}
%
\newcommand{\egref}[1]{example~\ref{eg:#1}\xspace}
\newcommand{\Egref}[1]{Example~\ref{eg:#1}\xspace}
%
\newcommand{\scrref}[1]{script~\ref{scr:#1}\xspace}
\newcommand{\Scrref}[1]{Script~\ref{scr:#1}\xspace}
\newcommand{\tscrref}[1]{the script~\ref{scr:#1}\xspace}
\newcommand{\Tscrref}[1]{The script~\ref{scr:#1}\xspace}
%
\newcommand{\mthref}[1]{method~\ref{mth:#1}\xspace}
\newcommand{\mthsref}[1]{methods~\ref{mth:#1}\xspace}
\newcommand{\Mthref}[1]{Method~\ref{mth:#1}\xspace}
\newcommand{\tmthref}[1]{the method~\ref{mth:#1}\xspace}
\newcommand{\Tmthref}[1]{The method~\ref{mth:#1}\xspace}
%
\newcommand{\clsref}[1]{class~\ref{cls:#1}\xspace}
\newcommand{\tclsref}[1]{the class~\ref{cls:#1}\xspace}
\newcommand{\Tclsref}[1]{The class~\ref{cls:#1}\xspace}

\newcommand{\chalabel}[1]{\label{cha:#1}}
\newcommand{\seclabel}[1]{\label{sec:#1}}
\newcommand{\figlabel}[1]{\label{fig:#1}}
\newcommand{\tablabel}[1]{\label{tab:#1}}
\newcommand{\rulelabel}[1]{\label{rule:#1}}
\newcommand{\eglabel}[1]{\label{eg:#1}}
\newcommand{\scrlabel}[1]{\label{scr:#1}}
\newcommand{\mthlabel}[1]{\label{mth:#1}}
\newcommand{\clslabel}[1]{\label{cls:#1}}
\newcommand{\faqlabel}[1]{\label{faq:#1}}
%=============================================================
%:Menu item macro
% for menu items, so we can change our minds on how to print them! (apb)
\definecolor{lightgray}{gray}{0.89}
\newcommand{\menu}[1]{{%
	\setlength{\fboxsep}{0pt}%
	\colorbox{lightgray}{{{\upshape\sffamily\strut \,#1\,}}}}}
\newcommand{\link}[1]{{%
	\fontfamily{lmr}\selectfont
 	\upshape{\sffamily \underline{#1}}}}
% For submenu items:
\newcommand{\go}{\,$\triangleright$\,}
% \newcommand{\go}{\,$\blacktriangleright$\,}
% For keyboard shortcuts:
%\newcommand{\short}[1]{\mbox{$\langle${\sc CMD}$\rangle$-#1}\xspace}
\newcommand{\short}[1]{\mbox{{\sc cmd}\hspace{0.08em}--\hspace{0.09em}#1}\xspace}
% For buttons:
\newcommand{\button}[1]{{%
	\setlength{\fboxsep}{0pt}%
	\fbox{{\upshape\sffamily\strut \,#1\,}}}}
% NB: The button macro does not work within captions -- incompatible with xcolor package :-(
\newcommand{\toolsflap}{\textit{Tools} flap\xspace}
%=============================================================
%:Mouse clicks
\newcommand{\click}{click\xspace} % RED
\newcommand{\actclick}{action-click\xspace} % YELLOW
\newcommand{\metaclick}{meta-click\xspace} % BLUE
\newcommand{\Click}{Click\xspace} % RED
\newcommand{\Actclick}{Action-click\xspace} % YELLOW
\newcommand{\Metaclick}{Meta-click\xspace} % BLUE
%=============================================================
%:ToSh macros
\newboolean{tosh}
\setboolean{tosh}{false}
\newcommand{\iftoshelse}[2]{\ifthenelse{\boolean{tosh}}{#1}{#2}}
%=============================================================
%:ToSh colors
%\newcommand{\highlightcolor}{\color{blue!65}}
%\newcommand{\boxcolor}{\color{gray!25}}
\newcommand{\highlight}[1]{\textcolor{blue!65}{#1}}
%\newcommand{\codecolor}{\color{blue!65}}
%%\setlength{\fboxrule}{2pt}
%\newcommand{\asPict}[1]{%
%	{\Large\highlight{#1}}}
%=============================================================
%:Reader cues (do this)
%
% Indicate something the reader should try out.
% \newcommand{\dothisicon}{\raisebox{-.5ex}{\includegraphics[width=1.4em]{squeak-logo}}}
\iftoshelse{
	\usepackage{marginnote}
		\renewcommand*{\marginfont}{\footnotesize}
	\newcommand{\vartriangleout}{\ifthenelse{\isodd{\thepage}}{\vartriangleright}{\vartriangleleft}}
	\newcommand{\dothisicon}{\fcolorbox{blue!65}{white}{\highlight{$\vartriangleout$}}}
	\newcommand{\dothis}[1]{%
		\noindent\par\noindent
		{\reversemarginpar
			\marginnote{\fcolorbox{blue!65}{white}{\highlight{$\vartriangleout$}}}}
		%\MarginLabel{do this}
		\noindent\emph{#1}
		\nopagebreak}
}{
	\newcommand{\dothisicon}{\raisebox{-.5ex}{\includegraphics[height=1.2em]{pharo}}}
	\newcommand{\dothis}[1]{%
		\medskip
		\noindent\dothisicon
		\ifx#1\empty\else\quad\emph{#1}\fi
		\par\smallskip\nopagebreak}
}
%===> NEW VERSION <===
% NB: To use this in an individual chapter, you must set:
%\graphicspath{{figures/} {../figures/}}
% at the head of the chapter.  Don't forget the final /
%=============================================================
%:Reader hints (hint)
%
% Indicates a non-obvious consequence 
\newcommand{\hint}[1]{\vspace{1ex}\noindent\fbox{\textsc{Hint}} \emph{#1}}
%=================================================================
% graphics for Morphic handles
\newcommand{\grabHandle}{\raisebox{-0.2ex}{\includegraphics[width=1em]{blackHandle}}}
\newcommand{\moveHandle}{\raisebox{-0.2ex}{\includegraphics[width=1em]{moveHandle}}}
\newcommand{\debugHandle}{\raisebox{-0.2ex}{\includegraphics[width=1em]{debugHandle}}}
%=============================================================
%:Highlighting Important stuff (doublebox)
%
% From Seaside book ...
\newsavebox{\SavedText}
\newlength{\InnerBoxRule}\setlength{\InnerBoxRule}{.75\fboxrule}
\newlength{\OuterBoxRule}\setlength{\OuterBoxRule}{1.5\fboxrule}
\newlength{\BoxSeparation}\setlength{\BoxSeparation}{1.5\fboxrule}
\addtolength{\BoxSeparation}{.5pt}
\newlength{\SaveBoxSep}\setlength{\SaveBoxSep}{2\fboxsep}
%
\newenvironment{doublebox}{\begin{lrbox}{\SavedText}
    \begin{minipage}{.75\textwidth}}
    {\end{minipage}\end{lrbox}\begin{center}
    \setlength{\fboxsep}{\BoxSeparation}\setlength{\fboxrule}{\OuterBoxRule}
    \fbox{\setlength{\fboxsep}{\SaveBoxSep}\setlength{\fboxrule}{\InnerBoxRule}%
      \fbox{\usebox{\SavedText}}}
  \end{center}}
% Use this:
\newcommand{\important}[1]{\begin{doublebox}#1\end{doublebox}}
%=============================================================
%:Section depth
\setcounter{secnumdepth}{2}
%% for this to happen start the file with
%\ifx\wholebook\relax\else
%% $Author$
% $Date$
% $Revision$

% HISTORY:
% 2006-10-31 - Oscar code macros
% ...

%=============================================================
% NB: documentclass must be set in main document.
% Allows book to be generated in multiple formats.
%=============================================================
%:Packages
\usepackage[T1]{fontenc}  %%%%%% really important to get the code directly in the text!
\usepackage{lmodern}
%\usepackage[scaled=0.85]{bookmanx} % needs another scale factor if used with \renewcommand{\sfdefault}{cmbr}
\usepackage{palatino}
\usepackage[scaled=0.85]{helvet}
\usepackage[protrusion,expansion=false]{microtype}
\usepackage{graphicx}
\usepackage{theorem}
\usepackage[english]{babel}
% ON: pdfsync breaks the use of p{width} for tabular columns!
\ifdefined\usepdfsync\usepackage{pdfsync}\fi % Requires texlive 2007
%=============================================================
%:More packages
%Stef should check which ones are used!
%\usepackage{picinpar}
%\usepackage{layout}
%\usepackage{color}
%\usepackage{enum}
%\usepackage{a4wide}
% \usepackage{fancyhdr}
\usepackage{ifthen}
\usepackage{float}
\usepackage{longtable}
\usepackage{makeidx}
\usepackage[nottoc]{tocbibind}
\usepackage{multicol}
\usepackage{booktabs}	% book-style tables
\usepackage{topcapt}	% enables \topcaption
\usepackage{multirow}
\usepackage{tabularx}
%\usepackage[bottom]{footmisc}
\usepackage{xspace}
\usepackage{alltt}
\usepackage{amssymb,textcomp}
\usepackage[usenames,dvipsnames]{color}
%\usepackage{colortbl}
\usepackage[hang]{subfigure}\makeatletter\def\p@subfigure{\thefigure\,}\makeatother
\usepackage{rotating}
\usepackage{enumitem}	% apb: allows more control over tags in enumerations
\usepackage{verbatim}     % for comment environment
\usepackage{varioref}	% for page references that work
\labelformat{footnote}{\thechapter--#1} % to distinguish citations from jurabib
\usepackage{needspace}
\usepackage{isodateo} % enable \isodate
\usepackage[newparttoc]{titlesec}
\usepackage{titletoc}
\usepackage{wrapfig}
\usepackage[
	super,
	citefull=first,
	authorformat={allreversed,and},
	titleformat={commasep,italic}
]{jurabib} % citations as footnotes
\usepackage[
	colorlinks=true,
	linkcolor=black,
	urlcolor=black,
	citecolor=black
]{hyperref}   % should come last
%=============================================================
%:PDF version
\pdfminorversion=3 % Set PDF to 1.3 for Lulu
%=============================================================
%:URL style
\makeatletter
\def\url@leostyle{%
  \@ifundefined{selectfont}{\def\UrlFont{\sf}}{\def\UrlFont{\sffamily}}}
\makeatother
% Now actually use the newly defined style.
\urlstyle{leo}
%=============================================================
%:Booleans
\newboolean{lulu}
\setboolean{lulu}{false}
\newcommand{\ifluluelse}[2]{\ifthenelse{\boolean{lulu}}{#1}{#2}}
%=============================================================
%:Names
\newcommand{\SUnit}{SUnit\xspace}
\newcommand{\sunit}{SUnit\xspace}
\newcommand{\xUnit}{$x$Unit\xspace}
\newcommand{\JUnit}{JUnit\xspace}
\newcommand{\st}{Smalltalk\xspace}
\newcommand{\pharo}{Pharo\xspace} % Use this, not \Pharo
%\newcommand{\sqmap}{SqueakMap\xspace}
\newcommand{\squeak}{Squeak\xspace} % use this, not \Squeak or \sq
\newcommand{\sqsrc}{SqueakSource\xspace}
\newcommand{\sbe}{\url{http://SqueakByExample.org}\xspace}
\newcommand{\pharoweb}{\url{http://pharo-project.org}\xspace}
\newcommand{\pbe}{\url{http://PharoByExample.org}\xspace}
\newcommand{\sba}{\url{http://SquareBracketAssociates.org}\xspace}
\newcommand{\bam}{\lct{Bounc\-ing\-Atoms\-Morph}\xspace}
%=============================================================
%:Markup macros for proof-reading
\usepackage[normalem]{ulem} % for \sout
\usepackage{xcolor}
\newcommand{\ra}{$\rightarrow$}
\newcommand{\ugh}[1]{\textcolor{red}{\uwave{#1}}} % please rephrase
\newcommand{\ins}[1]{\textcolor{blue}{\uline{#1}}} % please insert
\newcommand{\del}[1]{\textcolor{red}{\sout{#1}}} % please delete
\newcommand{\chg}[2]{\textcolor{red}{\sout{#1}}{\ra}\textcolor{blue}{\uline{#2}}} % please change
%=============================================================
%:Editorial comment macros
%\newcommand{\nnbb}[2]{
%    % \fbox{\bfseries\sffamily\scriptsize#1}
%    \fcolorbox{gray}{yellow}{\bfseries\sffamily\scriptsize#1}
%    {\sf\small$\blacktriangleright$\textit{#2}$\blacktriangleleft$}
%   }
\newcommand{\yellowbox}[1]{\fcolorbox{gray}{yellow}{\bfseries\sffamily\scriptsize#1}}
\newcommand{\triangles}[1]{{\sf\small$\blacktriangleright$\textit{#1}$\blacktriangleleft$}}
\newcommand{\nnbb}[2]{\yellowbox{#1} \triangles{#2}}
\newcommand{\fix}{\yellowbox{FIX!}}
\newcommand{\here}{\yellowbox{CONTINUE HERE!}}
% editor macros
\newcommand{\apl}[1]{\nnbb{Alain}{#1}} % Alain
\newcommand{\ab}[1]{\nnbb{Andrew}{#1}} % Black
\newcommand{\sd}[1]{\nnbb{St\'{e}f}{#1}} % Ducasse
\newcommand{\dc}[1]{\nnbb{DamienC}{#1}} % Ducasse
\newcommand{\md}[1]{\nnbb{Marcus}{#1}} % Denker
\newcommand{\on}[1]{\nnbb{Oscar}{#1}} % Nierstrasz
\newcommand{\damien}[1]{\nnbb{Damien}{#1}} % Pollet
\newcommand{\lr}[1]{\nnbb{Lukas}{#1}} % Renggli
\newcommand{\orla}[1]{\nnbb{Orla}{#1}} % Greevy
\newcommand{\alex}[1]{\nnbb{Alex}{#1}} % Bergel
\newcommand{\alx}[1]{\nnbb{Alex}{#1}} % Bergel
\newcommand{\dr}[1]{\nnbb{David}{#1}} % Roethlisberger
\newcommand{\ja}[1]{\nnbb{Jannik}{#1}} % Laval
\newcommand{\jr}[1]{\nnbb{Jorge}{#1}} % Ressia
\newcommand{\jb}[1]{\nnbb{JB}{#1}} % JB
\newcommand{\fp}[1]{\nnbb{Fabrizio}{#1}} % Perin
\newcommand{\michael}[1]{\nnbb{Michael}{#1}} % Davies
\newcommand{\ew}[1]{\nnbb{Erwann}{#1}} % Wernli
\newcommand{\mb}[1]{\nnbb{Martial}{#1}} % Boniou
\newcommand{\hw}[1]{\nnbb{Hernan}{#1}} % Wilkinson
\newcommand{\ben}[1]{\nnbb{Benjamin}{#1}} % Benjamin Van Ryseghem
\newcommand{\mmp}[1]{\nnbb{Mariano}{#1}} % Mariano Martinez Peck
%=============================================================
%:Abbreviation macros
\newcommand{\ie}{\emph{i.e.},\xspace}
\newcommand{\eg}{\emph{e.g.},\xspace}
\newcommand{\etc}{etc.\xspace}
%=============================================================
%:Cross reference macros
\newcommand{\charef}[1]{Chapter~\ref{cha:#1}\xspace}
\newcommand{\secref}[1]{Section~\ref{sec:#1}\xspace}
\newcommand{\figref}[1]{Figure~\ref{fig:#1}\xspace}
\newcommand{\Figref}[1]{Figure~\ref{fig:#1}\xspace}
\newcommand{\appref}[1]{Appendix~\ref{app:#1}\xspace}
\newcommand{\tabref}[1]{Table~\ref{tab:#1}\xspace}
\newcommand{\faqref}[1]{FAQ~\ref{faq:#1}, p.~\pageref{faq:#1}\xspace}
% APB: I removed trailing \xspace commands from these macros because
% \xspace mostly doesn't work.  If you want a space after your
% references, type one!
% ON: xspace has always worked just fine for me!  Please leave them in.
%
\newcommand{\ruleref}[1]{\ref{rule:#1}\xspace}
%
\newcommand{\egref}[1]{example~\ref{eg:#1}\xspace}
\newcommand{\Egref}[1]{Example~\ref{eg:#1}\xspace}
%
\newcommand{\scrref}[1]{script~\ref{scr:#1}\xspace}
\newcommand{\Scrref}[1]{Script~\ref{scr:#1}\xspace}
\newcommand{\tscrref}[1]{the script~\ref{scr:#1}\xspace}
\newcommand{\Tscrref}[1]{The script~\ref{scr:#1}\xspace}
%
\newcommand{\mthref}[1]{method~\ref{mth:#1}\xspace}
\newcommand{\mthsref}[1]{methods~\ref{mth:#1}\xspace}
\newcommand{\Mthref}[1]{Method~\ref{mth:#1}\xspace}
\newcommand{\tmthref}[1]{the method~\ref{mth:#1}\xspace}
\newcommand{\Tmthref}[1]{The method~\ref{mth:#1}\xspace}
%
\newcommand{\clsref}[1]{class~\ref{cls:#1}\xspace}
\newcommand{\tclsref}[1]{the class~\ref{cls:#1}\xspace}
\newcommand{\Tclsref}[1]{The class~\ref{cls:#1}\xspace}

\newcommand{\chalabel}[1]{\label{cha:#1}}
\newcommand{\seclabel}[1]{\label{sec:#1}}
\newcommand{\figlabel}[1]{\label{fig:#1}}
\newcommand{\tablabel}[1]{\label{tab:#1}}
\newcommand{\rulelabel}[1]{\label{rule:#1}}
\newcommand{\eglabel}[1]{\label{eg:#1}}
\newcommand{\scrlabel}[1]{\label{scr:#1}}
\newcommand{\mthlabel}[1]{\label{mth:#1}}
\newcommand{\clslabel}[1]{\label{cls:#1}}
\newcommand{\faqlabel}[1]{\label{faq:#1}}
%=============================================================
%:Menu item macro
% for menu items, so we can change our minds on how to print them! (apb)
\definecolor{lightgray}{gray}{0.89}
\newcommand{\menu}[1]{{%
	\setlength{\fboxsep}{0pt}%
	\colorbox{lightgray}{{{\upshape\sffamily\strut \,#1\,}}}}}
\newcommand{\link}[1]{{%
	\fontfamily{lmr}\selectfont
 	\upshape{\sffamily \underline{#1}}}}
% For submenu items:
\newcommand{\go}{\,$\triangleright$\,}
% \newcommand{\go}{\,$\blacktriangleright$\,}
% For keyboard shortcuts:
%\newcommand{\short}[1]{\mbox{$\langle${\sc CMD}$\rangle$-#1}\xspace}
\newcommand{\short}[1]{\mbox{{\sc cmd}\hspace{0.08em}--\hspace{0.09em}#1}\xspace}
% For buttons:
\newcommand{\button}[1]{{%
	\setlength{\fboxsep}{0pt}%
	\fbox{{\upshape\sffamily\strut \,#1\,}}}}
% NB: The button macro does not work within captions -- incompatible with xcolor package :-(
\newcommand{\toolsflap}{\textit{Tools} flap\xspace}
%=============================================================
%:Mouse clicks
\newcommand{\click}{click\xspace} % RED
\newcommand{\actclick}{action-click\xspace} % YELLOW
\newcommand{\metaclick}{meta-click\xspace} % BLUE
\newcommand{\Click}{Click\xspace} % RED
\newcommand{\Actclick}{Action-click\xspace} % YELLOW
\newcommand{\Metaclick}{Meta-click\xspace} % BLUE
%=============================================================
%:ToSh macros
\newboolean{tosh}
\setboolean{tosh}{false}
\newcommand{\iftoshelse}[2]{\ifthenelse{\boolean{tosh}}{#1}{#2}}
%=============================================================
%:ToSh colors
%\newcommand{\highlightcolor}{\color{blue!65}}
%\newcommand{\boxcolor}{\color{gray!25}}
\newcommand{\highlight}[1]{\textcolor{blue!65}{#1}}
%\newcommand{\codecolor}{\color{blue!65}}
%%\setlength{\fboxrule}{2pt}
%\newcommand{\asPict}[1]{%
%	{\Large\highlight{#1}}}
%=============================================================
%:Reader cues (do this)
%
% Indicate something the reader should try out.
% \newcommand{\dothisicon}{\raisebox{-.5ex}{\includegraphics[width=1.4em]{squeak-logo}}}
\iftoshelse{
	\usepackage{marginnote}
		\renewcommand*{\marginfont}{\footnotesize}
	\newcommand{\vartriangleout}{\ifthenelse{\isodd{\thepage}}{\vartriangleright}{\vartriangleleft}}
	\newcommand{\dothisicon}{\fcolorbox{blue!65}{white}{\highlight{$\vartriangleout$}}}
	\newcommand{\dothis}[1]{%
		\noindent\par\noindent
		{\reversemarginpar
			\marginnote{\fcolorbox{blue!65}{white}{\highlight{$\vartriangleout$}}}}
		%\MarginLabel{do this}
		\noindent\emph{#1}
		\nopagebreak}
}{
	\newcommand{\dothisicon}{\raisebox{-.5ex}{\includegraphics[height=1.2em]{pharo}}}
	\newcommand{\dothis}[1]{%
		\medskip
		\noindent\dothisicon
		\ifx#1\empty\else\quad\emph{#1}\fi
		\par\smallskip\nopagebreak}
}
%===> NEW VERSION <===
% NB: To use this in an individual chapter, you must set:
%\graphicspath{{figures/} {../figures/}}
% at the head of the chapter.  Don't forget the final /
%=============================================================
%:Reader hints (hint)
%
% Indicates a non-obvious consequence 
\newcommand{\hint}[1]{\vspace{1ex}\noindent\fbox{\textsc{Hint}} \emph{#1}}
%=================================================================
% graphics for Morphic handles
\newcommand{\grabHandle}{\raisebox{-0.2ex}{\includegraphics[width=1em]{blackHandle}}}
\newcommand{\moveHandle}{\raisebox{-0.2ex}{\includegraphics[width=1em]{moveHandle}}}
\newcommand{\debugHandle}{\raisebox{-0.2ex}{\includegraphics[width=1em]{debugHandle}}}
%=============================================================
%:Highlighting Important stuff (doublebox)
%
% From Seaside book ...
\newsavebox{\SavedText}
\newlength{\InnerBoxRule}\setlength{\InnerBoxRule}{.75\fboxrule}
\newlength{\OuterBoxRule}\setlength{\OuterBoxRule}{1.5\fboxrule}
\newlength{\BoxSeparation}\setlength{\BoxSeparation}{1.5\fboxrule}
\addtolength{\BoxSeparation}{.5pt}
\newlength{\SaveBoxSep}\setlength{\SaveBoxSep}{2\fboxsep}
%
\newenvironment{doublebox}{\begin{lrbox}{\SavedText}
    \begin{minipage}{.75\textwidth}}
    {\end{minipage}\end{lrbox}\begin{center}
    \setlength{\fboxsep}{\BoxSeparation}\setlength{\fboxrule}{\OuterBoxRule}
    \fbox{\setlength{\fboxsep}{\SaveBoxSep}\setlength{\fboxrule}{\InnerBoxRule}%
      \fbox{\usebox{\SavedText}}}
  \end{center}}
% Use this:
\newcommand{\important}[1]{\begin{doublebox}#1\end{doublebox}}
%=============================================================
%:Section depth
\setcounter{secnumdepth}{2}
%% for this to happen start the file with
%\ifx\wholebook\relax\else
%\input{../common.tex}
%\begin{document}
%\fi
% and terminate by
% \ifx\wholebook\relax\else\end{document}\fi

\DeclareGraphicsExtensions{.pdf, .jpg, .png}
%=============================================================
%:PDF setup
\hypersetup{
%   a4paper,
%   pdfstartview=FitV,
%   colorlinks,
%   linkcolor=darkblue,
%   citecolor=darkblue,
   pdftitle={Pharo by Example},
   pdfauthor={Andrew P. Black, St\'ephane Ducasse,	Oscar Nierstrasz,
Damien Pollet},
   pdfkeywords={Smalltalk, Squeak, Object-Oriented Programming, OOP},
   pdfsubject={Computer Science}
}
%=============================================================
%:Page layout and appearance
%
% \renewcommand{\headrulewidth}{0pt}
\renewcommand{\chaptermark}[1]{\markboth{#1}{}}
\renewcommand{\sectionmark}[1]{\markright{\thesection\ #1}}
\renewpagestyle{plain}[\small\itshape]{%
	\setheadrule{0pt}%
	\sethead[][][]{}{}{}%
	\setfoot[][][]{}{}{}}
\renewpagestyle{headings}[\small\itshape]{%
	\setheadrule{0pt}%
	\setmarks{chapter}{section}%
	\sethead[\thepage][][\chaptertitle]{\sectiontitle}{}{\thepage}%
	\setfoot[][][]{}{}{}}
%=============================================================
%:Title section setup and TOC numbering depth
\setcounter{secnumdepth}{1}
\setcounter{tocdepth}{1}
\titleformat{\part}[display]{\centering}{\huge\partname\ \thepart}{1em}{\Huge\textbf}[]
\titleformat{\chapter}[display]{}{\huge\chaptertitlename\ \thechapter}{1em}{\Huge\raggedright\textbf}[]
\titlecontents{part}[3pc]{%
		\pagebreak[2]\addvspace{1em plus.4em minus.2em}%
		\leavevmode\large\bfseries}
	{\contentslabel{3pc}}{\hspace*{-3pc}}
	{}[\nopagebreak]
\titlecontents{chapter}[3pc]{%
		\pagebreak[0]\addvspace{1em plus.2em minus.2em}%
		\leavevmode\bfseries}
	{\contentslabel{3pc}}{}
	{\hfill\contentspage}[\nopagebreak]
\dottedcontents{section}[3pc]{}{3pc}{1pc}
\dottedcontents{subsection}[3pc]{}{0pc}{1pc}
% \dottedcontents{subsection}[4.5em]{}{0pt}{1pc}
% Make \cleardoublepage insert really blank pages http://www.tex.ac.uk/cgi-bin/texfaq2html?label=reallyblank
\let\origdoublepage\cleardoublepage
\newcommand{\clearemptydoublepage}{%
  \clearpage
  {\pagestyle{empty}\origdoublepage}}
\let\cleardoublepage\clearemptydoublepage % see http://www.tex.ac.uk/cgi-bin/texfaq2html?label=patch
%=============================================================
%:FAQ macros (for FAQ chapter)
\newtheorem{faq}{FAQ}
\newcommand{\answer}{\paragraph{Answer}\ }
%=============================================================
%:Listings package configuration
% \newcommand{\caret}{\makebox{\raisebox{0.4ex}{\footnotesize{$\wedge$}}}}
\newcommand{\caret}{\^\,}
\newcommand{\escape}{{\sf \textbackslash}}
\definecolor{source}{gray}{0.95}
\usepackage{listings}
\lstdefinelanguage{Smalltalk}{
%  morekeywords={self,super,true,false,nil,thisContext}, % This is overkill
  morestring=[d]',
  morecomment=[s]{"}{"},
  alsoletter={\#:},
  escapechar={!},
  literate=
    {BANG}{!}1
    {CARET}{\^}1
    {UNDERSCORE}{\_}1
    {\\st}{Smalltalk}9 % convenience -- in case \st occurs in code
    % {'}{{\textquotesingle}}1 % replaced by upquote=true in \lstset
    {_}{{$\leftarrow$}}1
    {>>>}{{\sep}}1
    {^}{{$\uparrow$}}1
    {~}{{$\sim$}}1
    {-}{{\texttt{-}}}1 %{\textminus}}1 %{-}{\hspace{-0.13em}}{-}}1  % the goal is to make - the same width as +
    % {+}{\sf+}1 %{\raisebox{0.08ex}{+}}}1      % and to raise + off the baseline to match -
    {-->}{{\quad$\longrightarrow$\quad}}3
	, % Don't forget the comma at the end!
  tabsize=4
}[keywords,comments,strings]

\lstset{language=Smalltalk,
	basicstyle=\sffamily,
	keywordstyle=\color{black}\bfseries,
	% stringstyle=\ttfamily, % Ugly! do we really want this? -- on
	mathescape=true,
	showstringspaces=false,
	keepspaces=true,
	breaklines=true,
	breakautoindent=true,
	backgroundcolor=\color{source},
	lineskip={-1pt}, % Ugly hack
	upquote=true, % straight quote; requires textcomp package
	columns=fullflexible} % no fixed width fonts
% In-line code (literal)
% Normally use this for all in-line code:
\newcommand{\ct}{\lstinline[mathescape=false,backgroundcolor=\color{white},basicstyle={\sffamily\upshape}]}
% apb 2007.8.28 added the \upshape declaration to avoid getting italicized code in \dothis{ } sections.
% In-line code (latex enabled)
% Use this only in special situations where \ct does not work
% (within section headings ...):
\newcommand{\lct}[1]{{\textsf{\textup{#1}}}}
% Use these for system categories and protocols:
\newcommand{\scat}[1]{\emph{\textsf{#1}}\xspace}
\newcommand{\pkg}[1]{\emph{\textsf{#1}}\xspace}
\newcommand{\prot}[1]{\emph{\textsf{#1}}\xspace}
% Code environments
% NB: the arg is for tests
% Only code and example environments may be tests
\lstnewenvironment{code}[1]{%
	\lstset{%
		% frame=lines,
		frame=single,
		framerule=0pt,
		mathescape=false
	}
}{}
\def\ignoredollar#1{}
%=============================================================
%:Code environments (method, script ...)
% NB: the third arg is for tests
% Only code and example environments may be tests
\lstnewenvironment{example}[3][defaultlabel]{%
	\renewcommand{\lstlistingname}{Example}%
	\lstset{
		% frame=lines,
		frame=single,
		framerule=0pt,
		mathescape=false,
		caption={\emph{#2}},
		label={eg:#1}
	}
}{}
\lstnewenvironment{script}[2][defaultlabel]{%
\renewcommand{\lstlistingname}{Script}%
	\lstset{
		% frame=lines,
		frame=single,
		framerule=0pt,
		mathescape=false,
		name={Script},
		caption={\emph{#2}},
		label={scr:#1}
	}
}{}
\lstnewenvironment{method}[2][defaultlabel]{%
	\renewcommand{\lstlistingname}{Method}%
	\lstset{
		% frame=lines,
		frame=single,
		framerule=0pt,
		mathescape=false,
		name={Method},
		caption={\emph{#2}},
		label={mth:#1}
	}
}{}
\lstnewenvironment{methods}[2][defaultlabel]{% just for multiple methods at once
	\renewcommand{\lstlistingname}{Methods}%
	\lstset{
		% frame=lines,
		frame=single,
		framerule=0pt,
		mathescape=false,
		name={Method},
		caption={\emph{#2}},
		label={mth:#1}
	}
}{}
\lstnewenvironment{numMethod}[2][defaultlabel]{%
	\renewcommand{\lstlistingname}{Method}%
	\lstset{
		numbers=left,
		numberstyle={\tiny\sffamily},
		% frame=lines,
		frame=single,
		framerule=0pt,
		mathescape=false,
		name={Method},
		caption={\emph{#2}},
		label={mth:#1}
	}
}{}
\lstnewenvironment{classdef}[2][defaultlabel]{%
	\renewcommand{\lstlistingname}{Class}%
	\lstset{
		% frame=lines,
		frame=single,
		framerule=0pt,
		mathescape=false,
		name={Class},
		caption={\emph{#2}},
		label={cls:#1}
	}
}{}
%=============================================================
%:Reserving space
% Usually need one more line than the actual lines of code
\newcommand{\needlines}[1]{\Needspace{#1\baselineskip}}
%=============================================================
%:Indexing macros
% Macros ending with "ind" generate text as well as an index entry
% Macros ending with "index" *only* generate an index entry
\newcommand{\ind}[1]{\index{#1}#1\xspace} % plain text
\newcommand{\subind}[2]{\index{#1!#2}#2\xspace} % show #2, subindex under #1
\newcommand{\emphind}[1]{\index{#1}\emph{#1}\xspace} % emph #1
\newcommand{\emphsubind}[2]{\index{#1!#2}\emph{#2}\xspace} % show emph #2, subindex inder #1
\newcommand{\scatind}[1]{\index{#1@\textsf{#1} (category)}\scat{#1}} % category
\newcommand{\pkgind}[1]{\index{#1@\textsf{#1} (package)}\pkg{#1}} % package
\newcommand{\protind}[1]{\index{#1@\textsf{#1} (protocol)}\prot{#1}} % protocol
\newcommand{\clsind}[1]{\index{#1@\textsf{#1} (class)}\ct{#1}\xspace}
% \newcommand{\clsind}[1]{\index{#1!\#@(class)}\ct{#1}\xspace} % class
\newcommand{\clsindplural}[1]{\index{#1!\#@(class)}\ct{#1}s\xspace} % class
\newcommand{\cvind}[1]{\index{#1@\textsf{#1} (class variable)}\ct{#1}\xspace} % class var
\newcommand{\glbind}[1]{\index{#1@\textsf{#1} (global)}\ct{#1}\xspace} % global
\newcommand{\patind}[1]{\index{#1@#1 (pattern)}\ct{#1}\xspace} % pattern
\newcommand{\pvind}[1]{\index{#1@\textsf{#1} (pseudo variable)}\ct{#1}\xspace} % pseudo var
\newcommand{\clsmthind}[2]{\index{#1!#2@\ct{#2}}\ct{#1>>>#2}\xspace} % class + method name
\newcommand{\mthind}[2]{\index{#1!#2@\ct{#2}}\ct{#2}\xspace} % show method name only
\newcommand{\lmthind}[2]{\index{#1!#2@\ct{#2}}\lct{#2}\xspace} % show method name only
\newcommand{\cmind}[2]{\index{#1!#2@\ct{#2}}\ct{#1>>>#2}\xspace} % show class>>method
\newcommand{\lcmind}[2]{\index{#1!#2@\ct{#2}}\lct{#1>>>#2}\xspace} % show class>>method
\newcommand{\toolsflapind}{\index{Tools flap}\toolsflap} % index tools flap
% The following only generate an index entry:
% \newcommand{\clsindex}[1]{\index{#1@\textsf{#1} (class)}}
\newcommand{\clsindex}[1]{\index{#1!\#@(class)}} % class
\newcommand{\mthindex}[2]{\index{#1!#2@\ct{#2}}} % method
\newcommand{\cmindex}[2]{\index{#1!#2@\ct{#2}}} % class>>method
\newcommand{\cvindex}[1]{\index{#1@\textsf{#1} (class variable)}} % class var
\newcommand{\glbindex}[1]{\index{#1@\textsf{#1} (global)}}% global
\newcommand{\pvindex}[1]{\index{#1@\textsf{#1} (pseudo variable)}}% pseudo var
\newcommand{\seeindex}[2]{\index{#1|see{#2}}} % #1, see #2
\newcommand{\scatindex}[1]{\index{#1@\textsf{#1} (category)}} % category
\newcommand{\pkgindex}[1]{\index{#1@\textsf{#1} (package)}} % package
\newcommand{\protindex}[1]{\index{#1@\textsf{#1} (protocol)}} % protocol
% How can we have the main entry page numbers in bold yet not break the hyperlink?
\newcommand{\boldidx}[1]{{\bf #1}} % breaks hyperlink
%\newcommand{\indmain}[1]{\index{#1|boldidx}#1\xspace} % plain text, main entry
%\newcommand{\emphsubindmain}[2]{\index{#1!#2|boldidx}\emph{#2}\xspace} % subindex, main entry
%\newcommand{\subindmain}[2]{\index{#1!#2|boldidx}#2\xspace} % subindex, main entry
%\newcommand{\clsindmain}[1]{\index{#1@\textsf{#1} (class)|boldidx}\ct{#1}\xspace}
%\newcommand{\clsindmain}[1]{\index{#1!\#@(class)|boldidx}\ct{#1}\xspace} % class main
%\newcommand{\indexmain}[1]{\index{#1|boldidx}} % main index entry only
\newcommand{\indmain}[1]{\index{#1}#1\xspace} % The main index entry for this item
\newcommand{\emphsubindmain}[2]{\index{#1!#2}\emph{#2}\xspace} % subindex, main entry
\newcommand{\subindmain}[2]{\index{#1!#2}#2\xspace} % subindex, main entry
%\newcommand{\clsindmain}[1]{\index{#1@\textsf{#1} (class)}\ct{#1}\xspace}
\newcommand{\clsindmain}[1]{\index{#1!\#@(class)}\ct{#1}\xspace} % class main
\newcommand{\clsindexmain}[1]{\index{#1!\#@(class)}} % class main index only
\newcommand{\indexmain}[1]{\index{#1}} 
%=============================================================
%:Code macros
% some constants
\newcommand{\codesize}{\small}
\newcommand{\codefont}{\sffamily}
%\newcommand{\cat}[1]{\textit{In category #1}}%%To remove later
\newlength{\scriptindent}
\setlength{\scriptindent}{.3cm}
%% Method presentation constants
\newlength{\methodindent}
\newlength{\methodwordlength}
\newlength{\aftermethod}
\setlength{\methodindent}{0.2cm}
\settowidth{\methodwordlength}{\ M\'ethode\ }
%=============================================================
%:Smalltalk macros
%\newcommand{\sep}{{$\gg$}}
\newcommand{\sep}{\mbox{>>}}
\newcommand{\self}{\lct{self}\xspace}
\newcommand{\super}{\lct{super}\xspace}
\newcommand{\nil}{\lct{nil}\xspace}
%=============================================================
% be less conservative about float placement
% these commands are from http://www.tex.ac.uk/cgi-bin/texfaq2html?label=floats
\renewcommand{\topfraction}{.9}
\renewcommand{\bottomfraction}{.9}
\renewcommand{\textfraction}{.1}
\renewcommand{\floatpagefraction}{.85}
\renewcommand{\dbltopfraction}{.66}
\renewcommand{\dblfloatpagefraction}{.85}
\setcounter{topnumber}{9}
\setcounter{bottomnumber}{9}
\setcounter{totalnumber}{20}
\setcounter{dbltopnumber}{9}
%=============================================================
% Give information from each chapter's author
\newcommand{\contact}[2]{\textbf{#1} \textsf{(#2)}}

\newcommand{\chapterauthor}[1]{\emph{with the participation of:\\#1}\\}

\newcommand{\authornoury}{\contact{Noury Bouraqadi}{Noury.Bouraqadi@mines-douai.fr}}
\newcommand{\authordamienc}{\contact{Damien Cassou}{damien.cassou@gmail.com}}
\newcommand{\authoroscar}{\contact{Oscar Nierstrasz}{oscar.nierstrasz@acm.org}}
\newcommand{\authorsteph}{\contact{St\'ephane Ducasse}{stephane.ducasse@inria.fr}}
\newcommand{\authoralex}{\contact{Alexandre Bergel}{alexandre@bergel.eu}}
\newcommand{\authorolivier}{\contact{Olivier Auverlot}{olivier.auverlot@inria.fr}}
\newcommand{\authornicolas}{\contact{Nicolas Cellier}{nicolas.cellier.aka.nice@gmail.com}}
\newcommand{\authormarcus}{\contact{Marcus Denker}{marcus.denker@inria.fr}}
\newcommand{\authoralain}{\contact{Alain Plantec}{alain.plantec@univ-brest.fr}}
\newcommand{\authordale}{\contact{Dale Henrichs}{dale.henrichs@gemstone.com}}
\newcommand{\authormariano}{\contact{Mariano Martinez Peck}{marianopeck@gmail.com}}
\newcommand{\authorsven}{\contact{Sven Van Caekenberghe}{sven@beta9.be}}
\newcommand{\authorlukas}{\contact{Lukas Renggli}{renggli@gmail.com}}

%=============================================================
% apb doesn't like paragraphs to run in to each other without a break
\parskip 1ex
%=============================================================
%:Stuff to check, merge or deprecate
%\setlength{\marginparsep}{2mm}
%\renewcommand{\baselinestretch}{1.1}
%=============================================================

%\begin{document}
%\fi
% and terminate by
% \ifx\wholebook\relax\else\end{document}\fi

\DeclareGraphicsExtensions{.pdf, .jpg, .png}
%=============================================================
%:PDF setup
\hypersetup{
%   a4paper,
%   pdfstartview=FitV,
%   colorlinks,
%   linkcolor=darkblue,
%   citecolor=darkblue,
   pdftitle={Pharo by Example},
   pdfauthor={Andrew P. Black, St\'ephane Ducasse,	Oscar Nierstrasz,
Damien Pollet},
   pdfkeywords={Smalltalk, Squeak, Object-Oriented Programming, OOP},
   pdfsubject={Computer Science}
}
%=============================================================
%:Page layout and appearance
%
% \renewcommand{\headrulewidth}{0pt}
\renewcommand{\chaptermark}[1]{\markboth{#1}{}}
\renewcommand{\sectionmark}[1]{\markright{\thesection\ #1}}
\renewpagestyle{plain}[\small\itshape]{%
	\setheadrule{0pt}%
	\sethead[][][]{}{}{}%
	\setfoot[][][]{}{}{}}
\renewpagestyle{headings}[\small\itshape]{%
	\setheadrule{0pt}%
	\setmarks{chapter}{section}%
	\sethead[\thepage][][\chaptertitle]{\sectiontitle}{}{\thepage}%
	\setfoot[][][]{}{}{}}
%=============================================================
%:Title section setup and TOC numbering depth
\setcounter{secnumdepth}{1}
\setcounter{tocdepth}{1}
\titleformat{\part}[display]{\centering}{\huge\partname\ \thepart}{1em}{\Huge\textbf}[]
\titleformat{\chapter}[display]{}{\huge\chaptertitlename\ \thechapter}{1em}{\Huge\raggedright\textbf}[]
\titlecontents{part}[3pc]{%
		\pagebreak[2]\addvspace{1em plus.4em minus.2em}%
		\leavevmode\large\bfseries}
	{\contentslabel{3pc}}{\hspace*{-3pc}}
	{}[\nopagebreak]
\titlecontents{chapter}[3pc]{%
		\pagebreak[0]\addvspace{1em plus.2em minus.2em}%
		\leavevmode\bfseries}
	{\contentslabel{3pc}}{}
	{\hfill\contentspage}[\nopagebreak]
\dottedcontents{section}[3pc]{}{3pc}{1pc}
\dottedcontents{subsection}[3pc]{}{0pc}{1pc}
% \dottedcontents{subsection}[4.5em]{}{0pt}{1pc}
% Make \cleardoublepage insert really blank pages http://www.tex.ac.uk/cgi-bin/texfaq2html?label=reallyblank
\let\origdoublepage\cleardoublepage
\newcommand{\clearemptydoublepage}{%
  \clearpage
  {\pagestyle{empty}\origdoublepage}}
\let\cleardoublepage\clearemptydoublepage % see http://www.tex.ac.uk/cgi-bin/texfaq2html?label=patch
%=============================================================
%:FAQ macros (for FAQ chapter)
\newtheorem{faq}{FAQ}
\newcommand{\answer}{\paragraph{Answer}\ }
%=============================================================
%:Listings package configuration
% \newcommand{\caret}{\makebox{\raisebox{0.4ex}{\footnotesize{$\wedge$}}}}
\newcommand{\caret}{\^\,}
\newcommand{\escape}{{\sf \textbackslash}}
\definecolor{source}{gray}{0.95}
\usepackage{listings}
\lstdefinelanguage{Smalltalk}{
%  morekeywords={self,super,true,false,nil,thisContext}, % This is overkill
  morestring=[d]',
  morecomment=[s]{"}{"},
  alsoletter={\#:},
  escapechar={!},
  literate=
    {BANG}{!}1
    {CARET}{\^}1
    {UNDERSCORE}{\_}1
    {\\st}{Smalltalk}9 % convenience -- in case \st occurs in code
    % {'}{{\textquotesingle}}1 % replaced by upquote=true in \lstset
    {_}{{$\leftarrow$}}1
    {>>>}{{\sep}}1
    {^}{{$\uparrow$}}1
    {~}{{$\sim$}}1
    {-}{{\texttt{-}}}1 %{\textminus}}1 %{-}{\hspace{-0.13em}}{-}}1  % the goal is to make - the same width as +
    % {+}{\sf+}1 %{\raisebox{0.08ex}{+}}}1      % and to raise + off the baseline to match -
    {-->}{{\quad$\longrightarrow$\quad}}3
	, % Don't forget the comma at the end!
  tabsize=4
}[keywords,comments,strings]

\lstset{language=Smalltalk,
	basicstyle=\sffamily,
	keywordstyle=\color{black}\bfseries,
	% stringstyle=\ttfamily, % Ugly! do we really want this? -- on
	mathescape=true,
	showstringspaces=false,
	keepspaces=true,
	breaklines=true,
	breakautoindent=true,
	backgroundcolor=\color{source},
	lineskip={-1pt}, % Ugly hack
	upquote=true, % straight quote; requires textcomp package
	columns=fullflexible} % no fixed width fonts
% In-line code (literal)
% Normally use this for all in-line code:
\newcommand{\ct}{\lstinline[mathescape=false,backgroundcolor=\color{white},basicstyle={\sffamily\upshape}]}
% apb 2007.8.28 added the \upshape declaration to avoid getting italicized code in \dothis{ } sections.
% In-line code (latex enabled)
% Use this only in special situations where \ct does not work
% (within section headings ...):
\newcommand{\lct}[1]{{\textsf{\textup{#1}}}}
% Use these for system categories and protocols:
\newcommand{\scat}[1]{\emph{\textsf{#1}}\xspace}
\newcommand{\pkg}[1]{\emph{\textsf{#1}}\xspace}
\newcommand{\prot}[1]{\emph{\textsf{#1}}\xspace}
% Code environments
% NB: the arg is for tests
% Only code and example environments may be tests
\lstnewenvironment{code}[1]{%
	\lstset{%
		% frame=lines,
		frame=single,
		framerule=0pt,
		mathescape=false
	}
}{}
\def\ignoredollar#1{}
%=============================================================
%:Code environments (method, script ...)
% NB: the third arg is for tests
% Only code and example environments may be tests
\lstnewenvironment{example}[3][defaultlabel]{%
	\renewcommand{\lstlistingname}{Example}%
	\lstset{
		% frame=lines,
		frame=single,
		framerule=0pt,
		mathescape=false,
		caption={\emph{#2}},
		label={eg:#1}
	}
}{}
\lstnewenvironment{script}[2][defaultlabel]{%
\renewcommand{\lstlistingname}{Script}%
	\lstset{
		% frame=lines,
		frame=single,
		framerule=0pt,
		mathescape=false,
		name={Script},
		caption={\emph{#2}},
		label={scr:#1}
	}
}{}
\lstnewenvironment{method}[2][defaultlabel]{%
	\renewcommand{\lstlistingname}{Method}%
	\lstset{
		% frame=lines,
		frame=single,
		framerule=0pt,
		mathescape=false,
		name={Method},
		caption={\emph{#2}},
		label={mth:#1}
	}
}{}
\lstnewenvironment{methods}[2][defaultlabel]{% just for multiple methods at once
	\renewcommand{\lstlistingname}{Methods}%
	\lstset{
		% frame=lines,
		frame=single,
		framerule=0pt,
		mathescape=false,
		name={Method},
		caption={\emph{#2}},
		label={mth:#1}
	}
}{}
\lstnewenvironment{numMethod}[2][defaultlabel]{%
	\renewcommand{\lstlistingname}{Method}%
	\lstset{
		numbers=left,
		numberstyle={\tiny\sffamily},
		% frame=lines,
		frame=single,
		framerule=0pt,
		mathescape=false,
		name={Method},
		caption={\emph{#2}},
		label={mth:#1}
	}
}{}
\lstnewenvironment{classdef}[2][defaultlabel]{%
	\renewcommand{\lstlistingname}{Class}%
	\lstset{
		% frame=lines,
		frame=single,
		framerule=0pt,
		mathescape=false,
		name={Class},
		caption={\emph{#2}},
		label={cls:#1}
	}
}{}
%=============================================================
%:Reserving space
% Usually need one more line than the actual lines of code
\newcommand{\needlines}[1]{\Needspace{#1\baselineskip}}
%=============================================================
%:Indexing macros
% Macros ending with "ind" generate text as well as an index entry
% Macros ending with "index" *only* generate an index entry
\newcommand{\ind}[1]{\index{#1}#1\xspace} % plain text
\newcommand{\subind}[2]{\index{#1!#2}#2\xspace} % show #2, subindex under #1
\newcommand{\emphind}[1]{\index{#1}\emph{#1}\xspace} % emph #1
\newcommand{\emphsubind}[2]{\index{#1!#2}\emph{#2}\xspace} % show emph #2, subindex inder #1
\newcommand{\scatind}[1]{\index{#1@\textsf{#1} (category)}\scat{#1}} % category
\newcommand{\pkgind}[1]{\index{#1@\textsf{#1} (package)}\pkg{#1}} % package
\newcommand{\protind}[1]{\index{#1@\textsf{#1} (protocol)}\prot{#1}} % protocol
\newcommand{\clsind}[1]{\index{#1@\textsf{#1} (class)}\ct{#1}\xspace}
% \newcommand{\clsind}[1]{\index{#1!\#@(class)}\ct{#1}\xspace} % class
\newcommand{\clsindplural}[1]{\index{#1!\#@(class)}\ct{#1}s\xspace} % class
\newcommand{\cvind}[1]{\index{#1@\textsf{#1} (class variable)}\ct{#1}\xspace} % class var
\newcommand{\glbind}[1]{\index{#1@\textsf{#1} (global)}\ct{#1}\xspace} % global
\newcommand{\patind}[1]{\index{#1@#1 (pattern)}\ct{#1}\xspace} % pattern
\newcommand{\pvind}[1]{\index{#1@\textsf{#1} (pseudo variable)}\ct{#1}\xspace} % pseudo var
\newcommand{\clsmthind}[2]{\index{#1!#2@\ct{#2}}\ct{#1>>>#2}\xspace} % class + method name
\newcommand{\mthind}[2]{\index{#1!#2@\ct{#2}}\ct{#2}\xspace} % show method name only
\newcommand{\lmthind}[2]{\index{#1!#2@\ct{#2}}\lct{#2}\xspace} % show method name only
\newcommand{\cmind}[2]{\index{#1!#2@\ct{#2}}\ct{#1>>>#2}\xspace} % show class>>method
\newcommand{\lcmind}[2]{\index{#1!#2@\ct{#2}}\lct{#1>>>#2}\xspace} % show class>>method
\newcommand{\toolsflapind}{\index{Tools flap}\toolsflap} % index tools flap
% The following only generate an index entry:
% \newcommand{\clsindex}[1]{\index{#1@\textsf{#1} (class)}}
\newcommand{\clsindex}[1]{\index{#1!\#@(class)}} % class
\newcommand{\mthindex}[2]{\index{#1!#2@\ct{#2}}} % method
\newcommand{\cmindex}[2]{\index{#1!#2@\ct{#2}}} % class>>method
\newcommand{\cvindex}[1]{\index{#1@\textsf{#1} (class variable)}} % class var
\newcommand{\glbindex}[1]{\index{#1@\textsf{#1} (global)}}% global
\newcommand{\pvindex}[1]{\index{#1@\textsf{#1} (pseudo variable)}}% pseudo var
\newcommand{\seeindex}[2]{\index{#1|see{#2}}} % #1, see #2
\newcommand{\scatindex}[1]{\index{#1@\textsf{#1} (category)}} % category
\newcommand{\pkgindex}[1]{\index{#1@\textsf{#1} (package)}} % package
\newcommand{\protindex}[1]{\index{#1@\textsf{#1} (protocol)}} % protocol
% How can we have the main entry page numbers in bold yet not break the hyperlink?
\newcommand{\boldidx}[1]{{\bf #1}} % breaks hyperlink
%\newcommand{\indmain}[1]{\index{#1|boldidx}#1\xspace} % plain text, main entry
%\newcommand{\emphsubindmain}[2]{\index{#1!#2|boldidx}\emph{#2}\xspace} % subindex, main entry
%\newcommand{\subindmain}[2]{\index{#1!#2|boldidx}#2\xspace} % subindex, main entry
%\newcommand{\clsindmain}[1]{\index{#1@\textsf{#1} (class)|boldidx}\ct{#1}\xspace}
%\newcommand{\clsindmain}[1]{\index{#1!\#@(class)|boldidx}\ct{#1}\xspace} % class main
%\newcommand{\indexmain}[1]{\index{#1|boldidx}} % main index entry only
\newcommand{\indmain}[1]{\index{#1}#1\xspace} % The main index entry for this item
\newcommand{\emphsubindmain}[2]{\index{#1!#2}\emph{#2}\xspace} % subindex, main entry
\newcommand{\subindmain}[2]{\index{#1!#2}#2\xspace} % subindex, main entry
%\newcommand{\clsindmain}[1]{\index{#1@\textsf{#1} (class)}\ct{#1}\xspace}
\newcommand{\clsindmain}[1]{\index{#1!\#@(class)}\ct{#1}\xspace} % class main
\newcommand{\clsindexmain}[1]{\index{#1!\#@(class)}} % class main index only
\newcommand{\indexmain}[1]{\index{#1}} 
%=============================================================
%:Code macros
% some constants
\newcommand{\codesize}{\small}
\newcommand{\codefont}{\sffamily}
%\newcommand{\cat}[1]{\textit{In category #1}}%%To remove later
\newlength{\scriptindent}
\setlength{\scriptindent}{.3cm}
%% Method presentation constants
\newlength{\methodindent}
\newlength{\methodwordlength}
\newlength{\aftermethod}
\setlength{\methodindent}{0.2cm}
\settowidth{\methodwordlength}{\ M\'ethode\ }
%=============================================================
%:Smalltalk macros
%\newcommand{\sep}{{$\gg$}}
\newcommand{\sep}{\mbox{>>}}
\newcommand{\self}{\lct{self}\xspace}
\newcommand{\super}{\lct{super}\xspace}
\newcommand{\nil}{\lct{nil}\xspace}
%=============================================================
% be less conservative about float placement
% these commands are from http://www.tex.ac.uk/cgi-bin/texfaq2html?label=floats
\renewcommand{\topfraction}{.9}
\renewcommand{\bottomfraction}{.9}
\renewcommand{\textfraction}{.1}
\renewcommand{\floatpagefraction}{.85}
\renewcommand{\dbltopfraction}{.66}
\renewcommand{\dblfloatpagefraction}{.85}
\setcounter{topnumber}{9}
\setcounter{bottomnumber}{9}
\setcounter{totalnumber}{20}
\setcounter{dbltopnumber}{9}
%=============================================================
% Give information from each chapter's author
\newcommand{\contact}[2]{\textbf{#1} \textsf{(#2)}}

\newcommand{\chapterauthor}[1]{\emph{with the participation of:\\#1}\\}

\newcommand{\authornoury}{\contact{Noury Bouraqadi}{Noury.Bouraqadi@mines-douai.fr}}
\newcommand{\authordamienc}{\contact{Damien Cassou}{damien.cassou@gmail.com}}
\newcommand{\authoroscar}{\contact{Oscar Nierstrasz}{oscar.nierstrasz@acm.org}}
\newcommand{\authorsteph}{\contact{St\'ephane Ducasse}{stephane.ducasse@inria.fr}}
\newcommand{\authoralex}{\contact{Alexandre Bergel}{alexandre@bergel.eu}}
\newcommand{\authorolivier}{\contact{Olivier Auverlot}{olivier.auverlot@inria.fr}}
\newcommand{\authornicolas}{\contact{Nicolas Cellier}{nicolas.cellier.aka.nice@gmail.com}}
\newcommand{\authormarcus}{\contact{Marcus Denker}{marcus.denker@inria.fr}}
\newcommand{\authoralain}{\contact{Alain Plantec}{alain.plantec@univ-brest.fr}}
\newcommand{\authordale}{\contact{Dale Henrichs}{dale.henrichs@gemstone.com}}
\newcommand{\authormariano}{\contact{Mariano Martinez Peck}{marianopeck@gmail.com}}
\newcommand{\authorsven}{\contact{Sven Van Caekenberghe}{sven@beta9.be}}
\newcommand{\authorlukas}{\contact{Lukas Renggli}{renggli@gmail.com}}

%=============================================================
% apb doesn't like paragraphs to run in to each other without a break
\parskip 1ex
%=============================================================
%:Stuff to check, merge or deprecate
%\setlength{\marginparsep}{2mm}
%\renewcommand{\baselinestretch}{1.1}
%=============================================================

%\begin{document}
%\fi
% and terminate by
% \ifx\wholebook\relax\else\end{document}\fi

\DeclareGraphicsExtensions{.pdf, .jpg, .png}
%=============================================================
%:PDF setup
\hypersetup{
%   a4paper,
%   pdfstartview=FitV,
%   colorlinks,
%   linkcolor=darkblue,
%   citecolor=darkblue,
   pdftitle={Pharo by Example},
   pdfauthor={Andrew P. Black, St\'ephane Ducasse,	Oscar Nierstrasz,
Damien Pollet},
   pdfkeywords={Smalltalk, Squeak, Object-Oriented Programming, OOP},
   pdfsubject={Computer Science}
}
%=============================================================
%:Page layout and appearance
%
% \renewcommand{\headrulewidth}{0pt}
\renewcommand{\chaptermark}[1]{\markboth{#1}{}}
\renewcommand{\sectionmark}[1]{\markright{\thesection\ #1}}
\renewpagestyle{plain}[\small\itshape]{%
	\setheadrule{0pt}%
	\sethead[][][]{}{}{}%
	\setfoot[][][]{}{}{}}
\renewpagestyle{headings}[\small\itshape]{%
	\setheadrule{0pt}%
	\setmarks{chapter}{section}%
	\sethead[\thepage][][\chaptertitle]{\sectiontitle}{}{\thepage}%
	\setfoot[][][]{}{}{}}
%=============================================================
%:Title section setup and TOC numbering depth
\setcounter{secnumdepth}{1}
\setcounter{tocdepth}{1}
\titleformat{\part}[display]{\centering}{\huge\partname\ \thepart}{1em}{\Huge\textbf}[]
\titleformat{\chapter}[display]{}{\huge\chaptertitlename\ \thechapter}{1em}{\Huge\raggedright\textbf}[]
\titlecontents{part}[3pc]{%
		\pagebreak[2]\addvspace{1em plus.4em minus.2em}%
		\leavevmode\large\bfseries}
	{\contentslabel{3pc}}{\hspace*{-3pc}}
	{}[\nopagebreak]
\titlecontents{chapter}[3pc]{%
		\pagebreak[0]\addvspace{1em plus.2em minus.2em}%
		\leavevmode\bfseries}
	{\contentslabel{3pc}}{}
	{\hfill\contentspage}[\nopagebreak]
\dottedcontents{section}[3pc]{}{3pc}{1pc}
\dottedcontents{subsection}[3pc]{}{0pc}{1pc}
% \dottedcontents{subsection}[4.5em]{}{0pt}{1pc}
% Make \cleardoublepage insert really blank pages http://www.tex.ac.uk/cgi-bin/texfaq2html?label=reallyblank
\let\origdoublepage\cleardoublepage
\newcommand{\clearemptydoublepage}{%
  \clearpage
  {\pagestyle{empty}\origdoublepage}}
\let\cleardoublepage\clearemptydoublepage % see http://www.tex.ac.uk/cgi-bin/texfaq2html?label=patch
%=============================================================
%:FAQ macros (for FAQ chapter)
\newtheorem{faq}{FAQ}
\newcommand{\answer}{\paragraph{Answer}\ }
%=============================================================
%:Listings package configuration
% \newcommand{\caret}{\makebox{\raisebox{0.4ex}{\footnotesize{$\wedge$}}}}
\newcommand{\caret}{\^\,}
\newcommand{\escape}{{\sf \textbackslash}}
\definecolor{source}{gray}{0.95}
\usepackage{listings}
\lstdefinelanguage{Smalltalk}{
%  morekeywords={self,super,true,false,nil,thisContext}, % This is overkill
  morestring=[d]',
  morecomment=[s]{"}{"},
  alsoletter={\#:},
  escapechar={!},
  literate=
    {BANG}{!}1
    {CARET}{\^}1
    {UNDERSCORE}{\_}1
    {\\st}{Smalltalk}9 % convenience -- in case \st occurs in code
    % {'}{{\textquotesingle}}1 % replaced by upquote=true in \lstset
    {_}{{$\leftarrow$}}1
    {>>>}{{\sep}}1
    {^}{{$\uparrow$}}1
    {~}{{$\sim$}}1
    {-}{{\texttt{-}}}1 %{\textminus}}1 %{-}{\hspace{-0.13em}}{-}}1  % the goal is to make - the same width as +
    % {+}{\sf+}1 %{\raisebox{0.08ex}{+}}}1      % and to raise + off the baseline to match -
    {-->}{{\quad$\longrightarrow$\quad}}3
	, % Don't forget the comma at the end!
  tabsize=4
}[keywords,comments,strings]

\lstset{language=Smalltalk,
	basicstyle=\sffamily,
	keywordstyle=\color{black}\bfseries,
	% stringstyle=\ttfamily, % Ugly! do we really want this? -- on
	mathescape=true,
	showstringspaces=false,
	keepspaces=true,
	breaklines=true,
	breakautoindent=true,
	backgroundcolor=\color{source},
	lineskip={-1pt}, % Ugly hack
	upquote=true, % straight quote; requires textcomp package
	columns=fullflexible} % no fixed width fonts
% In-line code (literal)
% Normally use this for all in-line code:
\newcommand{\ct}{\lstinline[mathescape=false,backgroundcolor=\color{white},basicstyle={\sffamily\upshape}]}
% apb 2007.8.28 added the \upshape declaration to avoid getting italicized code in \dothis{ } sections.
% In-line code (latex enabled)
% Use this only in special situations where \ct does not work
% (within section headings ...):
\newcommand{\lct}[1]{{\textsf{\textup{#1}}}}
% Use these for system categories and protocols:
\newcommand{\scat}[1]{\emph{\textsf{#1}}\xspace}
\newcommand{\pkg}[1]{\emph{\textsf{#1}}\xspace}
\newcommand{\prot}[1]{\emph{\textsf{#1}}\xspace}
% Code environments
% NB: the arg is for tests
% Only code and example environments may be tests
\lstnewenvironment{code}[1]{%
	\lstset{%
		% frame=lines,
		frame=single,
		framerule=0pt,
		mathescape=false
	}
}{}
\def\ignoredollar#1{}
%=============================================================
%:Code environments (method, script ...)
% NB: the third arg is for tests
% Only code and example environments may be tests
\lstnewenvironment{example}[3][defaultlabel]{%
	\renewcommand{\lstlistingname}{Example}%
	\lstset{
		% frame=lines,
		frame=single,
		framerule=0pt,
		mathescape=false,
		caption={\emph{#2}},
		label={eg:#1}
	}
}{}
\lstnewenvironment{script}[2][defaultlabel]{%
\renewcommand{\lstlistingname}{Script}%
	\lstset{
		% frame=lines,
		frame=single,
		framerule=0pt,
		mathescape=false,
		name={Script},
		caption={\emph{#2}},
		label={scr:#1}
	}
}{}
\lstnewenvironment{method}[2][defaultlabel]{%
	\renewcommand{\lstlistingname}{Method}%
	\lstset{
		% frame=lines,
		frame=single,
		framerule=0pt,
		mathescape=false,
		name={Method},
		caption={\emph{#2}},
		label={mth:#1}
	}
}{}
\lstnewenvironment{methods}[2][defaultlabel]{% just for multiple methods at once
	\renewcommand{\lstlistingname}{Methods}%
	\lstset{
		% frame=lines,
		frame=single,
		framerule=0pt,
		mathescape=false,
		name={Method},
		caption={\emph{#2}},
		label={mth:#1}
	}
}{}
\lstnewenvironment{numMethod}[2][defaultlabel]{%
	\renewcommand{\lstlistingname}{Method}%
	\lstset{
		numbers=left,
		numberstyle={\tiny\sffamily},
		% frame=lines,
		frame=single,
		framerule=0pt,
		mathescape=false,
		name={Method},
		caption={\emph{#2}},
		label={mth:#1}
	}
}{}
\lstnewenvironment{classdef}[2][defaultlabel]{%
	\renewcommand{\lstlistingname}{Class}%
	\lstset{
		% frame=lines,
		frame=single,
		framerule=0pt,
		mathescape=false,
		name={Class},
		caption={\emph{#2}},
		label={cls:#1}
	}
}{}
%=============================================================
%:Reserving space
% Usually need one more line than the actual lines of code
\newcommand{\needlines}[1]{\Needspace{#1\baselineskip}}
%=============================================================
%:Indexing macros
% Macros ending with "ind" generate text as well as an index entry
% Macros ending with "index" *only* generate an index entry
\newcommand{\ind}[1]{\index{#1}#1\xspace} % plain text
\newcommand{\subind}[2]{\index{#1!#2}#2\xspace} % show #2, subindex under #1
\newcommand{\emphind}[1]{\index{#1}\emph{#1}\xspace} % emph #1
\newcommand{\emphsubind}[2]{\index{#1!#2}\emph{#2}\xspace} % show emph #2, subindex inder #1
\newcommand{\scatind}[1]{\index{#1@\textsf{#1} (category)}\scat{#1}} % category
\newcommand{\pkgind}[1]{\index{#1@\textsf{#1} (package)}\pkg{#1}} % package
\newcommand{\protind}[1]{\index{#1@\textsf{#1} (protocol)}\prot{#1}} % protocol
\newcommand{\clsind}[1]{\index{#1@\textsf{#1} (class)}\ct{#1}\xspace}
% \newcommand{\clsind}[1]{\index{#1!\#@(class)}\ct{#1}\xspace} % class
\newcommand{\clsindplural}[1]{\index{#1!\#@(class)}\ct{#1}s\xspace} % class
\newcommand{\cvind}[1]{\index{#1@\textsf{#1} (class variable)}\ct{#1}\xspace} % class var
\newcommand{\glbind}[1]{\index{#1@\textsf{#1} (global)}\ct{#1}\xspace} % global
\newcommand{\patind}[1]{\index{#1@#1 (pattern)}\ct{#1}\xspace} % pattern
\newcommand{\pvind}[1]{\index{#1@\textsf{#1} (pseudo variable)}\ct{#1}\xspace} % pseudo var
\newcommand{\clsmthind}[2]{\index{#1!#2@\ct{#2}}\ct{#1>>>#2}\xspace} % class + method name
\newcommand{\mthind}[2]{\index{#1!#2@\ct{#2}}\ct{#2}\xspace} % show method name only
\newcommand{\lmthind}[2]{\index{#1!#2@\ct{#2}}\lct{#2}\xspace} % show method name only
\newcommand{\cmind}[2]{\index{#1!#2@\ct{#2}}\ct{#1>>>#2}\xspace} % show class>>method
\newcommand{\lcmind}[2]{\index{#1!#2@\ct{#2}}\lct{#1>>>#2}\xspace} % show class>>method
\newcommand{\toolsflapind}{\index{Tools flap}\toolsflap} % index tools flap
% The following only generate an index entry:
% \newcommand{\clsindex}[1]{\index{#1@\textsf{#1} (class)}}
\newcommand{\clsindex}[1]{\index{#1!\#@(class)}} % class
\newcommand{\mthindex}[2]{\index{#1!#2@\ct{#2}}} % method
\newcommand{\cmindex}[2]{\index{#1!#2@\ct{#2}}} % class>>method
\newcommand{\cvindex}[1]{\index{#1@\textsf{#1} (class variable)}} % class var
\newcommand{\glbindex}[1]{\index{#1@\textsf{#1} (global)}}% global
\newcommand{\pvindex}[1]{\index{#1@\textsf{#1} (pseudo variable)}}% pseudo var
\newcommand{\seeindex}[2]{\index{#1|see{#2}}} % #1, see #2
\newcommand{\scatindex}[1]{\index{#1@\textsf{#1} (category)}} % category
\newcommand{\pkgindex}[1]{\index{#1@\textsf{#1} (package)}} % package
\newcommand{\protindex}[1]{\index{#1@\textsf{#1} (protocol)}} % protocol
% How can we have the main entry page numbers in bold yet not break the hyperlink?
\newcommand{\boldidx}[1]{{\bf #1}} % breaks hyperlink
%\newcommand{\indmain}[1]{\index{#1|boldidx}#1\xspace} % plain text, main entry
%\newcommand{\emphsubindmain}[2]{\index{#1!#2|boldidx}\emph{#2}\xspace} % subindex, main entry
%\newcommand{\subindmain}[2]{\index{#1!#2|boldidx}#2\xspace} % subindex, main entry
%\newcommand{\clsindmain}[1]{\index{#1@\textsf{#1} (class)|boldidx}\ct{#1}\xspace}
%\newcommand{\clsindmain}[1]{\index{#1!\#@(class)|boldidx}\ct{#1}\xspace} % class main
%\newcommand{\indexmain}[1]{\index{#1|boldidx}} % main index entry only
\newcommand{\indmain}[1]{\index{#1}#1\xspace} % The main index entry for this item
\newcommand{\emphsubindmain}[2]{\index{#1!#2}\emph{#2}\xspace} % subindex, main entry
\newcommand{\subindmain}[2]{\index{#1!#2}#2\xspace} % subindex, main entry
%\newcommand{\clsindmain}[1]{\index{#1@\textsf{#1} (class)}\ct{#1}\xspace}
\newcommand{\clsindmain}[1]{\index{#1!\#@(class)}\ct{#1}\xspace} % class main
\newcommand{\clsindexmain}[1]{\index{#1!\#@(class)}} % class main index only
\newcommand{\indexmain}[1]{\index{#1}} 
%=============================================================
%:Code macros
% some constants
\newcommand{\codesize}{\small}
\newcommand{\codefont}{\sffamily}
%\newcommand{\cat}[1]{\textit{In category #1}}%%To remove later
\newlength{\scriptindent}
\setlength{\scriptindent}{.3cm}
%% Method presentation constants
\newlength{\methodindent}
\newlength{\methodwordlength}
\newlength{\aftermethod}
\setlength{\methodindent}{0.2cm}
\settowidth{\methodwordlength}{\ M\'ethode\ }
%=============================================================
%:Smalltalk macros
%\newcommand{\sep}{{$\gg$}}
\newcommand{\sep}{\mbox{>>}}
\newcommand{\self}{\lct{self}\xspace}
\newcommand{\super}{\lct{super}\xspace}
\newcommand{\nil}{\lct{nil}\xspace}
%=============================================================
% be less conservative about float placement
% these commands are from http://www.tex.ac.uk/cgi-bin/texfaq2html?label=floats
\renewcommand{\topfraction}{.9}
\renewcommand{\bottomfraction}{.9}
\renewcommand{\textfraction}{.1}
\renewcommand{\floatpagefraction}{.85}
\renewcommand{\dbltopfraction}{.66}
\renewcommand{\dblfloatpagefraction}{.85}
\setcounter{topnumber}{9}
\setcounter{bottomnumber}{9}
\setcounter{totalnumber}{20}
\setcounter{dbltopnumber}{9}
%=============================================================
% Give information from each chapter's author
\newcommand{\contact}[2]{\textbf{#1} \textsf{(#2)}}

\newcommand{\chapterauthor}[1]{\emph{with the participation of:\\#1}\\}

\newcommand{\authornoury}{\contact{Noury Bouraqadi}{Noury.Bouraqadi@mines-douai.fr}}
\newcommand{\authordamienc}{\contact{Damien Cassou}{damien.cassou@gmail.com}}
\newcommand{\authoroscar}{\contact{Oscar Nierstrasz}{oscar.nierstrasz@acm.org}}
\newcommand{\authorsteph}{\contact{St\'ephane Ducasse}{stephane.ducasse@inria.fr}}
\newcommand{\authoralex}{\contact{Alexandre Bergel}{alexandre@bergel.eu}}
\newcommand{\authorolivier}{\contact{Olivier Auverlot}{olivier.auverlot@inria.fr}}
\newcommand{\authornicolas}{\contact{Nicolas Cellier}{nicolas.cellier.aka.nice@gmail.com}}
\newcommand{\authormarcus}{\contact{Marcus Denker}{marcus.denker@inria.fr}}
\newcommand{\authoralain}{\contact{Alain Plantec}{alain.plantec@univ-brest.fr}}
\newcommand{\authordale}{\contact{Dale Henrichs}{dale.henrichs@gemstone.com}}
\newcommand{\authormariano}{\contact{Mariano Martinez Peck}{marianopeck@gmail.com}}
\newcommand{\authorsven}{\contact{Sven Van Caekenberghe}{sven@beta9.be}}
\newcommand{\authorlukas}{\contact{Lukas Renggli}{renggli@gmail.com}}

%=============================================================
% apb doesn't like paragraphs to run in to each other without a break
\parskip 1ex
%=============================================================
%:Stuff to check, merge or deprecate
%\setlength{\marginparsep}{2mm}
%\renewcommand{\baselinestretch}{1.1}
%=============================================================

	\pagestyle{headings}
	\setboolean{lulu}{true}
% --------------------------------------------
% A4:
%	\documentclass[a4paper,11pt,twoside]{book}
%	% $Author$
% $Date$
% $Revision$

% HISTORY:
% 2006-10-31 - Oscar code macros
% ...

%=============================================================
% NB: documentclass must be set in main document.
% Allows book to be generated in multiple formats.
%=============================================================
%:Packages
\usepackage[T1]{fontenc}  %%%%%% really important to get the code directly in the text!
\usepackage{lmodern}
%\usepackage[scaled=0.85]{bookmanx} % needs another scale factor if used with \renewcommand{\sfdefault}{cmbr}
\usepackage{palatino}
\usepackage[scaled=0.85]{helvet}
\usepackage[protrusion,expansion=false]{microtype}
\usepackage{graphicx}
\usepackage{theorem}
\usepackage[english]{babel}
% ON: pdfsync breaks the use of p{width} for tabular columns!
\ifdefined\usepdfsync\usepackage{pdfsync}\fi % Requires texlive 2007
%=============================================================
%:More packages
%Stef should check which ones are used!
%\usepackage{picinpar}
%\usepackage{layout}
%\usepackage{color}
%\usepackage{enum}
%\usepackage{a4wide}
% \usepackage{fancyhdr}
\usepackage{ifthen}
\usepackage{float}
\usepackage{longtable}
\usepackage{makeidx}
\usepackage[nottoc]{tocbibind}
\usepackage{multicol}
\usepackage{booktabs}	% book-style tables
\usepackage{topcapt}	% enables \topcaption
\usepackage{multirow}
\usepackage{tabularx}
%\usepackage[bottom]{footmisc}
\usepackage{xspace}
\usepackage{alltt}
\usepackage{amssymb,textcomp}
\usepackage[usenames,dvipsnames]{color}
%\usepackage{colortbl}
\usepackage[hang]{subfigure}\makeatletter\def\p@subfigure{\thefigure\,}\makeatother
\usepackage{rotating}
\usepackage{enumitem}	% apb: allows more control over tags in enumerations
\usepackage{verbatim}     % for comment environment
\usepackage{varioref}	% for page references that work
\labelformat{footnote}{\thechapter--#1} % to distinguish citations from jurabib
\usepackage{needspace}
\usepackage{isodateo} % enable \isodate
\usepackage[newparttoc]{titlesec}
\usepackage{titletoc}
\usepackage{wrapfig}
\usepackage[
	super,
	citefull=first,
	authorformat={allreversed,and},
	titleformat={commasep,italic}
]{jurabib} % citations as footnotes
\usepackage[
	colorlinks=true,
	linkcolor=black,
	urlcolor=black,
	citecolor=black
]{hyperref}   % should come last
%=============================================================
%:PDF version
\pdfminorversion=3 % Set PDF to 1.3 for Lulu
%=============================================================
%:URL style
\makeatletter
\def\url@leostyle{%
  \@ifundefined{selectfont}{\def\UrlFont{\sf}}{\def\UrlFont{\sffamily}}}
\makeatother
% Now actually use the newly defined style.
\urlstyle{leo}
%=============================================================
%:Booleans
\newboolean{lulu}
\setboolean{lulu}{false}
\newcommand{\ifluluelse}[2]{\ifthenelse{\boolean{lulu}}{#1}{#2}}
%=============================================================
%:Names
\newcommand{\SUnit}{SUnit\xspace}
\newcommand{\sunit}{SUnit\xspace}
\newcommand{\xUnit}{$x$Unit\xspace}
\newcommand{\JUnit}{JUnit\xspace}
\newcommand{\st}{Smalltalk\xspace}
\newcommand{\pharo}{Pharo\xspace} % Use this, not \Pharo
%\newcommand{\sqmap}{SqueakMap\xspace}
\newcommand{\squeak}{Squeak\xspace} % use this, not \Squeak or \sq
\newcommand{\sqsrc}{SqueakSource\xspace}
\newcommand{\sbe}{\url{http://SqueakByExample.org}\xspace}
\newcommand{\pharoweb}{\url{http://pharo-project.org}\xspace}
\newcommand{\pbe}{\url{http://PharoByExample.org}\xspace}
\newcommand{\sba}{\url{http://SquareBracketAssociates.org}\xspace}
\newcommand{\bam}{\lct{Bounc\-ing\-Atoms\-Morph}\xspace}
%=============================================================
%:Markup macros for proof-reading
\usepackage[normalem]{ulem} % for \sout
\usepackage{xcolor}
\newcommand{\ra}{$\rightarrow$}
\newcommand{\ugh}[1]{\textcolor{red}{\uwave{#1}}} % please rephrase
\newcommand{\ins}[1]{\textcolor{blue}{\uline{#1}}} % please insert
\newcommand{\del}[1]{\textcolor{red}{\sout{#1}}} % please delete
\newcommand{\chg}[2]{\textcolor{red}{\sout{#1}}{\ra}\textcolor{blue}{\uline{#2}}} % please change
%=============================================================
%:Editorial comment macros
%\newcommand{\nnbb}[2]{
%    % \fbox{\bfseries\sffamily\scriptsize#1}
%    \fcolorbox{gray}{yellow}{\bfseries\sffamily\scriptsize#1}
%    {\sf\small$\blacktriangleright$\textit{#2}$\blacktriangleleft$}
%   }
\newcommand{\yellowbox}[1]{\fcolorbox{gray}{yellow}{\bfseries\sffamily\scriptsize#1}}
\newcommand{\triangles}[1]{{\sf\small$\blacktriangleright$\textit{#1}$\blacktriangleleft$}}
\newcommand{\nnbb}[2]{\yellowbox{#1} \triangles{#2}}
\newcommand{\fix}{\yellowbox{FIX!}}
\newcommand{\here}{\yellowbox{CONTINUE HERE!}}
% editor macros
\newcommand{\apl}[1]{\nnbb{Alain}{#1}} % Alain
\newcommand{\ab}[1]{\nnbb{Andrew}{#1}} % Black
\newcommand{\sd}[1]{\nnbb{St\'{e}f}{#1}} % Ducasse
\newcommand{\dc}[1]{\nnbb{DamienC}{#1}} % Ducasse
\newcommand{\md}[1]{\nnbb{Marcus}{#1}} % Denker
\newcommand{\on}[1]{\nnbb{Oscar}{#1}} % Nierstrasz
\newcommand{\damien}[1]{\nnbb{Damien}{#1}} % Pollet
\newcommand{\lr}[1]{\nnbb{Lukas}{#1}} % Renggli
\newcommand{\orla}[1]{\nnbb{Orla}{#1}} % Greevy
\newcommand{\alex}[1]{\nnbb{Alex}{#1}} % Bergel
\newcommand{\alx}[1]{\nnbb{Alex}{#1}} % Bergel
\newcommand{\dr}[1]{\nnbb{David}{#1}} % Roethlisberger
\newcommand{\ja}[1]{\nnbb{Jannik}{#1}} % Laval
\newcommand{\jr}[1]{\nnbb{Jorge}{#1}} % Ressia
\newcommand{\jb}[1]{\nnbb{JB}{#1}} % JB
\newcommand{\fp}[1]{\nnbb{Fabrizio}{#1}} % Perin
\newcommand{\michael}[1]{\nnbb{Michael}{#1}} % Davies
\newcommand{\ew}[1]{\nnbb{Erwann}{#1}} % Wernli
\newcommand{\mb}[1]{\nnbb{Martial}{#1}} % Boniou
\newcommand{\hw}[1]{\nnbb{Hernan}{#1}} % Wilkinson
\newcommand{\ben}[1]{\nnbb{Benjamin}{#1}} % Benjamin Van Ryseghem
\newcommand{\mmp}[1]{\nnbb{Mariano}{#1}} % Mariano Martinez Peck
%=============================================================
%:Abbreviation macros
\newcommand{\ie}{\emph{i.e.},\xspace}
\newcommand{\eg}{\emph{e.g.},\xspace}
\newcommand{\etc}{etc.\xspace}
%=============================================================
%:Cross reference macros
\newcommand{\charef}[1]{Chapter~\ref{cha:#1}\xspace}
\newcommand{\secref}[1]{Section~\ref{sec:#1}\xspace}
\newcommand{\figref}[1]{Figure~\ref{fig:#1}\xspace}
\newcommand{\Figref}[1]{Figure~\ref{fig:#1}\xspace}
\newcommand{\appref}[1]{Appendix~\ref{app:#1}\xspace}
\newcommand{\tabref}[1]{Table~\ref{tab:#1}\xspace}
\newcommand{\faqref}[1]{FAQ~\ref{faq:#1}, p.~\pageref{faq:#1}\xspace}
% APB: I removed trailing \xspace commands from these macros because
% \xspace mostly doesn't work.  If you want a space after your
% references, type one!
% ON: xspace has always worked just fine for me!  Please leave them in.
%
\newcommand{\ruleref}[1]{\ref{rule:#1}\xspace}
%
\newcommand{\egref}[1]{example~\ref{eg:#1}\xspace}
\newcommand{\Egref}[1]{Example~\ref{eg:#1}\xspace}
%
\newcommand{\scrref}[1]{script~\ref{scr:#1}\xspace}
\newcommand{\Scrref}[1]{Script~\ref{scr:#1}\xspace}
\newcommand{\tscrref}[1]{the script~\ref{scr:#1}\xspace}
\newcommand{\Tscrref}[1]{The script~\ref{scr:#1}\xspace}
%
\newcommand{\mthref}[1]{method~\ref{mth:#1}\xspace}
\newcommand{\mthsref}[1]{methods~\ref{mth:#1}\xspace}
\newcommand{\Mthref}[1]{Method~\ref{mth:#1}\xspace}
\newcommand{\tmthref}[1]{the method~\ref{mth:#1}\xspace}
\newcommand{\Tmthref}[1]{The method~\ref{mth:#1}\xspace}
%
\newcommand{\clsref}[1]{class~\ref{cls:#1}\xspace}
\newcommand{\tclsref}[1]{the class~\ref{cls:#1}\xspace}
\newcommand{\Tclsref}[1]{The class~\ref{cls:#1}\xspace}

\newcommand{\chalabel}[1]{\label{cha:#1}}
\newcommand{\seclabel}[1]{\label{sec:#1}}
\newcommand{\figlabel}[1]{\label{fig:#1}}
\newcommand{\tablabel}[1]{\label{tab:#1}}
\newcommand{\rulelabel}[1]{\label{rule:#1}}
\newcommand{\eglabel}[1]{\label{eg:#1}}
\newcommand{\scrlabel}[1]{\label{scr:#1}}
\newcommand{\mthlabel}[1]{\label{mth:#1}}
\newcommand{\clslabel}[1]{\label{cls:#1}}
\newcommand{\faqlabel}[1]{\label{faq:#1}}
%=============================================================
%:Menu item macro
% for menu items, so we can change our minds on how to print them! (apb)
\definecolor{lightgray}{gray}{0.89}
\newcommand{\menu}[1]{{%
	\setlength{\fboxsep}{0pt}%
	\colorbox{lightgray}{{{\upshape\sffamily\strut \,#1\,}}}}}
\newcommand{\link}[1]{{%
	\fontfamily{lmr}\selectfont
 	\upshape{\sffamily \underline{#1}}}}
% For submenu items:
\newcommand{\go}{\,$\triangleright$\,}
% \newcommand{\go}{\,$\blacktriangleright$\,}
% For keyboard shortcuts:
%\newcommand{\short}[1]{\mbox{$\langle${\sc CMD}$\rangle$-#1}\xspace}
\newcommand{\short}[1]{\mbox{{\sc cmd}\hspace{0.08em}--\hspace{0.09em}#1}\xspace}
% For buttons:
\newcommand{\button}[1]{{%
	\setlength{\fboxsep}{0pt}%
	\fbox{{\upshape\sffamily\strut \,#1\,}}}}
% NB: The button macro does not work within captions -- incompatible with xcolor package :-(
\newcommand{\toolsflap}{\textit{Tools} flap\xspace}
%=============================================================
%:Mouse clicks
\newcommand{\click}{click\xspace} % RED
\newcommand{\actclick}{action-click\xspace} % YELLOW
\newcommand{\metaclick}{meta-click\xspace} % BLUE
\newcommand{\Click}{Click\xspace} % RED
\newcommand{\Actclick}{Action-click\xspace} % YELLOW
\newcommand{\Metaclick}{Meta-click\xspace} % BLUE
%=============================================================
%:ToSh macros
\newboolean{tosh}
\setboolean{tosh}{false}
\newcommand{\iftoshelse}[2]{\ifthenelse{\boolean{tosh}}{#1}{#2}}
%=============================================================
%:ToSh colors
%\newcommand{\highlightcolor}{\color{blue!65}}
%\newcommand{\boxcolor}{\color{gray!25}}
\newcommand{\highlight}[1]{\textcolor{blue!65}{#1}}
%\newcommand{\codecolor}{\color{blue!65}}
%%\setlength{\fboxrule}{2pt}
%\newcommand{\asPict}[1]{%
%	{\Large\highlight{#1}}}
%=============================================================
%:Reader cues (do this)
%
% Indicate something the reader should try out.
% \newcommand{\dothisicon}{\raisebox{-.5ex}{\includegraphics[width=1.4em]{squeak-logo}}}
\iftoshelse{
	\usepackage{marginnote}
		\renewcommand*{\marginfont}{\footnotesize}
	\newcommand{\vartriangleout}{\ifthenelse{\isodd{\thepage}}{\vartriangleright}{\vartriangleleft}}
	\newcommand{\dothisicon}{\fcolorbox{blue!65}{white}{\highlight{$\vartriangleout$}}}
	\newcommand{\dothis}[1]{%
		\noindent\par\noindent
		{\reversemarginpar
			\marginnote{\fcolorbox{blue!65}{white}{\highlight{$\vartriangleout$}}}}
		%\MarginLabel{do this}
		\noindent\emph{#1}
		\nopagebreak}
}{
	\newcommand{\dothisicon}{\raisebox{-.5ex}{\includegraphics[height=1.2em]{pharo}}}
	\newcommand{\dothis}[1]{%
		\medskip
		\noindent\dothisicon
		\ifx#1\empty\else\quad\emph{#1}\fi
		\par\smallskip\nopagebreak}
}
%===> NEW VERSION <===
% NB: To use this in an individual chapter, you must set:
%\graphicspath{{figures/} {../figures/}}
% at the head of the chapter.  Don't forget the final /
%=============================================================
%:Reader hints (hint)
%
% Indicates a non-obvious consequence 
\newcommand{\hint}[1]{\vspace{1ex}\noindent\fbox{\textsc{Hint}} \emph{#1}}
%=================================================================
% graphics for Morphic handles
\newcommand{\grabHandle}{\raisebox{-0.2ex}{\includegraphics[width=1em]{blackHandle}}}
\newcommand{\moveHandle}{\raisebox{-0.2ex}{\includegraphics[width=1em]{moveHandle}}}
\newcommand{\debugHandle}{\raisebox{-0.2ex}{\includegraphics[width=1em]{debugHandle}}}
%=============================================================
%:Highlighting Important stuff (doublebox)
%
% From Seaside book ...
\newsavebox{\SavedText}
\newlength{\InnerBoxRule}\setlength{\InnerBoxRule}{.75\fboxrule}
\newlength{\OuterBoxRule}\setlength{\OuterBoxRule}{1.5\fboxrule}
\newlength{\BoxSeparation}\setlength{\BoxSeparation}{1.5\fboxrule}
\addtolength{\BoxSeparation}{.5pt}
\newlength{\SaveBoxSep}\setlength{\SaveBoxSep}{2\fboxsep}
%
\newenvironment{doublebox}{\begin{lrbox}{\SavedText}
    \begin{minipage}{.75\textwidth}}
    {\end{minipage}\end{lrbox}\begin{center}
    \setlength{\fboxsep}{\BoxSeparation}\setlength{\fboxrule}{\OuterBoxRule}
    \fbox{\setlength{\fboxsep}{\SaveBoxSep}\setlength{\fboxrule}{\InnerBoxRule}%
      \fbox{\usebox{\SavedText}}}
  \end{center}}
% Use this:
\newcommand{\important}[1]{\begin{doublebox}#1\end{doublebox}}
%=============================================================
%:Section depth
\setcounter{secnumdepth}{2}
%% for this to happen start the file with
%\ifx\wholebook\relax\else
%% $Author$
% $Date$
% $Revision$

% HISTORY:
% 2006-10-31 - Oscar code macros
% ...

%=============================================================
% NB: documentclass must be set in main document.
% Allows book to be generated in multiple formats.
%=============================================================
%:Packages
\usepackage[T1]{fontenc}  %%%%%% really important to get the code directly in the text!
\usepackage{lmodern}
%\usepackage[scaled=0.85]{bookmanx} % needs another scale factor if used with \renewcommand{\sfdefault}{cmbr}
\usepackage{palatino}
\usepackage[scaled=0.85]{helvet}
\usepackage[protrusion,expansion=false]{microtype}
\usepackage{graphicx}
\usepackage{theorem}
\usepackage[english]{babel}
% ON: pdfsync breaks the use of p{width} for tabular columns!
\ifdefined\usepdfsync\usepackage{pdfsync}\fi % Requires texlive 2007
%=============================================================
%:More packages
%Stef should check which ones are used!
%\usepackage{picinpar}
%\usepackage{layout}
%\usepackage{color}
%\usepackage{enum}
%\usepackage{a4wide}
% \usepackage{fancyhdr}
\usepackage{ifthen}
\usepackage{float}
\usepackage{longtable}
\usepackage{makeidx}
\usepackage[nottoc]{tocbibind}
\usepackage{multicol}
\usepackage{booktabs}	% book-style tables
\usepackage{topcapt}	% enables \topcaption
\usepackage{multirow}
\usepackage{tabularx}
%\usepackage[bottom]{footmisc}
\usepackage{xspace}
\usepackage{alltt}
\usepackage{amssymb,textcomp}
\usepackage[usenames,dvipsnames]{color}
%\usepackage{colortbl}
\usepackage[hang]{subfigure}\makeatletter\def\p@subfigure{\thefigure\,}\makeatother
\usepackage{rotating}
\usepackage{enumitem}	% apb: allows more control over tags in enumerations
\usepackage{verbatim}     % for comment environment
\usepackage{varioref}	% for page references that work
\labelformat{footnote}{\thechapter--#1} % to distinguish citations from jurabib
\usepackage{needspace}
\usepackage{isodateo} % enable \isodate
\usepackage[newparttoc]{titlesec}
\usepackage{titletoc}
\usepackage{wrapfig}
\usepackage[
	super,
	citefull=first,
	authorformat={allreversed,and},
	titleformat={commasep,italic}
]{jurabib} % citations as footnotes
\usepackage[
	colorlinks=true,
	linkcolor=black,
	urlcolor=black,
	citecolor=black
]{hyperref}   % should come last
%=============================================================
%:PDF version
\pdfminorversion=3 % Set PDF to 1.3 for Lulu
%=============================================================
%:URL style
\makeatletter
\def\url@leostyle{%
  \@ifundefined{selectfont}{\def\UrlFont{\sf}}{\def\UrlFont{\sffamily}}}
\makeatother
% Now actually use the newly defined style.
\urlstyle{leo}
%=============================================================
%:Booleans
\newboolean{lulu}
\setboolean{lulu}{false}
\newcommand{\ifluluelse}[2]{\ifthenelse{\boolean{lulu}}{#1}{#2}}
%=============================================================
%:Names
\newcommand{\SUnit}{SUnit\xspace}
\newcommand{\sunit}{SUnit\xspace}
\newcommand{\xUnit}{$x$Unit\xspace}
\newcommand{\JUnit}{JUnit\xspace}
\newcommand{\st}{Smalltalk\xspace}
\newcommand{\pharo}{Pharo\xspace} % Use this, not \Pharo
%\newcommand{\sqmap}{SqueakMap\xspace}
\newcommand{\squeak}{Squeak\xspace} % use this, not \Squeak or \sq
\newcommand{\sqsrc}{SqueakSource\xspace}
\newcommand{\sbe}{\url{http://SqueakByExample.org}\xspace}
\newcommand{\pharoweb}{\url{http://pharo-project.org}\xspace}
\newcommand{\pbe}{\url{http://PharoByExample.org}\xspace}
\newcommand{\sba}{\url{http://SquareBracketAssociates.org}\xspace}
\newcommand{\bam}{\lct{Bounc\-ing\-Atoms\-Morph}\xspace}
%=============================================================
%:Markup macros for proof-reading
\usepackage[normalem]{ulem} % for \sout
\usepackage{xcolor}
\newcommand{\ra}{$\rightarrow$}
\newcommand{\ugh}[1]{\textcolor{red}{\uwave{#1}}} % please rephrase
\newcommand{\ins}[1]{\textcolor{blue}{\uline{#1}}} % please insert
\newcommand{\del}[1]{\textcolor{red}{\sout{#1}}} % please delete
\newcommand{\chg}[2]{\textcolor{red}{\sout{#1}}{\ra}\textcolor{blue}{\uline{#2}}} % please change
%=============================================================
%:Editorial comment macros
%\newcommand{\nnbb}[2]{
%    % \fbox{\bfseries\sffamily\scriptsize#1}
%    \fcolorbox{gray}{yellow}{\bfseries\sffamily\scriptsize#1}
%    {\sf\small$\blacktriangleright$\textit{#2}$\blacktriangleleft$}
%   }
\newcommand{\yellowbox}[1]{\fcolorbox{gray}{yellow}{\bfseries\sffamily\scriptsize#1}}
\newcommand{\triangles}[1]{{\sf\small$\blacktriangleright$\textit{#1}$\blacktriangleleft$}}
\newcommand{\nnbb}[2]{\yellowbox{#1} \triangles{#2}}
\newcommand{\fix}{\yellowbox{FIX!}}
\newcommand{\here}{\yellowbox{CONTINUE HERE!}}
% editor macros
\newcommand{\apl}[1]{\nnbb{Alain}{#1}} % Alain
\newcommand{\ab}[1]{\nnbb{Andrew}{#1}} % Black
\newcommand{\sd}[1]{\nnbb{St\'{e}f}{#1}} % Ducasse
\newcommand{\dc}[1]{\nnbb{DamienC}{#1}} % Ducasse
\newcommand{\md}[1]{\nnbb{Marcus}{#1}} % Denker
\newcommand{\on}[1]{\nnbb{Oscar}{#1}} % Nierstrasz
\newcommand{\damien}[1]{\nnbb{Damien}{#1}} % Pollet
\newcommand{\lr}[1]{\nnbb{Lukas}{#1}} % Renggli
\newcommand{\orla}[1]{\nnbb{Orla}{#1}} % Greevy
\newcommand{\alex}[1]{\nnbb{Alex}{#1}} % Bergel
\newcommand{\alx}[1]{\nnbb{Alex}{#1}} % Bergel
\newcommand{\dr}[1]{\nnbb{David}{#1}} % Roethlisberger
\newcommand{\ja}[1]{\nnbb{Jannik}{#1}} % Laval
\newcommand{\jr}[1]{\nnbb{Jorge}{#1}} % Ressia
\newcommand{\jb}[1]{\nnbb{JB}{#1}} % JB
\newcommand{\fp}[1]{\nnbb{Fabrizio}{#1}} % Perin
\newcommand{\michael}[1]{\nnbb{Michael}{#1}} % Davies
\newcommand{\ew}[1]{\nnbb{Erwann}{#1}} % Wernli
\newcommand{\mb}[1]{\nnbb{Martial}{#1}} % Boniou
\newcommand{\hw}[1]{\nnbb{Hernan}{#1}} % Wilkinson
\newcommand{\ben}[1]{\nnbb{Benjamin}{#1}} % Benjamin Van Ryseghem
\newcommand{\mmp}[1]{\nnbb{Mariano}{#1}} % Mariano Martinez Peck
%=============================================================
%:Abbreviation macros
\newcommand{\ie}{\emph{i.e.},\xspace}
\newcommand{\eg}{\emph{e.g.},\xspace}
\newcommand{\etc}{etc.\xspace}
%=============================================================
%:Cross reference macros
\newcommand{\charef}[1]{Chapter~\ref{cha:#1}\xspace}
\newcommand{\secref}[1]{Section~\ref{sec:#1}\xspace}
\newcommand{\figref}[1]{Figure~\ref{fig:#1}\xspace}
\newcommand{\Figref}[1]{Figure~\ref{fig:#1}\xspace}
\newcommand{\appref}[1]{Appendix~\ref{app:#1}\xspace}
\newcommand{\tabref}[1]{Table~\ref{tab:#1}\xspace}
\newcommand{\faqref}[1]{FAQ~\ref{faq:#1}, p.~\pageref{faq:#1}\xspace}
% APB: I removed trailing \xspace commands from these macros because
% \xspace mostly doesn't work.  If you want a space after your
% references, type one!
% ON: xspace has always worked just fine for me!  Please leave them in.
%
\newcommand{\ruleref}[1]{\ref{rule:#1}\xspace}
%
\newcommand{\egref}[1]{example~\ref{eg:#1}\xspace}
\newcommand{\Egref}[1]{Example~\ref{eg:#1}\xspace}
%
\newcommand{\scrref}[1]{script~\ref{scr:#1}\xspace}
\newcommand{\Scrref}[1]{Script~\ref{scr:#1}\xspace}
\newcommand{\tscrref}[1]{the script~\ref{scr:#1}\xspace}
\newcommand{\Tscrref}[1]{The script~\ref{scr:#1}\xspace}
%
\newcommand{\mthref}[1]{method~\ref{mth:#1}\xspace}
\newcommand{\mthsref}[1]{methods~\ref{mth:#1}\xspace}
\newcommand{\Mthref}[1]{Method~\ref{mth:#1}\xspace}
\newcommand{\tmthref}[1]{the method~\ref{mth:#1}\xspace}
\newcommand{\Tmthref}[1]{The method~\ref{mth:#1}\xspace}
%
\newcommand{\clsref}[1]{class~\ref{cls:#1}\xspace}
\newcommand{\tclsref}[1]{the class~\ref{cls:#1}\xspace}
\newcommand{\Tclsref}[1]{The class~\ref{cls:#1}\xspace}

\newcommand{\chalabel}[1]{\label{cha:#1}}
\newcommand{\seclabel}[1]{\label{sec:#1}}
\newcommand{\figlabel}[1]{\label{fig:#1}}
\newcommand{\tablabel}[1]{\label{tab:#1}}
\newcommand{\rulelabel}[1]{\label{rule:#1}}
\newcommand{\eglabel}[1]{\label{eg:#1}}
\newcommand{\scrlabel}[1]{\label{scr:#1}}
\newcommand{\mthlabel}[1]{\label{mth:#1}}
\newcommand{\clslabel}[1]{\label{cls:#1}}
\newcommand{\faqlabel}[1]{\label{faq:#1}}
%=============================================================
%:Menu item macro
% for menu items, so we can change our minds on how to print them! (apb)
\definecolor{lightgray}{gray}{0.89}
\newcommand{\menu}[1]{{%
	\setlength{\fboxsep}{0pt}%
	\colorbox{lightgray}{{{\upshape\sffamily\strut \,#1\,}}}}}
\newcommand{\link}[1]{{%
	\fontfamily{lmr}\selectfont
 	\upshape{\sffamily \underline{#1}}}}
% For submenu items:
\newcommand{\go}{\,$\triangleright$\,}
% \newcommand{\go}{\,$\blacktriangleright$\,}
% For keyboard shortcuts:
%\newcommand{\short}[1]{\mbox{$\langle${\sc CMD}$\rangle$-#1}\xspace}
\newcommand{\short}[1]{\mbox{{\sc cmd}\hspace{0.08em}--\hspace{0.09em}#1}\xspace}
% For buttons:
\newcommand{\button}[1]{{%
	\setlength{\fboxsep}{0pt}%
	\fbox{{\upshape\sffamily\strut \,#1\,}}}}
% NB: The button macro does not work within captions -- incompatible with xcolor package :-(
\newcommand{\toolsflap}{\textit{Tools} flap\xspace}
%=============================================================
%:Mouse clicks
\newcommand{\click}{click\xspace} % RED
\newcommand{\actclick}{action-click\xspace} % YELLOW
\newcommand{\metaclick}{meta-click\xspace} % BLUE
\newcommand{\Click}{Click\xspace} % RED
\newcommand{\Actclick}{Action-click\xspace} % YELLOW
\newcommand{\Metaclick}{Meta-click\xspace} % BLUE
%=============================================================
%:ToSh macros
\newboolean{tosh}
\setboolean{tosh}{false}
\newcommand{\iftoshelse}[2]{\ifthenelse{\boolean{tosh}}{#1}{#2}}
%=============================================================
%:ToSh colors
%\newcommand{\highlightcolor}{\color{blue!65}}
%\newcommand{\boxcolor}{\color{gray!25}}
\newcommand{\highlight}[1]{\textcolor{blue!65}{#1}}
%\newcommand{\codecolor}{\color{blue!65}}
%%\setlength{\fboxrule}{2pt}
%\newcommand{\asPict}[1]{%
%	{\Large\highlight{#1}}}
%=============================================================
%:Reader cues (do this)
%
% Indicate something the reader should try out.
% \newcommand{\dothisicon}{\raisebox{-.5ex}{\includegraphics[width=1.4em]{squeak-logo}}}
\iftoshelse{
	\usepackage{marginnote}
		\renewcommand*{\marginfont}{\footnotesize}
	\newcommand{\vartriangleout}{\ifthenelse{\isodd{\thepage}}{\vartriangleright}{\vartriangleleft}}
	\newcommand{\dothisicon}{\fcolorbox{blue!65}{white}{\highlight{$\vartriangleout$}}}
	\newcommand{\dothis}[1]{%
		\noindent\par\noindent
		{\reversemarginpar
			\marginnote{\fcolorbox{blue!65}{white}{\highlight{$\vartriangleout$}}}}
		%\MarginLabel{do this}
		\noindent\emph{#1}
		\nopagebreak}
}{
	\newcommand{\dothisicon}{\raisebox{-.5ex}{\includegraphics[height=1.2em]{pharo}}}
	\newcommand{\dothis}[1]{%
		\medskip
		\noindent\dothisicon
		\ifx#1\empty\else\quad\emph{#1}\fi
		\par\smallskip\nopagebreak}
}
%===> NEW VERSION <===
% NB: To use this in an individual chapter, you must set:
%\graphicspath{{figures/} {../figures/}}
% at the head of the chapter.  Don't forget the final /
%=============================================================
%:Reader hints (hint)
%
% Indicates a non-obvious consequence 
\newcommand{\hint}[1]{\vspace{1ex}\noindent\fbox{\textsc{Hint}} \emph{#1}}
%=================================================================
% graphics for Morphic handles
\newcommand{\grabHandle}{\raisebox{-0.2ex}{\includegraphics[width=1em]{blackHandle}}}
\newcommand{\moveHandle}{\raisebox{-0.2ex}{\includegraphics[width=1em]{moveHandle}}}
\newcommand{\debugHandle}{\raisebox{-0.2ex}{\includegraphics[width=1em]{debugHandle}}}
%=============================================================
%:Highlighting Important stuff (doublebox)
%
% From Seaside book ...
\newsavebox{\SavedText}
\newlength{\InnerBoxRule}\setlength{\InnerBoxRule}{.75\fboxrule}
\newlength{\OuterBoxRule}\setlength{\OuterBoxRule}{1.5\fboxrule}
\newlength{\BoxSeparation}\setlength{\BoxSeparation}{1.5\fboxrule}
\addtolength{\BoxSeparation}{.5pt}
\newlength{\SaveBoxSep}\setlength{\SaveBoxSep}{2\fboxsep}
%
\newenvironment{doublebox}{\begin{lrbox}{\SavedText}
    \begin{minipage}{.75\textwidth}}
    {\end{minipage}\end{lrbox}\begin{center}
    \setlength{\fboxsep}{\BoxSeparation}\setlength{\fboxrule}{\OuterBoxRule}
    \fbox{\setlength{\fboxsep}{\SaveBoxSep}\setlength{\fboxrule}{\InnerBoxRule}%
      \fbox{\usebox{\SavedText}}}
  \end{center}}
% Use this:
\newcommand{\important}[1]{\begin{doublebox}#1\end{doublebox}}
%=============================================================
%:Section depth
\setcounter{secnumdepth}{2}
%% for this to happen start the file with
%\ifx\wholebook\relax\else
%% $Author$
% $Date$
% $Revision$

% HISTORY:
% 2006-10-31 - Oscar code macros
% ...

%=============================================================
% NB: documentclass must be set in main document.
% Allows book to be generated in multiple formats.
%=============================================================
%:Packages
\usepackage[T1]{fontenc}  %%%%%% really important to get the code directly in the text!
\usepackage{lmodern}
%\usepackage[scaled=0.85]{bookmanx} % needs another scale factor if used with \renewcommand{\sfdefault}{cmbr}
\usepackage{palatino}
\usepackage[scaled=0.85]{helvet}
\usepackage[protrusion,expansion=false]{microtype}
\usepackage{graphicx}
\usepackage{theorem}
\usepackage[english]{babel}
% ON: pdfsync breaks the use of p{width} for tabular columns!
\ifdefined\usepdfsync\usepackage{pdfsync}\fi % Requires texlive 2007
%=============================================================
%:More packages
%Stef should check which ones are used!
%\usepackage{picinpar}
%\usepackage{layout}
%\usepackage{color}
%\usepackage{enum}
%\usepackage{a4wide}
% \usepackage{fancyhdr}
\usepackage{ifthen}
\usepackage{float}
\usepackage{longtable}
\usepackage{makeidx}
\usepackage[nottoc]{tocbibind}
\usepackage{multicol}
\usepackage{booktabs}	% book-style tables
\usepackage{topcapt}	% enables \topcaption
\usepackage{multirow}
\usepackage{tabularx}
%\usepackage[bottom]{footmisc}
\usepackage{xspace}
\usepackage{alltt}
\usepackage{amssymb,textcomp}
\usepackage[usenames,dvipsnames]{color}
%\usepackage{colortbl}
\usepackage[hang]{subfigure}\makeatletter\def\p@subfigure{\thefigure\,}\makeatother
\usepackage{rotating}
\usepackage{enumitem}	% apb: allows more control over tags in enumerations
\usepackage{verbatim}     % for comment environment
\usepackage{varioref}	% for page references that work
\labelformat{footnote}{\thechapter--#1} % to distinguish citations from jurabib
\usepackage{needspace}
\usepackage{isodateo} % enable \isodate
\usepackage[newparttoc]{titlesec}
\usepackage{titletoc}
\usepackage{wrapfig}
\usepackage[
	super,
	citefull=first,
	authorformat={allreversed,and},
	titleformat={commasep,italic}
]{jurabib} % citations as footnotes
\usepackage[
	colorlinks=true,
	linkcolor=black,
	urlcolor=black,
	citecolor=black
]{hyperref}   % should come last
%=============================================================
%:PDF version
\pdfminorversion=3 % Set PDF to 1.3 for Lulu
%=============================================================
%:URL style
\makeatletter
\def\url@leostyle{%
  \@ifundefined{selectfont}{\def\UrlFont{\sf}}{\def\UrlFont{\sffamily}}}
\makeatother
% Now actually use the newly defined style.
\urlstyle{leo}
%=============================================================
%:Booleans
\newboolean{lulu}
\setboolean{lulu}{false}
\newcommand{\ifluluelse}[2]{\ifthenelse{\boolean{lulu}}{#1}{#2}}
%=============================================================
%:Names
\newcommand{\SUnit}{SUnit\xspace}
\newcommand{\sunit}{SUnit\xspace}
\newcommand{\xUnit}{$x$Unit\xspace}
\newcommand{\JUnit}{JUnit\xspace}
\newcommand{\st}{Smalltalk\xspace}
\newcommand{\pharo}{Pharo\xspace} % Use this, not \Pharo
%\newcommand{\sqmap}{SqueakMap\xspace}
\newcommand{\squeak}{Squeak\xspace} % use this, not \Squeak or \sq
\newcommand{\sqsrc}{SqueakSource\xspace}
\newcommand{\sbe}{\url{http://SqueakByExample.org}\xspace}
\newcommand{\pharoweb}{\url{http://pharo-project.org}\xspace}
\newcommand{\pbe}{\url{http://PharoByExample.org}\xspace}
\newcommand{\sba}{\url{http://SquareBracketAssociates.org}\xspace}
\newcommand{\bam}{\lct{Bounc\-ing\-Atoms\-Morph}\xspace}
%=============================================================
%:Markup macros for proof-reading
\usepackage[normalem]{ulem} % for \sout
\usepackage{xcolor}
\newcommand{\ra}{$\rightarrow$}
\newcommand{\ugh}[1]{\textcolor{red}{\uwave{#1}}} % please rephrase
\newcommand{\ins}[1]{\textcolor{blue}{\uline{#1}}} % please insert
\newcommand{\del}[1]{\textcolor{red}{\sout{#1}}} % please delete
\newcommand{\chg}[2]{\textcolor{red}{\sout{#1}}{\ra}\textcolor{blue}{\uline{#2}}} % please change
%=============================================================
%:Editorial comment macros
%\newcommand{\nnbb}[2]{
%    % \fbox{\bfseries\sffamily\scriptsize#1}
%    \fcolorbox{gray}{yellow}{\bfseries\sffamily\scriptsize#1}
%    {\sf\small$\blacktriangleright$\textit{#2}$\blacktriangleleft$}
%   }
\newcommand{\yellowbox}[1]{\fcolorbox{gray}{yellow}{\bfseries\sffamily\scriptsize#1}}
\newcommand{\triangles}[1]{{\sf\small$\blacktriangleright$\textit{#1}$\blacktriangleleft$}}
\newcommand{\nnbb}[2]{\yellowbox{#1} \triangles{#2}}
\newcommand{\fix}{\yellowbox{FIX!}}
\newcommand{\here}{\yellowbox{CONTINUE HERE!}}
% editor macros
\newcommand{\apl}[1]{\nnbb{Alain}{#1}} % Alain
\newcommand{\ab}[1]{\nnbb{Andrew}{#1}} % Black
\newcommand{\sd}[1]{\nnbb{St\'{e}f}{#1}} % Ducasse
\newcommand{\dc}[1]{\nnbb{DamienC}{#1}} % Ducasse
\newcommand{\md}[1]{\nnbb{Marcus}{#1}} % Denker
\newcommand{\on}[1]{\nnbb{Oscar}{#1}} % Nierstrasz
\newcommand{\damien}[1]{\nnbb{Damien}{#1}} % Pollet
\newcommand{\lr}[1]{\nnbb{Lukas}{#1}} % Renggli
\newcommand{\orla}[1]{\nnbb{Orla}{#1}} % Greevy
\newcommand{\alex}[1]{\nnbb{Alex}{#1}} % Bergel
\newcommand{\alx}[1]{\nnbb{Alex}{#1}} % Bergel
\newcommand{\dr}[1]{\nnbb{David}{#1}} % Roethlisberger
\newcommand{\ja}[1]{\nnbb{Jannik}{#1}} % Laval
\newcommand{\jr}[1]{\nnbb{Jorge}{#1}} % Ressia
\newcommand{\jb}[1]{\nnbb{JB}{#1}} % JB
\newcommand{\fp}[1]{\nnbb{Fabrizio}{#1}} % Perin
\newcommand{\michael}[1]{\nnbb{Michael}{#1}} % Davies
\newcommand{\ew}[1]{\nnbb{Erwann}{#1}} % Wernli
\newcommand{\mb}[1]{\nnbb{Martial}{#1}} % Boniou
\newcommand{\hw}[1]{\nnbb{Hernan}{#1}} % Wilkinson
\newcommand{\ben}[1]{\nnbb{Benjamin}{#1}} % Benjamin Van Ryseghem
\newcommand{\mmp}[1]{\nnbb{Mariano}{#1}} % Mariano Martinez Peck
%=============================================================
%:Abbreviation macros
\newcommand{\ie}{\emph{i.e.},\xspace}
\newcommand{\eg}{\emph{e.g.},\xspace}
\newcommand{\etc}{etc.\xspace}
%=============================================================
%:Cross reference macros
\newcommand{\charef}[1]{Chapter~\ref{cha:#1}\xspace}
\newcommand{\secref}[1]{Section~\ref{sec:#1}\xspace}
\newcommand{\figref}[1]{Figure~\ref{fig:#1}\xspace}
\newcommand{\Figref}[1]{Figure~\ref{fig:#1}\xspace}
\newcommand{\appref}[1]{Appendix~\ref{app:#1}\xspace}
\newcommand{\tabref}[1]{Table~\ref{tab:#1}\xspace}
\newcommand{\faqref}[1]{FAQ~\ref{faq:#1}, p.~\pageref{faq:#1}\xspace}
% APB: I removed trailing \xspace commands from these macros because
% \xspace mostly doesn't work.  If you want a space after your
% references, type one!
% ON: xspace has always worked just fine for me!  Please leave them in.
%
\newcommand{\ruleref}[1]{\ref{rule:#1}\xspace}
%
\newcommand{\egref}[1]{example~\ref{eg:#1}\xspace}
\newcommand{\Egref}[1]{Example~\ref{eg:#1}\xspace}
%
\newcommand{\scrref}[1]{script~\ref{scr:#1}\xspace}
\newcommand{\Scrref}[1]{Script~\ref{scr:#1}\xspace}
\newcommand{\tscrref}[1]{the script~\ref{scr:#1}\xspace}
\newcommand{\Tscrref}[1]{The script~\ref{scr:#1}\xspace}
%
\newcommand{\mthref}[1]{method~\ref{mth:#1}\xspace}
\newcommand{\mthsref}[1]{methods~\ref{mth:#1}\xspace}
\newcommand{\Mthref}[1]{Method~\ref{mth:#1}\xspace}
\newcommand{\tmthref}[1]{the method~\ref{mth:#1}\xspace}
\newcommand{\Tmthref}[1]{The method~\ref{mth:#1}\xspace}
%
\newcommand{\clsref}[1]{class~\ref{cls:#1}\xspace}
\newcommand{\tclsref}[1]{the class~\ref{cls:#1}\xspace}
\newcommand{\Tclsref}[1]{The class~\ref{cls:#1}\xspace}

\newcommand{\chalabel}[1]{\label{cha:#1}}
\newcommand{\seclabel}[1]{\label{sec:#1}}
\newcommand{\figlabel}[1]{\label{fig:#1}}
\newcommand{\tablabel}[1]{\label{tab:#1}}
\newcommand{\rulelabel}[1]{\label{rule:#1}}
\newcommand{\eglabel}[1]{\label{eg:#1}}
\newcommand{\scrlabel}[1]{\label{scr:#1}}
\newcommand{\mthlabel}[1]{\label{mth:#1}}
\newcommand{\clslabel}[1]{\label{cls:#1}}
\newcommand{\faqlabel}[1]{\label{faq:#1}}
%=============================================================
%:Menu item macro
% for menu items, so we can change our minds on how to print them! (apb)
\definecolor{lightgray}{gray}{0.89}
\newcommand{\menu}[1]{{%
	\setlength{\fboxsep}{0pt}%
	\colorbox{lightgray}{{{\upshape\sffamily\strut \,#1\,}}}}}
\newcommand{\link}[1]{{%
	\fontfamily{lmr}\selectfont
 	\upshape{\sffamily \underline{#1}}}}
% For submenu items:
\newcommand{\go}{\,$\triangleright$\,}
% \newcommand{\go}{\,$\blacktriangleright$\,}
% For keyboard shortcuts:
%\newcommand{\short}[1]{\mbox{$\langle${\sc CMD}$\rangle$-#1}\xspace}
\newcommand{\short}[1]{\mbox{{\sc cmd}\hspace{0.08em}--\hspace{0.09em}#1}\xspace}
% For buttons:
\newcommand{\button}[1]{{%
	\setlength{\fboxsep}{0pt}%
	\fbox{{\upshape\sffamily\strut \,#1\,}}}}
% NB: The button macro does not work within captions -- incompatible with xcolor package :-(
\newcommand{\toolsflap}{\textit{Tools} flap\xspace}
%=============================================================
%:Mouse clicks
\newcommand{\click}{click\xspace} % RED
\newcommand{\actclick}{action-click\xspace} % YELLOW
\newcommand{\metaclick}{meta-click\xspace} % BLUE
\newcommand{\Click}{Click\xspace} % RED
\newcommand{\Actclick}{Action-click\xspace} % YELLOW
\newcommand{\Metaclick}{Meta-click\xspace} % BLUE
%=============================================================
%:ToSh macros
\newboolean{tosh}
\setboolean{tosh}{false}
\newcommand{\iftoshelse}[2]{\ifthenelse{\boolean{tosh}}{#1}{#2}}
%=============================================================
%:ToSh colors
%\newcommand{\highlightcolor}{\color{blue!65}}
%\newcommand{\boxcolor}{\color{gray!25}}
\newcommand{\highlight}[1]{\textcolor{blue!65}{#1}}
%\newcommand{\codecolor}{\color{blue!65}}
%%\setlength{\fboxrule}{2pt}
%\newcommand{\asPict}[1]{%
%	{\Large\highlight{#1}}}
%=============================================================
%:Reader cues (do this)
%
% Indicate something the reader should try out.
% \newcommand{\dothisicon}{\raisebox{-.5ex}{\includegraphics[width=1.4em]{squeak-logo}}}
\iftoshelse{
	\usepackage{marginnote}
		\renewcommand*{\marginfont}{\footnotesize}
	\newcommand{\vartriangleout}{\ifthenelse{\isodd{\thepage}}{\vartriangleright}{\vartriangleleft}}
	\newcommand{\dothisicon}{\fcolorbox{blue!65}{white}{\highlight{$\vartriangleout$}}}
	\newcommand{\dothis}[1]{%
		\noindent\par\noindent
		{\reversemarginpar
			\marginnote{\fcolorbox{blue!65}{white}{\highlight{$\vartriangleout$}}}}
		%\MarginLabel{do this}
		\noindent\emph{#1}
		\nopagebreak}
}{
	\newcommand{\dothisicon}{\raisebox{-.5ex}{\includegraphics[height=1.2em]{pharo}}}
	\newcommand{\dothis}[1]{%
		\medskip
		\noindent\dothisicon
		\ifx#1\empty\else\quad\emph{#1}\fi
		\par\smallskip\nopagebreak}
}
%===> NEW VERSION <===
% NB: To use this in an individual chapter, you must set:
%\graphicspath{{figures/} {../figures/}}
% at the head of the chapter.  Don't forget the final /
%=============================================================
%:Reader hints (hint)
%
% Indicates a non-obvious consequence 
\newcommand{\hint}[1]{\vspace{1ex}\noindent\fbox{\textsc{Hint}} \emph{#1}}
%=================================================================
% graphics for Morphic handles
\newcommand{\grabHandle}{\raisebox{-0.2ex}{\includegraphics[width=1em]{blackHandle}}}
\newcommand{\moveHandle}{\raisebox{-0.2ex}{\includegraphics[width=1em]{moveHandle}}}
\newcommand{\debugHandle}{\raisebox{-0.2ex}{\includegraphics[width=1em]{debugHandle}}}
%=============================================================
%:Highlighting Important stuff (doublebox)
%
% From Seaside book ...
\newsavebox{\SavedText}
\newlength{\InnerBoxRule}\setlength{\InnerBoxRule}{.75\fboxrule}
\newlength{\OuterBoxRule}\setlength{\OuterBoxRule}{1.5\fboxrule}
\newlength{\BoxSeparation}\setlength{\BoxSeparation}{1.5\fboxrule}
\addtolength{\BoxSeparation}{.5pt}
\newlength{\SaveBoxSep}\setlength{\SaveBoxSep}{2\fboxsep}
%
\newenvironment{doublebox}{\begin{lrbox}{\SavedText}
    \begin{minipage}{.75\textwidth}}
    {\end{minipage}\end{lrbox}\begin{center}
    \setlength{\fboxsep}{\BoxSeparation}\setlength{\fboxrule}{\OuterBoxRule}
    \fbox{\setlength{\fboxsep}{\SaveBoxSep}\setlength{\fboxrule}{\InnerBoxRule}%
      \fbox{\usebox{\SavedText}}}
  \end{center}}
% Use this:
\newcommand{\important}[1]{\begin{doublebox}#1\end{doublebox}}
%=============================================================
%:Section depth
\setcounter{secnumdepth}{2}
%% for this to happen start the file with
%\ifx\wholebook\relax\else
%\input{../common.tex}
%\begin{document}
%\fi
% and terminate by
% \ifx\wholebook\relax\else\end{document}\fi

\DeclareGraphicsExtensions{.pdf, .jpg, .png}
%=============================================================
%:PDF setup
\hypersetup{
%   a4paper,
%   pdfstartview=FitV,
%   colorlinks,
%   linkcolor=darkblue,
%   citecolor=darkblue,
   pdftitle={Pharo by Example},
   pdfauthor={Andrew P. Black, St\'ephane Ducasse,	Oscar Nierstrasz,
Damien Pollet},
   pdfkeywords={Smalltalk, Squeak, Object-Oriented Programming, OOP},
   pdfsubject={Computer Science}
}
%=============================================================
%:Page layout and appearance
%
% \renewcommand{\headrulewidth}{0pt}
\renewcommand{\chaptermark}[1]{\markboth{#1}{}}
\renewcommand{\sectionmark}[1]{\markright{\thesection\ #1}}
\renewpagestyle{plain}[\small\itshape]{%
	\setheadrule{0pt}%
	\sethead[][][]{}{}{}%
	\setfoot[][][]{}{}{}}
\renewpagestyle{headings}[\small\itshape]{%
	\setheadrule{0pt}%
	\setmarks{chapter}{section}%
	\sethead[\thepage][][\chaptertitle]{\sectiontitle}{}{\thepage}%
	\setfoot[][][]{}{}{}}
%=============================================================
%:Title section setup and TOC numbering depth
\setcounter{secnumdepth}{1}
\setcounter{tocdepth}{1}
\titleformat{\part}[display]{\centering}{\huge\partname\ \thepart}{1em}{\Huge\textbf}[]
\titleformat{\chapter}[display]{}{\huge\chaptertitlename\ \thechapter}{1em}{\Huge\raggedright\textbf}[]
\titlecontents{part}[3pc]{%
		\pagebreak[2]\addvspace{1em plus.4em minus.2em}%
		\leavevmode\large\bfseries}
	{\contentslabel{3pc}}{\hspace*{-3pc}}
	{}[\nopagebreak]
\titlecontents{chapter}[3pc]{%
		\pagebreak[0]\addvspace{1em plus.2em minus.2em}%
		\leavevmode\bfseries}
	{\contentslabel{3pc}}{}
	{\hfill\contentspage}[\nopagebreak]
\dottedcontents{section}[3pc]{}{3pc}{1pc}
\dottedcontents{subsection}[3pc]{}{0pc}{1pc}
% \dottedcontents{subsection}[4.5em]{}{0pt}{1pc}
% Make \cleardoublepage insert really blank pages http://www.tex.ac.uk/cgi-bin/texfaq2html?label=reallyblank
\let\origdoublepage\cleardoublepage
\newcommand{\clearemptydoublepage}{%
  \clearpage
  {\pagestyle{empty}\origdoublepage}}
\let\cleardoublepage\clearemptydoublepage % see http://www.tex.ac.uk/cgi-bin/texfaq2html?label=patch
%=============================================================
%:FAQ macros (for FAQ chapter)
\newtheorem{faq}{FAQ}
\newcommand{\answer}{\paragraph{Answer}\ }
%=============================================================
%:Listings package configuration
% \newcommand{\caret}{\makebox{\raisebox{0.4ex}{\footnotesize{$\wedge$}}}}
\newcommand{\caret}{\^\,}
\newcommand{\escape}{{\sf \textbackslash}}
\definecolor{source}{gray}{0.95}
\usepackage{listings}
\lstdefinelanguage{Smalltalk}{
%  morekeywords={self,super,true,false,nil,thisContext}, % This is overkill
  morestring=[d]',
  morecomment=[s]{"}{"},
  alsoletter={\#:},
  escapechar={!},
  literate=
    {BANG}{!}1
    {CARET}{\^}1
    {UNDERSCORE}{\_}1
    {\\st}{Smalltalk}9 % convenience -- in case \st occurs in code
    % {'}{{\textquotesingle}}1 % replaced by upquote=true in \lstset
    {_}{{$\leftarrow$}}1
    {>>>}{{\sep}}1
    {^}{{$\uparrow$}}1
    {~}{{$\sim$}}1
    {-}{{\texttt{-}}}1 %{\textminus}}1 %{-}{\hspace{-0.13em}}{-}}1  % the goal is to make - the same width as +
    % {+}{\sf+}1 %{\raisebox{0.08ex}{+}}}1      % and to raise + off the baseline to match -
    {-->}{{\quad$\longrightarrow$\quad}}3
	, % Don't forget the comma at the end!
  tabsize=4
}[keywords,comments,strings]

\lstset{language=Smalltalk,
	basicstyle=\sffamily,
	keywordstyle=\color{black}\bfseries,
	% stringstyle=\ttfamily, % Ugly! do we really want this? -- on
	mathescape=true,
	showstringspaces=false,
	keepspaces=true,
	breaklines=true,
	breakautoindent=true,
	backgroundcolor=\color{source},
	lineskip={-1pt}, % Ugly hack
	upquote=true, % straight quote; requires textcomp package
	columns=fullflexible} % no fixed width fonts
% In-line code (literal)
% Normally use this for all in-line code:
\newcommand{\ct}{\lstinline[mathescape=false,backgroundcolor=\color{white},basicstyle={\sffamily\upshape}]}
% apb 2007.8.28 added the \upshape declaration to avoid getting italicized code in \dothis{ } sections.
% In-line code (latex enabled)
% Use this only in special situations where \ct does not work
% (within section headings ...):
\newcommand{\lct}[1]{{\textsf{\textup{#1}}}}
% Use these for system categories and protocols:
\newcommand{\scat}[1]{\emph{\textsf{#1}}\xspace}
\newcommand{\pkg}[1]{\emph{\textsf{#1}}\xspace}
\newcommand{\prot}[1]{\emph{\textsf{#1}}\xspace}
% Code environments
% NB: the arg is for tests
% Only code and example environments may be tests
\lstnewenvironment{code}[1]{%
	\lstset{%
		% frame=lines,
		frame=single,
		framerule=0pt,
		mathescape=false
	}
}{}
\def\ignoredollar#1{}
%=============================================================
%:Code environments (method, script ...)
% NB: the third arg is for tests
% Only code and example environments may be tests
\lstnewenvironment{example}[3][defaultlabel]{%
	\renewcommand{\lstlistingname}{Example}%
	\lstset{
		% frame=lines,
		frame=single,
		framerule=0pt,
		mathescape=false,
		caption={\emph{#2}},
		label={eg:#1}
	}
}{}
\lstnewenvironment{script}[2][defaultlabel]{%
\renewcommand{\lstlistingname}{Script}%
	\lstset{
		% frame=lines,
		frame=single,
		framerule=0pt,
		mathescape=false,
		name={Script},
		caption={\emph{#2}},
		label={scr:#1}
	}
}{}
\lstnewenvironment{method}[2][defaultlabel]{%
	\renewcommand{\lstlistingname}{Method}%
	\lstset{
		% frame=lines,
		frame=single,
		framerule=0pt,
		mathescape=false,
		name={Method},
		caption={\emph{#2}},
		label={mth:#1}
	}
}{}
\lstnewenvironment{methods}[2][defaultlabel]{% just for multiple methods at once
	\renewcommand{\lstlistingname}{Methods}%
	\lstset{
		% frame=lines,
		frame=single,
		framerule=0pt,
		mathescape=false,
		name={Method},
		caption={\emph{#2}},
		label={mth:#1}
	}
}{}
\lstnewenvironment{numMethod}[2][defaultlabel]{%
	\renewcommand{\lstlistingname}{Method}%
	\lstset{
		numbers=left,
		numberstyle={\tiny\sffamily},
		% frame=lines,
		frame=single,
		framerule=0pt,
		mathescape=false,
		name={Method},
		caption={\emph{#2}},
		label={mth:#1}
	}
}{}
\lstnewenvironment{classdef}[2][defaultlabel]{%
	\renewcommand{\lstlistingname}{Class}%
	\lstset{
		% frame=lines,
		frame=single,
		framerule=0pt,
		mathescape=false,
		name={Class},
		caption={\emph{#2}},
		label={cls:#1}
	}
}{}
%=============================================================
%:Reserving space
% Usually need one more line than the actual lines of code
\newcommand{\needlines}[1]{\Needspace{#1\baselineskip}}
%=============================================================
%:Indexing macros
% Macros ending with "ind" generate text as well as an index entry
% Macros ending with "index" *only* generate an index entry
\newcommand{\ind}[1]{\index{#1}#1\xspace} % plain text
\newcommand{\subind}[2]{\index{#1!#2}#2\xspace} % show #2, subindex under #1
\newcommand{\emphind}[1]{\index{#1}\emph{#1}\xspace} % emph #1
\newcommand{\emphsubind}[2]{\index{#1!#2}\emph{#2}\xspace} % show emph #2, subindex inder #1
\newcommand{\scatind}[1]{\index{#1@\textsf{#1} (category)}\scat{#1}} % category
\newcommand{\pkgind}[1]{\index{#1@\textsf{#1} (package)}\pkg{#1}} % package
\newcommand{\protind}[1]{\index{#1@\textsf{#1} (protocol)}\prot{#1}} % protocol
\newcommand{\clsind}[1]{\index{#1@\textsf{#1} (class)}\ct{#1}\xspace}
% \newcommand{\clsind}[1]{\index{#1!\#@(class)}\ct{#1}\xspace} % class
\newcommand{\clsindplural}[1]{\index{#1!\#@(class)}\ct{#1}s\xspace} % class
\newcommand{\cvind}[1]{\index{#1@\textsf{#1} (class variable)}\ct{#1}\xspace} % class var
\newcommand{\glbind}[1]{\index{#1@\textsf{#1} (global)}\ct{#1}\xspace} % global
\newcommand{\patind}[1]{\index{#1@#1 (pattern)}\ct{#1}\xspace} % pattern
\newcommand{\pvind}[1]{\index{#1@\textsf{#1} (pseudo variable)}\ct{#1}\xspace} % pseudo var
\newcommand{\clsmthind}[2]{\index{#1!#2@\ct{#2}}\ct{#1>>>#2}\xspace} % class + method name
\newcommand{\mthind}[2]{\index{#1!#2@\ct{#2}}\ct{#2}\xspace} % show method name only
\newcommand{\lmthind}[2]{\index{#1!#2@\ct{#2}}\lct{#2}\xspace} % show method name only
\newcommand{\cmind}[2]{\index{#1!#2@\ct{#2}}\ct{#1>>>#2}\xspace} % show class>>method
\newcommand{\lcmind}[2]{\index{#1!#2@\ct{#2}}\lct{#1>>>#2}\xspace} % show class>>method
\newcommand{\toolsflapind}{\index{Tools flap}\toolsflap} % index tools flap
% The following only generate an index entry:
% \newcommand{\clsindex}[1]{\index{#1@\textsf{#1} (class)}}
\newcommand{\clsindex}[1]{\index{#1!\#@(class)}} % class
\newcommand{\mthindex}[2]{\index{#1!#2@\ct{#2}}} % method
\newcommand{\cmindex}[2]{\index{#1!#2@\ct{#2}}} % class>>method
\newcommand{\cvindex}[1]{\index{#1@\textsf{#1} (class variable)}} % class var
\newcommand{\glbindex}[1]{\index{#1@\textsf{#1} (global)}}% global
\newcommand{\pvindex}[1]{\index{#1@\textsf{#1} (pseudo variable)}}% pseudo var
\newcommand{\seeindex}[2]{\index{#1|see{#2}}} % #1, see #2
\newcommand{\scatindex}[1]{\index{#1@\textsf{#1} (category)}} % category
\newcommand{\pkgindex}[1]{\index{#1@\textsf{#1} (package)}} % package
\newcommand{\protindex}[1]{\index{#1@\textsf{#1} (protocol)}} % protocol
% How can we have the main entry page numbers in bold yet not break the hyperlink?
\newcommand{\boldidx}[1]{{\bf #1}} % breaks hyperlink
%\newcommand{\indmain}[1]{\index{#1|boldidx}#1\xspace} % plain text, main entry
%\newcommand{\emphsubindmain}[2]{\index{#1!#2|boldidx}\emph{#2}\xspace} % subindex, main entry
%\newcommand{\subindmain}[2]{\index{#1!#2|boldidx}#2\xspace} % subindex, main entry
%\newcommand{\clsindmain}[1]{\index{#1@\textsf{#1} (class)|boldidx}\ct{#1}\xspace}
%\newcommand{\clsindmain}[1]{\index{#1!\#@(class)|boldidx}\ct{#1}\xspace} % class main
%\newcommand{\indexmain}[1]{\index{#1|boldidx}} % main index entry only
\newcommand{\indmain}[1]{\index{#1}#1\xspace} % The main index entry for this item
\newcommand{\emphsubindmain}[2]{\index{#1!#2}\emph{#2}\xspace} % subindex, main entry
\newcommand{\subindmain}[2]{\index{#1!#2}#2\xspace} % subindex, main entry
%\newcommand{\clsindmain}[1]{\index{#1@\textsf{#1} (class)}\ct{#1}\xspace}
\newcommand{\clsindmain}[1]{\index{#1!\#@(class)}\ct{#1}\xspace} % class main
\newcommand{\clsindexmain}[1]{\index{#1!\#@(class)}} % class main index only
\newcommand{\indexmain}[1]{\index{#1}} 
%=============================================================
%:Code macros
% some constants
\newcommand{\codesize}{\small}
\newcommand{\codefont}{\sffamily}
%\newcommand{\cat}[1]{\textit{In category #1}}%%To remove later
\newlength{\scriptindent}
\setlength{\scriptindent}{.3cm}
%% Method presentation constants
\newlength{\methodindent}
\newlength{\methodwordlength}
\newlength{\aftermethod}
\setlength{\methodindent}{0.2cm}
\settowidth{\methodwordlength}{\ M\'ethode\ }
%=============================================================
%:Smalltalk macros
%\newcommand{\sep}{{$\gg$}}
\newcommand{\sep}{\mbox{>>}}
\newcommand{\self}{\lct{self}\xspace}
\newcommand{\super}{\lct{super}\xspace}
\newcommand{\nil}{\lct{nil}\xspace}
%=============================================================
% be less conservative about float placement
% these commands are from http://www.tex.ac.uk/cgi-bin/texfaq2html?label=floats
\renewcommand{\topfraction}{.9}
\renewcommand{\bottomfraction}{.9}
\renewcommand{\textfraction}{.1}
\renewcommand{\floatpagefraction}{.85}
\renewcommand{\dbltopfraction}{.66}
\renewcommand{\dblfloatpagefraction}{.85}
\setcounter{topnumber}{9}
\setcounter{bottomnumber}{9}
\setcounter{totalnumber}{20}
\setcounter{dbltopnumber}{9}
%=============================================================
% Give information from each chapter's author
\newcommand{\contact}[2]{\textbf{#1} \textsf{(#2)}}

\newcommand{\chapterauthor}[1]{\emph{with the participation of:\\#1}\\}

\newcommand{\authornoury}{\contact{Noury Bouraqadi}{Noury.Bouraqadi@mines-douai.fr}}
\newcommand{\authordamienc}{\contact{Damien Cassou}{damien.cassou@gmail.com}}
\newcommand{\authoroscar}{\contact{Oscar Nierstrasz}{oscar.nierstrasz@acm.org}}
\newcommand{\authorsteph}{\contact{St\'ephane Ducasse}{stephane.ducasse@inria.fr}}
\newcommand{\authoralex}{\contact{Alexandre Bergel}{alexandre@bergel.eu}}
\newcommand{\authorolivier}{\contact{Olivier Auverlot}{olivier.auverlot@inria.fr}}
\newcommand{\authornicolas}{\contact{Nicolas Cellier}{nicolas.cellier.aka.nice@gmail.com}}
\newcommand{\authormarcus}{\contact{Marcus Denker}{marcus.denker@inria.fr}}
\newcommand{\authoralain}{\contact{Alain Plantec}{alain.plantec@univ-brest.fr}}
\newcommand{\authordale}{\contact{Dale Henrichs}{dale.henrichs@gemstone.com}}
\newcommand{\authormariano}{\contact{Mariano Martinez Peck}{marianopeck@gmail.com}}
\newcommand{\authorsven}{\contact{Sven Van Caekenberghe}{sven@beta9.be}}
\newcommand{\authorlukas}{\contact{Lukas Renggli}{renggli@gmail.com}}

%=============================================================
% apb doesn't like paragraphs to run in to each other without a break
\parskip 1ex
%=============================================================
%:Stuff to check, merge or deprecate
%\setlength{\marginparsep}{2mm}
%\renewcommand{\baselinestretch}{1.1}
%=============================================================

%\begin{document}
%\fi
% and terminate by
% \ifx\wholebook\relax\else\end{document}\fi

\DeclareGraphicsExtensions{.pdf, .jpg, .png}
%=============================================================
%:PDF setup
\hypersetup{
%   a4paper,
%   pdfstartview=FitV,
%   colorlinks,
%   linkcolor=darkblue,
%   citecolor=darkblue,
   pdftitle={Pharo by Example},
   pdfauthor={Andrew P. Black, St\'ephane Ducasse,	Oscar Nierstrasz,
Damien Pollet},
   pdfkeywords={Smalltalk, Squeak, Object-Oriented Programming, OOP},
   pdfsubject={Computer Science}
}
%=============================================================
%:Page layout and appearance
%
% \renewcommand{\headrulewidth}{0pt}
\renewcommand{\chaptermark}[1]{\markboth{#1}{}}
\renewcommand{\sectionmark}[1]{\markright{\thesection\ #1}}
\renewpagestyle{plain}[\small\itshape]{%
	\setheadrule{0pt}%
	\sethead[][][]{}{}{}%
	\setfoot[][][]{}{}{}}
\renewpagestyle{headings}[\small\itshape]{%
	\setheadrule{0pt}%
	\setmarks{chapter}{section}%
	\sethead[\thepage][][\chaptertitle]{\sectiontitle}{}{\thepage}%
	\setfoot[][][]{}{}{}}
%=============================================================
%:Title section setup and TOC numbering depth
\setcounter{secnumdepth}{1}
\setcounter{tocdepth}{1}
\titleformat{\part}[display]{\centering}{\huge\partname\ \thepart}{1em}{\Huge\textbf}[]
\titleformat{\chapter}[display]{}{\huge\chaptertitlename\ \thechapter}{1em}{\Huge\raggedright\textbf}[]
\titlecontents{part}[3pc]{%
		\pagebreak[2]\addvspace{1em plus.4em minus.2em}%
		\leavevmode\large\bfseries}
	{\contentslabel{3pc}}{\hspace*{-3pc}}
	{}[\nopagebreak]
\titlecontents{chapter}[3pc]{%
		\pagebreak[0]\addvspace{1em plus.2em minus.2em}%
		\leavevmode\bfseries}
	{\contentslabel{3pc}}{}
	{\hfill\contentspage}[\nopagebreak]
\dottedcontents{section}[3pc]{}{3pc}{1pc}
\dottedcontents{subsection}[3pc]{}{0pc}{1pc}
% \dottedcontents{subsection}[4.5em]{}{0pt}{1pc}
% Make \cleardoublepage insert really blank pages http://www.tex.ac.uk/cgi-bin/texfaq2html?label=reallyblank
\let\origdoublepage\cleardoublepage
\newcommand{\clearemptydoublepage}{%
  \clearpage
  {\pagestyle{empty}\origdoublepage}}
\let\cleardoublepage\clearemptydoublepage % see http://www.tex.ac.uk/cgi-bin/texfaq2html?label=patch
%=============================================================
%:FAQ macros (for FAQ chapter)
\newtheorem{faq}{FAQ}
\newcommand{\answer}{\paragraph{Answer}\ }
%=============================================================
%:Listings package configuration
% \newcommand{\caret}{\makebox{\raisebox{0.4ex}{\footnotesize{$\wedge$}}}}
\newcommand{\caret}{\^\,}
\newcommand{\escape}{{\sf \textbackslash}}
\definecolor{source}{gray}{0.95}
\usepackage{listings}
\lstdefinelanguage{Smalltalk}{
%  morekeywords={self,super,true,false,nil,thisContext}, % This is overkill
  morestring=[d]',
  morecomment=[s]{"}{"},
  alsoletter={\#:},
  escapechar={!},
  literate=
    {BANG}{!}1
    {CARET}{\^}1
    {UNDERSCORE}{\_}1
    {\\st}{Smalltalk}9 % convenience -- in case \st occurs in code
    % {'}{{\textquotesingle}}1 % replaced by upquote=true in \lstset
    {_}{{$\leftarrow$}}1
    {>>>}{{\sep}}1
    {^}{{$\uparrow$}}1
    {~}{{$\sim$}}1
    {-}{{\texttt{-}}}1 %{\textminus}}1 %{-}{\hspace{-0.13em}}{-}}1  % the goal is to make - the same width as +
    % {+}{\sf+}1 %{\raisebox{0.08ex}{+}}}1      % and to raise + off the baseline to match -
    {-->}{{\quad$\longrightarrow$\quad}}3
	, % Don't forget the comma at the end!
  tabsize=4
}[keywords,comments,strings]

\lstset{language=Smalltalk,
	basicstyle=\sffamily,
	keywordstyle=\color{black}\bfseries,
	% stringstyle=\ttfamily, % Ugly! do we really want this? -- on
	mathescape=true,
	showstringspaces=false,
	keepspaces=true,
	breaklines=true,
	breakautoindent=true,
	backgroundcolor=\color{source},
	lineskip={-1pt}, % Ugly hack
	upquote=true, % straight quote; requires textcomp package
	columns=fullflexible} % no fixed width fonts
% In-line code (literal)
% Normally use this for all in-line code:
\newcommand{\ct}{\lstinline[mathescape=false,backgroundcolor=\color{white},basicstyle={\sffamily\upshape}]}
% apb 2007.8.28 added the \upshape declaration to avoid getting italicized code in \dothis{ } sections.
% In-line code (latex enabled)
% Use this only in special situations where \ct does not work
% (within section headings ...):
\newcommand{\lct}[1]{{\textsf{\textup{#1}}}}
% Use these for system categories and protocols:
\newcommand{\scat}[1]{\emph{\textsf{#1}}\xspace}
\newcommand{\pkg}[1]{\emph{\textsf{#1}}\xspace}
\newcommand{\prot}[1]{\emph{\textsf{#1}}\xspace}
% Code environments
% NB: the arg is for tests
% Only code and example environments may be tests
\lstnewenvironment{code}[1]{%
	\lstset{%
		% frame=lines,
		frame=single,
		framerule=0pt,
		mathescape=false
	}
}{}
\def\ignoredollar#1{}
%=============================================================
%:Code environments (method, script ...)
% NB: the third arg is for tests
% Only code and example environments may be tests
\lstnewenvironment{example}[3][defaultlabel]{%
	\renewcommand{\lstlistingname}{Example}%
	\lstset{
		% frame=lines,
		frame=single,
		framerule=0pt,
		mathescape=false,
		caption={\emph{#2}},
		label={eg:#1}
	}
}{}
\lstnewenvironment{script}[2][defaultlabel]{%
\renewcommand{\lstlistingname}{Script}%
	\lstset{
		% frame=lines,
		frame=single,
		framerule=0pt,
		mathescape=false,
		name={Script},
		caption={\emph{#2}},
		label={scr:#1}
	}
}{}
\lstnewenvironment{method}[2][defaultlabel]{%
	\renewcommand{\lstlistingname}{Method}%
	\lstset{
		% frame=lines,
		frame=single,
		framerule=0pt,
		mathescape=false,
		name={Method},
		caption={\emph{#2}},
		label={mth:#1}
	}
}{}
\lstnewenvironment{methods}[2][defaultlabel]{% just for multiple methods at once
	\renewcommand{\lstlistingname}{Methods}%
	\lstset{
		% frame=lines,
		frame=single,
		framerule=0pt,
		mathescape=false,
		name={Method},
		caption={\emph{#2}},
		label={mth:#1}
	}
}{}
\lstnewenvironment{numMethod}[2][defaultlabel]{%
	\renewcommand{\lstlistingname}{Method}%
	\lstset{
		numbers=left,
		numberstyle={\tiny\sffamily},
		% frame=lines,
		frame=single,
		framerule=0pt,
		mathescape=false,
		name={Method},
		caption={\emph{#2}},
		label={mth:#1}
	}
}{}
\lstnewenvironment{classdef}[2][defaultlabel]{%
	\renewcommand{\lstlistingname}{Class}%
	\lstset{
		% frame=lines,
		frame=single,
		framerule=0pt,
		mathescape=false,
		name={Class},
		caption={\emph{#2}},
		label={cls:#1}
	}
}{}
%=============================================================
%:Reserving space
% Usually need one more line than the actual lines of code
\newcommand{\needlines}[1]{\Needspace{#1\baselineskip}}
%=============================================================
%:Indexing macros
% Macros ending with "ind" generate text as well as an index entry
% Macros ending with "index" *only* generate an index entry
\newcommand{\ind}[1]{\index{#1}#1\xspace} % plain text
\newcommand{\subind}[2]{\index{#1!#2}#2\xspace} % show #2, subindex under #1
\newcommand{\emphind}[1]{\index{#1}\emph{#1}\xspace} % emph #1
\newcommand{\emphsubind}[2]{\index{#1!#2}\emph{#2}\xspace} % show emph #2, subindex inder #1
\newcommand{\scatind}[1]{\index{#1@\textsf{#1} (category)}\scat{#1}} % category
\newcommand{\pkgind}[1]{\index{#1@\textsf{#1} (package)}\pkg{#1}} % package
\newcommand{\protind}[1]{\index{#1@\textsf{#1} (protocol)}\prot{#1}} % protocol
\newcommand{\clsind}[1]{\index{#1@\textsf{#1} (class)}\ct{#1}\xspace}
% \newcommand{\clsind}[1]{\index{#1!\#@(class)}\ct{#1}\xspace} % class
\newcommand{\clsindplural}[1]{\index{#1!\#@(class)}\ct{#1}s\xspace} % class
\newcommand{\cvind}[1]{\index{#1@\textsf{#1} (class variable)}\ct{#1}\xspace} % class var
\newcommand{\glbind}[1]{\index{#1@\textsf{#1} (global)}\ct{#1}\xspace} % global
\newcommand{\patind}[1]{\index{#1@#1 (pattern)}\ct{#1}\xspace} % pattern
\newcommand{\pvind}[1]{\index{#1@\textsf{#1} (pseudo variable)}\ct{#1}\xspace} % pseudo var
\newcommand{\clsmthind}[2]{\index{#1!#2@\ct{#2}}\ct{#1>>>#2}\xspace} % class + method name
\newcommand{\mthind}[2]{\index{#1!#2@\ct{#2}}\ct{#2}\xspace} % show method name only
\newcommand{\lmthind}[2]{\index{#1!#2@\ct{#2}}\lct{#2}\xspace} % show method name only
\newcommand{\cmind}[2]{\index{#1!#2@\ct{#2}}\ct{#1>>>#2}\xspace} % show class>>method
\newcommand{\lcmind}[2]{\index{#1!#2@\ct{#2}}\lct{#1>>>#2}\xspace} % show class>>method
\newcommand{\toolsflapind}{\index{Tools flap}\toolsflap} % index tools flap
% The following only generate an index entry:
% \newcommand{\clsindex}[1]{\index{#1@\textsf{#1} (class)}}
\newcommand{\clsindex}[1]{\index{#1!\#@(class)}} % class
\newcommand{\mthindex}[2]{\index{#1!#2@\ct{#2}}} % method
\newcommand{\cmindex}[2]{\index{#1!#2@\ct{#2}}} % class>>method
\newcommand{\cvindex}[1]{\index{#1@\textsf{#1} (class variable)}} % class var
\newcommand{\glbindex}[1]{\index{#1@\textsf{#1} (global)}}% global
\newcommand{\pvindex}[1]{\index{#1@\textsf{#1} (pseudo variable)}}% pseudo var
\newcommand{\seeindex}[2]{\index{#1|see{#2}}} % #1, see #2
\newcommand{\scatindex}[1]{\index{#1@\textsf{#1} (category)}} % category
\newcommand{\pkgindex}[1]{\index{#1@\textsf{#1} (package)}} % package
\newcommand{\protindex}[1]{\index{#1@\textsf{#1} (protocol)}} % protocol
% How can we have the main entry page numbers in bold yet not break the hyperlink?
\newcommand{\boldidx}[1]{{\bf #1}} % breaks hyperlink
%\newcommand{\indmain}[1]{\index{#1|boldidx}#1\xspace} % plain text, main entry
%\newcommand{\emphsubindmain}[2]{\index{#1!#2|boldidx}\emph{#2}\xspace} % subindex, main entry
%\newcommand{\subindmain}[2]{\index{#1!#2|boldidx}#2\xspace} % subindex, main entry
%\newcommand{\clsindmain}[1]{\index{#1@\textsf{#1} (class)|boldidx}\ct{#1}\xspace}
%\newcommand{\clsindmain}[1]{\index{#1!\#@(class)|boldidx}\ct{#1}\xspace} % class main
%\newcommand{\indexmain}[1]{\index{#1|boldidx}} % main index entry only
\newcommand{\indmain}[1]{\index{#1}#1\xspace} % The main index entry for this item
\newcommand{\emphsubindmain}[2]{\index{#1!#2}\emph{#2}\xspace} % subindex, main entry
\newcommand{\subindmain}[2]{\index{#1!#2}#2\xspace} % subindex, main entry
%\newcommand{\clsindmain}[1]{\index{#1@\textsf{#1} (class)}\ct{#1}\xspace}
\newcommand{\clsindmain}[1]{\index{#1!\#@(class)}\ct{#1}\xspace} % class main
\newcommand{\clsindexmain}[1]{\index{#1!\#@(class)}} % class main index only
\newcommand{\indexmain}[1]{\index{#1}} 
%=============================================================
%:Code macros
% some constants
\newcommand{\codesize}{\small}
\newcommand{\codefont}{\sffamily}
%\newcommand{\cat}[1]{\textit{In category #1}}%%To remove later
\newlength{\scriptindent}
\setlength{\scriptindent}{.3cm}
%% Method presentation constants
\newlength{\methodindent}
\newlength{\methodwordlength}
\newlength{\aftermethod}
\setlength{\methodindent}{0.2cm}
\settowidth{\methodwordlength}{\ M\'ethode\ }
%=============================================================
%:Smalltalk macros
%\newcommand{\sep}{{$\gg$}}
\newcommand{\sep}{\mbox{>>}}
\newcommand{\self}{\lct{self}\xspace}
\newcommand{\super}{\lct{super}\xspace}
\newcommand{\nil}{\lct{nil}\xspace}
%=============================================================
% be less conservative about float placement
% these commands are from http://www.tex.ac.uk/cgi-bin/texfaq2html?label=floats
\renewcommand{\topfraction}{.9}
\renewcommand{\bottomfraction}{.9}
\renewcommand{\textfraction}{.1}
\renewcommand{\floatpagefraction}{.85}
\renewcommand{\dbltopfraction}{.66}
\renewcommand{\dblfloatpagefraction}{.85}
\setcounter{topnumber}{9}
\setcounter{bottomnumber}{9}
\setcounter{totalnumber}{20}
\setcounter{dbltopnumber}{9}
%=============================================================
% Give information from each chapter's author
\newcommand{\contact}[2]{\textbf{#1} \textsf{(#2)}}

\newcommand{\chapterauthor}[1]{\emph{with the participation of:\\#1}\\}

\newcommand{\authornoury}{\contact{Noury Bouraqadi}{Noury.Bouraqadi@mines-douai.fr}}
\newcommand{\authordamienc}{\contact{Damien Cassou}{damien.cassou@gmail.com}}
\newcommand{\authoroscar}{\contact{Oscar Nierstrasz}{oscar.nierstrasz@acm.org}}
\newcommand{\authorsteph}{\contact{St\'ephane Ducasse}{stephane.ducasse@inria.fr}}
\newcommand{\authoralex}{\contact{Alexandre Bergel}{alexandre@bergel.eu}}
\newcommand{\authorolivier}{\contact{Olivier Auverlot}{olivier.auverlot@inria.fr}}
\newcommand{\authornicolas}{\contact{Nicolas Cellier}{nicolas.cellier.aka.nice@gmail.com}}
\newcommand{\authormarcus}{\contact{Marcus Denker}{marcus.denker@inria.fr}}
\newcommand{\authoralain}{\contact{Alain Plantec}{alain.plantec@univ-brest.fr}}
\newcommand{\authordale}{\contact{Dale Henrichs}{dale.henrichs@gemstone.com}}
\newcommand{\authormariano}{\contact{Mariano Martinez Peck}{marianopeck@gmail.com}}
\newcommand{\authorsven}{\contact{Sven Van Caekenberghe}{sven@beta9.be}}
\newcommand{\authorlukas}{\contact{Lukas Renggli}{renggli@gmail.com}}

%=============================================================
% apb doesn't like paragraphs to run in to each other without a break
\parskip 1ex
%=============================================================
%:Stuff to check, merge or deprecate
%\setlength{\marginparsep}{2mm}
%\renewcommand{\baselinestretch}{1.1}
%=============================================================

%\begin{document}
%\fi
% and terminate by
% \ifx\wholebook\relax\else\end{document}\fi

\DeclareGraphicsExtensions{.pdf, .jpg, .png}
%=============================================================
%:PDF setup
\hypersetup{
%   a4paper,
%   pdfstartview=FitV,
%   colorlinks,
%   linkcolor=darkblue,
%   citecolor=darkblue,
   pdftitle={Pharo by Example},
   pdfauthor={Andrew P. Black, St\'ephane Ducasse,	Oscar Nierstrasz,
Damien Pollet},
   pdfkeywords={Smalltalk, Squeak, Object-Oriented Programming, OOP},
   pdfsubject={Computer Science}
}
%=============================================================
%:Page layout and appearance
%
% \renewcommand{\headrulewidth}{0pt}
\renewcommand{\chaptermark}[1]{\markboth{#1}{}}
\renewcommand{\sectionmark}[1]{\markright{\thesection\ #1}}
\renewpagestyle{plain}[\small\itshape]{%
	\setheadrule{0pt}%
	\sethead[][][]{}{}{}%
	\setfoot[][][]{}{}{}}
\renewpagestyle{headings}[\small\itshape]{%
	\setheadrule{0pt}%
	\setmarks{chapter}{section}%
	\sethead[\thepage][][\chaptertitle]{\sectiontitle}{}{\thepage}%
	\setfoot[][][]{}{}{}}
%=============================================================
%:Title section setup and TOC numbering depth
\setcounter{secnumdepth}{1}
\setcounter{tocdepth}{1}
\titleformat{\part}[display]{\centering}{\huge\partname\ \thepart}{1em}{\Huge\textbf}[]
\titleformat{\chapter}[display]{}{\huge\chaptertitlename\ \thechapter}{1em}{\Huge\raggedright\textbf}[]
\titlecontents{part}[3pc]{%
		\pagebreak[2]\addvspace{1em plus.4em minus.2em}%
		\leavevmode\large\bfseries}
	{\contentslabel{3pc}}{\hspace*{-3pc}}
	{}[\nopagebreak]
\titlecontents{chapter}[3pc]{%
		\pagebreak[0]\addvspace{1em plus.2em minus.2em}%
		\leavevmode\bfseries}
	{\contentslabel{3pc}}{}
	{\hfill\contentspage}[\nopagebreak]
\dottedcontents{section}[3pc]{}{3pc}{1pc}
\dottedcontents{subsection}[3pc]{}{0pc}{1pc}
% \dottedcontents{subsection}[4.5em]{}{0pt}{1pc}
% Make \cleardoublepage insert really blank pages http://www.tex.ac.uk/cgi-bin/texfaq2html?label=reallyblank
\let\origdoublepage\cleardoublepage
\newcommand{\clearemptydoublepage}{%
  \clearpage
  {\pagestyle{empty}\origdoublepage}}
\let\cleardoublepage\clearemptydoublepage % see http://www.tex.ac.uk/cgi-bin/texfaq2html?label=patch
%=============================================================
%:FAQ macros (for FAQ chapter)
\newtheorem{faq}{FAQ}
\newcommand{\answer}{\paragraph{Answer}\ }
%=============================================================
%:Listings package configuration
% \newcommand{\caret}{\makebox{\raisebox{0.4ex}{\footnotesize{$\wedge$}}}}
\newcommand{\caret}{\^\,}
\newcommand{\escape}{{\sf \textbackslash}}
\definecolor{source}{gray}{0.95}
\usepackage{listings}
\lstdefinelanguage{Smalltalk}{
%  morekeywords={self,super,true,false,nil,thisContext}, % This is overkill
  morestring=[d]',
  morecomment=[s]{"}{"},
  alsoletter={\#:},
  escapechar={!},
  literate=
    {BANG}{!}1
    {CARET}{\^}1
    {UNDERSCORE}{\_}1
    {\\st}{Smalltalk}9 % convenience -- in case \st occurs in code
    % {'}{{\textquotesingle}}1 % replaced by upquote=true in \lstset
    {_}{{$\leftarrow$}}1
    {>>>}{{\sep}}1
    {^}{{$\uparrow$}}1
    {~}{{$\sim$}}1
    {-}{{\texttt{-}}}1 %{\textminus}}1 %{-}{\hspace{-0.13em}}{-}}1  % the goal is to make - the same width as +
    % {+}{\sf+}1 %{\raisebox{0.08ex}{+}}}1      % and to raise + off the baseline to match -
    {-->}{{\quad$\longrightarrow$\quad}}3
	, % Don't forget the comma at the end!
  tabsize=4
}[keywords,comments,strings]

\lstset{language=Smalltalk,
	basicstyle=\sffamily,
	keywordstyle=\color{black}\bfseries,
	% stringstyle=\ttfamily, % Ugly! do we really want this? -- on
	mathescape=true,
	showstringspaces=false,
	keepspaces=true,
	breaklines=true,
	breakautoindent=true,
	backgroundcolor=\color{source},
	lineskip={-1pt}, % Ugly hack
	upquote=true, % straight quote; requires textcomp package
	columns=fullflexible} % no fixed width fonts
% In-line code (literal)
% Normally use this for all in-line code:
\newcommand{\ct}{\lstinline[mathescape=false,backgroundcolor=\color{white},basicstyle={\sffamily\upshape}]}
% apb 2007.8.28 added the \upshape declaration to avoid getting italicized code in \dothis{ } sections.
% In-line code (latex enabled)
% Use this only in special situations where \ct does not work
% (within section headings ...):
\newcommand{\lct}[1]{{\textsf{\textup{#1}}}}
% Use these for system categories and protocols:
\newcommand{\scat}[1]{\emph{\textsf{#1}}\xspace}
\newcommand{\pkg}[1]{\emph{\textsf{#1}}\xspace}
\newcommand{\prot}[1]{\emph{\textsf{#1}}\xspace}
% Code environments
% NB: the arg is for tests
% Only code and example environments may be tests
\lstnewenvironment{code}[1]{%
	\lstset{%
		% frame=lines,
		frame=single,
		framerule=0pt,
		mathescape=false
	}
}{}
\def\ignoredollar#1{}
%=============================================================
%:Code environments (method, script ...)
% NB: the third arg is for tests
% Only code and example environments may be tests
\lstnewenvironment{example}[3][defaultlabel]{%
	\renewcommand{\lstlistingname}{Example}%
	\lstset{
		% frame=lines,
		frame=single,
		framerule=0pt,
		mathescape=false,
		caption={\emph{#2}},
		label={eg:#1}
	}
}{}
\lstnewenvironment{script}[2][defaultlabel]{%
\renewcommand{\lstlistingname}{Script}%
	\lstset{
		% frame=lines,
		frame=single,
		framerule=0pt,
		mathescape=false,
		name={Script},
		caption={\emph{#2}},
		label={scr:#1}
	}
}{}
\lstnewenvironment{method}[2][defaultlabel]{%
	\renewcommand{\lstlistingname}{Method}%
	\lstset{
		% frame=lines,
		frame=single,
		framerule=0pt,
		mathescape=false,
		name={Method},
		caption={\emph{#2}},
		label={mth:#1}
	}
}{}
\lstnewenvironment{methods}[2][defaultlabel]{% just for multiple methods at once
	\renewcommand{\lstlistingname}{Methods}%
	\lstset{
		% frame=lines,
		frame=single,
		framerule=0pt,
		mathescape=false,
		name={Method},
		caption={\emph{#2}},
		label={mth:#1}
	}
}{}
\lstnewenvironment{numMethod}[2][defaultlabel]{%
	\renewcommand{\lstlistingname}{Method}%
	\lstset{
		numbers=left,
		numberstyle={\tiny\sffamily},
		% frame=lines,
		frame=single,
		framerule=0pt,
		mathescape=false,
		name={Method},
		caption={\emph{#2}},
		label={mth:#1}
	}
}{}
\lstnewenvironment{classdef}[2][defaultlabel]{%
	\renewcommand{\lstlistingname}{Class}%
	\lstset{
		% frame=lines,
		frame=single,
		framerule=0pt,
		mathescape=false,
		name={Class},
		caption={\emph{#2}},
		label={cls:#1}
	}
}{}
%=============================================================
%:Reserving space
% Usually need one more line than the actual lines of code
\newcommand{\needlines}[1]{\Needspace{#1\baselineskip}}
%=============================================================
%:Indexing macros
% Macros ending with "ind" generate text as well as an index entry
% Macros ending with "index" *only* generate an index entry
\newcommand{\ind}[1]{\index{#1}#1\xspace} % plain text
\newcommand{\subind}[2]{\index{#1!#2}#2\xspace} % show #2, subindex under #1
\newcommand{\emphind}[1]{\index{#1}\emph{#1}\xspace} % emph #1
\newcommand{\emphsubind}[2]{\index{#1!#2}\emph{#2}\xspace} % show emph #2, subindex inder #1
\newcommand{\scatind}[1]{\index{#1@\textsf{#1} (category)}\scat{#1}} % category
\newcommand{\pkgind}[1]{\index{#1@\textsf{#1} (package)}\pkg{#1}} % package
\newcommand{\protind}[1]{\index{#1@\textsf{#1} (protocol)}\prot{#1}} % protocol
\newcommand{\clsind}[1]{\index{#1@\textsf{#1} (class)}\ct{#1}\xspace}
% \newcommand{\clsind}[1]{\index{#1!\#@(class)}\ct{#1}\xspace} % class
\newcommand{\clsindplural}[1]{\index{#1!\#@(class)}\ct{#1}s\xspace} % class
\newcommand{\cvind}[1]{\index{#1@\textsf{#1} (class variable)}\ct{#1}\xspace} % class var
\newcommand{\glbind}[1]{\index{#1@\textsf{#1} (global)}\ct{#1}\xspace} % global
\newcommand{\patind}[1]{\index{#1@#1 (pattern)}\ct{#1}\xspace} % pattern
\newcommand{\pvind}[1]{\index{#1@\textsf{#1} (pseudo variable)}\ct{#1}\xspace} % pseudo var
\newcommand{\clsmthind}[2]{\index{#1!#2@\ct{#2}}\ct{#1>>>#2}\xspace} % class + method name
\newcommand{\mthind}[2]{\index{#1!#2@\ct{#2}}\ct{#2}\xspace} % show method name only
\newcommand{\lmthind}[2]{\index{#1!#2@\ct{#2}}\lct{#2}\xspace} % show method name only
\newcommand{\cmind}[2]{\index{#1!#2@\ct{#2}}\ct{#1>>>#2}\xspace} % show class>>method
\newcommand{\lcmind}[2]{\index{#1!#2@\ct{#2}}\lct{#1>>>#2}\xspace} % show class>>method
\newcommand{\toolsflapind}{\index{Tools flap}\toolsflap} % index tools flap
% The following only generate an index entry:
% \newcommand{\clsindex}[1]{\index{#1@\textsf{#1} (class)}}
\newcommand{\clsindex}[1]{\index{#1!\#@(class)}} % class
\newcommand{\mthindex}[2]{\index{#1!#2@\ct{#2}}} % method
\newcommand{\cmindex}[2]{\index{#1!#2@\ct{#2}}} % class>>method
\newcommand{\cvindex}[1]{\index{#1@\textsf{#1} (class variable)}} % class var
\newcommand{\glbindex}[1]{\index{#1@\textsf{#1} (global)}}% global
\newcommand{\pvindex}[1]{\index{#1@\textsf{#1} (pseudo variable)}}% pseudo var
\newcommand{\seeindex}[2]{\index{#1|see{#2}}} % #1, see #2
\newcommand{\scatindex}[1]{\index{#1@\textsf{#1} (category)}} % category
\newcommand{\pkgindex}[1]{\index{#1@\textsf{#1} (package)}} % package
\newcommand{\protindex}[1]{\index{#1@\textsf{#1} (protocol)}} % protocol
% How can we have the main entry page numbers in bold yet not break the hyperlink?
\newcommand{\boldidx}[1]{{\bf #1}} % breaks hyperlink
%\newcommand{\indmain}[1]{\index{#1|boldidx}#1\xspace} % plain text, main entry
%\newcommand{\emphsubindmain}[2]{\index{#1!#2|boldidx}\emph{#2}\xspace} % subindex, main entry
%\newcommand{\subindmain}[2]{\index{#1!#2|boldidx}#2\xspace} % subindex, main entry
%\newcommand{\clsindmain}[1]{\index{#1@\textsf{#1} (class)|boldidx}\ct{#1}\xspace}
%\newcommand{\clsindmain}[1]{\index{#1!\#@(class)|boldidx}\ct{#1}\xspace} % class main
%\newcommand{\indexmain}[1]{\index{#1|boldidx}} % main index entry only
\newcommand{\indmain}[1]{\index{#1}#1\xspace} % The main index entry for this item
\newcommand{\emphsubindmain}[2]{\index{#1!#2}\emph{#2}\xspace} % subindex, main entry
\newcommand{\subindmain}[2]{\index{#1!#2}#2\xspace} % subindex, main entry
%\newcommand{\clsindmain}[1]{\index{#1@\textsf{#1} (class)}\ct{#1}\xspace}
\newcommand{\clsindmain}[1]{\index{#1!\#@(class)}\ct{#1}\xspace} % class main
\newcommand{\clsindexmain}[1]{\index{#1!\#@(class)}} % class main index only
\newcommand{\indexmain}[1]{\index{#1}} 
%=============================================================
%:Code macros
% some constants
\newcommand{\codesize}{\small}
\newcommand{\codefont}{\sffamily}
%\newcommand{\cat}[1]{\textit{In category #1}}%%To remove later
\newlength{\scriptindent}
\setlength{\scriptindent}{.3cm}
%% Method presentation constants
\newlength{\methodindent}
\newlength{\methodwordlength}
\newlength{\aftermethod}
\setlength{\methodindent}{0.2cm}
\settowidth{\methodwordlength}{\ M\'ethode\ }
%=============================================================
%:Smalltalk macros
%\newcommand{\sep}{{$\gg$}}
\newcommand{\sep}{\mbox{>>}}
\newcommand{\self}{\lct{self}\xspace}
\newcommand{\super}{\lct{super}\xspace}
\newcommand{\nil}{\lct{nil}\xspace}
%=============================================================
% be less conservative about float placement
% these commands are from http://www.tex.ac.uk/cgi-bin/texfaq2html?label=floats
\renewcommand{\topfraction}{.9}
\renewcommand{\bottomfraction}{.9}
\renewcommand{\textfraction}{.1}
\renewcommand{\floatpagefraction}{.85}
\renewcommand{\dbltopfraction}{.66}
\renewcommand{\dblfloatpagefraction}{.85}
\setcounter{topnumber}{9}
\setcounter{bottomnumber}{9}
\setcounter{totalnumber}{20}
\setcounter{dbltopnumber}{9}
%=============================================================
% Give information from each chapter's author
\newcommand{\contact}[2]{\textbf{#1} \textsf{(#2)}}

\newcommand{\chapterauthor}[1]{\emph{with the participation of:\\#1}\\}

\newcommand{\authornoury}{\contact{Noury Bouraqadi}{Noury.Bouraqadi@mines-douai.fr}}
\newcommand{\authordamienc}{\contact{Damien Cassou}{damien.cassou@gmail.com}}
\newcommand{\authoroscar}{\contact{Oscar Nierstrasz}{oscar.nierstrasz@acm.org}}
\newcommand{\authorsteph}{\contact{St\'ephane Ducasse}{stephane.ducasse@inria.fr}}
\newcommand{\authoralex}{\contact{Alexandre Bergel}{alexandre@bergel.eu}}
\newcommand{\authorolivier}{\contact{Olivier Auverlot}{olivier.auverlot@inria.fr}}
\newcommand{\authornicolas}{\contact{Nicolas Cellier}{nicolas.cellier.aka.nice@gmail.com}}
\newcommand{\authormarcus}{\contact{Marcus Denker}{marcus.denker@inria.fr}}
\newcommand{\authoralain}{\contact{Alain Plantec}{alain.plantec@univ-brest.fr}}
\newcommand{\authordale}{\contact{Dale Henrichs}{dale.henrichs@gemstone.com}}
\newcommand{\authormariano}{\contact{Mariano Martinez Peck}{marianopeck@gmail.com}}
\newcommand{\authorsven}{\contact{Sven Van Caekenberghe}{sven@beta9.be}}
\newcommand{\authorlukas}{\contact{Lukas Renggli}{renggli@gmail.com}}

%=============================================================
% apb doesn't like paragraphs to run in to each other without a break
\parskip 1ex
%=============================================================
%:Stuff to check, merge or deprecate
%\setlength{\marginparsep}{2mm}
%\renewcommand{\baselinestretch}{1.1}
%=============================================================

%	\usepackage{a4wide}
% --------------------------------------------
    \graphicspath{{figures/} {../figures/}}
	\begin{document}
	\renewcommand{\nnbb}[2]{} % Disable editorial comments
	\sloppy
	
\fi

	\newcommand{\Mont}{Monticello\xspace}
	\newcommand{\MCB}{\Mont browser\xspace}
	\newcommand{\RI}{repository inspector\xspace}

%=================================================================
\chapter{Versioning your code with \Mont}
\chapterauthor{\authoroscar{}}

\alex{Maybe a section on MC Configuration maps may be given}
\on{what are they?}

\on{Instead of a chapter on Installer, wouldn't a short section on Scripting Monticello suffice?}


\lr{
Subject: Re: [Pharo-project] Create a changeset for an entire package?\\
Reply-To: Pharo-project@lists.gforge.inria.fr\\
a) I don't wan't to include everything I changed from the package.
There are some things that I changed for debugging. So, I can't
fileout everything and I don't know if monticello will let me do
what I want.\\
It is possible, but it requires some manual work to get a custom
change-set from A to B:\\
1. Make sure that both versions A and B are stored as Monticello
Packages (probably to a local directory).\\
2. Load version A.\\
3. Load version B.\\
4. In the changes-browser there is now a new change-set named A->B.\\
5. Open a change browser on it and edit it as you like.\\
6. File-out the change-set.\\
What I usually do is different and I think goes better with the whole
workflow of Monticello:\\
1. In an image where I have the modified code I display the changes
between the previous version (A) and the current image (B).\\
2. Then I selectively revert the changes that I don't want to go into B.\\
3. And commit the package B.\\
b) Supose the change includes 3 separate packages, that should be
seen as a one. Loading only one package only would break things, so
I'd like the change to include all 3 in one operation.\\
It is possible to create custom PackageInfo subclasses (Seaside <= 2.8
did this), but I have rarely seen this and don't find it particularly
useful. I would stick to the original packages to make merging for the
maintainers easy.\\
Lukas
}

\mb{- in PBE, it would be nice to define the package from the start with
OB. The section is made for Squeak w/o package browser\\
- 'You should now be able to verify that the only the original (red)
tests are loaded.' typo before 'Branching' subsection\\
- in Branching subsection: 'Once again the tests should be green,
though our implementation of perfect numbers is slightly different.'
but there is only one test now so: 'The current test should be green...'\\
- (replace the merge tool by the current one in Pharo / add explanation)\\
- In Change subsection: the Patch Browser is mentionned in the figure
not in the text. At the end of the first paragraph, we should add 'via
the Patch Browser.'}

A versioning system helps you to store and log multiple versions of your code. In addition it may help you to manage concurrent accesses to a common source code repository. It keeps track of all changes to a set of documents and enables several developers to collaborate. As soon as the size of your software increases beyond a few classes, you probably need a versioning system.

Many different versioning systems are available. \ind{CVS}\footnote{\url{http://www.nongnu.org/cvs}} and \ind{Subversion}\footnote{\url{http://subversion.tigris.org}} are probably the most popular.
In principle you could use them to manage the development of \pharo software projects, but such a practice would disconnect the versioning system from the \pharo environment.
In addition, CVS-like tools only version plain text files and not individual packages, classes or methods. We would therefore lack the ability to track changes at the appropriate level of granularity. 
If the versioning tools know that you store classes and methods instead of plain text,
they can do a better job of supporting the development process.

%\ab{It would be better to say concretely why this would be a bad idea, rather than resorting to name-calling.  Many people think that files are good.  Let's tell the reader about the disadvantages!}

%\indmain{Monticello}
\index{SqueakSource}
\emph{\indmain{\Mont{}}} is  a versioning system for \pharo in which classes and methods, rather than lines of text,
are the units of change. \emph{\sqsrc{}} is a central online repository in which you can store versions of your applications using \Mont. \sqsrc is the equivalent of \ind{GForge}, and \Mont the equivalent of CVS. 

In this chapter, you will learn how to use use \Mont and \sqsrc to manage your software. We have already met \Mont briefly in earlier chapters\footnote{``A first application'' and ``The Pharo programming environment''}.
% These chapters are in book1, so we can't cross-reference them!
% (\charef{firstApp} and \charef{env}).
This chapter delves into the details of \Mont{} and describes some additional features that are useful for versioning large applications.

%=================================================================
\section{Basic usage}

We will start by reviewing the basics of creating a package and committing changes, and then we will see how to update and merge changes. 

%-----------------------------------------------------------------
\subsection{Running example --- perfect numbers}

We will use a small running example of perfect numbers\footnote{Perfect numbers were discovered by \ind{Euclid}. A perfect number is a positive integer that is the sum of its proper divisors. $6 = 1 + 2 + 3$ is the first perfect number.} in this chapter to illustrate the features of \Mont. We will start our project by defining some simple tests.

\dothis{Define a subclass of \clsind{TestCase} called \ct{PerfectTest} in the category \ct{Perfect}, and define the following test methods in the protocol \ct{running}:}
\begin{code}{}
PerfectTest>>>testPerfect
	self assert: 6 isPerfect.
	self assert: 7 isPerfect not.
	self assert: 28 isPerfect.
\end{code}

Of course these tests will fail as we have not yet implemented the \ct{isPerfect} method for integers. We would like to put this code under the control of \Mont as we revise and extend it.

%-----------------------------------------------------------------
\subsection{Launching \Mont}

\Mont is included in the standard \pharo distribution.
We will assume that \Mont is already installed in your image.
\menu{\Mont Browser} can be selected from the \emph{World} menu.
\index{Monticello!browser}

\begin{figure}[ht]\centering
	\includegraphics[width=\textwidth]{monticelloBrowser}
	\caption{The \Mont Browser.\figlabel{monticelloMain}}
\end{figure}

In \figref{monticelloMain} we see that the \Mont Browser consists of two list panes and one button pane. The left pane lists installed packages and the right panes shows known repositories.
Various operations may be performed via the button pane and the menus of the two list panes.

%-----------------------------------------------------------------
\subsection{Creating a package}

\index{Monticello!creating packages}
\Mont manages versions of \emph{packages}.  A \ind{package} is essentially a named set of classes and methods.
In fact, a package is an object\,---\,an instance of \clsind{PackageInfo}\,---\,that knows how to identify the classes and methods that belong to it.

We would like to version our \ct{PerfectTest} class. The right way to do this is to define a package\,---\,called \ct{Perfect}\,---\,containing \ct{PerfectTest} and all the related classes and methods we will introduce later. For the moment, no such package exists. We only have a \emph{category} called (not coincidentally) \ct{Perfect}. This is perfect, % (ugh!), 
since \Mont will map categories to packages for us.

\dothis{Press the \button{+Package} in the \MCB and enter \scat{Perfect}.}

\emph{Voil\`a!} You have just created the \pkg{Perfect} \Mont package. 

\begin{figure}[ht]\centering
	\includegraphics[width=\textwidth]{perfectPackage}
	\caption{Creating the Perfect package.\figlabel{perfect}}
\end{figure}

\Mont packages follow a number of important naming conventions for class and method categories.
Our new package named \pkg{Perfect} contains:

\begin{itemize}

\item All classes in the category \scat{Perfect}, or in categories whose names start with \scat{Perfect-}. For now this includes only our \ct{PerfectTest} class. \sd{Ask lukas because I do not see why Perfect- would be any different from PerfectZork - is not a special character.}

\item All methods belonging to \emph{any} class (in any category) that are defined in a protocol named \prot{*perfect} or \prot{*Perfect}, or in protocols whose names start with \prot{*perfect-} or \prot{*Perfect-}. Such methods are known as \emph{extensions}. We don't have any yet, but we will define some very soon.

\item All methods belonging to any classes in the category \scat{Perfect}, or in categories whose names begin with \scat{Perfect-}, \emph{except} those in protocols whose names start with \prot{*} (\ie those belonging to \emph{other} packages). This includes our \ct{testPerfect} method, since it belongs to the protocol \ct{running}.

\end{itemize}

%-----------------------------------------------------------------
\subsection{Committing changes}

\index{Monticello!committing changes}
Note in \figref{perfect} that the \button{Save} button is disabled (greyed out).

Before we save our \ct{Perfect} package, we need to specify \emph{where} we want to save it. A \emph{repository} is a package container, which may either be local to your machine or remote (accessed over the network). 
Various protocols may be used to establish a connection between your \pharo image and a repository. As we will see later (\secref{monti:repositories}), \Mont supports a large choice of repositories, though the most commonly used is HTTP, since this is the one used by \sqsrc.

\index{package-cache}
At least one repository, called \ct{package-cache}, is set up by default, and is shown as the first entry in the list of repositories on the right-hand side of your \MCB (see \figref{monticelloMain}).
The package-cache is created automatically in the local directory where your \pharo image is located. It will contain a copy of all the packages you download from remote repositories. 
By default, copies of your packages are also saved in the package-cache when you save them to a remote server.

Each package knows which repositories it can be saved to.
To add a new repository to the selected package, press the \button{+Repository} button. This will offer a number of choices of different kinds of repository, including HTTP. For the rest of the chapter we will work with the \ct{package-cache} repository, as this is all we need to explore the features of \Mont.

\dothis{Select the directory repository named \lct{package cache}, press \button{Save}, enter an appropriate log message, and \button{Accept} to save the changes.}

\begin{figure}[ht]\centering
	\includegraphics[width=.6\textwidth]{saving}
	\caption{You may set a new version name and a commit message when you save a version of a package.\figlabel{saving}}
\end{figure}

The \ct{Perfect} package is now saved in \ct{package-cache}, which is nothing more than a directory contained in the same directory as your \pharo image. Note, however, that if you use any other kind or repository (\eg{} HTTP, FTP, another local directory), a copy of your package will also be saved in the package-cache. 

\dothis{Use your favorite file browser (\eg Windows Explorer, Finder or XTerm) to confirm that a file \ct{Perfect-XX.1.mcz} was created in your package cache. \ct{XX} corresponds to your name or initials.\footnote{In the past, the convention was for developers to log their changes using only their initials. Now, with many developers sharing identical initials, the convention is to use an identifier based on the full name, such as ``apblack'' or ``AndrewBlack''.}}

%You may change the version name and add a comment.
%\ab{Experience with my students is that changing the version name is a really bad idea, since it can easily break \Mont.}
%\alex{I was not able to find a satisfying explanation, So I prefer to leave this point out}

A \emphind{version} is an immutable snapshot of a package that has been written to a repository. Each version has a unique version number to identify it in a repository.
Be aware, however, that this number is \emph{not} globally unique\,---\,in another repository you might have the same file identifier for a \emph{different snapshot}. For example, \ct{Perfect-onierstrasz.1.mcz} in another repository might be the \emph{final}, deployed version of our project!
When saving a version into a repository, the next available number is automatically assigned to the version, but you can change this number if you wish.
Note that version branches do not interfere with the numbering scheme (as with CVS or Subversion). As we shall see later, versions are by default ordered by their version number when viewing a repository.

%-----------------------------------------------------------------
\subsection{Class extensions}

Let's implement the methods that will make our tests green.

\dothis{Define the following two methods in the class \ct{Integer}, and put each method in a protocol called \ct{*perfect}. Also add the new boundary tests.  Check that the tests are now green.}

\begin{code}{}
Integer>>>isPerfect
	^ self > 1 and: [self divisors sum = self]

Integer>>>divisors
	^ (1 to: self - 1 ) select: [ :each | (self rem: each) = 0 ]

PerfectTest>>>testPerfectBoundary
	self assert: 0 isPerfect not.
	self assert: 1 isPerfect not.
\end{code}

Although the methods on \ct{Integer} do not belong to the \scat{Perfect} category, they \emph{do} belong to the \ct{Perfect} package since they are in a protocol whose name starts with \ct{*} and matches the package name. Such methods are known as \emphind{class extensions}, since they extend existing classes. These methods will be available \emph{only} to someone who loads the \ct{Perfect} package.

\alex{We could give a reference to the Cmd-p command in the OB browser. This keystroke is used to turn a method into a class-extension.}
\on{you mean it moves a method to an extension package}

%-----------------------------------------------------------------
\subsection{``Clean'' and ``Dirty'' packages}

\index{package!clean and dirty}
Modifying the code in a package with any of the development tools makes that package \emph{dirty}.
This means that the version of the package in the image is different from the version that has been saved or loaded. 

\begin{figure}[ht]\centering
	\includegraphics[width=\textwidth]{dirty}
	\caption{Modifying our Perfect package will ``dirty'' it.\figlabel{dirty}}
\end{figure}

In the \MCB,  a dirty package can be recognized by an asterix (\ct{*}) preceding its name.  This indicates which packages have uncommitted changes, and therefore need to be saved into a repository if those changes are not to be lost. Saving a dirty package cleans it. \ab{Why is it that when I click \button{Changes} on a dirty package, I sometimes get a dialog back that says ``no changes''?  I found this very confusing.  How can I see \emph{why} \Mont thinks a package is dirty?  How can I see \emph{which} repository it was originally loaded from or saved to?}

\dothis{Try the \button{Browse}, \button{History} and \button{Changes} buttons to see what they do\footnote{At the time of this writing, the \button{Scripts} button does not work.}.
\button{Save} the changes to the \ct{Perfect} package. Confirm that the package is now ``clean'' again.}

%\on{What does \button{Scripts} do?!}
%\subsection{Scripts}
% Script do not work!

%-----------------------------------------------------------------
\subsection{The Repository inspector} 

\index{Repository inspector|see{Monticello, repository inspector}}
\index{Monticello!repository inspector}
The contents of a repository can be explored using a \RI, which is launched using the \button{Open} button of \Mont (cf \figref{repositoryinspector}).

\index{package-cache}
\dothis{Select the \ct{package-cache} repository and open it. You should see something like \figref{repositoryinspector}.}

\begin{figure}[ht]\centering
	\includegraphics[width=\textwidth]{{repositoryinspector}}
	\caption{A \RI.\figlabel{repositoryinspector}}
\end{figure}

All the packages in the repository are listed on the left-hand side of the inspector:

\begin{itemize}
\item an \underline{underlined} package name means that this package is installed in the image;
\item a \underline{\bf bold underlined} name means that the package is installed, but that there is a more recent version in the repository;
\item a name in a normal typeface means that the package is not installed in the image.
\end{itemize}

\noindent
Once a package is selected, the right-hand pane lists the versions of the selected package:

\begin{itemize}
\item an \underline{underlined} version name means that this version is installed in the image;
\item a {\bf bold} version name means that this version is not an ancestor of the installed version. This may mean that it is a newer version, or that it belongs to a different branch from the installed version;
\item a version name displayed with a normal typeface shows an older version than the installed current one.
\end{itemize}

{\Actclick}ing the right-hand side of the inspector opens a menu with different sorting options. The \menu{unchanged} entry in the menu discards any particular sorting. It uses the order given by the repository.

%-----------------------------------------------------------------
\subsection{Loading, unloading and updating packages}

\index{package-cache}
\index{Monticello!loading, unloading, updating packages}
At present we have two versions of the \ct{Perfect} package stored safely in our \ct{package-cache} repository. We will now see how to unload this package, load an earlier version, and finally update it.

\dothis{Select the \ct{Perfect} package and its repository in the \MCB. \Actclick on the package name and select \menu{unload package}.}

\begin{figure}[th]\centering
	\includegraphics[width=\textwidth]{{unload}}
	\caption{Unloading a package.\figlabel{unload}}
\end{figure}

You should now be able to confirm that the \ct{Perfect} package has vanished from your image!

\dothis{In the \MCB, select the \ct{package-cache} in the repository pane, without selecting anything in the package pane, and \button{Open} the \RI.
Scroll down and select the \ct{Perfect} package. It should be displayed in a normal typeface, indicated that it is not installed.
Now select version 1 of the package and \button{Load} it.}

\begin{figure}[ht]\centering
	\includegraphics[width=\textwidth]{{loading}}
	\caption{Loading an earlier version.\figlabel{loading}}
\end{figure}

You should now be able to verify that the only the original (red) tests are loaded.

\dothis{Select the second version of the \ct{Perfect} package in the \RI and \button{Load} it.
You have now \emph{updated} the package to the latest version.}

Now the tests should be green again.

%-----------------------------------------------------------------
\subsection{Branching}

\index{Monticello!branching}
A \emphind{branch} is a line of development versions that exists independently of another line, yet still shares a common ancestor version if you look far enough back in time.

You may create new version branch when saving your package. Branching is useful when you want to have a new parallel development. For example, suppose your job is to maintain a software in your company. One day a different division asks you for the same software, but with a few parts tweaked for them, since they do things slightly differently. The way to deal with this situation is to create a second branch of your program that incorporate the tweaks, while leaving the first branch unmodified.

\dothis{From the \RI, select version 1 of the \ct{Perfect} package and \button{Load} it. Version 2 should again be displayed in bold, indicating that it no longer loaded (since it is not an ancestor of version 1).
Now implement the following two \ct{Integer} methods and place them in the \ct{*perfect} protocol, and also modify the existing \ct{PerfectTest} test method as follows:}

\begin{code}{}
Integer>>>isPerfect
	self < 2 ifTrue: [ ^ false ].
	^ self divisors sum = self

Integer>>>divisors
	^ (1 to: self - 1 ) select: [ :each | (self \\ each) = 0]

PerfectTest>>>testPerfect
	self assert: 2 isPerfect not.
	self assert: 6 isPerfect.
	self assert: 7 isPerfect not.
	self assert: 28 isPerfect.
\end{code}

Once again the tests should be green, though our implementation of perfect numbers is slightly different.

\dothis{Attempt to load version 2 of the \ct{Perfect} package.}

Now you should get a warning that you have unsaved changes.

\begin{figure}[ht]\centering
	\includegraphics[width=.8\textwidth]{{unsavedWarning}}
	\caption{Unsaved changes warning.\figlabel{unsavedWarning}}
\end{figure}

\dothis{Select \button{Abandon} to avoid overwriting your new methods.
Now \button{Save} your changes. You will get another warning that there may be newer versions.
Select \menu{Yes}, enter your log message, and \button{Accept} the new version.
}

\begin{figure}[ht]\centering
	\includegraphics[width=\textwidth]{{newerWarning}}
	\caption{Newer versions warning.\figlabel{newerWarning}}
\end{figure}

Congratulations! You have now created a new branch of the \ct{Perfect} package.

\begin{figure}[ht]\centering
	\includegraphics[width=\textwidth]{{branch}}
	\caption{Versions 2 and 3 are separate branches of version 1.\figlabel{branch}}
\end{figure}

\dothis{If you still have the \RI open, \button{Refresh} it to see the new version (\figref{branch}).}

%-----------------------------------------------------------------
\subsection{Merging}

\on{Should we be describing the new strategy instead?}

\index{Monticello!merging}
\index{Monticello!merge tool}
This section describes the conventional merging facility of Monticello.
To use it, make sure that the preference \ct{useNewDiffToolsForMC} is disabled.
You can either use the Preference Browser, or you can do this programmatically by evaluating the following expression:
\begin{code}{}
Preferences disable: #useNewDiffToolsForMC
\end{code}

% \alex{This subsection will need a major overhaul when turning the chapter for Pharo. The merging tool is radically different}\sd{Is it not. There is a preference to get the one shown in this section. We should just say useDiff....}
You can merge one version of a package with another using the \button{Merge} button in the \MCB. Typically you will want to do this when (i) you discover that you have been working on a out-of-date version, or (ii) branches that were previously independent have to be re-integrated. Both scenarios are common when multiple developers are working on the same package.

%\ap{Here I stopped}

Consider the current situation with our \ct{Perfect} package, as illustrated at the left of \figref{branching-merging}. We have published a new version 3 that is based on version 1.
Since version 2 is also based on version 1, versions 2 and 3 constitute independent branches.

\begin{figure}[ht]\centering
	\includegraphics[width=\textwidth]{branching-merging}
	\caption{Branching (left) and merging (right).\figlabel{branching-merging}}
\end{figure}

At this point we realize that there are changes in version 2 that we would like to merge with our changes from version 3.  Since we have version 3 currently loaded, we would like to merge in changes from version 2, and publish a new, merged version 4, as illustrated at the right of \figref{branching-merging}.

%\begin{figure}[ht]\centering
%	\includegraphics[width=.5\textwidth]{branching}
%	\caption{Version 3 is the current branch of version 1.\figlabel{branching}}
%\end{figure}


%\begin{figure}[ht]\centering
%	\includegraphics[width=.5\textwidth]{merging}
%	\caption{Version 4 merges changes from versions 2 and 3.\figlabel{merging}}
%\end{figure}

\begin{figure}[ht]\centering
	\includegraphics[width=\textwidth]{mergeButton}
	\caption{Select a separate branch (in bold) to be merged.\figlabel{mergeButton}}
\end{figure}

\dothis{Select version 2 in the repository browser, as shown in \figref{mergeButton}, and \click the \button{Merge} button.}

The merge tool is a tool that allows for fine-grained package version merging. Elements contained in the package to-be-merged are listed in the upper text pane. The lower text pane shows the definition of a selected element. 

\begin{figure}[ht]\centering
	\includegraphics[width=.8\textwidth]{mergeTool}
	\caption{Version 2 of the \ct{Perfect} package being merged with the current version 3.
	\on{This is a major fail! The old merge tool used to show deleted text striked through.
	Now, colour blind people will see no difference at all!}
	\figlabel{mergeTool}}
\end{figure}

In \figref{mergeTool} we see the three differences between versions 2 and 3 of the \ct{Perfect} package. The method \ct{PerfectTest>>>testPerfectBoundary} is new, and the two indicated methods of \ct{Integer} have been changed. In the lower pane we see the old and new versions of the source code of \ct{Integer>>>isPerfect}.
New code is displayed in red, removed code is barred and displayed in blue, and unchanged code is shown in black. 

A method or a class is in conflict if its definition has been altered. \figref{mergeTool} shows 2 conflicting methods in the class \ct{Integer}: \ct{isPerfect} and \ct{divisors}. A conflicting package element is indicated by being \underline{underlined}, \sout{barred}, or {\bf bold}. The full set of typeface conventions is as follows:

\begin{description}
\item[Plain=No Conflict.] A plain typeface indicates the definition is non-conflicting. For example, the method \ct{PerfectTest>>>testPerfectBoundary} does not conflict with an existing method, and can be installed.
\item[Bold=A method is conflicting.] A decision needs to be taken to keep the proposed change or reject it. The proposed method \ct{Integer>>>>isPerfect} is in conflict with an existing definition in the image. The conflict can be resolved by clicking \button{Keep} or \button{Reject}.
\item[Underlined=Repository replace current.] An \underline{underlined} element will be kept and replace the current element in the image. In \figref{keepReject} we see that \ct{Integer>>>isPerfect} from version 2 has been kept.
\item[Barred=Repository version rejected.] A \sout{barred} element has been rejected, and the local definition will not be replaced. In \figref{keepReject} \ct{Integer>>>divisors} from version 2 is rejected, so the definition from version 3 will remain.
\on{This appears to no longer be true! -- a rejected element is simply shown in plain font! -- should be fixed.}
\end{description}

\begin{figure}[ht]\centering
	\includegraphics[width=.8\textwidth]{keepReject}
	\caption{Keeping and rejecting changes.
	\figlabel{keepReject}}
\end{figure}

Note that the merge tool offers buttons to select all newer or all older changes, or to select all local or all remote changes that are still in conflict.

\dothis{Keep \ct{Integer>>>>isPerfect} and reject \ct{Integer>>>divisors}, and click the \button{Merge} button. Confirm that the tests are all green. Commit the new merged version of \ct{Perfect} as version 4.}

\on{Something is wrong here -- it complains that version 3 might be newer. When I saved it however everything seems fine.}

\begin{figure}[ht]\centering
	\includegraphics[width=0.8\textwidth]{merged}
	\caption{All older versions are now ancestors of merged version 4.
	\figlabel{merged}}
\end{figure}

If you now refresh the \RI, you will see that there are no more versions shown in bold, \ie all versions are ancestors of the currently loaded version 4 (\figref{merged}).

%=================================================================
\section{Exploring \Mont repositories}
\seclabel{monti:exploring}

\Mont has many other useful features. As we can see in \figref{monticelloMain}, the \MCB window has nine buttons. We have already used four of them\,---\,\button{+Package}, \button{Save}, \button{+Repository} and \button{Open}.
We will now look at \button{Browse}, \button{Changes} and \button{History}, which are used to explore the state and history of repositories

%-----------------------------------------------------------------
\subsection{Browse}

\index{Monticello!snapshot browser}
The \button{Browse} button opens a ``snapshot browser'' to display the contents of a package.
The advantage of the snapshot browser over the browser is its ability to display class extensions.

\dothis{Select the \ct{Perfect} package and click the \button{Browse} button.}

\begin{figure}[ht]
\centering
	\includegraphics[width=\textwidth]{{packageviewer}}
	\caption{The snapshot browser reveals that the \ct{Perfect} package extends the class \ct{Integer} with 2 methods.\figlabel{packageviewer}}
\end{figure}

\index{class extension}
For example, \figref{packageviewer} shows the class extensions defined in the \pkg{Perfect} package. Note that code cannot be edited here, though by {\actclick}ing, if your environment has been set up accordingly) on a class or a method name you can open a regular browser.

It is a good practice to always browse the code of your package before publishing it, to ensure that it really contains what you think it does.

\alex{David's browser shows class extensions. Although it does not supersede the snapshot browser, it should be mentioned here}


% \sd{could we get rid of this yellow and red clicking. it makes no sense. I remember that I propose a good naming convention for the mouse button -- it was also in one of the freebooks - select = left = you select a window to move it around, operate you want to get the menus and act on something and  meta is for the rest}

%-----------------------------------------------------------------
\subsection{Changes} 

\index{Monticello!patch browser}
The \button{Changes} button computes the difference between the code in the image and the most recent version of the package in the repository. 

\dothis{Make the following changes to \ct{PerfectTest}, and then click the \button{Changes} button in the \MCB.}

\begin{code}{}
PerfectTest>>>testPerfect
	self assert: 2 isPerfect not.
	self assert: 6 isPerfect.
	self assert: 7 isPerfect not.
	self assert: 496 isPerfect.

PerfectTest>>>testPerfectTo1000
	self assert: ((1 to: 1000) select: [:each | each isPerfect]) = #(6 28 496)
\end{code}

\begin{figure}[ht]\centering
	\includegraphics[width=\textwidth]{{patchbrowser}}
	\caption{The patch browser shows the difference between the code in the image and the most recently committed version.\figlabel{patchbrowser}}
\end{figure}

\figref{patchbrowser} shows that the \ct{Perfect} package has been locally modified with one changed method and one new method.
As usual, {\actclick}ing on a change offers you a choice of contextual operations.

%-----------------------------------------------------------------
\subsection{History} 

\index{Monticello!history browser}
The \button{History} button opens a version history viewer that displays the comments committed along with each version of the selected package (see \figref{historyviewer}).  
The versions of the package, in this case \ct{Perfect}, are listed on the left,
while information about the selected version is displayed on the right.

\dothis{Select the \ct{Perfect} package and click the \button{History} button.}

\begin{figure}[ht]\centering
	\includegraphics[width=\textwidth]{{historyviewer}}
	\caption{The version history viewer provides information about the various versions of a package.\figlabel{historyviewer}}
\end{figure}

By {\actclick}ing on a particular version, you can explore the changes with respect to the current working copy of the package loaded in the image, or spawn a new history browser relative to the selected version.

%=================================================================
\section{Advanced topics}
\seclabel{monti:advanced}

Now we will have a look at several advanced topics, including backporting, managing dependencies, and class initialization.

%-----------------------------------------------------------------
\subsection{Backporting}

\alex{I am wondering how useful this is. I personally never used it}

\index{Monticello!backporting}
Sometimes we want to port changes from one branch to another, without actually being forced to merge those branches.
Backporting is a process of applying selected changes from one version of a package to an ancestor so that these changes can be merged into later branches.  This is especially useful when corrections to software defects must be merged into multiple branches.

The process is illustrated in \figref{backport}.
Suppose that the main branch of our software system consists of versions A and B, maintained by Manny.
A contributor, Conny, has developed a separate experimental branch, C, with changes X and Y.
Change X fixes a nasty problem in versions A and B, so Manny asks Conny to prepare a backported branch D containing \emph{only} change X.
Now Manny can merge B and D to produce a new version E that fixes the defect.
Conny can continue to further develop her independent branch C.

\begin{figure}[ht]
\centering
	\includegraphics[width=\textwidth]{{backport}}
	\caption{Change X is backported from version C to version A, producing a new branch D. D can then be merged into B, without affecting C.\figlabel{backport}}
\end{figure}

The system records the fact that this new version was backported from a later version, and will make use of that information when merging.

To use \button{Backport}, you must have just saved your package\,---\,if your package is marked with the modified *, \button{Backport} is disabled.  When you press \button{Backport}, you will first be asked to pick the ancestor version you want to backport to.  You will then be presented with a multi-select list of all the changes between that ancestor and the current version.   Choose only the changes you want to backport, and then press \button{Select}.

Let us see how this works in practice. Recall that we earlier rejected the implementation of \ct{isPerfect} when we merged versions 2 and 3 of the \ct{Perfect} package. Now we will recover that change as a backport to version 1.  (Versions 1, 2 and 3 play the roles of versions A, B and C, respectively, in \figref{backport}.)

\dothis{Unload the \ct{Perfect} package. Now open a \RI on your \ct{package-cache} and load \ct{Perfect} version 3. In the \MCB select \ct{Perfect} and click on the \button{Backport} button. Select version 1 as the ancestor. You should be able to browse the changes between version 3 and 1, as shown in \figref{changes}.  Now select the \ct{Integer>>>isPerfect} method and click on \button{Select}.}

\begin{figure}[ht]
\centering
	\includegraphics[width=0.8\textwidth]{{changes}}
	\caption{Backporting changes from version 3 to version 1 of the \ct{Perfect} package.\figlabel{changes}}
\end{figure}


Congratulations! You have now backported the \ct{isPerfect} method from version 3 to version 1 of \ct{Perfect}.
Any changes you didn't select were reverted; that is, your image will now contain only the code from the ancestor version 1, plus the changes that you chose.
In the \MCB you should see that the currently loaded version of \ct{Perfect} is now version 1 (not version 3). If you click on \button{Changes}, you will see that the only change is the \ct{isPerfect} method.
You can now save this backported version, merge it into something else, or whatever you like.

%-----------------------------------------------------------------
\subsection{Dependencies}

\index{Monticello!dependency}
\index{package!required|see{Monticello, dependency}}
Most applications cannot live on their own and typically require the presence of other packages in order to work properly. For example, let us have a look at Pier\footnote{\url{http://source.lukas-renggli.ch/pier}}, a meta-described content management system. Pier is a large piece of software with many facets (tools, documentations, blog, catch strategies, security, ...). Each facet is implemented by a separate package. Most Pier packages cannot be used in isolation since they refer to methods and classes defined in other packages. \Mont provides a dependency mechanism for declaring the \emph{required packages} of a given package to ensure that it will be correctly loaded.

Essentially, the dependency mechanism ensures that all required packages of a package are loaded before the package is loaded itself. Since required packages may themselves require other packages, the process is applied recursively to a tree of dependencies, ensuring that the leaves of the tree are loaded before any branches that depend on them.
Whenever new versions of required packages are checked in, then new versions of the packages that depend on them will automatically depend on the new versions.

\important{\emph{Dependencies cannot be expressed across repositories}. All requiring and required packages must live in the same repository.}

\figref{dependencies} illustrates how this works in \ind{Pier}.
Package \ct{Pier-All} is an \emph{empty package} that acts as a kind of umbrella.
It requires \ct{Pier-Blog}, \ct{Pier-Caching} and all the other Pier packages.

\begin{figure}[ht]
\centering
	\includegraphics[width=\textwidth]{{dependencies}}
	\caption{Dependencies in Pier.\figlabel{dependencies}}
\end{figure}

Because of these dependencies, installing \ct{Pier-All} causes all the other Pier packages to be installed. Furthermore, when developing, the only package that needs to be saved is \ct{Pier-All}; all dependent dirty packages are saved automatically.

Let us see how this works in practice.  Our \ct{Perfect} package currently bundles the tests together with the implementation.  Suppose we would like instead to separate these into separate packages, so that the implementation can be loaded without the tests.  By default, however, we would like to load everything.

\dothis{Take the following steps:
\begin{itemize}\setlength{\itemsep}{0pt}
\item Load version 4 of the \ct{Perfect} package from the package cache
\item Create a new package in the browser called \ct{NewPerfect-Tests} and drag the class \ct{PerfectTest} to this package
% \item \sout{Rename the \ct{Perfect} category to \ct{Perfect-Tests} ({\actclick} on the category in the browser to rename it)} \on{There is no menu item to rename a package!}
\item Rename the \ct{*perfect} protocol of the \ct{Integer} class to \ct{*newperfect-extensions} (\actclick to rename it)
\item In the \MCB, add the packages \ct{NewPerfect-All} and \ct{NewPerfect-Extensions}.
\item Add \ct{NewPerfect-Extensions} and \ct{NewPerfect-Tests} as required packages to \ct{NewPerfect-All} ({\actclick} on \ct{NewPerfect-All})
\item Save package \ct{NewPerfect-All} in the package-cache repository.
	Note that \Mont prompts for comments to save the required packages too.
\item Check that all three packages have been saved in the package cache.
\item \Mont thinks that \ct{Perfect} is still loaded. Unload it and then load \ct{NewPerfect-All} from the \RI. This will cause \ct{NewPerfect-Extensions} and \ct{NewPerfect-Tests} to be loaded as well as required packages.
\item Check that all tests run.
\end{itemize}
}
\index{package-cache}

Note that when \ct{NewPerfect-All} is selected in the \MCB, the dependent packages are displayed in bold (see \figref{perfectDependencies}).

\begin{figure}[ht]
\centering
	\includegraphics[width=\textwidth]{{perfectDependencies}}
	\caption{\ct{NewPerfect-All} requires \ct{NewPerfect-Extensions} and \ct{NewPerfect-Tests}.\figlabel{perfectDependencies}}
\end{figure}

\important{If you further develop the \ct{Perfect} package, you should only load or save \ct{NewPerfect-All}, not its required packages.}

Here is the reason why:

\begin{itemize}
\item If you load \ct{NewPerfect-All} from a repository (package-cache, or anywhere else), this will cause \ct{NewPerfect-Extensions} and \ct{NewPerfect-Tests} to be loaded from the same repository.
\item If you modify the \ct{PerfectTest} class, this will cause the \ct{NewPerfect-Tests} and \ct{NewPerfect-All} packages to both become dirty (but not \ct{NewPerfect-Extensions}).
\item To commit the change, you should save \ct{NewPerfect-All}. This will commit a new version of \ct{NewPerfect-All} which then requires the new version of \ct{NewPerfect-Tests}. (It will also depend on the existing, unmodified version of \ct{NewPerfect-Extensions}.)  Loading the latest version of \ct{NewPerfect-All} will also load the latest version of the required packages.
\item If instead you save \ct{NewPerfect-Tests}, this will \emph{not} cause \ct{NewPerfect-All} to be saved.  This is bad because you effectively break the dependency.  If you then load the latest version of \ct{NewPerfect-All} you will not get the latest versions of the required packages. Don't do it!
\end{itemize}

\important{Do not name your top level package with a suffix (\eg \ct{Perfect}) that could match your subpackages. Do not define \ct{Perfect} as a required package of \ct{Perfect-Extensions} or \ct{PerfectTest}. You would get in trouble since \Mont would save all the classes for three packages while you only want two packages and an empty one at the top level.}
%-----------------------------------------------------------------
\subsection{Class initialization}

\index{class!initialization}
\index{Monticello!class initialization}
When \Mont loads a package into the image, any class that defines an \ct{initialize} method on the class side will be sent the \ct{initialize} message. The message is sent \emph{only} to classes that define this method on the class side. A class that does not define this method will not be initialized, even if \ct{initialize} is defined by one of its superclasses. NB: the initialize method is not invoked when you merely reload a package!
% martial: invoked method?? -> better with 'message \ct{initialize} is not sent??

%\on{OK, I confirmed this with a simple test}

Class initialization can be used to perform any number of checks or special actions.
A particularly useful application is to add new instance variables to a class.

Class extensions are strictly limited to adding new methods to a class.
Sometimes, however, extension methods may need new instance variables to exist.

Suppose, for example, that we want to extend the \ct{TestCase} class of SUnit with methods to keep track of the history of the last time the test was red.  We would need to store that information somewhere, but unfortunately we cannot define instance variables as part of our extension.

A solution would be to define an \ct{initialize} method on the class side of one of the classes:

\begin{code}{}
TestCaseExtension class>>initialize
	(TestCase instVarNames includes: 'lastRedRun') 
		ifFalse: [TestCase addInstVarName: 'lastRedRun']
\end{code}

When our package is loaded, this code will be evaluated and the instance variable will be added, if it does not already exist.


%As an illustration, let us suppose an extension of the Lights Out game to support unchangeable cells. Those cells are located randomly on the board. Generate numbers need to be generated. This can be done by adding a variable \ct{randomNumber} to the class \ct{LOGame}. This kind of extension is not well supported by \Mont. A package cannot define the variable \ct{randomNumber} as an extension of \ct{LOGame}. \Mont supports method extension only.
%\ab{Tell us what the problem is, before telling us the solution.  With an Example!}


%One way to circumvent this limitation is to define an initialize method on the class side, and manually add this variable. We could then define the following \ct{initialize} method on a class \ct{GameExtension} contained in the extending package:

%\begin{code}{}
%GameExtension class>>initialize
%	(LOGame instVarNames includes: 'randomNumber') 
%		ifFalse: [LOGame addInstVarName: 'randomNumber']
%\end{code}

%In the case that \ct{randomNumber} is not defined in \ct{LOGame}, this variable is added to this class.

%\ab{Isn't this more appropriate for a post-load \emph{do-it} than an \ct{initialize} method?  Do such things not exist (they did with change sets \ldots}
%\alex{Probably, but this seems to be broken with \Mont}

%=================================================================
\section{Getting a change set from two versions}
%:\alex{Change set were seen in Chapter 6, page 135 of the version 1 of PBE}
A Monticello version is the snapshot of one or more packages. A version contains the complete set of class and method definitions that constitute the underlying packages. Sometimes, it is useful to have a ``patch'' from two versions. A patch is the set of all necessary side effect in the system to go from one version A to another version B. 

\emph{Change set} is a \pharo built-in mechanism to define system patches. A change set is composed of global side effects on the system. New change set may be created and edited from the \emph{Change Sorter}. This tool is available from the \menu{World\go{}Tools} entry.

The difference between two Monticello versions may be easily captured by creating a new change set before loading a second version of a package. As an illustration, we will capture the differences between version 1 and 2 of the \pkg{Perfect} package:
\begin{enumerate}
\item Load version 1 of \pkg{Perfect} from the Monticello browser
\item Open a change sorter and create a new change set. Let's name it \lct{DiffPerfect}
\item Load version 2
\item In the change sorter, you should now see the difference between version 1 and 2. The change set may be saved on the filesystem by \actclick{ing} on it and selecting \menu{file out}. A \lct{DiffPerfect.X.cs} file is now located next to your \pharo image.
\end{enumerate}


%=================================================================
\section{Kinds of repositories}
\seclabel{monti:repositories}

\index{Monticello!repository}
Several kinds of repositories are supported by \Mont, each with different characteristics and uses. Repositories can be read-only, write-only or read-write. Access rights may be defined globally or can be tied to a particular user (as in \sqsrc, for example).
% and are related to the way of storing data used.

\index{Monticello!HTTP repository}
\paragraph{HTTP.} HTTP repositories are probably the most popular kind of repository since this is the kind supported by \sqsrc. % (Such servers can also be configured for read-only access. Saving versions via HTTP uses the PUT method \ab{Is ``put'' the right word}, which must be enabled on the server.)

The nice thing about HTTP repositories is that it's easy to link directly to specific versions from web sites. With a little configuration work on the HTTP server, HTTP repositories can be made browsable by ordinary web browsers, WebDAV clients, and so on.

HTTP repositories may be used with an HTTP server other than \ind{\sqsrc}. For example, a simple configuration\footnote{\url{http://www.visoracle.com/squeak/faq/monticello-1.html}} turns \ind{Apache} into a \Mont repository with restricted access rights:

\begin{code}{}
"My apache2 install worked as a Monticello repository right out of the box on my
RedHat 7.2 server.  For posterity's sake, here's all I had to add to my apache2 config:"
Alias /monticello/ /var/monticello/
<Directory /var/monticello>
  DAV on
  Options indexes
  Order allow,deny
  Allow from all
  AllowOverride None
  # Limit write permission to list of valid users.
  <LimitExcept GET PROPFIND OPTIONS REPORT>
    AuthName "Authorization Realm"
    AuthUserFile /etc/monticello-auth
    AuthType Basic
    Require valid-user
  </LimitExcept>
</Directory>
"This gives a world-readable, authorized-user-writable Monticello repository in
/var/monticello.  I created /etc/monticello-auth with htpasswd and off I went.
I love Monticello and look forward to future improvements."
\end{code}

\index{Monticello!FTP repository}
\paragraph{FTP.} This is similar to an HTTP repository, except that it uses an FTP server instead. An FTP server may also offer restricted access right and different FTP clients may be used to browse such \Mont repository.

\index{Monticello!GOODS repository}
\paragraph{GOODS.}
This repository type stores versions in a GOODS object database.
GOODS is a fully distributed object-oriented database management system that uses an active client model\footnote{\url{http://www.garret.ru/goods.html}}.
It's a read-write repository, so it makes a good ``working'' repository where versions can be saved and retreived. Because of the transaction support, journaling and replication capabilities offered by GOODS, it is suitable for large repositories used by many clients.  

\index{Monticello!directory repository}
\index{package-cache}
\paragraph{Directory.} A directory repository stores versions in a directory in the local file system. Since it requires very little work to set up, it's handy for private projects; since it requires no network connection, it's the only option for disconnected development. The \ct{package-cache} we have been using in the exercises for this chapter is an example of this kind of repository. Versions in a directory repository may be copied to a public or shared repository at a later time. \sqsrc supports this feature by allowing package versions (.mcz files) to be imported for a given project. Simply log in to \sqsrc, navigate to the project, and click on the \menu{Import Versions} link.

\paragraph{Directory with Subdirectories.}  A ``directory with subdirectories'' is very similar to ``directory'' except that it looks in subdirectories to retrieve list of available packages. Instead of having a flat directory that contains all package versions, such as repository may be hierarchically  structured with  subdirectories.

\index{Monticello!SMTP repository}
\paragraph{SMTP.} SMTP repositories are useful for sending versions by mail. When creating an SMTP repository, you specify a destination email address. This could be the address of another developer\,---\,the package's maintainer, for example\,---\,or a mailing list such as pharo-project. Any versions saved in such a repository will be emailed to this address.  SMTP repositories are write-only.

\paragraph{Programmatically adding repositories} For particular purposes, it may be necessary to programmatically add new repositories. This happens when managing configuration and large set of distributed monticello packages or simply customizing the entries available in the Monticello browser. For example, the following code snippet programmatically adds new directory repositories

\begin{code}
| repo |
{'/path/to/repositories/project-1/'. 
'/path/to/repositories/project-2/'. 
'/path/to/repositories/project-3/'. } do: 
[ :path |
	repo := MCDirectoryRepository new directory: 
		(FileDirectory on: path).
	MCRepositoryGroup default addRepository: repo ].
\end{code}

%\paragraph{\sqmap Release} This is a write-only repository used for publishing releases of a package to \sqmap. To configure the repository enter the name of the package on \sqmap, your \sqmap initials and your \sqmap password. Now any versions saved to the repository will be uploaded to your \sqmap account, and registered as a new release with \sqmap.  

%You need an account on \sqmap to add packages. New accounts may be created online\footnote{\url{http://map.squeak.org}}. The list of packages you add will appear on the \sqmap package list displayed when \sqmap is open.

%\alex{I added few sentences about adding package on squeakmap. But not many people are using it now. It is progressively being replaced by universe.}
%\ab{We havn't talked bout how to put stuff on \sqmap, only about how to load it.  So, we should add that material, either here, or in a subsequent subsection.}

%\paragraph{\sqmap Cache} When packages are installed via \sqmap, downloaded files are stored in a dedicated subdirectory in your working directory that acts as a cache. In order to make these files available to \Mont for loading, merging, \etc, a \sqmap Cache repository is created when these files are loaded for the first time.

%\on{why don't you first talk about the package cache?  I see you commented it out.}
%\alex{Because it does not appear in \pharo anymore. Maybe it was buggy or simply obsolete}

%\ab{One thing to say is how to get to versions in some \emph{other} package cache}
%\alex{I am not sure about what you mean.}. 

%\paragraph{package-cache}

%The package cache is a special repository that \Mont creates automatically. Like a directory repository, the package cache stores files in a directory on your local filesystem. See Elements of \Mont for more information.

%\ab{what about ``Directory with Subdirectories''?} 
%\alex{I added a new paragraph on it above}

%-----------------------------------------------------------------
\subsection{Using \sqsrc}

\indmain{\sqsrc} is a online repository that you can use to store your \Mont packages. An instance is running and accessible from \url{http://www.squeaksource.com}. At the time this chapter is being written, over 1500 projects are registered on \sqsrc and nearly 2000 people have an account. \figref{squeaksource} shows the main web page.  

\begin{figure}[ht]\centering
	\includegraphics[width=\textwidth]{squeaksource2}
	\caption{\sqsrc, the online \Mont code repository.\figlabel{squeaksource}}
\end{figure}

\dothis{Use a web browser to visit the \pbe project at
  \url{http://www.squeaksource.com/PharoByExample.html}. This project
  contains the Lights Out project from the first volume of this book.
  In the registration section on that web page you should see this
  \emph{repository expression:}}

\begin{code}{}
MCHttpRepository
    location: 'http://www.squeaksource.com/PharoByExample'
    user: ''
    password: ''
\end{code}
\noindent
\emph{Add this repository to \Mont by clicking \button{+Repository},
  and then selecting \menu{HTTP}. Fill out the template with the URL
  corresponding to the Lights Out project\,---\,you can copy the above
  repository expression from the web page and paste it into the
  template. Since you are not going to commit new versions of this
  package, you do not need to fill in the user and password.
  \button{Open} the repository, select the latest version and click
  \button{Load}.}

Pressing the \link{Register Member} link on the \sqsrc home page will probably be your first step if you do not have a \sqsrc account. 
Once you are a member, \link{Register Project} allows you to create a new project. 


\begin{figure}[ht]\centering
	\includegraphics[width=\textwidth]{squeaksourcesetting}
	\caption{Repositories under \sqsrc are highly configurable.\figlabel{squeaksourcesetting}}
\end{figure}

\Mont offers a large range of options (cf. \figref{squeaksourcesetting}) to configure a project repository: tags may be assigned, a license may be chosen, access for people who are not part of the project may be restricted (read/write, read, no access), emails may be sent upon commits, mailing list may be managed, and users may be defined to be members of the project (as administrator, developer, or guest).

% \on{This section seems too painfully obvious to include.}
%\paragraph{Troubleshooting} Not being able to remotely save your package may be due to a large range of causes. You will have to run through the usually network debugging routine:

%\begin{enumerate}
%\item Is your computer turned on?
%\item Is your network cable plugged in?
%\item Can you reach \url{http://squeaksource.com/} with a web browser from the same host as the image?
%\item Are people on IRC (or any other forum) ranting about \sqsrc being down again?
%\item Is \sqsrc accessible from other sites?\footnote{A handy link to check this is: \url{http://downforeveryoneorjustme.com/squeaksource.com}}
%\item Can you ping \url{http://squeaksource.com/} from Pharo?:\\
%   \ct{Socket pingPorts: #(80) on: 'squeaksource.com' timeOutSecs: 30}.
%\item Can you load packages using \Mont from \\\url{http://squeaksource.com/}?
%\end{enumerate}



%=================================================================
\section{The .mcz file format}

\index{Monticello!version}
\index{Monticello!mcz format}
\index{mcz format|see{Monticello, mcz format}}
Versions are stored in repositories as binary files.
These files are commonly call ``mcz files'' as they carry the extension .mcz.
This stands for ``\Mont zip'' since an mcz file is simply a zipped file containing the source code and other meta-data.

\important{An mcz file can be dragged and dropped onto an open image file, just like a change set. \pharo will then prompt you to ask if you want to load the package it contains. \Mont will not know which repository the package came from, however, so do not use this technique for development.}

% \sd{but you lose the fact that this is a package so don't do that.}

You may try to unzip such a file, for example to view the source code directly, but normally end users should not need to unzip these files themselves.
If you unzip it, you will find the following members of the mcz file.

\paragraph{File contents}
Mcz files are actually ZIP archives that follow certain conventions. Conceptually a version contains four things:

\begin{itemize}
\index{Monticello!package}
\item \emph{Package}. A version is related to a particular package. Each mcz file contains a file called ``package'' that contains information about the package's name.

\item \emph{VersionInfo}. This is the meta-data about the snapshot. It contains the author initials, date and time the snapshot was taken, and the ancestry of the snapshot. Each mcz file contains a member called ``version'' which contains this information.

A version doesn't contain a full history of the source code. It's a snapshot of the code at a single point in time, with a UUID identifying that snapshot, and a record of the UUIDs of all the previous snapshots it's descended from.

\index{Monticello!snapshot}
\item \emph{Snapshot}. A Snapshot is a record of the state of the package at a particular time. Each mcz file contains a directory named ``snapshot/''. All the members in this directory contain definitions of program elements, which when combined form the Snapshot. Current versions of \Mont only create one member in this directory, called ``source.st''.

\index{Monticello!dependency}
\item \emph{Dependencies}. A version may depend on specific version of other packages. An mcz file may contain a ``dependencies/'' directory with a member for each dependency. These members will be named after each package the \Mont package depends upon. For example, a \ct{Pier-All} mcz file will contains files named \ct{Pier-Blog} and \ct{Pier-Caching} in its dependencies directory.
\end{itemize}

\paragraph{Source code encoding}

The member named ``snapshot/source.st'' contains a standard fileout of the code that belongs to the package.

\paragraph{Metadata encoding}

The other members of the zip archive are encoded using S-expressions. Conceptually, the expressions represent nestable dictionaries. Each pair of elements in a list represent a key and value. For example, the following is an excerpt of a ``version'' file of a package named \ct{AA}:

\ct{(name 'AA-ab.3' message 'empty log message' date '10 January 2008' time '10:31:06 am' author 'ab' ancestors ((name 'AA-ab.2' message...)))}

It basically says that the version \ct{AA-ab.3} has an empty log message, was created on \ct{January 10, 2008}, by \ct{ab}, and has an ancestor named \ct{AA-ab.2}, ...

%\ab{butif it does?}

%\paragraph{Distributing mcz files}

%The metadata for a version ends up being fairly compact, so it's not unreasonable to distribute it with a release. \ab{What's a release?} It's also important that it be present if somebody decides to start hacking on your package \on{?}. Then they can create a mcz with their version of your package and it will have the correct ancestry information, enabling you to easily and correctly merge it back into your work.

%Stated another way,  So it's a perfect thing to distribute. \ab{This paragraph I understand.  Maybe just delete the previous one?  What is the connection with the paragraph title?}

%=================================================================
\section{Chapter Summary}

%This chapter explained how to use \Mont and \sqsrc to manage the source code of your application.

This chapter has presented the functionality of \Mont in detail.
The following points were covered:

\begin{itemize}
\item \Mont are mapped to Smalltalk categories and method protocols.
	If you add a package called \ct{Foo} to \Mont, it will include all classes in categories called \ct{Foo} or starting with \ct{Foo-}. It will also include all methods in those categories, except those in protocols starting with \ct{*}. Finally it will include all \emph{class extension} methods in protocols called \ct{*foo} or starting with \ct{*foo-} anywhere else in the system.

\item When you modify any methods or classes in a package, it will be marked as ``dirty'' in \Mont, and can be saved to a repository.

\item There are many kinds of repositories, the most popular being HTTP repositories, such as those hosted by \sqsrc.

\index{package-cache}
\item Saved packages are caches locally in a directory called \ct{package-cache}.

\item The \Mont \RI can be used to browse a repository. You can select which versions of packages to load or unload.

\item You can create a new \emph{branch} of a package by basing a new version on another version which is earlier than the latest version. The \RI keeps track of the ancestry of packages and can tell you which versions belong to separate branches.

\item Branches can be \emph{merged}. \Mont offers a fine degree of control over the resolution of conflicts between merged versions. The merged version will have as its ancestor the two versions it was merged from.

\item Alternatively, selected changes of a branch can be \emph{backported} to an arbitrary earlier version. This will create a new version that can be merged with any other version that needs those changes. The original backported branch remains independent in this case.

\item \Mont can keep track of dependencies between packages. When a package with dependencies to required packages is saved, a new version of that package is created, which then depends on the latest versions of all the required packages.

\item If classes in your packages have class-side \ct{initialize} methods, then \ct{initialize} will be sent to those classes when your package is loaded. This mechanism can be used to perform various checks or start-up actions. A particularly useful application is to add new instance variables to classes for which you are defining extension methods.

\item \Mont stores package versions in a special zipped file with the file extension \ct{.mcz}. The mcz file contains a snapshot of the complete source code of that version of your package, as well as files containing other important metadata, such a package dependencies.

\item You can drag and drop an mcz file onto your image as a quick way to load it.

%\item \Mont is used to manage your packages. \Mont is part of the Pharo standard distribution.

%\item \sqsrc is a remote online central repository in which you can store your code.

%\item \Mont packages are mapped to categories. 

%\item After having created a package in \Mont, you may store it in a remote repository such as that offered by \sqsrc.

%\item Merging and backporting are feature provided by \Mont that allow changes to cross different version line and to move along a same package version line.

%\item \Mont allows a package version different from the local version to be merged, creating a new branch and changing the local version of your package.
%\on{confusing sentence --- not sure what you want to say}
%\ab{ditto}
\end{itemize}

%ACK: Dale Henrichs
%=================================================================

%\section{Packages in \Mont: PackageInfo}

%The PackageInfo system is a simple, lightweight way of organizing Smalltalk source: it is nothing more than a naming convention, which uses (or abuses) the existing categorization mechanisms to group related code. Let me give you an example: say that you are developing a framework named PharoLink to facilitate using relational databases from Pharo. You will probably have a series of categories to contain all of your classes (e.g., category \cat{PharoLink-Connections} containing the classes \ct{OracleConnection}, \ct{MySQLConnection} and \ct{PostgresConnection})
%(\cat{PharoLink-Model} containing \ct{DBTable}, \ct{DBRow}  and \ct{DBQuery}) and so on. But not all of your code will reside in these classes\,---\,you may also have, for example, a series of methods to convert objects into an SQL friendly format: \mthind{Object}{asSQL},  \mthind{String}{asSQL} and \mthind{Date}{asSQL}.

%These methods belong in the same package as the classes in \cat{PharoLink-Connections} and \cat{PharoLink-Model}. You mark this by placing those methods in a method category (of \ct{Object}, \ct{String}, \ct{Date}, and so on) named \cat{*squeaklink} (note the initial star). The combination of the \cat{PharoLink-...} categories and the \cat{*squeaklink} method categories forms a package named "PharoLink".

%
%\section{Getting Started}

%\paragraph{Installing}

%

%\paragraph{Creating a Working Copy}

%The first thing you need to do is tell \Mont about the package you are interested in versioning. You do this by creating a Working Copy.

%\paragraph{From an .mcz version file}
%Open a FileList and navigate to the version file. Click on the 'Load' button to load the package into your image.

%\paragraph{From scratch}

%Click on the '+Package' button, and enter the name of a PackageInfo package. It doesn't matter whether or not the code for the package already exists.

%Once the Working Copy has been created, the name of the package will appear in the package list on the left side of the \MCB. If you loaded an existing version, the version name will be displayed in parenthesis after the package name, otherwise the parenthesis will be empty, indicating that your working copy has no ancestors.

%\paragraph{Connecting to a Repository}

%If you've already got a Working Copy, click on the package name on the left side of the \MCB, so that your repository will be associated with your package. To connect to a repository, click on the '+Repository' button in the \MCB. A pop-up menu will appear, allowing you to select the type of repository you want to connect to.

%The simplest repository type is 'directory.' When you select this type of repository, \Mont will open a FileList2 to allow you to select an existing directory in which to store versions. Other types of repositories typically require more configuration, and will open a text pane to allow you to enter it.

%\paragraph{Saving Changes}

%Changes to your working copy are automatically logged in your changes file, so you only need to create a new version of your package when you want to share the changes with others. Select the package on the left side of the \MCB and the repository to save to on the right, then click the 'Save' button. See Repositories for discussion of how to publish to shared repositories.

%\paragraph{Merging Changes}

%If you or some other developer have made changes to the same version of a package, load one version as your working set and then select the repository containing the other version in the \MCB, open a Repository Inspector and select the other version. Clicking the 'Merge' button will automatically load all non-conflicting changes from the other version. If you need to control which changes to accept, you may instead click 'Changes' to browse every difference.

%

%\section{Elements of \Mont}

%\paragraph{Packages}

%Packages are the units of versioning used by \Mont; the classes and methods they contain are recorded and versioned together. \Mont uses the packages defined by PackageInfo.

%\paragraph{Snapshots}

%A Snapshot is the state of a Package at a particular point in time

%\paragraph{Versions}

%A Version is a Snapshot of a Package and it's associated metadata\,---\,author initials, the date and time the snapshot was taken, and the Version's ancestry\,---\,the list of Versions from which it is derived.

%A Version is the standard currency of the system. You save them, load them, give them to others, merge them, delete... you get the picture. Versions are often stored in mcz files\,---\,see File Format

%\paragraph{Working Copies}

%Each package in an image that is being versioned with \Mont has a Working Copy. The Working Copy represents the Version of the package that is currently active in the image, and which may be modified by the Smalltalk development tools.

%\paragraph{Repositories}

%These are places to store your Versions. Unlike CVS, in which a Package is associated with one Repository, a \Mont Package can have Versions in many repositories. When adding a new Repository to use, you can choose from \sqmap Cache, FTP, HTTP (webdav), \sqmap Release, SMTP, or a directory somewhere on your hard drive (or network drive).

%For example, if I have six versions of package Foo, I could have Foo versions 1-4 being on my local harddrive, and 5-6 being on an ftp server. You could download version 5, make some changes and commit a new version (7) to your WebDAV repository. I can download and merge that version with my own work to produce version 8, which I save to my ftp repository.

%This is a key element of \Mont's distributed development model.

%\paragraph{Package cache}

%The package-cache is a local repository the \Mont uses to cache any package that is loaded into a particular image in a directory. That means it is filled with .mcz files, whether it is a package you create in your image, or one you download from somewhere else.

%When you use images in different directories you will have multiple package-caches, and may hold many of the same packages. If \Mont is loaded into an image which is subsequently moved, \Mont will continue to use the package-cache in the directory the image was moved from. Otherwise \Mont creates a new package-cache in the local directory. This can become a real mess and so some have used symlinks on unix systems to centralize it.

%\paragraph{Why cache packages at all?}

%When a Version is loaded into the image, it is likely to become the ancestor of new versions that are created as part of the development process. During merges, \Mont needs to examine the Snapshots of these ancestors in order to detect conflicts. By caching these ancestors as it loads them, \Mont reduces the chance that the necessary version will be unavailable\,---\,either because the repository it's in is no longer available or because it was loaded directly from a file and isn't in any repository.

%\section{The Snapshot Browser}

%The Snapshot browser is much like the standard Smalltalk Browser except that it displays the contents of a Snapshot, rather than the code that is active in the image. Since Snapshots are immutable, the Snapshot browser does not allow editiing.

%One difference between the Snapshot Browser and the familiar browsers is that the Snapshot browser uses the special category '*Extensions' to categorize classes that do not belong to the package, but which have extension methods that do.
%
%\section{More on PackageInfo}
%To get a feel for this, try filing the Refactoring Browser. The Refactoring Browser code uses PackageInfo's naming conventions, using "Refactory" as the package name. In a workspace, create a model of this package with  \ct{refactory := PackageInfo named: 'Refactory'}. 

%It is now possible to introspect on this package; for example, refactory classes will return the long list of classes that make up the Refactoring Browser. refactory coreMethods will return a list of MethodReferences for all of the methods in those classes. refactory extensionMethods is perhaps one of the most interesting queries: it will return a list of all methods contained in the Refactory package but not contained within a Refactory class. This includes, for example, \mthind{String}{expandMacrosWithArguments:} and \mthind{Behavior}{parseTreeFor:}.

%Since the PackageInfo naming conventions are based on those used already by \pharo, it is possible to use it to perform analysis even of code that has not explicitly adapted to work with it. For example, (PackageInfo named: 'Collections') externalSubclasses will return a list of all Collection subclasses outside the Collections categories.

%You can send fileOut to an instance of PackageInfo to get a changeset of the entire package. For more sophisticated versioning of packages, see the \Mont project.

%=========================================================
\ifx\wholebook\relax\else
   \bibliographystyle{jurabib}
   \nobibliography{scg}
   \end{document}
\fi
%=========================================================

