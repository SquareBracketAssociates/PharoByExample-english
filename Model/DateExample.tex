% $Author$
% $Date$
% $Revision$
%=================================================================
\ifx\wholebook\relax\else
% --------------------------------------------
% Lulu:
	\documentclass[a4paper,10pt,twoside]{book}
	\usepackage[
		papersize={6in,9in},
		hmargin={.75in,.75in},
		vmargin={.75in,1in},
		ignoreheadfoot
	]{geometry}
	\input{../common.tex}
	\pagestyle{headings}
	\setboolean{lulu}{true}
% --------------------------------------------
% A4:
%	\documentclass[a4paper,11pt,twoside]{book}
%	\input{../common.tex}
%	\usepackage{a4wide}
% --------------------------------------------
    \graphicspath{{figures/} {../figures/}}
	\begin{document}
	\renewcommand{\nnbb}[2]{} % Disable editorial comments
	\sloppy
\fi
%=================================================================


\chapter{Case Study}

\subsubsection{A Case Study: The Class \ct{Date}.}
A date is an object representing the date. \ct{Date today} returns the object that represents the current day. If we ask 
this object to print itself, we obtain, for example, '27 November 2002'. %%%/// update this?
\figref{dateToday} shows a date object in
an inspector obtained using the expression \ct{Date today inspect}. What we see is that a date object  records only a number
of days. There are no instance variables representing the month names, the day names, the number of days per month, and so on.
In fact, such information is shared by all the instances of the class and is then represented by the class variables 
\ct{DaysInMonth FirstDayOfMonth MonthNames  SecondsInDay WeekDayNames} %%%/// should there be commas between these names?
 of the \ct{Date} class definition, as shown in  
\clsref{cls:dateclass}.


\begin{classdef}{}\label{cls:dateclass}
Magnitude subclass: \#Date
   instanceVariableNames: 'julianDayNumber '
   \ct{classVariableNames: 'DaysInMonth FirstDayOfMonth MonthNames 
     SecondsInDay WeekDayNames '}
   poolDictionaries: ''
   category: 'Kernel-Magnitudes'
\end{classdef}


All instance methods of the class \ct{Date} (and subclasses) can directly access the  class variables defined by the class 
\ct{Date}, as shown by Method \ct{mth:monthName}, where the method \ct{monthName} accesses the class variable
\ct{MonthNames}.

\begin{method}{}\label{mth:monthName}
Date>>>monthName
   "Answer the name of the month in which the receiver falls."

   ^MonthNames at: self monthIndex
\end{method}

In a similar fashion, all class methods can access class variables. The class \mthref{nameOfDay} \ct{nameOfDay:} 
accesses the class variable \ct{WeekDayNames}.

\begin{method}{}\label{mth:nameOfDay}
Date class>>>nameOfDay: dayIndex 
   "Answer a symbol representing the name of the day indexed by dayIndex, 1-7."

   ^WeekDayNames at: dayIndex
\end{method}


%%%/// Should this go after or in the paragraph below called "pool dictionaries"?
Pool dictionary are really static concepts, %%%/// Unclear. "A pool dictionary is a static concept"?  
 and they %%%/// "and it"? 
should be defined before a method
can use them.  %%%/// can use it?
We strongly encourage you not to use them. %%%/// not to use pool dictionaries?

%=================================================================
\ifx\wholebook\relax\else
\end{document}\fi
%=================================================================
